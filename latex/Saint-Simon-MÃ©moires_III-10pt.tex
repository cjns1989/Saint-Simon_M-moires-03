\PassOptionsToPackage{unicode=true}{hyperref} % options for packages loaded elsewhere
\PassOptionsToPackage{hyphens}{url}
%
\documentclass[oneside,10pt,french,]{extbook} % cjns1989 - 27112019 - added the oneside option: so that the text jumps left & right when reading on a tablet/ereader
\usepackage{lmodern}
\usepackage{amssymb,amsmath}
\usepackage{ifxetex,ifluatex}
\usepackage{fixltx2e} % provides \textsubscript
\ifnum 0\ifxetex 1\fi\ifluatex 1\fi=0 % if pdftex
  \usepackage[T1]{fontenc}
  \usepackage[utf8]{inputenc}
  \usepackage{textcomp} % provides euro and other symbols
\else % if luatex or xelatex
  \usepackage{unicode-math}
  \defaultfontfeatures{Ligatures=TeX,Scale=MatchLowercase}
%   \setmainfont[]{EBGaramond-Regular}
    \setmainfont[Numbers={OldStyle,Proportional}]{EBGaramond-Regular}      % cjns1989 - 20191129 - old style numbers 
\fi
% use upquote if available, for straight quotes in verbatim environments
\IfFileExists{upquote.sty}{\usepackage{upquote}}{}
% use microtype if available
\IfFileExists{microtype.sty}{%
\usepackage[]{microtype}
\UseMicrotypeSet[protrusion]{basicmath} % disable protrusion for tt fonts
}{}
\usepackage{hyperref}
\hypersetup{
            pdftitle={SAINT-SIMON},
            pdfauthor={Mémoires III},
            pdfborder={0 0 0},
            breaklinks=true}
\urlstyle{same}  % don't use monospace font for urls
\usepackage[papersize={4.80 in, 6.40  in},left=.5 in,right=.5 in]{geometry}
\setlength{\emergencystretch}{3em}  % prevent overfull lines
\providecommand{\tightlist}{%
  \setlength{\itemsep}{0pt}\setlength{\parskip}{0pt}}
\setcounter{secnumdepth}{0}

% set default figure placement to htbp
\makeatletter
\def\fps@figure{htbp}
\makeatother

\usepackage{ragged2e}
\usepackage{epigraph}
\renewcommand{\textflush}{flushepinormal}

\usepackage{indentfirst}
\usepackage{relsize}

\usepackage{fancyhdr}
\pagestyle{fancy}
\fancyhf{}
\fancyhead[R]{\thepage}
\renewcommand{\headrulewidth}{0pt}
\usepackage{quoting}
\usepackage{ragged2e}

\newlength\mylen
\settowidth\mylen{...................}

\usepackage{stackengine}
\usepackage{graphicx}
\def\asterism{\par\vspace{1em}{\centering\scalebox{.9}{%
  \stackon[-0.6pt]{\bfseries*~*}{\bfseries*}}\par}\vspace{.8em}\par}

\usepackage{titlesec}
\titleformat{\chapter}[display]
  {\normalfont\bfseries\filcenter}{}{0pt}{\Large}
\titleformat{\section}[display]
  {\normalfont\bfseries\filcenter}{}{0pt}{\Large}
\titleformat{\subsection}[display]
  {\normalfont\bfseries\filcenter}{}{0pt}{\Large}

\setcounter{secnumdepth}{1}
\ifnum 0\ifxetex 1\fi\ifluatex 1\fi=0 % if pdftex
  \usepackage[shorthands=off,main=french]{babel}
\else
  % load polyglossia as late as possible as it *could* call bidi if RTL lang (e.g. Hebrew or Arabic)
%   \usepackage{polyglossia}
%   \setmainlanguage[]{french}
%   \usepackage[french]{babel} % cjns1989 - 1.43 version of polyglossia on this system does not allow disabling the autospacing feature
\fi

\title{SAINT-SIMON}
\author{Mémoires III}
\date{}

\begin{document}
\maketitle

\hypertarget{chapitre-premier.}{%
\chapter{CHAPITRE PREMIER.}\label{chapitre-premier.}}

1700

~

{\textsc{Tallard à Fontainebleau.}} {\textsc{- Conseil d'État d'Espagne
et quelques autres seigneurs.}} {\textsc{- Réflexions et mesures de
quelques-uns des principaux seigneurs sur les suites de la mort
prochaine du roi d'Espagne.}} {\textsc{- Avis célèbre sur les
renonciations de la reine Marie-Thérèse.}} {\textsc{- Chute de la reine
d'Espagne.}} {\textsc{- Le pape consulté secrètement.}}

~

Les nouvelles d'Espagne devenaient de jour en jour plus intéressantes
depuis le départ du marquis d'Harcourt et son arrivée à Paris, où il
rongeait son frein de n'avoir pas eu la liberté de traiter avec la reine
par l'amirante\footnote{Amiral de Castille. Il se nommait Thomas
  Enriquez de Cabrera, comte de Melgar.}, et de s'ouvrir ainsi le chemin
d'une grande et prompte fortune, et enviait le bonheur de Tallard qui
était arrivé de la Haye à Paris pour aller bientôt après retrouver le
roi d'Angleterre à son retour de Hollande à Londres, et qui se donnait
l'honneur du traité de partage qu'il avait signé avec ce prince, comme
d'un chef-d'œuvre de politique dont il était venu à bout, tandis que le
roi d'Angleterre, qui se moquait de lui, s'applaudissait avec raison de
l'avoir imaginé, et d'être parvenu à le faire accepter à la France, et
d'y avoir engagé tous ses anciens alliés, excepté l'empereur qu'il
espérait toujours d'y ramener. Qui aurait en effet mis ce traité en
avant, et l'eût poussé jusqu'où il le fut dans les vues d'en tirer le
fruit prodigieux qu'il vint à produire, eût été un profond et habile
politique. Mais le roi d'Angleterre qui l'avait imaginé, quelque grand
homme d'État qu'il fût, était bien loin d'en attendre un succès si
funeste à ce qu'il s'en était proposé, et Tallard qui se faisait honneur
de l'invention d'autrui, et qui n'y avait eu d'autre part que celle d'en
avoir reçu les premières propositions en Angleterre, et sur le compte
qu'il en rendit d'avoir suivi les ordres qu'il reçut d'aller en avant,
et enfin de signer, était tout aussi éloigné de penser qu'il pouvait
produire autre chose que son exécution\,; et il faut avouer que ce sont
de ces secrets de la Providence toute seule qui dispose des empires,
comme, quand et en la manière qu'il lui plaît, par des voies si
profondes et si peu possibles à attendre par ceux même qui par degrés
les exécutent, qu'il ne faut pas s'étonner si toute vue et toute
prudence humaines est demeurée dans les plus épaisses ténèbres jusqu'au
moment de l'événement.

Harcourt, à qui on voulait éviter de commettre son caractère à quelque
chose peut-être de fâcheux, n'avait pas plutôt donné avis à Blécourt de
son entrée en France, que cet envoyé du roi alla faire à l'Escurial la
déclaration du traité de partage au roi d'Espagne. On a vu plus haut
l'extrême colère où ce prince entra à une nouvelle pour lui si odieuse,
les plaintes qu'il en fit retentir par ses ministres dans toute
l'Europe, et en particulier en quels termes son ambassadeur à Londres se
plaignit du roi d'Angleterre, lors en Hollande, et les suites de
l'aigreur de cette plainte. Le conseil d'Espagne s'assembla souvent pour
délibérer sur une déclaration si importante, qu'elle réveilla ceux qui
le composaient de cet assoupissement profond qui, hors Madrid et ce qui
s'y passe, rend les grands seigneurs espagnols indifférents à tout le
reste du monde. La première marque qu'il en donna fut de supplier le roi
d'Espagne de trouver bon que, pour ménager sa santé et n'entendre pas si
souvent discuter des choses qui ne pouvaient que lui faire peine, il
s'assemblât hors de sa présence aussi souvent qu'il le jugerait
nécessaire pour lui rendre un compte abrégé des résolutions qu'il
estimerait devoir être prises, et des ordres en conséquence à lui
demander.

Portocarrero\footnote{Cette énumération des membres du conseil d'État
  d'Espagne a été supprimée dans les précédentes éditions. Elle est
  cependant indispensable, comme le reconnaîtra tout lecteur attentif,
  pour comprendre la suite du récit de Saint-Simon\,; en effet il se
  réfère plusieurs fois au tableau qu'il a tracé du conseil d'État
  d'Espagne en 1700. Les éditeurs ont peut-être cru (car on doit
  chercher une explication raisonnable de cette étrange omission) que
  Saint-Simon traitait de ce conseil, en 1721, à l'époque de son
  ambassade en Espagne. Il parle, en effet, du conseil d'État de cette
  époque, mais il a soin d'ajouter à l'occasion d'Ubilla\,: «\,Il avait
  eu le sort commun à tous ceux à qui Philippe V avait obligation de sa
  couronne, que la princesse des Ursins fit chasser.\,» Aussi le conseil
  d'État, dont il est question à la date de 1721, n'a-t-il rien de
  commun avec celui de 1700.}, Génois de la maison Boccanegra, mais
depuis longtemps établie en Espagne par le mariage d'une héritière de la
maison Portocarrero, qui, suivant la coutume d'Espagne, lui avait imposé
son nom et ses arme, était à la tête de ce conseil comme cardinal,
archevêque de Tolède, primat et chancelier des Espagnes et diocésain de
Madrid. Il était oncle paternel du comte Palma, grand d'Espagne\,;

Don J. Thomas Enriquez, duc de Rioseco, comte de Melgar, amirante de
Castille, qui avait été gouverneur de Milan\,;

Don Fr.~Benavidès, comte de S. Estevan del Puerto, qui avait été
vice-roi de Sardaigne, de Sicile et de Naples\,;

Don Joseph Fréd. de Tolède, marquis de Villafranca, majordome-major du
roi, avait été vice-roi de Sicile\,;

Don Pierre-Emmanuel de Portugal-Colomb, duc se Veragua, chevalier de la
Toison d'or, vice-roi de Sardaigne et de Sicile, où il était lors.

Ces quatre, grands d'Espagne, et le cinquième à vie\,:

Don Antoine-Sébastien de Tolède, marquis de Mancera\,;

Don Manuel Arias, commandeur de Castille, de Saint-Jean de Jérusalem,
gouverneur du conseil de Castille\,;

Don Antonio Ubilla, secrétaire des dépêches universelles\,;

Le comte d'Oropesa, de la maison de Portugal, président des conseils de
Castille et d'Italie, était exilé, et le duc de Medina-Celi était
vice-roi de Naples.

Outre ces conseillers d'État, comme on parle en Espagne\footnote{On
  appelait alors en France les conseillers d'État \emph{conseillers du
  roi en ses conseils. Voy}. note à la fin du Ier volume de cette
  édition de \emph{Saint-Simon}.}, il faut parler ici de trois autres
grands d'Espagne et d'un seigneur de la maison de Guzman, marquis de
Villagarcias, vice-roi de Valence, qui se trouva lors à Madrid. Les
trois grands sont\,:

Le marquis de Villena, duc d'Escalona (don J. Fernandez d'Acuña
Pacheco), chevalier de la Toison d'or, qui avait été vice-roi de
Navarre, d'Aragon, de Catalogne, où nous l'avons vu bien battu sur le
Ter par M. de Noailles et encore après par M. de Vendôme pendant le
siège de Barcelone, enfin {[}vice-roi{]} de Sicile. Il est mort longues
années depuis majordome-major, et son fils lui a succédé dans cette
grande charge, chose très rare en Espagne. J'aurai lieu plus d'une fois
de parler de lui.

Le duc de Medina-Sidonia (don J. de Guzman), majordome-major du roi.

Le comte de Benavente (don Fr.~Ant. Pimentel), sommelier de corps.

Ces deux derniers, ainsi que le cardinal Portocarrero, ont eu depuis
l'ordre du Saint-Esprit. Il {[}Benavente{]} était aîné de la maison de
Pimentel.

Don Louis Fernandez Boccanegra, cardinal Portocarrero, promu par Clément
IX (5 août 1669), à trente-huit ans, et depuis archevêque de Tolède,
était un grand homme tout blanc, assez gros, de bonne mine, avec un air
vénérable, et toute sa figure noble et majestueuse, honnête, poli,
franc, libre, parlant vite, avec beaucoup de probité, de grandeur, de
noblesse. Le sens bon et droit, avec un esprit et une capacité fort
médiocres, une opiniâtreté entêtée, assez politique, excellent ami,
ennemi implacable\,; un grand amour pour sa maison et tous ses parents,
et voulant tout faire, tout gouverner, ardent en tout ce qu'il voulait,
et sur le tout dévot, haut et glorieux, et quoique grand autrichien,
ennemi de la reine et de tous les siens, et déclaré tel\footnote{Voy.,
  dans les notes à la fin du volume les portraits des principaux
  personnages d'Espagne.}.

L'amirante, dévoué à la fortune, avec beaucoup d'esprit, de monde et de
talents, mais décrié sur tous les chapitres, était l'homme d'Espagne le
plus attaché à la reine.

San-Estevan avait beaucoup d'esprit et de capacité et assez de droiture,
extrêmement rompu au monde et à la cour, et avait souvent des propos et
des reparties fort libres et fort plaisantes, d'un esprit fin, doux,
liant, et sans aucune haine ni vengeance, et d'une dévotion solide et
cachée, peu ou point attaché aux étiquettes d'Espagne ni à ses maximes.
Il avouait franchement sa passion extrême pour sa famille et pour ses
parents les plus éloignés. En tout, c'était un homme d'État. Son fils a
été plénipotentiaire d'Espagne à Cambrai, puis gouverneur et premier
ministre du roi de Naples, chevalier du Saint-Esprit, et maintenant en
Espagne président des ordres et grand écuyer du roi. Le père mourut
majordome-major de la reine-Savoie\footnote{Louise de Savoie, première
  femme de Philippe V.}.

Veragua, avec infiniment d'esprit, était un homme capable, mais d'une
avarice sordide, de peu de courage dans l'âme et à qui personne ne se
fiait, et qui lors était en Sicile vice-roi.

Villafranca, chef de la maison de Tolède, était un homme de soixante-dix
ans, Espagnol jusques aux dents, attaché aux maximes, aux coutumes, aux
mœurs, aux étiquettes d'Espagne jusqu'à la dernière minutie\,;
courageux, haut, fier, sévère, pétri d'honneur, de valeur, de probité,
de vertu, un personnage à l'antique, généralement aimé, considéré,
respecté, sans aucuns ennemis, fort révéré et aimé du peuple, et, avec
ce que j'en vais dire, d'un esprit médiocre.

Arias était monté à ce haut degré de conseiller d'État, le \emph{non
plus ultra} d'Espagne pour le personnel, par son esprit vaste, juste,
net, capable, ferme, hardi. C'était un vrai homme d'État, fort Espagnol
dans son goût et dans toutes ses manières, grand homme de bien, qui
aimait fort la justice, et en tout grand ennemi de toutes voies
obliques, et austère dans ses mœurs.

Ubilla était un homme de peu, comme tous ceux qui occupent les premières
secrétaireries en Espagne. Il était arrivé à l'Universelle\footnote{À la
  secrétairerie universelle, sorte de ministère d'État.} par s'être
distingué dans divers endroits importants. Il avait l'esprit souple,
poli, délié, fin, avec cela ferme, net et voyait clair avec grande
capacité et pénétration dans les affaires, intègre pour un homme élevé
dans ces emplois-là, et uniquement attaché au bien, à la grandeur et à
la conservation de la monarchie.

J'oubliais le vieux Mancera, de la maison de Tolède, qui avait été
ambassadeur à Venise et en Allemagne, puis vice-roi de la
Nouvelle-Espagne, à son retour majordome-major de la reine mère, enfin
conseiller d'État. C'était encore un personnage à l'antique, en mœurs,
en vertu, en désintéressement, en fidélité, en attachement à ses
devoirs, avec une piété effective et soutenue, sans qu'il y parût, doux,
accessible, poli, bon, avec l'austérité et l'amour de toutes les
étiquettes espagnoles. C'était un homme qui pesait tout avec jugement et
discernement, et qui, une fois déterminé par raison à un parti, y était
d'une fidélité à toute épreuve, savant avec beaucoup d'esprit et le plus
honnête homme qui fût en Espagne.

Outre ce conseil d'État, que je n'ai pas rangé dans l'exactitude du rang
ni parlé de tous ses membres, il y avait encor\,; quelques seigneurs
dont les grands emplois ne permettaient pas qu'il se délibérât rien
d'aussi important sur la monarchie sans eux. Tels étaient le duc de
Medina-Sidonia, l'aîné des Guzman, majordome-major du roi\,; le comte de
Benavente, l'aîné des Pimentel, sommelier du corps\,; don Fernand de
Moncade, dit d'Aragon, duc de Montalte, président des conseils d'Aragon
et des Indes\,; don Nicolas Pignatelli, duc de Monteléon, chevalier de
la Toison, qui a été vice-roi de Sardaigne, et un des plus grands
seigneurs des royaumes de Naples et de Sicile\,; et le marquis de
Villena ou duc d'Escalona, par son rare mérite et les grands emplois par
lesquels il avait passé.

Medina-Sidonia\footnote{Nouveau passage supprimé par les précédents
  éditeurs.} était un homme très bien fait, d'environ soixante ans, qui
ne manquait pas d'esprit, vrai courtisa a, complaisant, liant, assidu,
fort haut, très glorieux\,; en même temps très poli, libéral,
magnifique, ambitieux à l'excès et d'une probité peu contraignante, de
ces hommes enfin à qui il ne manque rien pour cheminer et pour arriver
dans les cours, et grand autrichien. Il était aîné de la maison de
Guzman.

Benavente, fort bon homme et le meilleur des hommes, sans esprit, sans
talent aucun, mais plein d'honneur, de droiture, de probité et de piété.

Montalte, homme d'esprit, de courage, de capacité et d'une foi suspecte,
mais qui en savait plus qu'aucun, fort autrichien, profond dans ses vues
et dans ses voies, que tous regardaient mais sans se fier en lui.

Monteléon, italien jusque dans les moelles et autrichien de même,
c'est-à-dire tout plein d'esprit, de sens, de vues, et, au besoin, de
perfidie, avec beaucoup de capacité et des dehors fort agréables, mais
trop connu pour que personne osât lui faire aucune ouverture ni qu'on
pût jamais compter sur lui. Il avait épousé la petite-fille et héritière
de cette duchesse de Terranova qui fut camarera-mayor de la reine, fille
de Monsieur, à qui elle donna tant de déplaisirs, et qui à la fin se la
fit ôter, chose sans exemple en Espagne\,; et qui l'a fait duc de
Terranova.

Escalona, mais qui plus ordinairement portait le nom de Villena, était
la vertu, l'honneur, la probité la foi, la loyauté, la valeur, la piété,
l'ancienne chevalerie même, je dis celle de l'illustre Bayard, non pas
celle des romans et des romanesques. Avec cela beaucoup d'esprit, de
sens, de conduite, de hauteur et de sentiment, sans gloire et sans
arrogance, de la politesse, mais avec beaucoup de dignité, et par mérite
et sans usurpation, le dictateur perpétuel de ses amis, de sa famille,
de sa parenté, de ses alliances qui tous et toutes se ralliaient à lui.
Avec cela beaucoup de lecture, de savoir, de justesse et de discernement
dans l'esprit, sans opiniâtreté, mais avec fermeté, fort désintéressé,
toujours occupé, avec une belle bibliothèque et commerce avec force
savants dans tous les pays de l'Europe, attaché aux étiquettes et aux
manières d'Espagne, sans en être esclave, en un mot un homme du premier
mérite, et qui par là a toujours été compté, aimé, révéré beaucoup plus
que par ses grands emplois, et qui a été assez heureux pour n'avoir
contracté aucune tache de ses malheurs militaires en
Catalogne\footnote{Saint-Simon revient à l'époque de son ambassade en
  Espagne sur le marquis de Villena, mais sans répéter ce portrait, qui
  méritait bien d'être conservé.}.

Enfin Villagarcias, qui n'était ni grand ni conseiller d'État mais qui
était Guzman, vice-roi de Valence, homme de beaucoup d'esprit et de
talent, qui se trouvait lors à Madrid, et parent proche et ami de
confiance de plusieurs conseillers d'État\footnote{Fin du second passage
  supprimé dans les précédentes éditions.}.

Villafranca fut un des premiers qui ouvrit les yeux au seul parti qu'ils
avaient à prendre pour empêcher le démembrement de la monarchie, et se
conserver par là toute leur grandeur particulière à eux-mêmes, en
demeurant sujets d'un aussi grand roi, qui, retenant toutes les parties
de tant de vastes États, aurait à conférer les mêmes charges, les mêmes
vice-royautés, les mêmes grâces\,: il songea donc à faire tomber
l'entière succession au deuxième fils du fils unique de la reine, sueur
du roi d'Espagne. Il s'en ouvrit comme en tâtonnant à Medina-Sidonia,
quoiqu'il ne fût pas du conseil, mais par sa charge et son esprit, en
grande figure et en faveur, et avec qui il était en liaison
particulière. Celui-ci qui le respectait et qui le savait aussi
autrichien que lui-même, mais qui était gouverné par son intérêt, et
qui, par conséquent, craignait sur toutes choses le démembrement de la
monarchie, entra dans le sentiment de Villafranca, et l'y affermit même
par son esprit et ses raisons. Ces dernières étaient claires\,: la
puissance de la France était grande et en grande réputation en Europe,
contiguë par mer et par terre de tous les côtés à l'Espagne, en
situation par conséquent de l'attaquer ou de la soutenir avec succès et
promptitude, tout à fait frontière des Pays-Bas, et en état d'ailleurs
de soutenir le Milanais, Naples et Sicile contre l'empereur faible,
contigu à aucun de ces États, éloigné de tout, et pour qui le continent
de l'Espagne se trouvait hors de toute prise, tandis que de tous côtés
il l'était de plain-pied à la France. Ils communiquèrent leur pensée à
Villagarcias et à Villena qui y entrèrent tout d'abord. Ensuite ils
jugèrent qu'il fallait gagner San-Estevan qui était la meilleure tête du
conseil\,: Villena était son beau-frère, mari de sa sueur et son ami
intime\,; Villagarcias aussi très bien avec lui\,; ils s'en chargèrent
et ils réussirent.

Voilà donc cinq hommes très principaux résolus à donner leur couronne à
un de nos princes. Ils délibérèrent entre eux, et ils estimèrent qu'ils
ne pourraient rien faire sans l'autorité du cardinal Portocarrero qui
portait ces deux pour le conseil où il était le premier et pour la
conscience par ses qualités ecclésiastiques. La haine ouverte et
réciproque déclarée entre la reine et lui leur en fit bien espérer. Il
était de plus ami intime de Villafranca et de toute la maison de Tolède.
Celui-ci se chargea de le sonder, puis de lui parler\,; et il le fit si
bien, qu'il s'assura tout à fait de lui. Tout cela se pratiquait sans
que le roi ni personne en France songeât à rien moins, et sans que
Blécourt en eût la moindre connaissance, et se pratiquait par des
Espagnols qui n'avaient aucune liaison en France, et par des Espagnols,
la plupart fort autrichiens, mais qui aimaient mieux l'intégrité de leur
monarchie, et leur grandeur et leurs fortunes particulières à eux que la
maison d'Autriche, qui n'était pas à la même portée que la France de
maintenir l'une et de conserver les autres. Ils sentaient néanmoins deux
grandes difficultés\,: les renonciations si solennelles et si répétées
de notre reine par la paix des Pyrénées et par son contrat de mariage
avec le roi, et l'opposition naturelle du leur à priver sa propre
maison, dans l'adoration de laquelle il avait été élevé, et dans
laquelle il s'était nourri lui-même toute sa vie, et la priver en faveur
d'une maison ennemie et rivale de la sienne dans tous les temps. Ce
dernier obstacle, ils ne crurent personne en état de le lever que le
cardinal Portocarrero par le for de la conscience.

À l'égard de celui des renonciations, Villafranca ouvrit un avis qui en
trancha toute la difficulté. Il opina donc que les renonciations de
Marie-Thérèse étaient bonnes et valables, tant qu'elles ne sortaient que
l'effet qu'on avait eu pour objet en les exigeant et en les accordant\,;
que cet effet était d'empêcher, pour le repos de l'Europe, que les
couronnes\,; de France et d'Espagne ne se trouvassent réunies sur une
même tête, comme il arriveraient sans cette sage précaution au cas où
allait tomber dans la personne du Dauphin\,; mais que, maintenant que ce
prince avait trois fils, le second desquels pouvait être appelé à la
couronne d'Espagne, les renonciations de la reine sa grand'mère
devenaient caduques, comme ne sortissant plus l'effet pour lequel
uniquement elles avaient été faites, mais un autre inutile au repos de
l'Europe, et injuste en soi, en privant un prince particulier sans États
et pourtant héritier légitime, pour en revêtir ceux qui ne sont ni
héritiers ni en aucun titre à l'égard du fils de France, effet encore
qui n'allait à rien moins qu'à la dissipation et la destruction totale
d'une monarchie, pour la conservation de laquelle ces renonciations
avaient été faite. Cet avis célèbre fut approuvé de tous, et Villafranca
se chargea de l'ouvrir en plein conseil. Il n'y avait donc encore que
Portocarrero, Villafranca, Villena, San-Estevan, Medina-Sidonia et
Villagarcias dans ce secret. Ils estimèrent avec raison qu'il devait
être inviolablement gardé entre eux jusqu'à ce que le cardinal eût
persuadé le roi. Les difficultés en étaient extrêmes.

Outre cette passion démesurée et innée de la grandeur de la maison
d'Autriche dans le roi d'Espagne, il avait fait un testament en faveur
de l'archiduc de la totalité de tout ce qu'il possédait au monde. Il
fallait donc lui faire détruire son propre ouvrage, le chef-d'œuvre de
son cour, la consolation de la fin prématurée de ses grandeurs
temporelles, en les laissant dans sa maison qu'il branchait de nouveau,
à l'exemple de Charles-Quint\,; et sur cette destruction enter pour la
maison de France, l'émule et l'ennemie perpétuelle de celle d'Autriche,
la même grandeur, la même mi-partition qu'il avait faite pour la sienne,
qui était la détruire de ses propres mains en tout ce qui lui était
possible, pour enrichir son ennemie de ses dépouilles et de toutes les
couronnes que la maison d'Autriche avait accumulées sur la tête de son
aîné. Il fallait lutter contre tout le crédit et la puissance de la
reine, si grandement établie, et de nouveau ulcérée contre la France qui
n'avait pas voulu que Harcourt écoutât rien de sa part par l'amirante.
Enfin c'était une trame qu'il fallait ourdir sous les yeux du comte
d'Harrach, ambassadeur de l'empereur, qui avait sa brigue dès longtemps
formée et les yeux bien ouverts.

Quels que fussent ces obstacles, la grandeur de leur objet les roidit
contre. Ils commencèrent par attaquer la reine par l'autorité du
conseil, qui se joignit si puissamment à la voix publique contre la
faveur et les rapines de la Berlips, sa favorite, que cette Allemande
n'osa en soutenir le choc dans l'état de dépérissement où elle voyait le
roi d'Espagne, et se trouva heureuse d'emporter en Allemagne les trésors
qu'elle avait acquis, pour ne s'exposer point aux événements d'une
révolution en un pays où elle était si haïe, et d'emmener sa fille, à
qui le dernier effort du crédit de la reine fut de faire donner une
promesse du roi d'Espagne par écrit d'un collier de la Toison d'or à
quiconque elle épouserait. Avec cela la Berlips partit à la hâte,
traversa la France, et se retira de façon qu'on n'en entendit plus
parler. C'était un coup de partie.

La reine, bonne et peu capable, ne pouvait rien tirer d'elle-même. Il
lui fallait toujours quelqu'un qui la gouvernât. La Berlips, pour régner
sur elle à son aise, s'était bien gardée de la laisser approcher,
tellement que, privée de cette favorite, elle se trouvait sans conseil,
sans secours et sans ressource en elle-même, et le temps selon toute
apparence trop court pour qu'une autre eût le loisir de l'empaumer assez
pour la rendre embarrassante pendant le reste de la vie du roi. Ce fut
pour achever de se mettre en liberté à cet égard que, de concert encore
avec le public qui gémissait sous le poids des Allemands du prince de
Darmstadt qui maîtrisaient Madrid et les environs, le conseil fit encore
un tour de force en faisant remercier ce prince et licencier ce
régiment. Ces deux coups et si près à près atterrèrent la reine, et la
mirent hors de mesure pour tout le reste de a vie du roi. Portocarrero,
Villafranca et San-Estevan, les trois conseillers d'État seuls du
secret, induisirent habilement les autres à chasser la Berlips et le
prince de Darmstadt, qui pour la plupart s'y portèrent de haine pour la
reine et pour ses deux bras droits\,; et le peu qui lui étaient attachés
comme l'amirante par cabale et Veragua par politique, furent entraînés,
et apprirent à quitter doucement la reine par l'état où ce changement la
fit tomber. Ces deux grands pas faits, San-Estevan qui ne quitta jamais
le cardinal d'un moment, tant que cette grande affaire ne fut pas
consommée, le poussa à porter un autre coup, sans lequel ils ne crurent
pas qu'il y eut moyen de rien entreprendre avec succès. Ce fut de faire
chasser le confesseur du roi qui lui avait été donné par la reine, et
qui était un zélé autrichien.

Le cardinal prit si bien son temps et ses mesures qu'il fit coup
double\,: le confesseur fut renvoyé et Portocarrero en donna un autre
auquel il était assuré de faire dire et faire tout ce qu'il voudrait.
Alors il tint le roi d'Espagne par le for de la conscience, qui eut sur
lui d'autant plus de pouvoir qu'il commençait à ne regarder plus les
choses de ce monde qu'à la lueur de ce terrible flambeau qu'on allume
aux mourants. Portocarrero laissa ancrer un peu le confesseur, et quand
il jugea que l'état du roi d'Espagne le rendait susceptible de pouvoir
entendre mettre à la maison de France en parallèle avec celle
d'Autriche, le cardinal, toujours étayé et endoctriné par San-Estevan,
attaqua le roi d'Espagne avec toute l'autorité qu'il recevait de son
caractère, de son concert avec le confesseur, et de l'avis de ce peu de
personnages, mais si principaux qui étaient du secret, auxquels
l'importance et les conjonctures ne permettaient pas qu'on en joignît
d'autres. Ce prince exténué de maux, et dont la santé, faible toute sa
vie, avait rendu son esprit peu vigoureux, pressé par de si grandes
raisons temporelles, effrayé du poids des spirituelles, tomba dans une
étrange perplexité. L'amour extrême de sa maison, l'aversion de sa
rivale, tant d'États et de puissances à remettre à l'une ou à l'autre,
ses affections les plus chères, le plus fomentées jusqu'alors, son
propre ouvrage en faveur de l'archiduc à détruire pour la grandeur d'une
maison de tout temps ennemie, le salut éternel, la justice, l'intérêt
pressant de sa monarchie, le vœu des seuls ministres ou principaux
seigneurs qui jusqu'alors pussent être sûrement consultés nul Autrichien
pour le soutenir dans ce combat\,; le cardinal et le confesseur sans
cesse à le presser parmi ces avis, aucun dont il pût se défier, aucun
qui eût de liaison en France ni avec nul Français, aucun qui ne fût
Espagnol naturel, aucun qui ne l'eût bien servi, aucun en qui il eût
jamais reconnu le moindre éloignement pour la maison d'Autriche\,; un
grand attachement, au contraire, pour elle en plusieurs d'eux\,: il n'en
fallut pas moins pour le jeter dans une incertitude assez grande pour ne
savoir à quoi se résoudre\,; enfin, flottant, irrésolu, déchiré en
soi-même, ne pouvant plus porter cet état et toutefois ne pouvant se
déterminer, il pensa à consulter le pape comme un oracle avec lequel il
ne pouvait faillir\,; il résolut donc de déposer en son sein paternel
toutes ses inquiétudes, et de suivre ce qu'il lui conseillerait. Il le
proposa au cardinal qui y consentit, persuadé que le pape aussi
impartial, aussi éclairé qu'il s'était montré depuis qu'il gouvernait
l'Église, et d'ailleurs aussi désintéressé et aussi pieux qu'il l'était,
prononcerait en faveur du parti le plus juste.

Cette résolution prise soulagea extrêmement le roi d'Espagne\,; elle
calma ses violentes agitations, qui avaient porté encore beaucoup sur sa
santé qui reprit quelque sorte de lueur. Il écrivit donc fort au long au
pape, et se reposa sur le cardinal du soin de faire rendre directement
sa lettre avec tout le secret qu'elle demandait. Alors il fallut bien
mettre Ubilla dans le secret. Ce ministre, tel que je l'ai dépeint
d'après ceux qui l'ont fort connu, et qui ont vécu avez lui en maniement
commun de toutes les affaires, n'eut pas peine à entrer dans les vues
favorables à la France. Il les trouva déjà si bien concertées, si à
l'abri de toutes contradictions intérieures par le reculement de la
reine, et si avancées en environs, qu'il se joignit de bonne foi aux
seigneurs du secret qui acquirent ainsi une bonne tête, et un ministère
qui s'étendait sur toute la monarchie, et duquel il leur eût été comme
impossible de se passer. Le pape reçut, directement la consultation du
roi d'Espagne, et ne le fit pas attendre pour la réponse et sa décision.
Il lui récrivit qu'étant lui-même en un état aussi proche que l'était Sa
Majesté Catholique d'aller rendre compte au souverain pasteur du
troupeau universel qu'il lui avait confié, il avait un intérêt aussi
grand et aussi pressant qu'elle-même de lui donner un conseil dont il ne
pût alors recevoir de reproches, qu'il pensât combien peu il devait se
laisser toucher aux intérêts de la maison d'Autriche en comparaison de
ceux de son éternité, et de ce compte terrible qu'il était si peu
éloigné d'aller rendre au souverain juge des rois qui ne reçoit point
d'excuses et ne fait acception de personne. Qu'il voyait bien lui-même
que les enfants du Dauphin étaient les vrais, les seuls et les légitimes
héritiers de sa monarchie, qui excluait tous autres, et du vivant
desquels et de leur postérité, l'archiduc, la sienne et toute la maison
d'Autriche n'avaient aucun droit, et étaient entièrement étrangers. Que
plus la succession était immense, plus l'injustice qu'il y commettrait
lui deviendrait terrible au jugement de Dieu\,; que c'était donc à lui à
n'oublier aucunes des précautions ni des mesures que toute sa sagesse
lui pourrait inspirer pour faire justice à qui il la devait, et pour
assurer autant qu'il lui serait possible l'entière totalité de sa
succession et de sa monarchie à un des fils de France. Le secret de la
consultation et de la réponse d'Innocent XII fut si profondément
enseveli qu'il n'a été su que depuis que Philippe V a été en Espagne.

\hypertarget{chapitre-ii.}{%
\chapter{CHAPITRE II.}\label{chapitre-ii.}}

1700

~

{\textsc{Testament du roi d'Espagne en faveur du duc d'Anjou.}}
{\textsc{- Mort du roi d'Espagne.}} {\textsc{- Harcourt à Bayonne
assemblant une armée\,; son ambition et son adresse.}} {\textsc{-
Ouverture du testament.}} {\textsc{- Plaisanterie cruelle du duc
d'Abrantès.}} {\textsc{- Deux conseils d'État chez M\textsuperscript{me}
de Maintenon en deux jours.}} {\textsc{- Avis partagés\,; raisons pour
s'en tenir au traité de partage\,; raisons pour accepter le testament.}}
{\textsc{- Monseigneur {[}parle{]} avec force pour accepter.}}
{\textsc{- Résolution d'accepter le testament.}} {\textsc{- Surprise du
roi et de ses ministres.}}

~

Cependant le roi d'Espagne était veillé et suivi de près, dans
l'espérance où était le cardinal pour le disposer à une parfaite et
prompte obéissance à la décision qu'il attendait, de manière que
lorsqu'elle arriva il n'y eut plus à vaincre que des restes impuissants
de répugnance et à mettre la main tout de bon à l'œuvre\,; Ubilla, uni à
ceux du secret, fit un autre testament en faveur du duc d'Anjou, et le
dressa avec les motifs et les clauses qui ont paru à tous les esprits
désintéressés si pleines d'équité, de prudence, de force et de
sagesse\,; et qui est devenu si public que je n'en dirai rien ici
davantage. Quand il fut achevé d'examiner par les conseillers d'État du
secret, Ubilla le porta au roi d'Espagne avec l'autre précédent fait en
faveur de l'archiduc\,; celui-là fut brûlé par lui en présence du roi
d'Espagne, du cardinal et du confesseur, et l'autre tout de suite signé
par le roi d'Espagne et un moment après authentiqué au-dessus, lorsqu'il
fut fermé, par les signatures du cardinal, d'Ubilla et de quelques
autres. Cela fait, Ubilla tint prêts les ordres et les expéditions
nécessaires en conséquence pour les divers paye de l'obéissance
d'Espagne avec un secret égal\,; on prétend qu'alors ils firent
pressentir le roi sans oser pourtant confier tout le secret à Castel dos
Rios, et que ce fut la matière de cette audience si singulière qu'elle
est sans exemple, dort il exclut Torcy, auquel, ni devant ni après, il
ne dit pas un mot de la matière qu'il avait à traiter seul avec le roi.

L'extrémité du roi d'Espagne se fit connaître plusieurs jours seulement
après la signature du testament. Le cardinal, aidé des principaux du
secret qui avaient les deux grandes charges, et du comte de Benavente
qui avait l'autre, par laquelle il était maître de l'appartement et de
la chambre du roi, empêcha la reine d'en approcher les derniers jours
sous divers prétextes. Benavente n'était pas du secret, mais il était
ami des principaux du peu de ceux qui en étaient, et il était aisément
gouverné, de sorte qu'il fit tort ce qu'ils voulurent. Ils y comptaient
si bien qu'ils l'avaient fait mettre dans le testament pour entrer comme
grand d'Espagne dans la junte qu'il établit pour gouverner en attendant
le successeur, et il savait aussi que le testament était fait, sans
toutefois être instruit de ce qu'il contenait. Il était tantôt temps de
parler au conseil. Des huit qui en étaient, quatre seulement étaient du
secret, Portocarrero, Villafranca, San-Estevan et Ubilla. Les autres
quatre étaient l'amirante, Veragua, Mancera et Arias. Des deux derniers
ils n'en étaient point en peine, mais l'attachement de l'amirante à la
reine, le peu de foi de Veragua, et la difficulté de leur faire garder
un si important secret, avaient toujours retardé jusque tout aux
derniers jours du roi d'Espagne d'en venir aux opinions dans le conseil,
sur la succession.

À la fin, le roi prêt à manquer à tous les moments, toutes les
précautions possibles prises, et n'y ayant guère à craindre, que ces
deux conseillers d'État seuls, et sans appui ni confiance de personne,
et la reine dans l'abandon, osassent révéler un secret si prêt à l'être,
et si inutilement pour eux, le cardinal assembla le conseil et y mit
tout de suite la grande affaire de la succession en délibération.
Villafranca tint parole, et opina avec grande force en la manière
qu'elle se trouve ci-dessus. San-Estevan suivit avec autorité.
L'amirante et Veragua, qui virent la partie faite, n'osèrent contredire.
Le second ne se souciait que de sa fortune, qu'il ne voulait pas exposer
dans des moments si critiques et dans une actuelle impuissance de la
cour de Vienne par son éloignement, et la même raison retint l'amirante
malgré son attachement pour elle lancera, galant homme et qui ne voulait
que le bien, mais effrayé d'avoir à prendre son parti sur-le-champ en
chose de telle importance, demanda vingt-quatre heures pour y penser, et
au bout desquelles il opina pour la France. Arias s'y rendit d'abord, à
qui on avait dit le mot à l'oreille un peu auparavant. Ubilla, après que
le cardinal eut opiné et conclu, dressa sur la table même ce célèbre
résultat\,; ils le signèrent et jurèrent d'en garder un inviolable
secret, jusqu'à ce qu'après la mort du roi il fût temps d'agir en
conséquence de ce qui venait d'être résolu entre eux. En effet, ni
l'amirante ni Veragua n'osèrent en laisser échapper quoi que ce fût, et
l'amirante même fut impénétrable là-dessus à la reine et au comte
d'Harrach, qui ignorèrent toujours si le conseil avait pris une
résolution. Très peu après le roi d'Espagne mourut, le jour de la
Toussaint, auquel il était né quarante-deux ans auparavant\,; il mourut,
dis-je, à trois heures après midi dans le palais de Madrid.

Sur les nouvelles de l'état mourant du roi d'Espagne, dont Blécourt
avait grand soin d'informer le roi, il donna ordre au marquis d'Harcourt
de se tenir prêt pour aller assembler une armée à Bayonne, pour laquelle
on fit toutes les dispositions nécessaires, et Harcourt partit le 23
octobre avec le projet de prendre les places de cette frontière, comme
Fontarabie et les autres, et d'entrer par là en Espagne. Le Guipuscoa
était à la France par le traité de partage\,; ainsi jusque-là il n'y
avait rien à dire. Comme tout changea subitement de face, je n'ai point
su quels étaient les projets après avoir réduit cette petite province.
Mais, en attendant qu'Harcourt fît les affaires du roi, il profita de la
conjoncture et fit les siennes. Beuvron, son père, avait été plus que
très bien avec M\textsuperscript{me} de Maintenon dans ses jeunes
années. C'est ce qui fit la duchesse d'Arpajon, sa sueur, dame d'honneur
de M\textsuperscript{me} la Dauphine-Bavière, arrivant pour un procès au
conseil, de Languedoc où elle était depuis vingt ans, et sans qu'elle,
ni son frère, ni pas un des siens eût imaginé d'y songer. On a vu que
M\textsuperscript{me} de Maintenon n'a jamais oublié ces sortes d'amis.
C'est ce qui a fait la fortune d'Harcourt, de Villars et de bien
d'autres.

Harcourt sut en profiter en homme d'infiniment d'esprit et de sens qu'il
était. Il la courtisa dès qu'il put pointer, et la cultiva toujours sur
le pied d'en tout attendre, et quoiqu'il frappât avec jugement aux
bonnes portes, il se donna toujours pour ne rien espérer que par elle.
Il capitula donc par son moyen sans que le roi le trouvât mauvais, et il
partit avec assurance de n'attendre pas longtemps à être fait duc
héréditaire. La porte alors était entièrement fermée à la pairie.
J'aurai lieu d'expliquer cette anecdote ailleurs. Arriver là était toute
l'ambition d'Harcourt. Elle était telle que, longtemps avant cette
conjoncture, étant à Calais, pour passer avec le roi Jacques en
Angleterre, il ne craignit pais de s'en expliquer tout haut. On le
félicitait de commander à une entreprise dont le succès lui acquerrait
le bâton. Il ne balança point et répondit tout haut que tout son but
était d'être duc, et que, s'il savait sûrement devenir maréchal de
France et jamais duc, il quitterait le service tout à l'heure et se
retirerait chez lui.

Dès que le roi d'Espagne fut expiré, il fut question d'ouvrir son
testament. Le conseil d'État s'assembla, et tous les grands d'Espagne
qui se trouvèrent à Madrid y entrèrent. La curiosité de la grandeur d'un
événement si rare, et qui intéressait tant de millions d'hommes, attira
tout Madrid au palais, en sorte qu'on s'étouffait dans les pièces
voisines de celle où les grands et le conseil ouvraient le testament.
Tous les ministres étrangers en assiégeaient la porte. C'était à qui
saurait le premier le choix du roi qui venait de mourir, pour en
informer sa cour le premier. Blécourt était là comme les autres sans
savoir rien plus qu'eux, et le comte d'Harrach, ambassadeur de
l'empereur, qui espérait tout, et qui comptait sur le testament en
faveur de l'archiduc, était vis-à-vis la porte et tout proche avec un
air triomphant. Cela dura assez longtemps pour exciter l'impatience.
Enfin la porte s'ouvrit et se referma. Le duc d'Abrantès, qui était un
homme de beaucoup d'esprit, plaisant, mais à craindre, voulut se donner
le plaisir d'annoncer le choix du successeur, sitôt qu'il eut vu tous
les grands et le conseil y acquiescer et prendre leurs résolutions en
conséquence. Il se trouva investi aussitôt qu'il parut. Il jeta les yeux
de tous côtés en gardant gravement le silence. Blécourt s'avança, il le
regarda bien fixement, puis tournant la tête fit semblant de chercher ce
qu'il avait presque devant lui. Cette action surprit Blécourt et fut
interprétée mauvaise pour la France\,; puis tout à coup, faisant comme
s'il n'avait pas aperçu le comte d'Harrach et qu'il s'offrît
premièrement à sa vue, il prit un air de joie, lui saute au cou, et lui
dit en espagnol, fort haut\,: «\,Monsieur, c'est avec beaucoup de
plaisir\ldots.\,» et faisant une pause pour l'embrasser mieux, ajouta\,:
«\,Oui, monsieur, c'est avec une extrême joie que pour toute ma
vie\ldots\,» et redoublant d'embrassades pour s'arrêter encore, puis
acheva\,: «\,et avec le plus grand contentement que je me sépare de vous
et prends congé de la très auguste maison d'Autriche.\,» Puis perce la
foule, chacun courant après pour savoir qui était le successeur.
L'étonnement et l'indignation du comte d'Harrach lui fermèrent
entièrement la bouche, mais parurent sur son visage dans toute leur
étendue. Il demeura là encore quelques moments\,; il laissa des gens à
lui pour lui venir dire des nouvelles à la sortie du conseil, et s'alla
enfermer chez lui dans une confusion d'autant plus grande qu'il avait
été la dupe des accolades et de la cruelle tromperie du compliment du
duc d'Abrantès.

Blécourt, de son côté, n'en demanda pas davantage. Il courut chez lui
écrire pour dépêcher son courrier. Comme il était après, Ubilla lui
envoya un extrait du testament qu'il tenait tout prêt, et que Blécourt
n'eut qu'à mettre dans son paquet. Harcourt, qui était à Bayonne, avait
ordre d'ouvrir tous les paquets du roi, afin d'agir suivant les
nouvelles, sans perdre le temps à attendre les ordres de la cour qu'il
avait d'avance pour tous les cas prévus. Le courrier de Blécourt arriva
malade à Bayonne, de sorte qu'Harcourt en prit occasion d'en dépêcher un
à lui avec ordre de rendre à son ami Barbezieux les quatre mots qu'il
écrivit tant au roi qu'à lui, avant que de porter le paquet de Blécourt
à Torcy. Ce fut une galanterie qu'il fit à Barbezieux pour le faire
porteur de cette grande nouvelle. Barbezieux la reçut, et sur-le-champ
la porta au roi, qui était lors au conseil de finance, le mardi matin 9
novembre.

Le roi, qui devait aller tirer, contremanda la chasse, dîna à
l'ordinaire au petit couvert sans rien montrer sur son visage, déclara
la mort du roi d'Espagne, qu'il draperait\,; ajouta qu'il n'y aurait de
tout l'hiver ni appartement, ni comédies, ni aucuns divertissements à la
cour, et quand il fut rentré dans son cabinet, il manda aux ministres de
se trouver à trois heures chez M\textsuperscript{me} de Maintenon.
Monseigneur était revenu de courre le loup\,; il se trouva aussi à trois
heures chez M\textsuperscript{me} de Maintenon. Le conseil y dura
jusqu'après sept heures, en suite de quoi le roi y travailla jusqu'à dix
avec Torcy et Barbezieux, ensemble. M\textsuperscript{me} de Maintenon
avait toujours été présente au conseil\,; et la fut encore au travail
qui le suivit. Le lendemain mercredi, il y eut conseil d'État le matin
chez le roi à l'ordinaire, et au retour de la chasse il en tint un autre
comme la veille chez M\textsuperscript{me} de Maintenon, depuis six
heures du soir jusqu'à près de dix. Quelque accoutumé qu'on fût à la
cour à la faveur de M\textsuperscript{me} de Maintenon, on ne l'était
pas à la voir entrer publiquement dans les affaires, et la surprise fut
extrême de voir assembler deux conseils en forme chez elle, et pour la
plus grande et la plus importante délibération qui de tout ce long règne
et de beaucoup d'autres eût été mise sur le tapis.

Le roi, Monseigneur, le chancelier, le duc de Beauvilliers et Torcy, et
il n'y avait lors point d'autres ministres d'État que ces trois
derniers, furent les seuls qui délibérèrent sur cette grande affaire, et
M\textsuperscript{me} de Maintenon, avec eux, qui se taisait par
modestie, et que le roi força de dire son, avis après que tous eurent
opiné, excepté lui. Ils furent partagés\,: deux pour s'en tenir au
traité de partage, deux pour accepter le testament.

Les premiers soutenaient que la foi y était engagée, qu'il n'y avait
point de comparaison entre l'accroissement de la puissance et d'États
unis à la couronne, d'États contigus et aussi nécessaires que la
Lorraine, aussi importants que le Guipuscoa pour être une clef de
l'Espagne, aussi utiles au commerce que les places de Toscane, Naples et
Sicile\,; et la grandeur particulière d'un fils de France, dont tout au
plus loin la première postérité devenue espagnole par son intérêt, et
par ne connaître autre chose que l'Espagne, se montrerait aussi jalouse
de la puissance de la France que les rois d'Espagne autrichiens. Qu'en
acceptant le testament il fallait compter sur une longue et sanglante
guerre, par l'injure de la rupture du traité de partage, et par
l'intérêt de toute l'Europe à s'opposer à un colosse tel qu'allait
devenir la France pour un temps, si on lui laissait recueillir une
succession aussi vaste. Que la France épuisée d'une longue suite de
guerres, et qui n'avait pas eu loisir de respirer depuis la paix de
Ryswick, était hors d'état de s'y exposer\,; que l'Espagne l'était aussi
de longue main\,; qu'en l'acceptant tout le faix tombait sur la France,
qui, dans l'impuissance de soutenir le poids de tout ce qui s'allait
unir contre elle, aurait encore l'Espagne à supporter. Que c'était un
enchaînement dont on n'osait prévoir les suites\,; mais qui en gros se
montraient telles que toute la prudence humaine semblait conseiller de
ne s'y pas commettre. Qu'en se tenant au traité de partage, la France se
conciliait toute l'Europe par cette foi maintenue, et par ce grand
exemple de modération, elle qui n'avait eu toute l'Europe sur les bras
que par la persuasion, où sa conduite avait donné crédit, des calomnies
semées avec tant de succès qu'elle voulait tout envahir, et monter peu à
peu à la monarchie universelle tant reprochée autrefois à la maison
d'Autriche, dont l'acceptation du testament ne laisserait plus douter,
comme en étant un degré bien avancé. Que, se tenant au traité de
partage, elle s'attirerait la confiance de toute l'Europe dont elle
deviendrait la dictatrice, ce qu'elle ne pouvait espérer de ses armes,
et que l'intérieur du royaume, rétabli, par une longue paix, augmenté
aux dépens de l'Espagne, avec la clef du côté le plus jaloux et le plus
nu de ce royaume, et celle de tout le commerce du Levant, enfin
l'arrondissement si nécessaire de la Lorraine, qui réunit les Évêchés,
l'Alsace et la Franche-Comté, et délivre la Champagne qui n'a point de
frontières, formerait un État si puissant qu'il serait à l'avenir la
terreur ou le refuge de tous les autres, et en situation assurée de
faire tourner à son gré toutes les affaires générales de l'Europe. Torcy
ouvrit cet avis pour balancer et sans conclure, et le duc de
Beauvilliers le soutint puissamment.

Le chancelier, qui, pendant toute cette déduction s'était uniquement
appliqué à démêler l'inclination du roi, et qui crut l'avoir enfin
pénétrée, parla ensuite. Il établit d'abord qu'il était au choix du roi
de laisser brancher une seconde fois la maison d'Autriche à fort peu de
puissance près le ce qu'elle avait été depuis Philippe II, et dont on
avait vivement éprouvé la force et la puissance, ou de prendre le même
avantage pour la sienne\,; que cet avantage se trouvait fort supérieur à
celui dont la maison d'Autriche avait tiré de si grands avantages, par
la différence de la séparation des États des deux branches, qui ne se
pouvaient secourir que par des diversions de concert, et qui étaient
coupés par des États étrangers. Que l'une des deux n'avait ni mer ni
commerce, que sa puissance n'était qu'usurpation qui avait toujours
trouvé de la contradiction dans son propre sein, et souvent des révoltes
ouvertes, et dans ce vaste pays d'Allemagne où les diètes avaient
palpité tant qu'elles avaient pu, et où on avait pu sans messéance
fomenter les mécontentements par l'ancienne alliance de la France avec
le corps germanique, dont l'éloignement de l'Espagne ne recevait de
secours que difficilement, sans compter les inquiétudes de la part des
Turcs, dont les armes avaient souvent rendu celles des empereurs
inutiles à l'Espagne. Que les pays héréditaires dont l'empereur pouvait
disposer comme du sien, ne pouvaient entrer en comparaison avec les
moindres provinces de France. Que ce dernier royaume, le plus étendu, le
plus abondant, et le plus puissant de tous ceux de l'Europe, chaque État
considéré à part, avait l'avantage de ne dépendre de l'avis de qui que
ce soit, et de se remuer tout entier à la seule volonté de son roi, ce
qui en rendait les mouvements parfaitement secrets et tout à fait
rapides, et celui encore d'être contigu d'une mer à l'autre à l'Espagne,
et de plus par les deux mers d'avoir du commerce et une marine, et
d'être en état de protéger celle d'Espagne, et de profiter à l'avenir de
son union avec elle pour le commerce des Indes\,; par conséquent de
recueillir des fruits de cette union bien plus continuels, plus grands,
plus certains que n'avait pu faire la maison d'Autriche, qui, loin de
pouvoir compter mutuellement sur des secours précis, s'était souvent
trouvée embarrassée à faire passer ses simples courriers d'une branche à
l'autre, au lieu que la France et l'Espagne, par leur contiguïté, ne
faisaient, pour toutes ce importantes commodités, qu'une seule et même
province, et pouvait agir en tous temps à l'insu de tous ses voisins\,;
que ces avantages ne se trouvaient balancés que par ceux de
l'acquisition de la Lorraine, commode et importante à la vérité, mais
dont la possession n'augmenterait en rien le poids de la France dans les
affaires générales, tandis qu'unie avec l'Espagne, elle serait toujours
prépondérante et très supérieure à la plupart des puissances unies en
alliance, dont les divers intérêts ne pouvaient rendre ces unions
durables comme celui des frères et de la même maison. Que d'ailleurs en
se mettant à titre de nécessité au-dessus du scrupule de l'occupation de
la Lorraine désarmée, démantelée, enclavée comme elle était, ne l'avoir
pas était le plus petit inconvénient du monde, puisqu'on s'en saisirait
toujours au premier mouvement de guerre, comme on avait fait depuis si
longtemps, qu'en ces occasions on ne s'apercevait pas de différence
entre elle et une province du royaume.

À l'égard de Naples, Sicile, et des places de la côte de Toscane, il n'y
avait qu'à ouvrir les histoires pour voir combien souvent nos rois en
avaient été les maîtres, et avec ces États de celui de Milan, de Gènes
et d'autres petits d'Italie, et avec, quelle désastreuse et rapide
facilité ils les avaient toujours perdus. Que le traité de partage avait
été accepté faute de pouvoir espérer mieux dès qu'on ne voulait pas se
jeter dans les conquêtes\,; mais qu'en l'acceptant ç'aurait été se
tromper de méconnaître l'inimitié de tant d'années de l'habile main qui
l'avait dressé pour nous donner des noms sans nous donner de choses, ou
plutôt des choses impossibles à conserver par leur éloignement et leur
épuisement, et qui ne seraient bonnes qu'à consumer notre argent et
partager nos forces, et à nous tenir dans une contraint et une brassière
perpétuelles. Que pour le Guipuscoa c'était un leurre de le prendre pour
une clef d'Espagne\,; qu'il n'en fallait qu'appeler à nous-mêmes qui
avions été plus de trente ans en guerre avec l'Espagne, et toujours en
état de prendre les places et les ports de cette province, puisque le
roi avait bien conquis celles de Flandre, de la Meuse et du Rhin. Mais
que la stérilité affreuse d'un vaste pays, et la difficulté des Pyrénées
avaient toujours détourné la guerre de ce côté-là, et permis même dans
leur plus fort une sorte de commerce entre les deux frontières sous
prétexte de tolérance sans qu'il s'y fit jamais commis aucune hostilité.
Qu'enfin les places de la côte de Toscane seraient toujours en prise du
souverain du Milanais qui pouvait faire ses préparatifs à son aise et en
secret, tomber, dessus subitement et de plain-pied, et s'en être emparé
avant l'arrivée d'un secours par mer qui ne pouvait partir que des ports
de Provence. Que pour ce qui était du danger d'avoir les rois d'Espagne
français pour ennemis, comme ceux de la maison d'Autriche, cette
identité ne pouvait jamais avoir lieu, puisqu'au moins n'étant pas de
cette maison, mais de celle de France, tout ce qui ne serait pas
l'intérêt même d'Espagne ne serait jamais le leur, comme au contraire,
dès qu'il y aurait identité de maison, il y aurait identité d'intérêts,
dont, pour ne parler maintenant que de l'extérieur, l'abaissement de
l'empereur et la diminution du commerce et de l'accroissement des
colonies des Anglais et des Hollandais aux Indes, ferait toujours un tel
intérêt commun qu'il dominerait tous les autres. Que pour l'intérieur,
il n'y avait qu'à prendre exemple sur la maison d'Autriche, que rien
n'avait pu diviser depuis Charles V, quoique si souvent pleine de
\emph{riottes}\footnote{Querelles.} domestiques. Que le désir de
s'étendre en Flandre était un point que le moindre grain de sagesse et
de politique ferait toujours céder à tout ce que l'union de deux si
puissantes monarchies et si contiguës partout pouvait opérer, qui
n'allait à rien moins pour la nôtre qu'à s'enrichir par le commerce des
Indes, et pour toutes les deux à donner le branle, le poids et avec le
temps le ton à toutes les affaires de l'Europe\,; que cet intérêt était
si grand et si palpable, et le occasions de division entre les deux rois
de même sang si médiocres en eux-mêmes et si anéanties en comparaison de
ceux-là, qu'il n'y avait point de division raisonnable à en craindre.
Qu'il y a voit à espérer que le roi vivrait assez longtemps non
seulement pour l'établir, et Monseigneur, après lui, entre ses deux
fils, qu'il n'y avait pas moins lier d'en espérer la continuation dans
les deux frères si unis et si affermis de longue main dans ces
principes, qu'ils feraient passer aux cousins germains, ce qui montrait
déjà une longue suite d'années\,; qu'enfin si le malheur venait assez à
surmonter toute raison pour faire naître des guerres, il fallait
toujours qu'il y eût un roi d'Espagne, et qu'une guerre se pousserait
moins et se terminerait toujours plus aisément et plus heureusement avec
un roi de même sang, qu'avec un étranger et de la maison d'Autriche.

Après cet exposé, le chancelier vint à ce qui regardait la rupture du
traité de partage. Après en avoir remis le frauduleux, le captieux, le
dangereux, il prétendit que la face des choses, entièrement changée du
temps auquel il avait été signé, mettait de plein droit le roi en
liberté, sans pouvoir être accusé de manquer de foi\,; que par ce traité
il ne s'était engagé qu'à ce qu'il portait\,; qu'on n'y trouverait point
de stipulation d'aucun refus de ce qui serait donné pas la volonté du
roi d'Espagne, et volonté pure, sans sollicitation, et même à l'insu du
roi, et de ce qui serait offert par le vœu universel de tous les
seigneurs et les peuples d'Espagne\,; que le premier était arrivé, que
le second allait suivre, selon toute apparence\,; que le refuser contre
tout intérêt, comme il croyait l'avoir démontré, attirerait moins la
confiance avec qui le traité de partage avait été signé, que leur
mépris, que la persuasion d'une impuissance qui les enhardirait à
essayer de dépouiller bientôt la France de ce qui ne lui avait été donné
en distance si éloignée et de si fâcheuse garde, que pour le lui ôter à
la première occasion\,; et que, bien loin de devenir la dictatrice de
l'Europe par une modération si étrange et que nulle équité ne
prétextait, la France acquerrait une réputation de pusillanimité qui
serait attribuée aux dangers de la dernière guerre et à l'exténuation
qui lui en serait restée, et qu'elle deviendrait la risée de ses faux
amis avec bien plus de raison que Louis XII et François Ier ne l'avaient
été de Ferdinand le Catholique, de Charles V, des papes et des
Vénitiens, par leur rare attachement à leur foi et à leurs paroles
positives desquelles ici il n'y a rien qui puisse être pris en la
moindre parité\,; enfin qu'il convenait qu'une si riche succession ne se
recueillerait pas sans guerre, mais qu'il fallait lui accorder aussi que
l'empereur ne souffrirait pas plus paisiblement l'exécution du traité de
partage que celle du testament\,; que jamais il n'avait voulu y
consentir, qu'il avait tout tenté pour s'y opposer, qu'il n'était occupé
qu'à des levées et à des alliances\,; que guerre pour guerre, il valait
mieux la faire à mains garnies et ne se pas montrer à la face de
l'univers indignes de la plus haute fortune et la moins imaginée.

Ces deux avis, dont je ne donne ici que le précis, furent beaucoup plus
étendus de part et d'autre, et fort disputés par force répliques des
deux côtés. Monseigneur, tout noyé qu'il fit dans la graisse et dans
l'apathie, parut un autre homme dans tous ces deux conseils, à la grande
surprise du roi et des assistants. Quand ce fut à lui à parler, les
ripostes finies, il s'expliqua avec force pour l'acceptation du
testament, et reprit une partie des meilleures raisons du chancelier.
Puis se tournant vers le roi d'un air respectueux, mais ferme, il lui
dit qu'après avoir dit son avis comme les autres, il prenait la liberté
de lui demander son héritage, puisqu'il était en état de l'accepter\,;
que la monarchie d'Espagne était le bien de la reine sa mère, par
conséquent le sien, et pour la tranquillité de l'Europe celui de son
second fils, à qui il le cédait de tout son cœur, mais qu'il n'en
quitterait pas un seul pouce de terre à nul autre\,; que sa demande
était juste et conforme à l'honneur du roi, et à l'intérêt et à la
grandeur de sa couronne, et qu'il espérait bien aussi qu'elle ne lui
serait pas refusée. Cela dit d'un visage enflammé surprit à l'excès. Le
roi l'écouta fort attentivement, puis dit à M\textsuperscript{me} de
Maintenon\,: «\,Et vous, madame, que dites-vous sur tout ceci\,?» Elle à
faire la modeste\,; mais enfin pressée et même commandée, elle dit deux
mots d'un bienséant embarras, puis en peu de paroles se mit sur les
louanges de Monseigneur qu'elle craignait et n'aimait guère, ni lui
elle, et fut enfin d'avis d'accepter le testament.

Le roi conclut sans s'ouvrir. Il dit qu'il avait tout bien ouï, et
compris tout ce qui avait été dit de part et d'autre\,; qu'il y avait de
grandes raisons des deux côtés, que l'affaire méritait bien de dormir
dessus et d'attendre vingt-quatre heures ce qui pourrait venir
d'Espagne, et si les Espagnols seraient du même avis que leur roi. Il
congédia le conseil, à qui il ordonna de se retrouver le lendemain au
soir au même lieu et finit sa journée, comme on l'a cuit, entre
M\textsuperscript{me} de Maintenon, Torcy qu'il fit rester, et
Barbezieux qu'il envoya chercher.

Le mercredi 10 novembre, il arriva plusieurs courriers d'Espagne, dont
un ne fit que passer portant des ordres à l'électeur de Bavière à
Bruxelles. On eut par eux tout ce qui pouvait achever de déterminer le
roi à l'acceptation du testament, c'est-à-dire le vœu des seigneurs et
des peuples, autant que la brièveté du temps le pouvait permettre\,; de
sorte que, tout ayant été lu et discuté chez M\textsuperscript{me} de
Maintenon au conseil que le roi au retour de la chasse y tint comme la
veille, il s'y détermina à l'acceptation. Le lendemain matin, jeudi, le
roi, entre son lever et sa messe, donna audience à l'ambassadeur
d'Espagne, à laquelle Monseigneur et Torcy furent présents.
L'ambassadeur présenta, de la part de la reine et de la junte, une copie
authentique du testament. On n'a pas douté depuis qu'en cette audience,
le roi, sans s'expliquer nettement, n'eût donné de grandes espérances
d'acceptation à l'ambassadeur, à la sortie duquel le roi fit entrer Mgr
le duc de Bourgogne, à qui il confia le secret du parti pris. Le
chancelier s'en alla à Paris l'après-dînée, et les autres ministres
eurent congé jusqu'à Versailles, de manière que personne ne douta que la
résolution, quelle qu'elle fût, ne fût prise et arrêtée.

La junte qui fut nommée par le testament pour gouverner en attendant le
successeur fut fort courte, et seulement composée de la reine, du
cardinal Portocarrero, de don Manuel Arias, gouverneur du conseil de
Castille, du grand inquisiteur, et pour grands d'Espagne, du comte de
Benavente et du comte d'Aguilar. Ceux qui firent faire le testament
n'osèrent pas exclure la reine, et ne voulurent pas s'y mettre pour
éviter jalousie. Ils n'étaient pas moins sûrs de leur fait, dès que le
choix du successeur serait passé à l'ouverture du testament, ni de la
gestion, par la présence du cardinal, du comte de Benavente et d'Arias,
dont ils étaient sûrs, et duquel la charge que j'aurai ailleurs occasion
d'expliquer donnait le plus grand pouvoir, appuyé surtout de l'autorité
du cardinal qui était comme le régent et le chef de la junte, tout le
crédit et la puissance de la reine se trouvant anéantis au point qu'elle
fut réduite à faire sa cour au cardinal et à ses amis, et que, sous
prétexte de sa douleur, elle n'assista à la junte que pour signer aux
premières et plus importantes résolutions toutes arrêtées sans elle, et
qu'elle s'en retira dans l'ordinaire et le courant, parce qu'elle
sentait qu'elle n'y serait que de montre. Aguilar était l'homme
d'Espagne le plus laid, qui avait le plus d'esprit, et peut-être encore
le plus de capacité, mais le plus perfide et le plus méchant. Il était
si bien connu pour tel qu'il en plaisantait lui-même, et qu'il disait
qu'il serait le plus méchant homme d'Espagne, sans son fils qui avait
joint à la laideur de son âme celle que lui-même avait en son corps.
Mais c'était en même temps un homme cauteleux, et qui, voyant le parti
pris, ne pensa qu'à sa fortune, à plaire aux maîtres des affaires, et à
préparer le successeur à le bien traiter. Ubilla, par son emploi, était
encore d'un grand et solide secours au cardinal et à Arias.

La suite nécessaire d'une narration si intéressante ne m'a pas permis de
l'interrompre. Maintenant qu'elle est conduite à un point de repos il
faut revenir quelque peu sur ses pas. Il n'est pas croyable l'étonnement
qu'eut Blécourt d'une disposition si peu attendue, et dont on s'était
caché de lui autant que du comte d'Harrach. La rage de celui-ci fit
extrême par la surprise, par l'anéantissement du testament en faveur de
l'archiduc, sur lequel il comptait entièrement, et par l'abandon et
l'impuissance où il se trouva tombé tout à coup, et lui et la reine à
qui il ne resta pas une créature, ni à lui un autrichien qui se l'osât
montrer. Harcourt, en ouvrant les dépêches du roi à Bayonne, demeura
interdit. Il sentit bien alors que les propositions que l'amirante lui
avait faites de la part de la reine étaient de gens clairvoyants, non
pas elle, mais lui, qui craignaient que les choses ne prissent ce tour
par le grand intérêt des principaux particuliers, et qui, à tout hasard
du succès, vouloir faire leur marché. Il eût bien alors redoublé les
regrets de son retour, et de la défense qu'il reçut d'entrer en rien
avec l'amirante, s'il n'eût habilement su tirer sur le temps, et
profiter de la protection de M\textsuperscript{me} de Maintenon pour
emporter à Bayonne une promesse dont il se mit à hâter
l'accomplissement.

La surprise du roi et de ses ministres fut sans pareille. Ni lui ni eux
ne pouvaient croire ce qu'ils lisaient dans la dépêche de Blécourt, et
il leur fallut plusieurs jours pour en revenir assez pour être en état
de délibérer sur une aussi importante matière. Dès que la nouvelle
devint publique, elle fit la même impression sur toute la cour, et les
ministres étrangers passèrent les nuits à conférer et à méditer sur le
parti que le roi prendrait, et sur les intérêts de leurs maîtres, et
gardaient à l'extérieur un grand silence. Le courtisan ne s'occupait
qu'à raisonner\,; et presque tous allaient à l'acceptation. La manière
ne laissa pas d'en être agitée dans les conseils, jusqu'à y raisonner de
donner la comédie au monde, et de faire disparaître le duc d'Anjou sous
la conduite du nonce Gualterio qui l'emmènerait en Espagne. Je le sus et
je songeai à être de la partie. Mais ce misérable biais fut aussitôt
rejeté, par la honte d'accepter à la dérobée tant de couronnes offertes,
et par la nécessité prompte de lever le masque pour soutenir l'Espagne
trop faible pour être laissée à ses propres forces. Comme on ne parlait
d'autre chose que du parti qu'il y avait à prendre, le roi se divertit
un soir dans son cabinet à en demander leur avis aux princesses. Elles
répondirent que c'était d'envoyer promptement M. le duc d'Anjou en
Espagne, et que c'était le sentiment général, par fout ce qu'elles en
entendaient dire à tout le monde. «\,Je suis sûr, leur répliqua le roi,
que quelque parti que je prenne, beaucoup de gens me condamneront.\,»

C'était le samedi 13 novembre. Le lendemain matin dimanche 14, veille du
départ de Fontainebleau, le roi entretint longtemps Torcy, qui avertit
ensuite l'ambassadeur d'Espagne, qui était demeuré à Fontainebleau, de
se trouver le lendemain au soir à Versailles. Cela se sut et donna un
grand éveil. Les gens alertes avaient su encore que le vendredi
précédent le roi avait parlé longtemps à M. le duc d'Anjou en présence
de Monseigneur et de Mgr le duc de Bourgogne, ce qui était si
extraordinaire qu'on commença à se douter que le testament serait
accepté. Ce même dimanche, veille du départ, un courrier espagnol du
comte d'Harrach passa à Fontainebleau allant à Vienne, vit le roi à son
souper, et dit publiquement qu'on attendait à Madrid M. le duc d'Anjou
avec beaucoup d'impatience, et ajouta qu'il y avait quatre grands nommés
pour aller au-devant de lui. Ce prince, à qui on parla du testament, ne
répondit que par sa reconnaissance pour le roi d'Espagne, et se
conduisit si uniment qu'il ne parut jamais qu'il sût ou se doutât de
rien jusqu'à l'instant de sa déclaration.

\hypertarget{chapitre-iii}{%
\chapter{CHAPITRE III}\label{chapitre-iii}}

1700

~

{\textsc{Retour de Fontainebleau.}} {\textsc{- Déclaration du roi
d'Espagne\,; son traitement.}} {\textsc{- M. de Beauvilliers seul en
chef, et M. de Noailles en supplément accompagnent les princes au
voyage.}} {\textsc{- Le nonce et l'ambassadeur de Venise félicitent les
deux rois.}} {\textsc{- Harcourt duc vérifié et ambassadeur en
Espagne.}} {\textsc{- Rage singulière de Tallart.}} {\textsc{-
L'électeur de Bavière fait proclamer Philippe V aux Pays-Bas, qui est
harangué par le parlement et tous les corps.}} {\textsc{- Plaintes des
Hollandais.}} {\textsc{- Bedmar à Marly.}} {\textsc{- Philippe V
proclamé à Milan.}} {\textsc{- Le roi d'Espagne fait Castel dos Rios
grand d'Espagne de la première classe et prend la Toison\,; manière de
la porter.}} {\textsc{- Départ du roi d'Espagne et des princes ses
frères.}} {\textsc{- Philippe V proclamé à Madrid, à Naples, en Sicile
et en Sardaigne.}} {\textsc{- Affaire de Vaïni à Rome.}} {\textsc{-
Albano pape (Clément XI).}} {\textsc{- Grâces pécuniaires.}} {\textsc{-
Chamillart ministre.}} {\textsc{- Électeur de Brandebourg se déclare roi
de Prusse\,; comment {[}la Prusse{]} entrée dans sa maison.}} {\textsc{-
Courlande.}} {\textsc{- Tessé à Milan et Colmenero à Versailles.}}
{\textsc{- Castel dos Rios.}} {\textsc{- Harcourt retourné à Madrid\,;
sa place à la junte.}} {\textsc{- Troubles du Nord.}}

~

Le lundi 15 novembre, le roi partit de Fontainebleau entre neuf et dix
heures, n'ayant dans son carrosse que Mgr le duc de Bourgogne,
M\textsuperscript{me} la duchesse de Bourgogne, M\textsuperscript{me} la
princesse de Conti, et la duchesse du Lude, mangea un morceau sans en
sortir, et arriva à Versailles sur les quatre heures. Monseigneur alla
dîner à Meudon pour y demeurer quelques jours\,; et Monsieur et Madame à
Paris. En chemin, l'ambassadeur d'Espagne reçut un courrier avec de
nouveaux ordres et de nouveaux empressements pour demander M. le duc
d'Anjou. La cour se trouva fort grosse à Versailles, que la curiosité y
avait rassemblée dès le jour même de l'arrivée du roi.

Le lendemain, mardi 16 novembre, le roi, au sortir de son lever, fit
entrer l'ambassadeur d'Espagne dans son cabinet, où M. le duc d'Anjou
s'était rendu par les derrières. Le roi, le lui montrant, lui dit qu'il
le pouvait saluer comme son roi. Aussitôt il se jeta à genoux à la
manière espagnole, et lui fit un assez long compliment en cette langue.
Le roi lui dit qu'il ne l'entendait pas encore, et que c'était à lui à
répondre pour son petit-fils. Tout aussitôt après, le roi fit, contre
toute coutume, ouvrir les deux battants de la porte de son cabinet, et
commanda à tout le monde qui était là presque en foule d'entrer\,; puis,
passant majestueusement les yeux sur la nombreuse compagnie :
«\,Messieurs, leur dit-il en montrant le duc d'Anjou, voilà le roi
d'Espagne. La naissance l'appelait à cette couronne, le feu roi aussi
par son testament, toute la nation l'a souhaité et me l'a demandé
instamment\,; c'était l'ordre du ciel\,; je l'ai accordé avec
plaisir.\,» Et se tournant à son petit-fils\,: «\,Soyez bon Espagnol,
c'est présentement votre premier devoir, mais souvenez-vous que vous
êtes né Français, pour entretenir l'union entre les deux nations\,;
c'est le moyen de les rendre heureuses et de conserver la paix de
l'Europe. » Montrant après du doigt son petit-fils à l'ambassadeur\,:
«\,S'il suit mes conseils, lui dit-il, vous serez grand seigneur, et
bientôt\,; il ne saurait mieux faire que de suivre vos avis.\,»

Ce premier brouhaha du courtisan passé, les deux autres fils de France
arrivèrent, et tous trois s'embrassèrent tendrement et les larmes aux
yeux à plusieurs reprises. Zinzendorf envoyé de l'empereur, qui a depuis
fait une grande fortune à Vienne, avait demandé audience dans
l'ignorance de ce qui se devait passer, et dans la même ignorance
attendait en bas dans la salle des ambassadeurs que l'introducteur le
vînt chercher pour donner part de la naissance de l'archiduc,
petits-fils de l'empereur, qui mourut bientôt après. Il monta donc sans
rien savoir de ce qui venait de se passer. Le roi fit passer le nouveau
monarque et l'ambassadeur d'Espagne dans ses arrière-cabinets, puis fit
entrer Zinzendorf, qui n'apprit qu'en sortant le fâcheux contretemps
dans lequel il était tombé. Ensuite le roi alla à la messe à la tribune,
à l'ordinaire, mais le roi d'Espagne avec lui et à sa droite. À la
tribune, la maison royale, c'est-à-dire jusqu'aux petits-fils de France
inclusivement, et non plus, se mettaient à la rangette et de suite sur
le drap de pied du roi\,; et comme là, à la différence du prie-Dieu, ils
étaient tous appuyés comme lui sur la balustrade couverte du tapis, il
n'y a voit que le roi seul qui eût un carreau par-dessus la banquette,
et eux tous étaient à genoux sur la banquette couverte du même drap de
pied, et tous sans carreau. Arrivant à la tribune, il ne se trouva que
le carreau du roi qui le prit et le présenta au roi d'Espagne, lequel
n'ayant pas voulu l'accepter, il fut mis à côté, et tous deux
entendirent la messe sans carreau. Mais après il yen eut toujours deux
quand ils allaient à la même messe, ce qui arriva fort souvent.

Revenant de la messe, le roi s'arrêta dans la pièce du lit du grand
appartement, et dit au roi d'Espagne que désormais ce serait le sien\,;
il y coucha dès le même soir, et il y reçut toute la cour qui en foule
alla lui rendre ses respects. Villequier, premier gentilhomme de la
chambre du roi, en survivance du duc d'Aumont, son père, eut ordre de le
servir\,; et le roi lui céda deux de ses cabinets, où on entre de cette
pièce, pour s'y tenir lorsqu'il serait en particulier, et ne pas rompre
la communication des deux ailes qui n'est que par ce grand appartement.

Dès le même jour on sut que le roi d'Espagne partirait le 1er
décembre\,; qu'il serait accompagné des deux princes, ses frères, qui
demandèrent d'aller jusqu'à la frontière\,; que M. de Beauvilliers
aurait l'autorité dans tout le voyage sur les princes et les courtisans,
et le commandement seul sur les gardes\,; les troupes, les officiers et
la suite, et qu'il réglerait, disposerait seul de toutes choses. Le
maréchal-duc de Noailles lui fut joint, non pour se mêler, ni ordonner
de quoi que ce soit en sa présence, quoique maréchal de France et
capitaine des gardes du corps, mais pour le suppléer en tout en cas de
maladie ou d'absence du lieu où seraient les princes. Toute la jeunesse
de la cour, de l'âge à peu près des princes, eut permission de faire le
voyage, et beaucoup y allèrent ou entre eux ou dans les carrosses de
suite. On sut encore que de Saint-Jean de Luz, après la séparation, les
deux princes iraient voir la Provence et le Languedoc, passant par un
coin du Dauphiné\,; qu'ils reviendraient par Lyon, et que le voyage
serait de quatre mois. Cent vingt gardes sous Vaudreuil, lieutenant, et
Montesson, enseigne, avec des exempts, furent commandés pour les suivre,
et MM. de Beauvilliers et de Noailles eurent chacun cinquante mille
livres pour leur voyage.

Monseigneur, qui savait l'heure que le roi s'était réglée pour la
déclaration du roi d'Espagne, l'apprit à ceux qui étaient à Meudon\,; et
Monsieur, qui en eut le secret en partant de Fontainebleau, se mit sous
sa pendule dans l'impatience de l'annoncer, et quelques minutes avant
l'heure ne put s'empêcher de dire à sa cour qu'elle allait apprendre une
grande nouvelle, qu'il leur dit, dès que l'aiguille arrivée sur l'heure
le lui permit. Dès le vendredi précédent, Mgr le duc de Bourgogne, M. le
duc d'Anjou et l'ambassadeur d'Espagne le surent, et en gardèrent si
bien le secret qu'il n'en transpira rien à leur air ni à leurs manières.
M\textsuperscript{me} la duchesse de Bourgogne le sut en arrivant de
Fontainebleau, et M. le duc de Berry le lundi matin. Leur joie fut
extrême, quoique mêlée de l'amertume de se séparer\,; ils étaient
tendrement unis, et si la vivacité et l'enfance excitaient quelquefois
de petites riottes entre le premier et le troisième, c'était toujours le
second, naturellement sage, froid et réservé, qui les raccommodait.

Aussitôt après la déclaration, le roi la manda par le premier écuyer au
roi et à la reine d'Angleterre. L'après-dînée le roi d'Espagne alla voir
Monseigneur à Meudon, qui le reçut à la portière et le conduisit de
même. Il le fit toujours passer devant lui partout, et lui donna de la
\emph{Majesté\,;} en public ils demeurèrent debout. Monseigneur parut
hors de lui de joie. Il répétait souvent que jamais homme ne s'était
trouvé en état de dire comme lui\,: Le roi mon père, et le roi mon fils.
S'il avait su la prophétie qui dès sa naissance avait dit de lui\,: Fils
de roi, père de roi, et jamais roi, et que tout le monde avait ouï
répéter mille fois, je pense que, quelque vaines que soient ces
prophéties, il ne s'en serait pas tant réjoui. Depuis cette déclaration,
le roi d'Espagne fut traité comme le roi d'Angleterre. Il avait à souper
un fauteuil et son cadenas à la droite du roi, Monseigneur et le reste
de la famille royale des ployants au bout, et au retour de la table à
l'ordinaire, pour boire, une soucoupe et un verre couvert, et l'essai
comme pour le roi. Ils ne se voyaient en public qu'à la chapelle, et
pour y aller et en revenir, et à souper, au sortir duquel le roi le
conduisait jusqu'à la porte de la galerie. Il vit le roi et la reine
d'Angleterre à Versailles et à Saint-Germain, et ils se traitèrent comme
le roi et le roi d'Angleterre en tout, mais les trois rois ne se
trouvèrent jamais nulle part tous trois ensemble. Dans le particulier,
c'est-à-dire dans les cabinets et chez M\textsuperscript{me} de
Maintenon, il vivait en duc d'Anjou avec le roi qui, au premier souper,
se tourna à l'ambassadeur d'Espagne, et lui dit qu'il croyait encore que
tout ceci était un songe. Il ne vit qu'une fois M\textsuperscript{me} la
duchesse de Bourgogne et Mgrs ses frères, en cérémonie, chez lui et chez
eux. La visite se passa comme la première du roi d'Angleterre, et de
même avec Monsieur et Madame qu'il alla voir à Paris. Quand il sortait
ou rentrait, la garde battait aux champs\,; en un mot toute égalité avec
le roi. Lorsque, allant ou venant de la messe, ils passaient ensemble le
grand appartement, le roi prenait la droite, et à la dernière pièce la
quittait au roi d'Espagne, parce qu'alors il n'était plus dans son
appartement. Les soirs il les passait chez M\textsuperscript{me} de
Maintenon, dans des pièces séparées de celles où elle était avec le roi,
et là il jouait à toutes sortes de jeux, et le plus ordinairement à
courre comme des enfants avec Mgrs ses frères, M\textsuperscript{me} la
duchesse de Bourgogne qui s'occupait fort de l'amuser et ce petit nombre
de dames à qui cet accès était permis.

Le nonce et l'ambassadeur de Venise, un moment après la déclaration,
fendirent la presse et allèrent témoigner leur joie au roi et au nouveau
roi, ce qui fut extrêmement remarqué. Les autres ministres étrangers se
tinrent sur la réserve, assez embarrassés\,; mais l'état de Zinzendorf,
qui demeura quelque temps dans le salon au sortir de son audience, fut
une chose tout à fait singulière et curieuse. Je pense qu'il eût acheté
cher un mot d'avis à temps d'être demeuré à Paris. Bientôt après
l'ambassadeur de Savoie, et tous les ministres des princes d'Italie,
vinrent saluer et féliciter le roi d'Espagne.

Le mercredi 17 novembre, Harcourt fut déclaré duc héréditaire et
ambassadeur en Espagne, avec ordre d'attendre le roi d'Espagne à Bayonne
et de l'accompagner à Madrid. Tallard était encore à Versailles sur son
départ pour retourner à Londres, où le roi d'Angleterre était arrivé de
Hollande. C'était l'homme du monde le plus rongé d'ambition et de
politique. Il fut si outré de voir son traité de partage renversé, et
Harcourt duc héréditaire, qu'il en pensa perdre l'esprit. On le voyait
des fenêtres du château se promener tout seul dans le jardin, sur les
parterres, ses bras en croix sur sa poitrine, son chapeau sur ses yeux,
parlant tout seul et gesticulant parfois comme un possédé. Il avait
voulu, comme nous l'avons vu, se donner l'honneur du traité de partage,
comme Harcourt laissait croire tant qu'il pouvait que le testament était
son ouvrage, dont il n'avait jamais su un mot que par l'ouverture de la
dépêche du roi à Bayonne, comme je l'ai raconté, ni Tallard n'avait eu
d'autre part au traité de partage que la signature. Dans cet état de
rage, ce dernier, arrivant pour dîner chez Torcy, trouva qu'on était à
table, et perçant dans une autre pièce sans dire mot, y jeta son chapeau
et sa perruque sur des sièges, et se mit à déclamer tout haut et tout
seul sur l'utilité du traité de partage, les dangers de l'acceptation du
testament, le bonheur d'Harcourt qui sans y avoir rien fait lui enlevait
sa récompense. Tout cela fut accompagné de tant de dépit, de jalousie,
mais surtout de grimaces et de postures si étranges, qu'à la fin il fut
ramené à lui-même par un éclat de rire dont le grand bruit le fit
soudainement retourner en tressaillant, et il vit alors sept ou huit
personnes à table, environnées de valets, qui mangeaient dans la même
pièce, et qui s'étant prolongé le plus qu'ils avaient pu le plaisir de
l'entendre, et celui de le voir par la glace vers laquelle il était
tourné debout à la cheminée, n'avaient pu y tenir plus longtemps,
avaient tous à la fois laissé échapper ce grand éclat de rire. On peut
juger de ce que devint Tallard à ce réveil, et tous les contes qui en
coururent par Versailles.

Le vendredi 19 novembre, le roi d'Espagne prit le grand deuil.
Villequier dans les appartements, et ailleurs un lieutenant des gardes,
portèrent la queue de son manteau. Deux jours après, le roi le prit en
violet à l'ordinaire et drapa ainsi que ceux qui drapent avec lui. Le
lundi 22 on eut des lettres de l'électeur de Bavière, de Bruxelles, pour
reconnaître le roi d'Espagne. Il le fit proclamer parmi les \emph{Te
Deum}, les illuminations et les réjouissances, et nomma le marquis de
Bedmar, mestre de camp général des Pays-Bas, pour venir ici de sa part.
Le même jour, le parlement en corps et en robes rouges, mais sans
fourrures ni mortiers, vint saluer le roi d'Espagne. Le premier
président le harangua, ensuite la chambre des comptes et les autres
cours, conduites par le grand maître des cérémonies. Le roi d'Espagne ne
se leva point de son fauteuil pour pas un de ces corps, mais il demeura
toujours découvert. Chez le prince de Galles à Saint-Germain, et chez
Monsieur à Paris, il ne s'assit point et fut reçu et conduit à sa
portière comme il avait été à Meudon. Le mercredi 24, le roi alla à
Marly jusqu'au samedi suivant\,; le roi d'Espagne fut du voyage. Tout
s'y passa comme à Versailles, excepté qu'il fut davantage parmi tout le
monde dans le salon. Il mangea toujours à la table du roi, dans un
fauteuil à sa droite.

L'ambassadeur de Hollande, contre tout usage des ministres étrangers,
alla par les derrières chez Torcy se plaindre amèrement de l'acceptation
du testament, de la part de ses maîtres. L'ambassadeur d'Espagne y amena
le marquis de Bedmar, que le roi vit longtemps seul dans son cabinet. Le
prince de Chimay, et quelques autres Espagnols et Flamands qui les
accompagnaient, saluèrent aussi les deux rois\,; le nôtre, les promena
dans les jardins, et leur en fit les honneurs en présence du roi
d'Espagne. Ils furent surpris de ce que le roi fit à l'ordinaire couvrir
tout le monde et eux-mêmes\,; il s'en aperçut, et leur dit que jamais on
ne se couvrait devant lui, mais qu'aux promenades il ne voulait pas que
personne s'enrhumât.

Le dimanche 28, l'ambassadeur d'Espagne apporta au roi des lettres de M.
de Vaudemont, gouverneur du Milanais, qui y avait fait proclamer le roi
d'Espagne, avec les mêmes démonstrations de joie qu'à Bruxelles, et qui
donnait les mêmes assurances de fidélité. Bedmar retourna en Flandre,
après avoir encore entretenu le roi, auquel il plut fort. Les courriers
d'Espagne pleuvaient, avec des remerciements et des joies non pareilles
dans les lettres de la junte. Le 1er décembre, le chancelier, à la tête
du conseil en corps, alla prendre congé du roi d'Espagne, mais sans
harangue, l'usage du conseil étant de ne haranguer pas même le roi. Le
lundi 2, le, roi d'Espagne fit grand d'Espagne de la première classe le
marquis de Castel dos Rios, ambassadeur d'Espagne, et prit sans
cérémonie la Toison d'or, conservant l'ordre du Saint-Esprit, qui par
ses statuts est compatible avec cet ordre et celui de la Jarretière
seulement. Il la porta avec un ruban noir cordonné, en attendant d'en
recevoir le collier en Espagne par le plus ancien chevalier. La manière
de porter la Toison a fort varié, et est maintenant fixée au ruban rouge
ondé au cou. D'abord ce fut pour tous les jours un petit collier léger
sur le modèle de celui des jours de cérémonie\,; il dégénéra en chaîne
ordinaire, puis se mit à la boutonnière par commodité. Un ruban succéda
à la chaîne, soit au cou, soit à la boutonnière, et comme il n'était pas
de l'institution, la couleur en fut indifférente\,; enfin la noire
prévalut par l'exemple et le nombre des chevaliers graves et âgés,
jusqu'à ce que l'électeur de Bavière, étant devenu gouverneur des
Pays-Bas, préféra le rouge comme d'un plus ancien usage et plus parant.
À son exemple, tous les chevaliers de la Toison des Pays-Bas et
d'Allemagne prirent le ruban rouge ondé, et le roi d'Espagne le prit de
même bientôt après l'avoir porté en noir, et personne depuis ne l'a plus
porté autrement, ni à la boutonnière, que pour la chasse.

La maison royale, les princes et princesses du sang, toute la cour, le
nonce, les ambassadeurs de Venise et de Savoie, les ministres des
princes d'Italie prirent congé du roi d'Espagne qui ne fit aucune visite
d'adieu. Le roi donna aux princes ses petits-fils vingt et une bourses
de mille louis chacune, pour leur poche et leurs menus plaisirs pendant
le voyage, et beaucoup d'argent d'ailleurs pour les libéralités.

Enfin le samedi 4 décembre, le roi d'Espagne alla chez le roi avant
aucune entrée, et y resta longtemps seul, puis descendit chez
Monseigneur avec qui il fut aussi seul longtemps. Tous entendirent la
messe ensemble à la tribune\,; la foule des courtisans était incroyable.
Au sortir de la messe ils montèrent tout de suite en carrosse\,:
M\textsuperscript{me} la duchesse de Bourgogne entre les deux rois au
fond, Monseigneur au-devant entre Mgrs ses autres deux fils, Monsieur à
une portière et Madame à l'autre, environnés en pompe de beaucoup plus
de gardes que d'ordinaire, des gens d'armes et des chevau-légers\,; tout
le chemin jusqu'à Sceaux jonché de carrosses et de peuple, et Sceaux, où
ils arrivèrent un peu après midi plein de dames et de courtisans, gardé
par les deux compagnies des mousquetaires. Dès qu'ils eurent mis pied à
terre, le roi traversa tout l'appartement bas, entra seul dans la
dernière pièce avec le roi d'Espagne, et fit demeurer tout le monde dans
le salon. Un quart d'heure après il appela Monseigneur qui était resté
aussi dans le salon, et quelque temps après l'ambassadeur d'Espagne qui
prit la congé du roi son maître. Un moment après il fit entrer ensemble
Monseigneur et M\textsuperscript{me} la duchesse de Bourgogne, M. le duc
de Berry, Monsieur et Madame, et après un court intervalle les princes
et les princesses du sang. La porte était ouverte à deux battants, et du
salon on les voyait tous pleurer avec amertume. Le roi dit au roi
d'Espagne, en lui présentant ces princes\,: «\,Voici les princes de mon
sang et du vôtre\,; les deux nations présentement ne doivent plus se
regarder que comme une même nation, ils doivent avoir les mêmes
intérêts\,; ainsi je souhaite que ces princes soient attachés à vous
comme à moi\,; vous ne sauriez avoir d'amis plus fidèles ni plus
assurés.\,» Tout cela dura bien une heure et demie. À la fin il fallut
se séparer. Le roi conduisit le roi d'Espagne jusqu'au bout de
l'appartement, et l'embrassa à plusieurs reprises et le tenant longtemps
dans ses bras, Monseigneur de même. Le spectacle fut extrêmement
touchant.

Le roi rentra quelque temps pour se remettre, Monseigneur monta seul en
calèche et s'en alla à Meudon, et le roi d'Espagne avec Mgrs ses frères
et M. de Noailles dans son carrosse pour aller coucher à Chartres. Le
roi se promena ensuite en calèche avec M\textsuperscript{me} la duchesse
de Bourgogne, Monsieur et Madame, puis retournèrent tous à Versailles.
Desgranges, maître des cérémonies, et Noblet un des premiers commis de
Torcy, pour servir de secrétaire, suivirent au voyage. Louville, de qui
j'ai souvent parlé, Montriel et Valouse pour écuyers, Hersent, premier
valet de garde-robe, et Laroche pour premier valet de chambre, suivirent
pour demeurer en Espagne, avec quelques menus domestiques de chambre et
de garde-robe, et quelques gens pour la bouche et de médecine.

M. de Beauvilliers, qui se crevait de quinquina pour arrêter une fièvre
opiniâtre accompagnée d'un fâcheux dévoiement, mena
M\textsuperscript{me} sa femme à qui M\textsuperscript{me}s de Cheverny
et de Rasilly tinrent compagnie. Le roi voulut absolument qu'il se mît,
en chemin et qu'il tâchât de faire le voyage. Il l'entretint longtemps
le lundi matin avant que personne fût entré ni lui sorti du lit, d'où M.
de Beauvilliers monta tout de suite en carrosse pour aller coucher à
Étampes et joindre le roi d'Espagne le lendemain à Orléans. Laissons-les
aller, et admirons la Providence qui se joue des pensées des hommes et
dispose des États. Qu'auraient dit Ferdinand et Isabelle, Charles-Quint
et Philippe II qui ont voulu envahir la France à tant de différentes
reprises, qui ont été si accusés d'aspirer à la monarchie universelle,
et Philippe IV même, avec toutes ses précautions au mariage du roi et à
la paix des Pyrénées, de voir un fils de France devenir roi d'Espagne
par le testament du dernier de leur sang en Espagne et par le vœu
universel de tous les Espagnols, sans dessein, sans intrigue, sans une
amorce tirée de notre part, et à l'insu du roi, à son extrême surprise
et de tous ses ministres, et qui n'eut que l'embarras de se déterminer
et la peine d'accepter\,? Que de grandes et sages réflexions à faire,
mais qui ne seraient pas en place dans ces Mémoires\,! Reprenons ce qui
s'est passé dont je n'ai pas voulu interrompre une suite si curieuse et
si intéressante.

Cependant on avait appris que la nouvelle de l'acceptation du testament
avait causé à Madrid la plus extrême joie, aux acclamations de laquelle
le nouveau roi Philippe V avait été proclamé à Madrid, où les seigneurs,
le bourgeois et le peuple donnaient tous les jours quelque marque
nouvelle de leur haine pour les Allemands et pour la reine que presque
tout son service avait abandonnée, et à qui on refusait les choses les
plus ordinaires de son entretien. On apprit par un autre courrier de
Naples dépêché par le duc de Medina-Celi, vice-roi, que le roi d'Espagne
y avait été reconnu et proclamé avec la même joie\,; il le fut de même
en Sicile et en Sardaigne.

Quelque temps auparavant, il était arrivé une aventure assez désagréable
à Rome pour ce beau M. Vaïni, à qui la bassesse de donner
l'\emph{altesse} au cardinal de Bouillon avait valu l'ordre sans que le
roi s'en fût douté. Sa naissance était très commune, son mérite ne la
relevait pas, et ses affaires délabrées étaient en prise à des
créanciers de mauvaise humeur qui lui lâchèrent des sbires aux trousses
pour l'arrêter, n'osant pas trop faire exécuter ses meubles, parce que
les armes du roi étaient sur la porte de son palais, car tout est palais
en Italie et il ne s'y parle point de maison. Vaïni attaqué se battit en
retraite, et fut poursuivi jusque chez lui, où M. de Monaco, averti de
cette bagarre, accourut lui-même, et dit au commandant des sbires de se
retirer d'un palais qui n'était plus celui de Vaïni, mais le sien à lui,
ambassadeur, puisqu'il y était présent. Le commandant voulut se retirer,
mais quelques sbires n'obéissant pas, des gentilshommes de la suite de
M. de Monaco les chassèrent à coups d'épée, lui leur recommandant de
n'en point blesser. Des sbires qui étaient dans la rue, voyant qu'on
chassait ainsi leurs camarades, firent une décharge qui blessa quelques
domestiques de M. de Monaco, et qui blessa à mort le gentilhomme sur
lequel il s'appuyait, qui tomba, et l'ambassadeur sur lui. Cela fit
grand bruit dans Rome et peu d'honneur à M. de Monaco, qui se commit là
fort mal à propos en personne avec des canailles, et pour ce Vaïni qu'il
fallait protéger autrement, et qui n'était bon qu'à attirer de mauvaises
affaires. Il fut là fort tiraillé même par son cordon bleu. M. de
Monaco, mécontent de la lenteur du sacré collège sur cette affaire,
sortit de Rome avec éclat, sur quoi les trois chefs d'ordre qui se
trouvèrent de jour et qui étaient Acciaïuoli, Colloredo, et San-Cesareo
écrivirent au roi pour lui demander pardon au nom du sacré collège, et
quelle justice et satisfaction il lui plaisait prescrire. Le roi,
content de la soumission, les en laissa les maîtres, et manda au
cardinal d'Estrées qu'il voulait qu'on fît grâce, si on en condamnait
quelqu'un à mort.

San-Cesareo était aussi camerlingue, et de la maison Spinola, et fut
fort sur les rangs pour être pape avec un autre cardinal, Spinola
Marescotti, et Albano qui eut enfin toutes les voix, et qui eut vraiment
peine et sans feintise à se résoudre d'accepter le pontificat. Il était
de Pezzaro dans le duché d'Urbin, fils d'un avocat consistorial
qu'Urbain VIII avait fait sénateur. Notre pape avait pris la route des
petits gouvernements d'où Innocent XI le tira pour le faire secrétaire
des brefs, et son successeur Alexandre VIII le fit cardinal en 1690,
qu'il n'avait que quarante ans. C'était un homme de bien, mais qui
n'ayant jamais été au dehors, ni dans les congrégations importantes
pendant sa prélature, apporta peu d'expérience et de capacité à son
pontificat. Les Français eurent beaucoup de part à son exaltation, et le
cardinal de Bouillon entre autres qui eut la meilleure conduite du monde
dans le conclave avec nos cardinaux, et la plus française avec tous. Il
essuya tous les dégoûts que les nôtres lui donnèrent sans se fâcher ni
se détourner d'un pas de les seconder de toutes ses forces\,; et il fut
d'autant plus aise de l'exaltation d'Albano qu'il était son ami, qu'il
l'avait toujours porté, qu'il eut grande part au succès, et que ce pape,
qui s'était fait prêtre fort peu de jours avant d'entrer au conclave,
n'était point évêque, et devait être sacré par ses mains comme doyen du
sacré collège, comme il le sacra en effet. Il espéra donc recueillir le
fruit de sa bonne conduite et de la puissante recommandation du pape qui
la lui accorda en effet. Mais la mesure était comble et la colère du roi
ne se put apaiser. Nos cardinaux eurent ordre de revenir, excepté
Janson, chargé des affaires du roi à Rome, et Estrées, qui alla à Venise
où nous le retrouverons. Je ne sais par quelle fantaisie ce pape prit le
nom de Clément XI, dont il fit faire des excuses au cardinal Ottoboni,
de l'oncle duquel il était créature\,; il fut élu {[}le 24 novembre
1700{]}.

Le roi fit payer quatre cent mille livres au cardinal Radziewski, qu'il
prétendait avoir avancées pour l'élection manquée de M. le prince de
Conti, donna une grosse confiscation de vaisseaux de Dantzig qu'il avait
fait arrêter à l'abbé de Polignac, pour ses équipages, que ceux de cette
ville lui avaient pris, et reçut après leurs soumissions et leurs
pardons. Il donna aussi douze mille livres de pension à
M\textsuperscript{me} de Lislebonne, sœur de M. de Vaudemont, cinq mille
livres à la femme de, Mansart, et quatre mille livres à
M\textsuperscript{lle} de Croissy, sœur de Torcy, et le 23 novembre, il
fit Chamillart ministre, et lui ordonna de venir le lendemain au conseil
d'État. Il fut d'autant plus touché de cette importante grâce qu'il n'y
songeait pas encore. Le roi, qui l'aimait et qui s'en accommodait de
plus en plus, fut bien aise de lui hâter cette joie, et d'augmenter sa
considération et son crédit parmi les financiers dans un temps où il
prévoyait qu'il pourrait avoir besoin d'argent. Barbezieux, ami de
Chamillart, mais son ancien, et supérieur à lui en tant de manières, ne
lui en sut point mauvais gré, mais il prit cette préférence avec la
dernière amertume, et Pontchartrain se fit moquer de soi d'en paraître
fâché, et d'y avoir prétendu, et blâmer jusque par son père.

Cependant l'empereur se préparait à la guerre, et à avoir une armée en
Italie sous le prince Eugène, et une autre sur le Rhin que le prince
Louis de Bade devait commander. Mais il venait de se joindre de plus en
plus aux opposants au neuvième électorat. L'empereur lui en avait écrit
avec force et hauteur, il y avait répondu de même et mis le marché à la
main sur sa charge de feld-maréchal général de ses armées et de celles
de l'empire. S'étant assuré de la maison de Brunswick par ce neuvième
électorat, il s'acquit encore celle de Brandebourg, en adhérant à la
fantaisie de cet électeur.

Il possédait la Prusse à un étrange titre. Les chevaliers de l'ordre
Teutonique, chassés de Syrie par les Sarrasins, ne savaient où se
retirer, et ils étaient trente mille, tous Allemands. Rome, l'empire, la
Pologne, convinrent de leur donner la Prusse à conquérir sur les peuples
barbares et idolâtres qui en étaient les habitants et les maîtres, et
qui avaient un roi et une forme d'État. La conquête fut difficile,
longue, sanglante\,; à la fin elle réussit, et l'ordre Teutonique devint
très puissant. Le grand maître y était absolu et traité en roi avec une,
cour et de grands revenus\,; il y avait un maître de l'ordre sous le
grand maître, qui avait son état à part et grand nombre de commanderies.
La religion y fleurit et l'ordre avec elle jusqu'à entreprendre des
conquêtes, et d'envahir la Samogitie et la Lituanie, ce qui causa de
longues et de cruelles guerres entre eux et les Polonais. Luther ayant
répandu sa commode doctrine en Allemagne, ces chevaliers s'y engagèrent,
et usurpèrent héréditairement leurs commanderies. Albert de Brandebourg
était lors grand maître\,; il ruina tous les droits et les privilèges de
l'ordre qui l'avait élu, s'en appropria les richesses communes, se moqua
du pape et de l'empereur, et, sous prétexte de terminer la guerre de
Pologne, partagea la Prusse avec elle, dont la part fut appelée Prusse
royale, et la sienne ducale, et lui duc de Prusse. À son exemple,
Gothard Kettler qui était en même temps maître de l'ordre, s'appropria
la Courlande en duché héréditaire, sous la mouvance de la Pologne, et sa
postérité l'a conservée jusqu'à nos jours, que le dernier mâle étant
mort, la czarine en a su récompenser les services amoureux de
Byron\footnote{Saint-Simon a écrit Byron. Ce personnage est connu sons
  le nom de Biren et a joué un rôle important au XVIIIe siècle.},
gentilhomme tout simple du pays. Frédéric était petit-fils, fils et
frère des trois premiers électeurs de Brandebourg de la maison
d'aujourd'hui. Il eut trois fils entre autres de la fille de Casimir,
roi de Pologne\,: Casimir, qui fit la branche de Culmbach\,; qui servit
fort utilement Charles V et Ferdinand son frère\,; il laissa un fils
unique, mort sans postérité\,; Georges, qui fit la branche d'Anspach
l'ancienne, qui s'éteignit aussi dans son fils\,; et Albert qui, de
grand maître de l'ordre Teutonique, secoua le joug de Rome, de ses vœux,
de l'empire, et se fit duc héréditaire de Prusse, dont il prit
l'investiture du roi de Pologne.

Ainsi, la Prusse, qui était province de Pologne, fut séparée en deux,
comme je viens de dire, en 1525. Ce fut cet Albert, qui érigea
l'université de Kœnigsberg, capitale de la Prusse ducale\,; il mourut en
mars 1568, il ne laissa qu'un fils Albert-Frédéric, duc de Prusse, mort
imbécile en 1618, en qui finirent les trois branches susdites. Il avait
épousé en 1573 Marie-Éléonore, fille aînée de Guillaume, duc de Clèves,
Juliers, Berg, etc., sœur de J. Guillaume\,; mort sans enfants, 15 mars
1609, d'Anne, mariée au palatin de Neubourg, de Madeleine, femme d'autre
palatin, duc des Deux-Ponts, de Sibylle, marquise de Bade, puis de
Burgau de la maison d'Autriche, mais morte sans enfants de ses deux
maris. J. Sigismond, électeur de Brandebourg, eut donc de sa femme Anne,
fille aînée d'Albert-Frédéric de Brandebourg\,; duc de Prusse, et de
Marie-Éléonore, fille aînée de Guillaume, duc de Clèves et de Juliers,
et sœur de J. Guillaume, dernier duc de Clèves et Juliers, etc., eut,
dis-je, la Prusse et la prétention sur la succession de Clèves, Berg,
Juliers, etc., qu'il partagea enfin provisionnellement avec le palatin
de Neubourg. Frédéric-Guillaume, électeur de Brandebourg, petit-fils de
ce mariage, eut quelque pensée de faire ériger sa Prusse ducale en
royaume, par l'empereur, sans pousser plus loin cette idée. Frédéric III
son fils et son successeur la suivit davantage, et servit bien
l'empereur Léopold en Hongrie et sur le Rhin, où il ouvrit la guerre de
1688, par les sièges de Kaiserswerth et de Bonn qu'il prit en personne.
S'étant toujours depuis rendu nécessaire à l'empereur, il s'assura de
lui sur son dessein, et dans cette conjoncture favorable où l'empereur
cherchait partout des troupes, de l'argent et des alliés pour disputer
la succession d'Espagne, l'électeur donna un repas aux principaux de sa
cour dans lequel il leur porta la santé de Frédéric III, roi de Prusse
et électeur de Brandebourg, et se déclara roi de cette manière. Il fut
aussitôt traité de \emph{Majesté} par les conviés et par tout ce qui
n'osa ou ne voulut pas se brouiller avec, lui, et s'alla bientôt après
installer lui-même en cette nouvelle dignité à Kœnigsberg par un nouvel
hommage de toute la Prusse ducale. C'est le père de celui qui vient de
mourir et le grand-père de celui d'aujourd'hui.

La conduite de l'empereur, le murmure des Hollandais, le silence profond
de l'Angleterre, firent songer ici à se mettre en état de soutenir le
testament partout. Tessé fut envoyé à Milan concerter avec le prince de
Vaudemont les choses militaires, et choisi pour commander les troupes
que le roi enverrait au Milanais aux ordres de Vaudemont. Celui-ci
envoya bientôt après Colmenero, son confident et général d'artillerie,
au Milanais, rendre compte au roi de toutes choses et presser l'envoi
des troupes. On se mit aussi au meilleur ordre qu'on put par mer, et on
fit partir un gros corps de troupes sous des officiers généraux pour
passer au Milanais, partie par mer, partie par terre, M. de Savoie ayant
accordé lé passage de bonne grâce.

Le duc d'Ossone, jeune grand d'Espagne, vint saluer le roi, et ne baisa
point M\textsuperscript{me} la duchesse de Bourgogne, les grands
d'Espagne n'ayant jamais eu de rang en France. Sa figure ne donna pas
idée à notre cour de celle d'Espagne, il fut fort festoyé. Il trouva le
roi d'Espagne à Amboise, et comme il était gentilhomme de la chambre, il
le voulut servir à son dîner\,; mais M. de Beauvilliers lui fit entendre
que ce prince serait fort aise qu'il fit sa charge auprès de lui, dès
qu'il aurait passé la Bidassoa, mais que tant qu'il serait en France, il
voulait être servi à l'ordinaire par des Français. M. de Beauvilliers,
comme premier gentilhomme de la chambre du roi et le sien particulier
pour avoir été son gouverneur, le servit toujours tant que sa santé le
lui permit dans le voyage. Il entendait une messe tous les jours
séparément des deux autres princes ses frères, recevait seul, et sans
qu'ils se trouvassent présents, les harangues et les honneurs qui lui
étaient faits, et mangea toujours seul, et lorsqu'ils se trouvaient
ensemble en public, c'était toujours debout, en sorte qu'ils ne se
voyaient familièrement qu'en carrosse ou à porte fermée, et que tout
cérémonial était évité entre eux. Je ne sais pourquoi cela fut
imaginé\,; en Espagne, les infants ont un fauteuil, même en cérémonie,
devant le roi et la reine, qui est toujours à la vérité d'une, étoffe
moins riche\,; il est vrai qu'en public ils ne mangent point avec eux,
mais en particulier. Plusieurs grands d'Espagne écrivirent au roi pour
le remercier de l'acceptation du testament. Le roi leur répondit à tous,
et leur donna à tous le \emph{cousin} qu'ils ont aussi des rois
d'Espagne.

Le roi, qui traita toujours le marquis de Castel dos Rios avec grande
distinction et beaucoup de familiarité depuis l'acceptation du
testament, lui envoya beaucoup d'argent à différentes reprises, dont il
manquait fort sans en jamais parler\,; il l'accepta comme du grand-père
de son maître, avec grâce. C'était un très bon, honnête et galant homme,
à qui la tête ne tourna ni ne manqua dans cette conjoncture si
extraordinaire et si brillante, poli et considéré, et qui se fit aimer
et estimer de tout le monde. Le roi lui procura, au sortir d'ici, la
vice-royauté du Pérou pour l'enrichir, où il mourut au bout de quelques
années dans un âge médiocrement avancé. Il reçut tous ses diplômes de
grand d'Espagne de première classe gratis, par un courrier, aussitôt
après l'arrivée du roi d'Espagne à Madrid.

Le duc d'Harcourt était retourné à Madrid par ordre du roi, où il fut
reçu avec la plus grande joie. La junte, qui désira qu'il y assistât
quelquefois, lui donna le choix de sa place, qu'il prit à la gauche de
la reine, le cardinal Portocarrero étant à droite, et après lui ceux qui
la composent, la place de la reine demeurant vide en son absence, et
elle ne s'y trouvait presque jamais. Cette junte supplia le roi de
donner ses ordres dans tous les États du roi son petit-fils, et lui
manda qu'elle avait envoyé ordre à l'électeur de Bavière, au duc de
Medina-Celi, au prince de Vaudemont, en un mot à tous les vice-rois et
gouverneurs généraux et particuliers, ambassadeurs et ministres
d'Espagne, de lui obéir en tout sans attendre d'autres ordres sur tout
ce qu'il lui plairait de commander, de même à tous les officiers de
finance et autres de la monarchie.

Le Nord était cependant fort troublé, au grand déplaisir de l'empereur
qui avait moyenné la paix entre la Suède et le Danemark, à qui le jeune
roi de Suède avait fait grand mal et encore plus de peur par ses
conquêtes en personne. Le roi y entra aussi plus pour l'honneur que pour
l'effet. De là ce jeune prince attaqua les Moscovites, qu'il battit avec
une poignée de troupes contre près de cent mille hommes\,; il força
leurs retranchements à Narva, leur fit lever des sièges, les chassa de
la Livonie et des provinces voisines, et s'irrita fort contre le roi de
Pologne, qui s'était allié avec eux pour soutenir sa guerre d'Elbing,
dans laquelle la Pologne avait refusé d'entrer, et où Oginski, à la tête
d'un grand parti contre les Sapieha, ou plutôt contre le roi de Pologne,
remportait de grands avantages, ce qui empêchait l'empereur d'espérer du
Nord les secours dont il s'était flatté pour augmenter ses troupes. Il
cherchait en même temps de tous côtés à en acheter, il en farcissait le
Tyrol, et se donna beaucoup de mouvements à Rome pour empêcher le pape
de donner l'investiture de Naples et de Sicile au nouveau roi d'Espagne.
Il y réussit, mais d'autre côté le pape admit les nominations des
bénéfices de ce royaume faites par ce prince comme en étant roi, et fit
dire dans l'un et dans l'autre, qu'encore qu'il eût des raisons de
retarder l'investiture, il le reconnaissait pour seul roi de Naples et
de Sicile, et voulait qu'il y fût reconnu pour tel sans difficulté.
J'avance de quelques mois ce procédé du pape pour n'avoir pas à y
revenir.

\hypertarget{chapitre-iv.}{%
\chapter{CHAPITRE IV.}\label{chapitre-iv.}}

1701

~

{\textsc{Année 1701.}} {\textsc{- Mesures en Italie\,: Tessé.}}
{\textsc{- Mort et caractère de Barbezieux.}} {\textsc{- Chamillart
secrétaire d'État\,; son caractère.}} {\textsc{- Torcy chancelier et
Saint-Pouange grand trésorier de l'ordre.}} {\textsc{- Mort de Rose,
secrétaire du cabinet.}} {\textsc{- La plume.}} {\textsc{- Caillières a
la plume.}} {\textsc{- Rose et M. le prince.}} {\textsc{- Rose et M. de
Duras.}} {\textsc{- Rose et les Portail.}} {\textsc{- Mort de Stoppa,
colonel des gardes suisses.}} {\textsc{- Mort du prince de Monaco,
ambassadeur à Rome.}} {\textsc{- Mort de Bontems.}} {\textsc{- Bloin.}}
{\textsc{- M. de Vendôme.}} {\textsc{- Bals particuliers à la cour.}}

~

Il était donc question de se préparer à une guerre vive en Italie, où
Tessé avait été envoyé comme un homme agréable à M. de Savoie et à ses
ministres, qui avait négocié à Turin la dernière paix et le mariage de
M\textsuperscript{me} la duchesse de Bourgogne. C'était un homme doux,
liant, insinuant, avec plus de manège que d'esprit ni de capacité, mais
heureux en tout au dernier point, avec une figure fort noble, et un
langage de cour qu'il savait tourner et retourner. On avait un besoin
continuel de M. de Savoie pour le passage et les vivres, on s'en voulait
assurer pour allié\,; Mantoue aussi par sa situation était un objet
principal, et Tessé connaissait fort M. de Mantoue. Il était donc parti
chargé de beaucoup d'instructions, et si Torcy y avait beaucoup
travaillé pour le politique, Barbezieux avait eu une grande besogne à
dresser pour tous les détails des troupes, des vivres et des différentes
parties et plans de la guerre. Au fort de ce travail, il eut la douleur
de voir, comme je l'ai dit, Chamillart ministre dans le temps où on s'y
attendait le moins. Ce fut pour lui un coup de foudre. Depuis plus de
soixante ans ses pères avaient eu, dans sa même place, une très
principale part au gouvernement de l'État, et lui-même, depuis près de
dix ans qu'il la remplissait, ne s'y était guère moins acquis de crédit
et d'autorité qu'eux. Chamillart, tout nouveau et depuis deux ans en
place, en était encore à rechercher de lui faire sa cour, après avoir
été souvent dans l'antichambre de son père et dans la sienne. Cette
préférence lui fut insupportable en elle-même, et encore par le coup de
caveçon qu'elle lui donnait, et qui lui fit bien sentir qu'il n'était
pas saison de s'en plaindre. Chamillart, qui n'avait pas imaginé d'être
appelé sitôt au conseil d'État, lit en homme modeste et en bon ami tout
ce qu'il put pour le consoler.

Barbezieux ne fut point piqué contre lui\,; mais outré de la chose il ne
put se laisser adoucir le courage haut, fier, et présomptueux à l'excès.
Sitôt qu'il eut expédié Tessé, il se livra avec ses amis à la débauche
plus que de coutume pour dissiper son chagrin. Il avait bâti entre
Versailles et Vaucresson, au bout du parc de Saint-Cloud, une maison en
plein champ, qu'on appela l'Étang, qui dans la plus triste situation du
monde, mais à portée de tout, lui avait coûté des millions. Il y allait
souvent, et c'était là qu'il tâchait de noyer ses déplaisirs avec ses
amis dans la bonne chère et les autres plaisirs secrets\,; mais le
chagrin surnageait, qui, joint à des plaisirs au-dessus de ses forces
dans lesquelles il se fiait trop, lui donna le coup mortel. Il revint au
bout de quatre jours de l'Étang à Versailles avec un grand mal de gorge
et une fièvre ardente qui, dans un tempérament d'athlète comme était le
sien et à son âge, demandait force saignées que la vie qu'il venait de
mener rendait fort dangereuses. La maladie le parut dès le premier
moment\,; elle {[}ne{]} dura que cinq jours. À peine eut-il le temps de
faire son testament et de se confesser quand l'archevêque de Reims
l'avertit du danger pressant, contre lequel il disputait contre Fagon
même. Il mourut tout en vie avec fermeté, au milieu de sa famille, et sa
porte ayant été continuellement assiégée de toute la cour. Elle venait
de partir pour Marly\,; c'était la veille des Rois. Il finit avant
trente-trois ans, dans la même chambre où son père était mort.

C'était un homme d'une figure frappante, extrêmement agréable, fort
mâle, avec un visage gracieux et aimable, et une physionomie forte\,;
beaucoup d'esprit, de pénétration, d'activité, de la justesse et une
facilité incroyable au travail, sur laquelle il se reposait pour prendre
ses plaisirs, et en faisait plus et mieux en deux heures qu'un autre en
un jour. Toute sa personne, son langage, ses manières et son énonciation
aisée, juste, choisie, mais naturelle, avec de la force et de
l'éloquence, tout en était gracieux. Personne n'avait autant l'air du
monde, les manières d'un grand seigneur, tel qu'il eût bien voulu être,
les façons les plus polies et, quand il lui plaisait, les plus
respectueuses, la galanterie la plus naturelle et la plus fine, et des
grâces répandues partout. Aussi quand il voulait plaire, il charmait\,;
et quand il obligeait, c'était au triple de qui que ce fût par les
manières. Nul homme ne rapportait mieux une affaire, ni ne possédait
plus pleinement tous les détails, ni ne les mandait plus aisément que
lui. Il sentait avec délicatesse toutes les différences des personnes,
et avec capacité toutes celles des affaires, de leurs gradations, de
leur plus ou moins d'importance, et il épuisait les affaires d'une
manière surprenante\,; mais orgueilleux à l'excès, entreprenant, hardi,
insolent, vindicatif au dernier point, facile à se blesser des moindres
choses, et très difficile à en revenir. Son humeur était terrible et
fréquente\,; il la connaissait, il s'en plaignait, il ne la pouvait
vaincre\,; naturellement brusque et dur, il devenait alors brutal et
capable de toutes les insultes et de tous les emportements imaginables,
qui lui ont ôté beaucoup d'amis. Il les choisissait mal, et dans ses
humeurs il les outrageait quels qu'ils fussent, et les plus proches et
les plus grands, et après il en était au désespoir\,; changeant avec
cela, mais le meilleur et le plus utile ami du monde tandis qu'il
l'était, et l'ennemi le plus dangereux, le plus terrible, le plus suivi,
le plus implacable, et naturellement féroce\,: c'était un homme qui ne
voulait trouver de résistance en rien, et dont l'audace était extrême.

Il avait accoutumé le roi à remettre son travail, quand il avait trop
bu, ou qu'il avait une partie qu'il ne voulait pas manquer et lui
mandait qu'il avait la fièvre. Le roi le souffrait par l'utilité et la
facilité de son travail et le plaisir de croire tout faire et de former
un ministre\,; mais il ne l'aimait point, et s'apercevait très bien de
ses absences et de ses fièvres factices\,; mais M\textsuperscript{me} de
Maintenon qui avait perdu son père trop puissant, et par des raisons
personnelles, protégeait le fils qui était en respect devant elle et
hors d'état d'en sortir à son égard. C'était à tout prendre de quoi
faire un grand ministre, mais étrangement dangereux. C'est même une
question si ce fut une perte pour l'État par l'excès de son ambition\,:
mais ce n'en fut pas une pour la cour et le monde qui gagna beaucoup à
la mort d'un homme que tous ses talents n'auraient rendu que plus
terrible à mesure de sa puissance, et dont la sûreté était très médiocre
dans le commerce et fort accusée dans les affaires de sa gestion, non
par avarice, car c'était la libéralité, la magnificence et la
prodigalité même, qui l'avaient déjà mené bien loin, mais pour servir ou
pour nuire, et surtout pour aller à son but. On a vu sur le siège de
Barcelone et sur M. de Noailles un échantillon de ce qu'il savait faire.

Aussitôt qu'il fut mort, Saint-Pouange le vint dire au roi à Marly qui,
deux heures auparavant, partant de Versailles, s'y était si bien
attendu, qu'il avait laissé La Vrillière pour mettre le scellé partout.
Fagon qui l'avait condamné d'abord, et qui ne l'aimait point, non plus
que son père, fut accusé de l'avoir trop saigné exprès. Du moins lui
échappa-t-il des paroles de joie de ce qu'il n'en reviendrait point, une
des deux dernières fois qu'il sortit de chez lui. Il désolait souvent
par ses réponses qu'il faisait toujours haut à ses audiences où on lui
parlait bas, et faisait attendre les principales personnes de la cour,
hommes et femmes, tandis qu'il se jouait avec ses chiens dans son
cabinet ou avec quelque bas complaisant, et après s'être fait longtemps
attendre sortait souvent par les derrières\,; ses beaux-frères même
étaient toujours en brassière de ses humeurs, et ses meilleurs amis ne
l'abordaient qu'en tâtant le pavé. Beaucoup de gens et force belles
dames perdirent beaucoup à sa mort. Aussi y en eut-il plusieurs fort
éplorées dans le salon de Marly\,; mais quand elles se mirent à table et
qu'on eut tiré le gâteau, le roi témoigna une joie qui parut vouloir
être imitée. Il ne se contenta pas de crier\,: \emph{la reine boit\,!}
mais, comme en franc cabaret, il frappa et fit frapper chacun de sa
cuiller et de sa fourchette sur son assiette, ce qui causa un charivari
fort étrange, et qui à reprises dura tout le souper. Les pleureuses y
firent plus de bruit que les autres, et de plus longs éclats de rire, et
les plus proches et les meilleures amies en firent encore davantage\,:
le lendemain il n'y parut plus\,: On fut deux jours à raisonner de la
vacance\,; je me sus bon gré de ne m'y être pas trompé.

Chamillart était allé faire les Rois chez lui à Montfermeil, d'où il
avait été mandé pour la place de contrôleur général\,; ce fut encore au
même lieu où le roi lui manda le 7 par un valet de chambre de
M\textsuperscript{me} de Maintenon de se trouver le lendemain à son
lever, à l'issue duquel il le fit entrer dans son cabinet, et lui donna
la charge de Barbezieux. Chamillart, en homme sage, lui voulut remettre
les finances, ne trouvant pas avec raison de comparaison entre la
périlleuse place de contrôleur général et celle de secrétaire d'État de
la guerre\,; et sur ce que le roi ne voulut point qu'il les quittât, il
lui représenta l'impossibilité de s'acquitter de deux emplois ensemble
qui séparément avaient occupé tout entiers Colbert et Louvois\,; mais
c'était précisément le souvenir de ces deux ministres et de leurs
débats, qui faisait vouloir obstinément au roi de réunir les deux
ministères, et qui le rendit sourd à tout ce que Chamillart lui put
dire.

C'était un bon et très honnête homme, à mains parfaitement nettes et
avec les meilleures intentions, poli, patient, obligeant, bon ami,
ennemi médiocre, aimant l'État, mais le roi sur toutes choses, et
extrêmement bien avec lui et avec M\textsuperscript{me} de Maintenon\,;
d'ailleurs très borné et, comme tous les gens de peu d'esprit et de
lumière, très opiniâtre, très entêté, riant jaune avec une douce
compassion à qui opposait, des raisons aux siennes et entièrement
incapable de les entendre\,; par conséquent dupe en amis, en affaires et
en tout, et gouverné par ceux dont à divers égards il s'était fait une
grande idée, ou qui avec un très léger poids étoient fort de ses amis.
Sa capacité était nulle, et il croyait tout savoir en tout genre, et
cela était d'autant plus pitoyable, que cela lui était venu avec ses
places, et que c'était moins présomption que sottise, et encore moins
vanité dont il n'avait aucune. Le rare est que le grand ressort de la
tendre affection du roi pour lui était cette incapacité même. Il
l'avouait au roi à chaque pas, et le roi se complaisait à le diriger et
à l'instruire\,; en sorte qu'il était jaloux de ses succès comme du sien
propre, et qu'il en excusait tout. Le monde aussi et la cour l'excusait
de même, charmé de la facilité de son abord, de sa joie d'accorder ou de
servir, de la douceur et de la douleur de ses refus et de son
infatigable patience à écouter. Sa mémoire lui représentait fort
nettement les gens et les choses malgré la multitude qui en passait par
ses mains, en sorte que chacun était ravi de voir que son affaire lui
était parfaitement présente quoique entamée et délaissée depuis
longtemps. Il écrivait aussi fort bien, et ce style net, et coulant, et
précis plaisait extrêmement au roi et à M\textsuperscript{me} de
Maintenon qui ne cessaient de le louer, de l'encourager et de
s'applaudir d'avoir mis sur de si faibles épaules deux fardeaux\,; dont
chacun eût suffi à accabler les plus fortes.

Torcy eut la charge de chancelier de l'ordre qu'avait Barbezieux\,; et
la sienne de grand trésorier de l'ordre, le roi en voulut récompenser
Saint-Pouange qui ne pouvait plus servir de principal commis à un
étranger, comme il avait fait sous ses plus proches, dont il avait
toujours eu le plus intime secret et souvent par là celui du roi sur les
choses de la guerre, avec lequel même il avait eu souvent occasion de
travailler. En même temps il vendit sa charge de secrétaire du cabinet à
Charmont, des Hennequin de Paris, qui se défit de sa charge de procureur
général du grand conseil, et qui fut ensuite ambassadeur à Venise, où il
ne réussit pas. Saint-Pouange, qui avait depuis longtemps la charge
d'intendant de l'ordre, la vendit à La Cour des Chiens, fameux
financier.

Rose, autre secrétaire du cabinet du roi et qui depuis cinquante ans
avait la plume, mourut en ce temps-ci à quatre-vingt-six ou sept ans,
avec toute sa tête et dans une santé parfaite jusqu'au bout. Il était
aussi président à la chambre des comptes, fort riche et fort avare, mais
c'était un homme de beaucoup d'esprit, et qui avait des saillies et des
reparties incomparables, beaucoup de lettres, une mémoire nette et
admirable, et un parfait répertoire de cour et d'affaires, gai, libre,
hardi, volontiers audacieux\,; mais à qui ne lui marchait point sur le
pied, poli, respectueux, tout à fait en sa place, et sentant extrêmement
la vieille cour. Il avait été au cardinal Mazarin et fort dans sa
privance et sa confiance, ce qui l'y avait mis avec la reine mère et
qu'il se sut toujours conserver avec elle et avec le roi jusqu'à sa mort
en sorte qu'il était compté et ménagé même par tous les ministres. Sa
plume l'avait entretenu dans une sorte de commerce avec le roi, et
quelquefois d'affaires qui demeuraient ignorées des ministres. Avoir la
plume, c'est être faussaire public, et faire par charge ce qui coûterait
la vie à tout autre. Cet exercice consiste à imiter si exactement
l'écriture du roi qu'elle ne se puisse distinguer de celle que la plume
contrefait, et d'écrire en cette sorte toutes les lettres que le roi
doit ou veut écrire de sa main et toutefois n'en veut pas prendre la
peine. Il y en a quantité aux souverains et à d'autres étrangers de haut
parage\,; il y en a aux sujets, comme généraux d'armée ou autres gens
principaux par secret d'affaires ou par marque de bonté ou de
distinction. Il n'est pas possible de faire parler un grand roi avec
plus de dignité que faisait Rose, ni plus convenablement à chacun, ni
sur chaque matière, que les lettres qu'il écrivait ainsi, et que le roi
signait toutes de sa main, et pour le caractère il était si semblable à
celui du roi qu'il ne s'y trouvait pas la moindre différence. Une
infinité de choses importantes avait passé par les mains de Rose, et il
y en passait encore quelquefois. Il était extrêmement fidèle et secret,
et le roi s'y fiait entièrement. Ainsi celui des quatre secrétaires du
cabinet qui a la plume en a toutes les fonctions, et les trois autres
n'en ont aucune, sinon leurs entrées.

Caillières eut la plume à la mort de Rose. Ce bonhomme était fin, rusé,
adroit et dangereux\,; il y a de lui des histoires sans nombre, dont je
rapporterai deux ou trois seulement, parce qu'elles le caractérisent lui
et ceux dont il s'y agit. Il avait fort près de Chantilly une belle
terre et bien bâtie qu'il aimait fort, et où il allait souvent\,; il
rendait force respects à M. le Prince (c'est du dernier mort dont je
parle), mais il était attentif à ne s'en pas laisser dominer chez lui.
M. le Prince, fatigué d'un voisinage qui le resserrait, et peut-être
plus que lui, ses officiers de chasse, fit proposer à Rose de l'en
accommoder\,; celui-ci n'y voulut jamais entendre ni s'en défaire pour
quoi que ce fût. À la fin M. le Prince, hors de cette espérance, se mit
à lui faire des niches pour le dégoûter et le résoudre\,; et de niche en
niche, il lui fit jeter trois ou quatre cents renards ou renardeaux,
qu'il fit prendre et venir de tous côtés, par-dessus les murailles de
son parc. On peut se représenter quel désordre y fit cette compagnie, et
la surprise extrême de Rose et de ses gens d'une fourmilière inépuisable
de renards venus là en une nuit.

Le bonhomme, qui était colère et véhément et qui connaissait bien M. le
Prince, ne se méprit pas à l'auteur du présent. Il s'en alla trouver le
roi dans son cabinet, et tout résolument lui demanda la permission de
lui faire une question peut-être un peu sauvage. Le roi fort accoutumé à
lui et à ses goguenarderies, car il était plaisant et fort salé, lui
demanda ce que c'était. «\,Ce que c'est, sire, lui répondit Rose d'un
visage enflammé, c'est que je vous prie de me dire si nous avons deux
rois en France. --- Qu'est-ce à dire\,? dit le roi surpris, et
rougissant à son tour. --- Qu'est-ce à dire\,? répliqua Rose, c'est que
si M. le Prince est roi comme vous, il faut pleurer et baisser la tête
sous ce tyran. S'il n'est que premier prince du sang, je vous en demande
justice, sire, car vous la devez à tous vos sujets, et vous ne devez pas
souffrir qu'ils soient la proie de M. le Prince.\,» Et de là lui conte
comme il l'a voulu obliger à lui vendre sa terre, et après l'y forcer en
le persécutant, et raconte enfin l'aventure des renards. Le roi lui
promit qu'il parlerait à M. le Prince de façon qu'il aurait repos
désormais. En effet, il lui ordonna de faire ôter par ses gens et à ses
frais jusqu'au dernier renard du parc du bonhomme, et de façon qu'il ne
s'y fit aucun dommage, et qu'il réparât ceux que les renards y avaient
faits\,; et pour l'avenir lui imposa si bien, que M. le Prince, plus bas
courtisan qu'homme du monde, se mit à rechercher Rose, qui se tint
longtemps sur son fier, et oncques depuis n'osa le troubler en la
moindre chose. Malgré tant d'avances, qu'il fallut bien enfin recevoir,
il la lui gardait toujours bonne, et lui lâchait volontiers quelque
brocard. Moi et cinquante autres en filmes un jour témoins.

Les jours de conseil, les ministres s'assemblaient dans la chambre du
roi sur la fin de la messe, pour entrer dans le cabinet quand on les
appelait pour le conseil, lorsque le roi était rentré par la galerie
droit dans ses cabinets. Il y avait toujours des courtisans à ces
heures-là dans la chambre du roi, ou qui avaient affaire aux ministres,
à qui ils parlaient là plus commodément quand ils avaient peu à leur
dire, ou pour causer avec eux. M. le Prince y venait souvent, et il
était vrai qu'il leur parlait à tous sans avoir rien à leur dire, avec
le maintien d'un client qui fait bassement sa cour. Rose, à qui rien
n'échappait, prit sa belle qu'il y avait beaucoup du meilleur de la cour
que le hasard y avait rassemblé ce jour-là, et que M. le Prince avait
cajolé les ministres avec beaucoup de souplesse et de flatterie. Tout
d'un coup le bonhomme, qui le voyait faire, s'en va droit à lui, et
clignant un œill avec un doigt dessous, qui était quelquefois son
geste\,: «\,Monsieur, lui dit-il tout haut, je vous vois faire ici un
manège avec tous ces messieurs, et depuis plusieurs jours, et ce n'est
pas pour rien\,; je connais ma cour et mes gens depuis longues années,
on ne m'en fera pas accroire je vois bien où cela va\,;» et avec des
tours et des inflexions de voix, qui embarrassaient tout à fait M. le
Prince, qui se défendait comme il pouvait. Ce dialogue amassa les
ministres, et ce qu'il y avait là de principal autour d'eux. Comme Rose
se vit bien environné et le conseil sur le point d'être appelé, il prend
respectueusement M. le Prince par le bout du bras avec un souris fin et
malin\,: «\,Serait-ce point, monsieur, lui dit-il, que vous voudriez
vous faire premier prince du sang\,?» et à l'instant fait la pirouette,
et s'écoule. Qui demeura stupéfait\,? ce fut M. le Prince, et toute
l'assistance à rire sans pouvoir s'en empêcher. C'était là, de ces tours
hardis de Rose\,; celui-là fit plusieurs jours l'amusement et
l'entretien de la cour. M. le Prince fut enragé\,; mais il ne put et
n'osa que dire. Il n'y avait guère plus d'un an de cette aventure,
lorsque ce bonhomme mourut.

Il n'avait jamais pardonné à M. de Duras un trait, qui en effet fut une
cruauté. C'était à un voyage de la cour\,; la voiture de Rose avait été,
je ne sais comment, déconfite. D'impatience, il avait pris un cheval. Il
n'était pas bon cavalier\,; lui et le cheval se brouillèrent, et le
cheval s'en défit dans un bourbier. Passa M. de Duras, à qui Rose cria à
l'aide de dessous son cheval au milieu du bourbier. M. de Duras, dont le
carrosse allait doucement dans cette fange, mit la tête à la portière,
et pour tout secours se mit à rire et à crier que c'était là un cheval
bien délicieux, de se rouler ainsi sur les roses\,; et continua son
chemin et le laissa là. Vint après le duc de Coislin, qui fut plus
charitable, et qui le ramassa\,; mais si furieux et si hors de soi de
colère, que la carrossée fut quelque temps sans pouvoir apprendre à qui
il en avait. Mais le pis fut à la couchée. M. de Duras, qui ne craignait
personne, et qui avait le bec aussi bon que Rose, en avait fait le conte
au roi et à toute la cour, qui en rit fort. Cela outra Rose à un point
qu'il n'a depuis jamais approché de M. de Duras, et n'en a parlé qu'en
furie, et quand quelquefois il hasardait devant le roi quelque lardon
sur lui, le roi se mettait à rire, et lui parlait du bourbier.

Sur la fin de sa vie, il avait marié sa petite-fille fort riche, et qui
attendait encore de plus grands biens de lui, à Portail, qui longtemps
depuis est mort premier président du parlement de Paris. Le mariage ne
fut point concordant\,; la jeune épouse, qui se sentait riche parti,
méprisait son mari, et disait qu'au lieu d'entrer en quelque bonne
maison elle était demeurée au portail. À la fin, le père, vieux
conseiller de grand'chambre, et le fils firent leurs plaintes au
bonhomme, d'abord il n'en tint pas grand compte, et comme elles
recommencèrent il leur promit de parler à sa petite-fille et n'en fit
rien. À la fin, lassé de ces plaintes\,: «\,Vous avez toute raison, leur
répondit-il en colère, c'est une impertinente, une coquine dont on ne
peut venir à bout, et si j'entends encore parler d'elle, je l'ai résolu,
je la déshériterai.\,» Ce fut la fin des plaintes. Rose était un petit
homme ni gras ni maigre, avec un assez beau visage, une physionomie
fine, des yeux perçants et pétillants d'esprit, un petit manteau, une
calotte de satin sur ses cheveux presque blancs, un petit rabat uni
presque d'abbé, et toujours son mouchoir entre son habit et sa veste. Il
disait qu'il était là plus près de son nez. Il m'avait pris en amitié,
se moquait très librement des princes étrangers, de leurs rangs, de
leurs prétentions, et appelait toujours les ducs avec qui il était
familier Votre Altesse Ducale\,: c'était pour rire de ces autres
prétendues Altesses. Il était extrêmement propre et gaillard et plein de
sens jusqu'à la fin\,: c'était une sorte de personnage.

Stoppa, colonel des gardes suisses et d'un autre régiment suisse de son
nom, mourut en même temps. Il avait amassé un bien immense pour un homme
de son état, avec une grosse maison pourtant et toujours grande chère.
Il avait toute la confiance du roi sur ce qui regardait les troupes
suisses et les cantons, au point que tant qu'il vécut, M. du Maine n'y
put et n'y fit aucune chose. Le roi s'était servi de lui en beaucoup de
choses secrètes, et de sa femme encore plus, qui, sans paraître, avait
toute la confiance de M\textsuperscript{me} de Maintenon, et était
extrêmement crainte et comptée, plus encore que son mari, quoiqu'il le
fût beaucoup. Il avait plus de quatre-vingts ans, avec le même sens, la
même privance du roi, la même pleine autorité sur sa nation en France,
et grand crédit en Suisse. Sa mort rendit M. du Maine effectivement
colonel général des Suisses avec pleine autorité, qu'il sut étendre en
même temps sur ce qu'il n'avait pu encore atteindre dans l'artillerie
avec M. de Barbezieux.

La mort d'un plus grand seigneur fit moins de bruit et de vide. Ce fut
celle de M. de Monaco, ambassadeur à Rome, qui y fut peu regretté, comme
il y avait été peu considéré\,; {[}il avait{]} très médiocrement soutenu
les affaires du roi, et {[}été{]} très peu soutenu de la cour. On en a
vu les raisons. C'était un Italien glorieux, fantasque, avare, fort bon
homme, mais qui n'était pas fait pour les affaires, avec cela gros comme
un muid, et ne voyait pas jusqu'à la pointe de son ventre. Il avait
passé sa vie en chagrins domestiques, d'abord de la belle
M\textsuperscript{me} de Monaco, sa femme, si amie de la première femme
de Monsieur, et si mêlée dans ses galanteries, et elle-même si galante
et qui, pour se tirer d'avec son mari, se fit surintendante de la maison
de Madame, la seule fille de France qui en ait jamais eu. Elle était
sœur de ce galant comte de Guiche et du duc de Grammont. Sa belle-fille
ne lui avait pas donné moins de peine, comme on a vu ici en son temps,
et le rang qu'elle lui avait valu le jeta dans des prétentions dont pas
une ne réussit, et qui l'accablèrent d'ennuis et de dégoûts qui
portèrent à plomb sur les affaires de son ambassade.

Bontems, le premier des quatre premiers valets de chambre du roi, et
gouverneur de Versailles et de Marly, dont il avait l'entière
administration des maisons, des chasses et de quantité de sortes de
dépenses, mourut aussi en ce temps-là. C'était de tous les valets
intérieurs celui qui avait la plus ancienne et la plus entière confiance
du roi pour toutes les choses intimes et personnelles. C'était un grand
homme, fort bien fait, qui était devenu fort gros et fort pesant, qui
avait près de quatre-vingts ans, et qui périt en quatre jours, le 17
janvier, d'une apoplexie. C'était l'homme le plus profondément secret,
le plus fidèle et le plus attaché au roi qu'il eût su trouver, et, pour
tout dire en un mot, qui avait disposé la messe nocturne dans les
cabinets du roi que dit le P. de La Chaise à Versailles, l'hiver de 1683
à 1684, que Bontems servit, et où le roi épousa M\textsuperscript{me} de
Maintenon en présence de l'archevêque de Paris, Harlay, Montchevreuil et
Louvois.

On peut dire de Bontems et du roi en ce genre\,: tel maître, tel
valet\,; car il était veuf, et avait chez lui à Versailles une
M\textsuperscript{lle} de La Roche, mère de La Roche qui suivit le roi
d'Espagne et fut son premier valet de chambre et eut son estampille
vingt-cinq ans jusqu'à sa mort. Cette M\textsuperscript{lle} de La Roche
ne paraissait nulle part, et assez peu même chez lui, dont elle ne
sortait point, et le gouvernait parfaitement sans presque le paraître.
Personne ne doutait que ce ne fût sa Maintenon et qu'il ne l'eût
épousée. Pourquoi ne le point déclarer\,? c'est ce qu'on n'a jamais su.
Bontems était rustre et brusque, avec cela respectueux et tout à fait à
sa place, qui n'était jamais que chez lui ou chez le roi, où il entrait
partout à toutes heures, et toujours par les derrières, et qui n'avait
d'esprit que pour bien servir son maître, à quoi il était tout entier
sans jamais sortir de sa sphère. Outre les fonctions si intimes de ces
deux emplois, c'était par lui que passaient tous les ordres et messages
secrets, les audiences ignorées qu'il introduisait chez le roi, les
lettres cachées au roi et du roi, et tout ce qui était mystère. C'était
bien de quoi gâter un homme qui était connu pour être depuis cinquante
ans dans cette intimité, et qui avait la cour à ses pieds, à commencer
par les enfants du roi et les ministres les plus accrédités, et à
continuer par les plus grands seigneurs. Jamais il ne sortit de son
état, et, sans comparaison, moins que les plus petits garçons bleus qui
tous étaient sous ses ordres. Il ne fit jamais mal à qui que ce soit, et
se servit toujours de son crédit pour obliger. Grand nombre de gens,
même de personnages lui durent leur fortune, sur quoi il était d'une
modestie à se brouiller avec eux, s'ils en avaient parlé jusqu'à
lui-même. Il aimait, voulait et procurait les grâces pour le seul
plaisir de bien faire, et il se peut dire de lui qu'il fut toute sa vie
le père des pauvres, la ressource des affligés et des disgraciés qu'il
connaissait le moins, et peut-être le meilleur des humains, avec des
mains non seulement parfaitement nettes, mais un désintéressement entier
et une application extrême à tout ce qui était sous sa charge. Aussi,
quoique fort diminué de crédit pour les autres par son âge et sa
pesanteur, sa perte causa un deuil public et à la cour et à Paris, et
dans les provinces\,; chacun en fut affligé comme d'une perte
particulière, et il est également innombrable et inouï tout ce qui fut
volontairement rendu à sa mémoire, et de services solennels célébrés
partout pour lui. J'y perdis un ami sûr, plein de respect et de
reconnaissance pour mon père, comme je l'ai dit ailleurs. Il laissa deux
fils qui ne lui ressemblèrent en rien\,; l'aîné ayant sa survivance de
premier valet de chambre, l'autre premier valet de garde-robe.

Bloin, autre premier valet de chambre, eut l'intendance de Versailles et
de Marly, au père de qui, pour cet emploi, Bontems avait succédé. Bloin
eut aussi la confiance des paquets secrets et des audiences inconnues.
C'était un homme de beaucoup d'esprit, qui était galant et particulier,
qui choisissait sa compagnie dans le meilleur de la cour, qui régnait
chez lui dans l'exquise chère, parmi un petit nombre de commensaux
grands seigneurs, ou de gens qui suppléaient d'ailleurs aux titres, qui
était froid, indifférent, inabordable, glorieux, suffisant et volontiers
impertinent\,; toutefois peu méchant, mais à qui pourtant il ne fallait
pas déplaire. Ce fut un vrai personnage et qui se fit valoir et
courtiser par les plus grands et par les ministres, qui savait bien
servir ses amis, mais rarement, et n'en servait point d'autres, et ne
laissait pas d'être en tout fort dangereux et de prendre en aversion
sans cause, et alors de nuire infiniment.

M. de Vendôme revint d'Anet après avoir passé encore une fois par le
grand remède. Il se comptait guéri, et ne le fut jamais. Il demeura plus
défiguré qu'il ne l'était auparavant cette deuxième dose, et assez pour
n'oser se montrer aux dames et aller à Marly. Bientôt il s'y accoutuma
et tâcha d'y accoutumer les autres. Ce ne fut pas sans dégoût, et sans
chercher sa physionomie et ses principaux traits, qui ne se retrouvèrent
plus\,; il paya d'audace, en homme qui se sent tout permis et qui se
veut tout permettre. Il avait de bons appuis. C'était en janvier, et il
y avait des bals à Marly\,; le roi s'en amusa tous les voyages jusqu'au
carême\,; et la maréchale de Noailles en donna souvent à
M\textsuperscript{me} la duchesse de Bourgogne, chez elle à Versailles,
qui avait l'air d'être en particulier.

\hypertarget{chapitre-v.}{%
\chapter{CHAPITRE V.}\label{chapitre-v.}}

1701

~

{\textsc{Plusieurs bonnes nouvelles.}} {\textsc{- D'Avaux ambassadeur en
Hollande, au lieu de Briord, fort malade.}} {\textsc{- Les troupes
françaises, introduites au même instant dans les places espagnoles des
Pays-Bas, y arrêtent et désarment les garnisons hollandaises, que le roi
fait relâcher.}} {\textsc{- Flottille arrivée.}} {\textsc{- Chocolat des
Jésuites.}} {\textsc{- Philippe V reconnu par le Danemark.}} {\textsc{-
Connétable de Castille ambassadeur extraordinaire à Paris.}} {\textsc{-
Philippe V à Bayonne\,; à Saint-Jean de Luz\,; séparation des princes.}}
{\textsc{- Comte d'Ayen passe en Espagne.}} {\textsc{- Duc de
Beauvilliers revient malade.}} {\textsc{- Lettres patentes de
conservation des droits à la couronne de Philippe V.}} {\textsc{- La
reine d'Espagne abandonnée et reléguée à Tolède.}} {\textsc{- Philippe V
reconnu par les Provinces-Unies.}} {\textsc{- Ouragan à Paris et par la
France.}} {\textsc{- Mort de l'évêque-comte de Noyon.}} {\textsc{- Abbé
Bignon, conseiller d'État d'Église.}} {\textsc{- Aubigny, évêque de
Noyon.}} {\textsc{- M\textsuperscript{lle} Rose, béate extraordinaire.}}
{\textsc{- M. Duguet.}} {\textsc{- M. de Saint-Louis retiré à la
Trappe.}} {\textsc{- Institution d'un prince, par M. Duguet.}}
{\textsc{- Helvétius à Saint-Aignan.}} {\textsc{- Retour du duc de
Beauvilliers.}} {\textsc{- Cardinal de Bouillon à Cluni, restitué en ses
revenus.}} {\textsc{- Exil du comte de Melford.}} {\textsc{- Roi Jacques
à Bourbon.}}

~

Plusieurs nouvelles agréables arrivèrent fort près à près. Le roi reçut
de milan un acte qu'on n'avait pas quoique connu\,: c'était
l'investiture de Charles V du duché de Milan et du comté de Pavie pour
tous les successeurs tant mâles que femelles\,; la certitude du passage
de ses troupes en Italie accordé par M. de Savoie en la forme qu'on
désirait, et un succès en Flandre qui tenait de la merveille et très
semblable à un changement de théâtre d'opéra. Briord, ambassadeur en
Hollande, était tombé dangereusement malade. Les affaires y étaient en
grand mouvement. Il demanda par plusieurs courriers un successeur, et
d'Avaux y fut envoyé. Les États, qui de concert avec l'Angleterre ne
cherchaient qu'à nous amuser en attendant que leur partie fût prête, ne
se lassaient point de négocier. Ils demandaient des conférences avec
d'autant plus d'empressement que Briord était hors d'état d'ouïr parler
d'affaires. Le roi d'Angleterre faisait presser le roi de les accorder.
Quelque désir qu'eut le roi d'entretenir la paix, il ne pouvait se
dissimuler les mouvements découverts de l'empereur et la mauvaise foi de
ses anciens alliés.

Les Hollandais avaient vingt-deux bataillons dans les places espagnoles
des Pays-Bas, sous les gouverneurs espagnols qui y avaient aussi
quelques troupes espagnoles en moindre nombre. Puységur travailla à un
projet là-dessus, par ordre du roi, qu'il approuva. Il fut communiqué au
maréchal de Boufflers, gouverneur de la Flandre française, et Puységur
alla à Bruxelles pour le concerter avec l'électeur de Bavière,
gouverneur général des Pays-Bas pour l'Espagne. Les mesures furent si
secrètes et si justes, et leur exécution si profonde, si exacte et si à
un point nommé, que le dimanche matin, 6 février, les troupes françaises
entrèrent toutes au même instant dans toutes les places espagnoles des
Pays-Bas à portes ouvrantes, s'en saisirent, prirent les troupes
hollandaises entièrement au dépourvu, les surprirent, les dépostèrent,
les désarmèrent, sans que dans pas une il fût tiré une seule amorce. Les
gouverneurs espagnols et les chefs de nos troupes leur déclarèrent
qu'ils n'avaient rien à craindre, mais que le roi d'Espagne voulait de
nos troupes au lieu des leurs, et qu'ils demeureraient ainsi arrêtés
jusqu'à ce qu'on eût reçu les ordres du roi. Ils furent très différents
de ce qu'ils attendaient et de ce qu'on devait faire. L'ardeur de la
paix fit croire au roi qu'en renvoyant ces troupes libres avec leurs
armes et toutes sortes de bons traitements, un procédé si pacifique
toucherait et rassurerait les Hollandais, qui avaient jeté les hauts
cris à la nouvelle de l'introduction de nos troupes, et leur
persuaderait d'entretenir la paix avec des, voisins, des bonnes
intentions desquels ils ne pouvaient plus douter après un si grand
effet. Il se trompa.

Ce fut vingt-deux, très bons bataillons tout armés et tout équipés qu'il
leur renvoya, qui leur auraient fait grande faute, qui les auraient mis
hors d'état de faire la guerre, et par conséquent fort déconcerté
l'Angleterre, l'empereur et toute cette grande alliance qui se bâtissait
et s'organisait contre les deux couronnes. Le vendredi 11 février,
c'est-à-dire six jours après l'occupation des places et la détention des
vingt-deux bataillons hollandais, l'ordre du roi partit, portant liberté
de s'en aller chez eux avec armes et bagages, dès qu'ils seraient
rappelés par les États. Ceux-ci, qui n'espéraient rien moins, reçurent
cette nouvelle avec, une joie inespérée et des marques de reconnaissance
qui servirent de couverture nouvelle encore plus spécieuse de leurs
mauvais desseins, et frémissant cependant du danger qu'ils avaient couru
n'en devinrent que plus ardents à la guerre, gouvernés par le roi
d'Angleterre, ennemi personnel du roi, qui avec eux se moqua d'une
simplicité si ingénue, et qui retraça à l'Europe celles de Louis XII et
de François Ier qui furent si funestes à la France. Celle-ci ne la fut
aussi guère moins.

Enfin, l'arrivée de la flottille couronna ce succès. Elle était riche de
plus de soixante millions en or ou argent, et de douze millions de
marchandises sans les fraudes et les pacotilles. J'avancerai à cette
occasion le récit d'une aventure qui n'arriva que depuis que le roi
d'Espagne fut à Madrid. En déchargeant les vaisseaux il se trouva huit
grandes caisses de chocolat dont le dessus était\,: \emph{chocolat pour
le très révérend père général de la compagnie de Jésus}. Ces caisses
pensèrent rompre les reins aux gens qui les déchargèrent et qui s'y
mirent au double de ce qu'il fallait à les transporter à proportion de
leur grandeur. L'extrême peine qu'ils y eurent encore avec ce renfort
donna curiosité de savoir quelle en pouvait être la cause. Toutes les
caisses arrivées dans les magasins de Cadix, ceux qui les régissaient en
ouvrirent une entre eux et n'y trouvèrent que de grandes et grosses
billes de chocolat, arrangées les unes sur les autres. Ils en prirent
une dont la pesanteur les surprit, puis une deuxième et une troisième
toujours également pesantes. Ils en rompirent une qui résista, mais le
chocolat s'éclata, et ayant redoublé ils trouvèrent que c'étaient toutes
billes d'or, revêtues d'un doigt d'épais de chocolat tout alentour\,;
car, après cet essai, ils visitèrent au hasard le reste de la caisse et
après toutes les autres. Ils en donnèrent avis à Madrid, où malgré le
crédit de la société on s'en voulut donner le plaisir. On fit avertir
les jésuites, mais en vain. Ces fins politiques se gardèrent bien de
réclamer un chocolat si précieux\,; et ils aimèrent mieux le perdre que
de l'avouer. Ils protestèrent donc d'injure qu'ils ne savaient ce que
c'était, et ils y persévérèrent avec tant de fermeté et d'unanimité que
l'or demeura au profit du roi, qui ne fut pas médiocre, et on en peut
juger par le volume de huit grandes caisses de grandes et grosses billes
solides d'or\,; et le chocolat qui les revêtait demeura à ceux qui
avaient découvert la galanterie.

Le Danemark reconnut le roi d'Espagne. Ce prince fut rencontré à
Bordeaux par le connétable de Castille, venant ambassadeur
extraordinaire pour remercier le roi de l'acceptation du testament. Il
s'appelait don Joseph-Fernandez de Velasco, duc de Frias. Il fut reçu au
Bourg-la-Reine par le baron de Breteuil, introducteur des ambassadeurs,
qui est un honneur qui de ce règne n'avait été fait à aucun autre qu'au
marquis de La Fuente, qui après l'affaire du maréchal d'Estrades et du
baron de Vatteville à Londres pour la préséance, vint ambassadeur
extraordinaire pour en faire excuse et déclarer en présence de tous les
autres ambassadeurs, en audience publique, que l'Espagne ni ses
ambassadeurs ne disputeraient jamais la préséance au roi ni à ses
ambassadeurs et la lui céderaient partout. Le connétable de Castille
parut avec une grande splendeur, et fut extrêmement accueilli et
festoyé. Le roi le distingua extrêmement et lui fit un présent très
considérable à son départ. Il ne fut pas longtemps en France, et il y
parut fort magnifique, fort galant et fort poli.

À Bayonne le roi trouva le marquis de Castanaga, dix ou douze autres
personnes de considération, et plus de quatre mille Espagnols accourus
pour le voir. Harcourt y était arrivé deux jours auparavant, de Madrid,
au-devant de lui. Le roi se mit dans un fauteuil à la porte de son
cabinet, ayant derrière lui M. de Beauvilliers, entre MM. de Noailles et
d'Harcourt. Le duc d'Ossone était plus en avant, pour marquer au roi
ceux qui étant gentilshommes pouvaient avoir l'honneur de lui baiser la
main. Tous, à l'espagnole, se mirent à genoux en se présentant `devant
lui. Il vit toute cette foule les uns après les autres, et les satisfit
tous ainsi au dernier point fort aisément. M. de Beauvilliers avait
souvent entretenu le roi d'Espagne tête à tête pendant le voyage. Il y
eut pendant le séjour de Bayonne, des conférences où le duc d'Harcourt
fut presque toujours en tiers, et quelquefois le duc de Noailles avec
eux. Ils allèrent à Saint-Jean de Luz, et le 22 janvier se fit la
séparation des princes avec des larmes qui allèrent jusqu'aux cris.

Après quantité d'embrassades réitérées au bord de la Bidassoa, au même
endroit des fameuses conférences de la paix des Pyrénées, le duc de
Noailles emmena le roi d'Espagne d'un côté, et le duc de Beauvilliers
les deux autres princes de l'autre, avec lesquels il remonta en
carrosse, et retournèrent à Saint-Jean de Luz. Il y avait un pont et de
très jolies barques galamment ajustées par ceux du pays. Le roi
d'Espagne passa dans une avec le duc d'Harcourt, le marquis de Quintana,
gentilhomme de la chambre, et le comte d'Ayen. La petite rivière qui
sépare les deux royaumes était bordée d'un peuple innombrable à perte de
vue des deux côtés. Les acclamations ne finissaient point et
redoublaient à tous moments. Au sortir de la barque le roi d'Espagne
marcha un peu à pied, pour contenter la curiosité de ses peuples, et
alla coucher à Irun. Il fut d'abord à l'église, où le \emph{Te Deum} fut
chanté. Et, dès le même soir, il commença à être servi et à vivre à
l'espagnole. Il fut visiter le lendemain Fontarabie, puis
Saint-Sébastien, et continua son voyage à Madrid, ayant toujours le duc
d'Harcourt dans son carrosse, un ou deux de ses officiers principaux
espagnols et le comte d'Ayen. Ce dernier fut trouvé là fort mauvais,
l'entrée du carrosse du roi n'étant que pour ses officiers les plus
principaux. Ce neveu de M\textsuperscript{me} de Maintenon, à qui
d'Harcourt faisait sa cour, avait une nombreuse suite et une musique
complète, dont il tâchait les soirs d'amuser le roi d'Espagne. Son âge,
sa faveur en France, l'imitation des airs libres et familiers et des
grands rires de sa mère, montrèrent à l'Espagne un fort jeune homme,
bien gâté, et qui les scandalisa infiniment par toutes ses manières avec
les seigneurs de cette cour, et par la familiarité surtout qu'il affecta
avec le roi d'Espagne. Il fut le seul jeune seigneur français qui passa
avec lui. Noblet fit deux journées en Espagne, puis vint rendre compte
au roi de ce qui s'était passé durant le voyage.

De Saint-Jean de Luz, les princes allèrent à Acqs\footnote{Acqs, ou Dax,
  ville du département des Landes. Les anciens éditeurs ont écrit Auch.
  Mais, outre le manuscrit de Saint-Simon, qui ne peut laisser aucun
  doute, nous trouvons la confirmation de cette leçon dans le passage
  suivant d'un journal qu'avait rédigé le duc de Bourgogne et qui a été
  publié dans le t. II, p.~93-250 des \emph{Curiosités historiques, ou
  Recueil de pièces utiles à l'histoire de France} (Amsterdam, 1759, 2
  vol.~in-18). «\,Le lundi 24 janvier, nous partîmes de Bayonne, à six
  heures.., nous arrivâmes à Dax (les éditeurs auront changé l'ancienne
  forme qui était \emph{Acqs}), à sept heures du soir\,; il plut tout le
  jour\,; les chemins étaient horriblement mauvais\ldots. Le mardi 25,
  les eaux augmentèrent de telle sorte, que l'on ne pouvait plus
  repasser le pont ni sortir de la ville\,; elles augmentèrent encore le
  mercredi 26 et le jeudi 27, en sorte que la campagne en était toute
  couverte, et qu'on ne voyait que la pointe des arbres.\,» C'est donc à
  Dax que les princes sont arrêtés par les eaux, et c'est ce que dit le
  texte véritable de Saint-Simon\,: «\,Les princes allèrent à Acqs, où
  ils demeurèrent huit ou dix jours assiégés par les eaux. »}, où ils
demeurèrent huit ou dix jours assiégés par les eaux. Là ils commencèrent
à vivre avec plus de liberté, à manger quelquefois avec les jeunes
seigneurs de leur cour et à se trouver affranchis de toutes les mesures
qu'imposait la présence du roi d'Espagne. Le duc de Noailles demeura
leur conducteur comme l'avait été jusque-là M. de Beauvilliers, qui, se
trouvant toujours plus mal, avait eu besoin de tout son courage pour
venir jusqu'à la frontière, d'où il revint droit par le plus court,
autant que sa santé le lui permit. Le roi d'Espagne emporta des lettres
patentes enregistrées, pour lui conserver et à sa postérité leurs droits
à la couronne, pareilles à celles qu'Henri III avait emportées en
Pologne, et qu'on en avait dressé de toutes prêtes pour y envoyer à M.
le prince de Conti.

La reine d'Espagne avait écrit au roi les lettres les plus fortes par le
connétable de Castille, par lesquelles elle demandait aux deux rois leur
protection et la punition du comte de San-Estevan et de ses dames, qui
l'avaient quittée et outragée. Le style en était fort romanesque. Il y
en eut aussi pour Madame, dont elle réclamait les bons offices par leur
parenté. Je ne sais qui put lui donner ce conseil\,; sa partialité
déclarée, et sa liaison avec tout ce peu qui ne voyait qu'à regret
succéder la maison de France à celle d'Autriche en Espagne, ne lui
devaient pas laisser espérer de succès. Aussi, le roi d'Espagne n'eut
pas beaucoup fait de journées en Espagne, qu'elle eut ordre de quitter
Madrid et de se retirer à Tolède, où elle demeura reléguée avec peu de
suite et encore moins de considération. La junte avait été de cet avis,
et en avait chargé le duc d'Harcourt pour en faire envoyer l'ordre par
le roi d'Espagne\,: ce fut un trait de vengeance de Portocarrero.

Ce prince n'était pas encore à Madrid qu'il fut reconnu par les
Hollandais. Ils n'en avaient pas moins résolu la guerre. Mais toutes les
machines de l'alliance n'étaient pas prêtes, et rie s'expliquer point
eût été s'expliquer, et découvrir des desseins qu'ils prenaient de si
grands soins de cacher.

Il y eut, le jour de la Chandeleur, un ouragan si furieux que personne
ne se souvint de rien qui eût approché d'une telle violence, dont les
désordres furent infinis par tout le royaume. Le haut de l'église de
Saint-Louis, dans l'île, à Paris, tomba\,; beaucoup de gens qui y
entendaient la messe furent tués ou blessés\,: entre autres Verderonne,
qui était dans la gendarmerie, en mourut le lendemain. Il s'appelait
L'Aubépine comme ma mère. Cet ouragan a été l'époque du dérangement des
saisons et de la fréquence des grands vents en toutes\,; le froid en
tout temps, la pluie, etc., ont été bien plus ordinaires depuis, et ces
mauvais temps n'ont fait qu'augmenter jusqu'à présent, en sorte qu'il y
a longtemps qu'il n'y a plus du tout de printemps, peu d'automne, et,
pour l'été quelques jours par-ci par là\,: c'est de quoi exercer les
astronomes.

M. de Noyon mourut en ce temps-ci à Paris à soixante-quatorze ans. Il
avait l'ordre, et s'était, à l'exemple de M. de Reims, laissé faire
conseiller d'État d'Église. J'ai tant parlé de ce prélat que je me
contenterai de dire qu'il mourut fort pieusement, après avoir très
soigneusement gouverné son diocèse. On trouva dans ses papiers des
brouillons de sa main pour servir à son oraison funèbre, tant la folie
de la vanité avait séduit ce prélat\,; d'ailleurs docte, fort honnête
homme, très homme de bien, bon évêque et de beaucoup d'esprit. Il ne
laissa pas d'être regretté, et beaucoup, dans son diocèse. Sa vanité eût
été étrangement mortifiée s'il eût prévu ses successeurs.

Le chancelier qui avait extrêmement aimé sa sœur, femme de Bignon,
conseiller d'État, et qui en avait comme adopté les enfants, était fort
embarrassé de l'abbé Bignon. C'était ce qui véritablement, et en bonne
part, se pouvait appeler un bel esprit, très savant, et qui avait prêché
avec beaucoup d'applaudissements\,; mais sa vie avait si peu répondu à
sa doctrine qu'il n'osait plus se montrer en chaire, et que le roi se
repentait des bénéfices qu'il lui avait donnés. Que faire donc d'un
prêtre à qui ses mœurs ont ôté toute espérance de l'épiscopat\,? Cette
place de conseiller d'État d'Église parut à son oncle toute propre à
l'en consoler et à le réhabiliter dans le monde, en lui donnant un état.
L'embarras était que ces places étaient destinées aux évêques les plus
distingués, et qu'il était bien baroque de faire succéder l'abbé Bignon
à M. de Tonnerre, évêque-comte de Noyon, pour le mettre en troisième
avec M. de Reims et M. de Meaux\,; c'est pourtant ce que le chancelier
obtint, et ce fut tout l'effort de son crédit. Il fit par là un tort à
l'épiscopat et une plaie au conseil, où pas un évêque n'a voulu entrer
depuis, par l'indécence d'y seoir après un homme du second ordre, ce qui
ne peut s'éviter que par des évêques pairs qui précèdent le doyen des
conseillers d'État, comme faisaient MM. de Reims et de Noyon. L'abbé
Bignon fut transporté de joie d'une distinction jusqu'à lui inouïe. Son
oncle le mit dans des bureaux en attendant qu'il lui en pût donner, et à
la tête de toutes les académies\,: ce dernier emploi était fait exprès
pour lui. Il était un des premiers hommes de lettres de l'Europe\,; et
il y brilla, et solidement. Il amassa plus de cinquante mille volumes,
que nombre d'années après il vendit au fameux Law qui cherchait à placer
de l'argent à tout. L'abbé Bignon n'en avait plus que faire. Il était
devenu doyen du conseil à la tête de quantité de bureaux et d'affaires,
et bibliothécaire du roi. Il se fit une île enchantée auprès de Meulan,
qui se put comparer en son genre à celle de Caprée\,; l'âge ni les
places ne l'ayant pas changé, et n'y ayant gagné qu'à faire estimer son
savoir et son esprit aux dépens de son cœur et de son âme. Noyon ne fut
pas mieux rempli, mais à la renverse de la place de conseiller d'État
par un homme de condition et de très saintes mœurs et vie, mais
d'ailleurs un butor.

M. de Chartres avait trouvé à Saint-Sulpice un gros et grand pied plat,
lourd, bête, ignorant, esprit de travers, mais très homme de bien, saint
prêtre pour desservir, non pas une cure, mais une chapelle\,; surtout
sulpicien excellent en toutes les minuties et les inutiles puérilités
qui y font loi, et qu'il mit toute sa vie à côté ou même au-dessus des
plus éminentes vertus. Ce garçon n'en savait pas davantage, et n'était
pas capable de rien apprendre de mieux\,; d'ailleurs pauvre, crasseux et
huileux à merveille. Ces dehors trop puissants sur M. de Chartres, et
qui par ses mauvais choix ont perdu notre épiscopat, l'engagèrent à
s'informer de lui. C'était un homme de bonne et ancienne noblesse
d'Anjou qui s'appelait d'Aubigny\,; ce nom le frappa encore plus, il le
prit ou le voulut prendre pour parent de M\textsuperscript{me} de
Maintenon qui était d'Aunis, et s'appelait d'Aubigné. Il lui en parla et
à ce pied plat aussi, qui, tout bête qu'il fût, ne l'était pas assez
pour ne sentir pas les avantages d'une telle parenté dont on lui faisait
toutes les avances\,; M\textsuperscript{me} de Maintenon se trouva ravie
de s'enter sur ces gens-là. Les armes, le nom, et peu après, pour tout
unir, la livrée, furent bientôt les mêmes. Le rustre noble fut présenté
à Saint-Cyr à sa prétendue cousine, qui ne l'était pas tant, mais qui
pouvait tout. Teligny, frère de l'abbé, qui languissait de misère dans
sa chaumine, accourut par le messager, et fit aussi connaissance avec le
prélat et sa royale pénitente. Celui-ci se trouva un compère délié,
entendu et fin, qui gouverna son frère et suppléa tant qu'il put à ses
bêtises. M. de Chartres, qui voulut décrasser son disciple, le prit avec
lui, le fit son grand vicaire, et ce bon gros garçon, sans avoir pu rien
apprendre en si bonne école que des choses extérieures, fut nommé évêque
de Noyon, où sa piété et sa bonté se firent estimer, et ses travers et
ses bêtises détester, quoique parés par son frère qui ne le quittait
point, et qui était son tuteur.

M. le cardinal de Noailles, depuis peu revenu de Rome, chassa de son
diocèse M\textsuperscript{lle} Rose, célèbre béate à extases, à visions,
à conduite fort extraordinaire, qui dirigeait ses directeurs, et qui fut
une vraie énigme. C'était une vieille Gasconne ou plutôt du Languedoc,
qui en avait le parler à l'excès, carrée, entre deux tailles, fort
maigre, le visage jaune, extrêmement laid, des yeux très vifs, une
physionomie ardente, mais qu'elle savait adoucir\,; vive, éloquente,
savante, avec un air prophétique qui imposait. Elle dormait peu et sur
la dure, ne mangeait presque rien, assez mal vêtue, pauvre et qui ne se
laissait voir qu'avec mystère. Cette créature a toujours été une énigme,
car il est vrai qu'elle était désintéressée, qu'elle a fait de grandes
et surprenantes conversions qui ont tenu, qu'elle a dit des choses fort
extraordinaires, les unes très cachées qui étaient {[}passées{]},
d'autres à venir qui sont arrivées, qu'elle a opéré des guérisons
surprenantes sans remède, qu'elle a eu pour elle des gens très sages,
très précautionnés, très savants, très pieux, d'un génie sublime, qui
n'avaient ni ne pouvaient rien gagner à cet attachement, et qui l'ont
conservé toute leur vie. Tel a été M. Duguet, si célèbre par ses
ouvrages, par la vaste étendue de son esprit et de son érudition qui se
peut dire universelle, par l'humilité sincère et la sainteté de sa vie,
et par les charmes et la solidité de sa conversation.

M\textsuperscript{lle} Rose, ayant longtemps vécu dans son pays, où elle
pansait les pauvres et où sa piété lui avait attaché des prosélytes,
vint à Paris, je ne sais à quelle occasion. De doctrine particulière
elle n'en avait point, seulement fort opposée à celle de
M\textsuperscript{me} Guyon, et tout à fait du côté janséniste. Je ne
sais encore comment elle fit connaissance avec ce M. Boileau qui avait
été congédié de l'archevêché pour le \emph{Problème} dont j'ai fait
l'histoire en son temps, et qui vivait claquemuré et le plus sauvagement
du monde dans son cloître Saint-Honoré. De là elle vit M. du Charmel et
d'autres, et enfin M. Duguet qui, pour en dire la vérité, ne s'en
éprirent guère moins tous trois que M. de Cambrai de
M\textsuperscript{me} Guyon. Après avoir mené assez longtemps une vie
assez cachée à Paris, M. Duguet et M. du Charmel eurent aussi bien
qu'elle un extrême désir de la faire voir à M. de la Trappe, soit pour
s'éclairer d'un si grand maître sur une personne si extraordinaire, soit
dans l'espérance d'en obtenir l'approbation, et de relever leur sainte
par un si grand témoignage. Ils partirent tous trois sans dire mot, et
s'en allèrent à la Trappe, où on ne savait rien de leur projet.

M. du Charmel se mit aux hôtes à l'ordinaire dans la maison, et M. de
Saint-Louis, qui occupait la maison abbatiale au dehors, ne put refuser
une chambre à M. Duguet, et une autre à sa béate, et de manger avec lui.
C'était un gentilhomme peu éloigné de la Trappe, qui avait servi toute
sa vie avec grande réputation, qui avait eu longtemps un régiment de
cavalerie et était devenu brigadier. M. de Turenne, le maréchal de
Créqui, et les généraux sous qui il avait servi, le roi même sous qui il
avait fait la guerre de Hollande et d'autres campagnes, l'estimaient
fort, et l'avaient toujours distingué. Le roi lui donnait une assez
forte pension, et avait conservé beaucoup de bonté pour lui. Il se
trouva presque aveugle, lorsqu'en 1684 la trêve de vingt ans fut
conclue\,; cela le fit retirer du service. Peu de mois après, Dieu le
toucha. Il connaissait M. de la Trappe par le voisinage, et avait même
été lui offrir ses services au commencement de sa réforme, sur ce qu'il
apprit que les anciens religieux, qui étaient de vrais bandits et qui
demeuraient encore à la Trappe, avaient résolu de le noyer dans leurs
étangs. Il avait conservé quelque commerce depuis avec M. de la Trappe.
Ce fut donc là où il se retira, et où il a mené plus de trente ans la
vie la plus retirée, la plus pénitente et la plus sainte. C'était un
vrai guerrier, sans lettres aucunes, avec peu d'esprit, mais avec un
sens le plus droit et le plus juste que j'aie vu à personne, un
excellent cœur, et une droiture, une franchise, une vérité, une fidélité
admirables.

Le hasard fit que j'allai aussi à la Trappe tandis qu'ils y étaient. Je
n'avais jamais vu M. Duguet ni sa dévote. Elle ne voyait personne à la
Trappe, et n'y sortait presque point de sa chambre que pour la messe à
la chapelle, où les femmes pouvaient l'entendre, joignant ce logis
abbatial du dehors. Du vivant de M. de la Trappe, j'y passais
d'ordinaire six jours, huit, et quelquefois dix. J'eus donc loisir de
voir M\textsuperscript{lle} Rose à plusieurs reprises et M. Duguet, qui
ne fut pas une petite faveur. J'avoue que je trouvai plus
d'extraordinaire que d'autre chose en M\textsuperscript{lle} Rose\,;
pour M. Duguet, j'en fus charmé. Nous nous promenions tous les jours
dans le jardin de l'abbatial\,; les matières de dévotion, où il
excellait, n'étaient pas les seules sur lesquelles nous y en avions\,;
une fleur, une herbe, une plante, la première chose venue, des arts, des
métiers, des étoffes, tout lui fournissait de quoi dire et instruire,
mais si naturellement, si aisément, si couramment, et avec une
simplicité si éloquente, et des termes si justes, si exacts, si propres,
qu'on était également enlevé des grâces de ses conversations, et en même
temps épouvanté de l'étendue de ses connaissances qui lui faisaient
expliquer toutes ces choses comme auraient pu faire les botanistes, les
droguistes, les artisans et les marchands les plus consommés dans tous
ces métiers. Son attention, sa vénération pour M\textsuperscript{lle}
Rose, sa complaisance, son épanouissement à tout ce peu qu'elle disait,
ne laissaient pas de me surprendre. M. de Saint-Louis, tout rond et tout
franc, ne la put jamais goûter, et le disait très librement à M. du
Charmel, et le laissait sentir à M. Duguet, qui en étoient affligés.

Mais ce qui les toucha bien autrement, fût la douce et polie fermeté
avec laquelle, six semaines durant qu'ils furent là, M. de la Trappe se
défendit de voir M\textsuperscript{lle} Rose, quoique en état encore de
pouvoir sortir et la voir au dehors. Aussi s'en excusa-t-il, moins sur
la possibilité que sur son éloignement de ces voies extraordinaires, sur
ce qu'il n'avait ni mission ni caractère pour ces sortes d'examen, sur
son état de mort à toutes choses et de vie pénitente et cachée qui
l'occupait assez pour ne se point distraire à des curiosités inutiles,
et qu'il valait mieux pour lui suspendre son jugement et prier Dieu pour
elle que de la voir et d'entrer dans une dissipation qui n'était point
de son état. Ils partirent donc comme ils étaient venus\,; très
mortifiés de n'avoir pu réussir au but qu'ils s'étaient proposé de ce
voyage. M\textsuperscript{lle} Rose se tint depuis assez cachée à Paris,
et chez des prosélytes dans le voisinage, jusqu'à ce que, le nombre s'en
étant fort augmenté, elle se produisit beaucoup davantage et devint une
directrice `qui fit du bruit. Le cardinal de Noailles la fit examiner,
je pense même que M. de Meaux la vit. Le beau fut qu'on la chassa. Elle
avait converti un grand jeune homme fort bien fait, dont le père bien
gentilhomme avait été autrefois major de Blaye, et qui avait du bien. Ce
jeune homme quitta le service et s'attacha {[}tellement{]} à elle qu'il
ne la quitta plus depuis\,; il s'appelait Gondé, et il s'en alla avec
elle à Annecy lorsqu'elle fut chassée de Paris, où on n'en a guère ouï
parler depuis, quoiqu'elle y ait vécu fort longtemps. J'avancerai ici le
court récit d'une anecdote qui le mérite. Le prétexte de ce voyage de la
Trappe de M\textsuperscript{lle} Rose fut la conversion, qu'elle avait
faite auprès de Toulouse, d'un curé fort bien fait, et qui ne vivait pas
trop en frère. Il était frère d'un M. Parasa, conseiller au parlement de
Toulouse. Elle persuada à ce curé de quitter son bénéfice, de venir à
Paris, et de se faire religieux de la Trappe. Ce dernier point, elle eut
une peine extrême à le gagner sur lui, et il a souvent dit, avant et
depuis, qu'il s'était fait moine de la Trappe malgré lui. Il le fut bon
pourtant, et si bon, que M. de Savoie, ayant longtemps depuis demandé à
M. de la Trappe, un de ses religieux par qui il pût faire réformer
l'abbaye de Tamiers, celui-ci fut envoyé pour exécuter ce projet et en
fut abbé. Il y réussit si bien, que M. de Savoie, atteint alors d'un
assez long accès de dévotion, le goûta fort, fit plusieurs retraites à
Tamiers et lui donna toute sa confiance.

De là est, pour ainsi dire, né cet admirable ouvrage de
\emph{l'Institution d'un prince} de M. Duguet, dont on voit le comment
dans le court avertissement qui se lit au-devant de ce livre. Il faut
ajouter que M. Duguet, réduit depuis à chercher sa liberté hors du
royaume, se retira un temps à Tamiers, et y vit M. de Savoie, sans que
ce prince se soit jamais douté qu'il fût l'auteur de cet ouvrage, ni
qu'il lui en ait jamais parlé\,; en quoi l'humilité de l'auteur est
peut-être plus admirable que le prodige de l'érudition, de l'étendue et
de la justesse de cette \emph{Institution}. Elle fut faite entre la mort
du prince électoral de Bavière, petit-fils de l'empereur Léopold, et la
mort du roi d'Espagne, Charles II, dans le temps que M. de Savoie se
flatta que cet immense succession regarderait le prince de Piémont qui
est mort avant lui\,; et toutefois à la lire, qui ne soupçonnerait
qu'elle est faite d'aujourd'hui\,? c'est-à-dire, vingt-cinq ans après la
mort de Louis XIV, qu'elle a commencé à paraître, quelques années depuis
la mort de l'auteur, et à l'instant défendue, pourchassée, et traitée
comme les ouvrages les plus pernicieux, qui toutefois n'en a été que
plus recherchée et plus universellement goûtée et admirée.

M. de Beauvilliers, dont le mal était un dévoiement qui le consumait
depuis longtemps et auquel la fièvre s'était jointe, eut bien de la
peine à gagner sa maison de Saint-Aignan, près de Loches, où il fut à
l'extrémité. J'avais su, depuis son départ, que Fagon l'avait condamné,
et ne l'avait envoyé à Bourbon, peu avant ce voyage, que par se trouver
à bout, sans espérance de succès, et pour se délivrer du spectacle en
l'envoyant finir au loin. À cette nouvelle de Saint-Aignan, je courus
chez le duc de Chevreuse, pour l'exhorter de mettre toute politique à
part et d'y envoyer diligemment Helvétius, et j'eus une grande joie
d'apprendre de lui qu'il en avait pris le parti, et qu'il partait
lui-même le lendemain avec Helvétius.

C'était un gros Hollandais qui, pour n'avoir pas pris les degrés de
médecine, était l'aversion des médecins, et en particulier l'horreur de
Fagon, dont le crédit était extrême auprès du roi, et la tyrannie
pareille sur la médecine et sur ceux qui avaient le malheur d'en avoir
besoin. Cela s'appelait donc un empirique dans leur langage, qui ne
méritait que mépris et persécution, et qui attirait la disgrâce, la
colère et les mauvais offices de Fagon sur qui s'en servait. Il y avait
pourtant longtemps qu'Helvétius était à Paris, guérissant beaucoup de
gens rebutés ou abandonnés des médecins, et surtout les pauvres, qu'il
traitait avec une grande charité. Il eu recevait tous les jours chez lui
à heure fixée tant qu'il en voulait venir, à qui il fournissait les
remèdes et souvent la nourriture. Il excellait particulièrement aux
dévoiements invétérés et aux dysenteries. C'est à lui qu'on est
redevable de l'usage et de la préparation diverse de l'ipécacuanha pour
les divers genres de ces maladies, et le discernement encore de celles
où ce spécifique n'est pas à temps ou même n'est point propre. C'est ce
qui donna la vogue à Helvétius, qui d'ailleurs était un bon et honnête
homme, homme de bien, droit et de bonne foi. Il était excellent encore
pour les petites véroles et les autres maladies de venin, d'ailleurs
médiocre médecin.

M. de Chevreuse dit au roi la résolution qu'il prenait\,; il l'approuva,
et le rare est que Fagon même en fut bien aise, qui, dans une autre
occasion, en serait entré en furie\,; mais comme il était bien persuadé
que M. de Beauvilliers ne pouvait échapper, et qu'il mourrait à
Saint-Aignan, il fut ravi que ce fût entre les mains d'Helvétius, pour
en triompher. Dieu merci, le contraire arriva. Helvétius le trouva au
plus mal\,; en sept ou huit jours il le mit en état de guérison certaine
et de pouvoir s'en revenir. Il arriva de fort bonne heure à Versailles,
le 8 mars. Je courus l'embrasser avec toute la joie la plus vive.
Revenant de chez lui, et traversant l'antichambre du roi, je vis un gros
de monde qui se pressait à un coin de la cheminée\,: j'allai voir ce que
c'était. Ce groupe de monde se fendit\,; je vis Fagon tout débraillé,
assis, la bouche ouverte, dans l'état d'un homme qui se meurt. C'était
une attaque d'épilepsie. Il en avait quelquefois, et c'est ce qui le
tenait si barricadé chez lui, et si court en visites chez le peu de
malades de la cour qu'il voyait, et chez lui jamais personne. Aussitôt
que j'eus aperçu ce qui assemblait ce monde, je continuai mon chemin
chez M. le maréchal de Lorges, où entrant avec l'air épanoui de joie, la
compagnie, qui y était toujours très nombreuse, me demanda d'où je
venais avec l'air satisfait. «\,D'où je viens\,? répondis-je,
d'embrasser un malade condamné qui se porte bien, et de voir le médecin
condamnant qui se meurt.\,» J'étais ravi de M. de Beauvilliers, et piqué
sur lui contre Fagon. On me demanda ce que c'était que cette énigme. Je
l'expliquai, et voilà chacun en rumeur sur l'état de Fagon, qui était à
la cour un personnage très considérable et des plus comptés, jusque par
les ministres et par tout l'intérieur du roi. M. {[}le maréchal{]} et
M\textsuperscript{me} la maréchale de Lorges me firent signe, de peur
que je n'en disse davantage, et me grondèrent après avec raison de mon
imprudence. Apparemment qu'elle ne fut pas jusqu'à Fanon, avec qui je
fus toujours fort bien.

On sut en même temps que le cardinal de Bouillon, à bout d'espérances
sur ses manèges et sur les démarches réitérées du pape en sa faveur,
était enfin parti de Rome, et s'était rendu à son exil de Cluni, où
bientôt après il eut mainlevée de la saisie de ses biens et de ses
bénéfices. Il n'avait pu se tenir, après avoir ouvert la porte sainte du
grand jubilé, d'en faire frapper des médailles où cette cérémonie était
d'un côté, lui de l'autre, avec son nom autour et la qualité de grand
aumônier de France, qu'il n'était plus alors. Cela avait irrité le roi
de nouveau contre lui, et eut peut-être part à la fermeté avec laquelle
il résista au pape sur le retour et l'exil du cardinal de Bouillon et à
tout ce qu'il employa pour s'en délivrer.

Milord Melford, chevalier de la Jarretière, qu'on a vu ci-devant exilé
de Saint-Germain, et revenu seulement à Paris, écrivit une lettre à
milord Perth son frère, gouverneur du prince de Galles, par laquelle il
paraissait qu'il y avait un parti considérable en Écosse en faveur du
roi Jacques, et qu'on songeait toujours ici à le rétablir et la religion
catholique en Angleterre. Je ne sais ni personne n'a su comment il
arriva que cette lettre, au lieu d'aller à Saint-Germain, fut à Londres.
Le roi Guillaume la fit communiquer au parlement et en fit grand usage
contre la France qui ne pensait à rien moins, et qui avait bien d'autres
affaires pour soutenir la succession d'Espagne, et d'ailleurs ce n'eût
pas été au comte de Melford qu'on se fût fié d'un dessein de cette
importance, dans la situation où il était avec sa propre cour et la
nôtre\,; mais il n'en fallait pas tant au roi Guillaume pour faire bien
du bruit, ni aux Anglais pour les animer contre nous dans la conjoncture
des affaires présentes. Melford fut pour sa peine envoyé à Angers et fut
fort soupçonné. Je ne sais si ce fut à tort ou non.

Peu de jours après, le roi Jacques se trouva fort mal et tomba en
paralysie d'une partie du corps, sans que la tête fut attaquée. Le roi,
et toute la cour à son exemple, lui rendit de grands devoirs. Fagon
l'envoya à Bourbon. La reine d'Angleterre l'y accompagna. Le roi fournit
magnifiquement à tout, chargea d'Urfé d'aller avec eux de sa part, et de
leur faire rendre partout les mêmes honneurs qu'à lui-même, quoiqu'ils
voulussent être sans cérémonies.

\hypertarget{chapitre-vi.}{%
\chapter{CHAPITRE VI.}\label{chapitre-vi.}}

1701

~

{\textsc{Philippe V à Madrid.}} {\textsc{- Exil de Mendoze, grand
inquisiteur.}} {\textsc{- Exil confirmé du comte d'Oropesa, président du
conseil de Castille.}} {\textsc{- Digression sur l'Espagne\,: branches
de la maison de Portugal établies en Espagne.}} {\textsc{- Oropesa,
Lémos, Veragua, {[}branche{]} cadette de Ferreira ou Cadaval.}}
{\textsc{- {[}Branche de{]} Cadaval restée en Portugal.}} {\textsc{-
Alencastro, duc d'Aveiro.}} {\textsc{- Duchesse d'Arcos, héritière
d'Aveiro.}} {\textsc{- Abrantès et Liñarès, cadets d'Aveiro.}}
{\textsc{- Justice et conseil d'Aragon.}} {\textsc{- Conseil de
Castille\,; son président ou gouverneur.}} {\textsc{- Corrégidors.}}
{\textsc{- Conseillers d'État.}} {\textsc{- Secrétaire des dépêches
universelles.}} {\textsc{- Secrétaires d'État.}} {\textsc{- Les trois
charges\,: majordome-major du roi et les majordomes\,; sommelier du
corps et gentilshommes de la chambre\,; grand écuyer et premier
écuyer.}} {\textsc{- Capitaine des hallebardiers.}} {\textsc{-
Patriarche des Indes.}} {\textsc{- Majordome-major et majordomes de la
reine.}} {\textsc{- Grand écuyer et premier écuyer de la reine.}}
{\textsc{- Camarera-mayor.}} {\textsc{- Dames du palais et dames
d'honneur.}} {\textsc{- Azafata et femmes de chambre.}} {\textsc{-
Marche en carrosse de cérémonie.}} {\textsc{- Gentilshommes de la
chambre avec et sans exercice.}} {\textsc{- Estampilla.}} {\textsc{- La
Roche.}}

~

Le roi d'Espagne arriva enfin, le 19 février, à Madrid, ayant eu partout
sur sa route une foule et des acclamations continuelles, et dans les
villes des fêtes, des combats de taureaux, et quantité de dames et de
noblesse des pays par où il passa. Il y eut une telle presse à son
arrivée à Madrid, qu'on y compta soixante personnes étouffées. Il trouva
hors la ville et dans les rues une infinité de carrosses qui bordaient
sa route, remplis de dames fort parées, et toute la cour et la noblesse
qui remplissait le Buen-Retiro, où il fut descendre et loger. La junte
et beaucoup de grands le reçurent à la portière, où le cardinal
Portocarrero se voulut jeter à ses pieds pour lui baiser la main. Le roi
ne le voulut pas permettre, il le releva et l'embrassa, et le traita
comme son père. Le cardinal pleurait de joie, et ne cessa de tout le
soir de le regarder. Enfin tous les conseils, tout ce qu'il y avait
d'illustre, une foule de gens de qualité, une noblesse infinie et toute
la maison espagnole du feu roi Charles II {[}le reçurent à la
portière{]}. Les rues de son passage avaient été tapissées à la mode
d'Espagne, chargées de gradins remplis de beaux tableaux et d'une
infinité d'argenterie, avec des arcs de triomphe magnifiques d'espace en
espace. Il n'est pas possible d'une plus grande ni plus générale
démonstration de joie.

Le roi était bien fait, dans la fleur de la première jeunesse, blond
comme le feu roi Charles et la reine sa grand'mère, grave, silencieux,
mesuré, retenu, tout fait pour être parmi les Espagnols. Avec cela fort
attentif à chacun, et connaissant déjà les distinctions des personnes
par l'instruction qu'il avait eu loisir de prendre d'Harcourt, le long
du voyage. Il ôtait le chapeau ou le soulevait presque à tout le monde,
jusque-là que les Espagnols s'en formalisèrent et en parlèrent au duc
d'Harcourt, qui leur répondit que, pour toutes les choses essentielles,
le roi se conformerait à tous les usages, mais que dans les autres il
fallait lui laisser la civilité Française. On ne saurait croire combien
ces bagatelles d'attention extérieure attachèrent les cœurs à ce prince.

Le cardinal Portocarrero était transporté de contentement\,; il
regardait cet événement comme son ouvrage et le fondement durable de sa
grandeur et de sa puissance. Il en jouissait en plein. Harcourt et lui,
sentant en habiles gens le besoin réciproque qu'ils auraient l'un de
l'autre, s'étaient intimement liés, et leur union s'était encore
cimentée pendant le voyage par l'exil de la reine à Tolède, que le
cardinal avait obtenu, et par celui de Mendoze, évêque de Ségovie, grand
inquisiteur, charge qui balance, et qui a quelquefois embarrassé
l'autorité royale, et que le pape confère sûr la présentation du roi.
Mendoze était un homme de qualité distinguée, mais un assez pauvre
homme, qui n'avait rien commis de répréhensible ni qui pût même donner
du soupçon. Il ne méritait pas une si grande place, mais il méritait
encore moins d'être chassé. Son crime était d'être parvenu à ce grand
poste par le crédit de la reine, qui avait fort maltraité le cardinal
durant son autorité, et après la chute de sa puissance et la mort de
Charles II, le grand inquisiteur avait tenu sa morgue avec le cardinal
qu'il n'avait pas salué assez bas dans l'éclat où il venait de monter.
Ce \emph{punto}\footnote{Ce mot signifie probablement ici point
  d'honneur.} espagnol qui pouvait être loué de grandeur de courage,
acheva d'allumer la colère du cardinal, ennemi de toutes les créatures
de la reine et passionné de le leur faire sentir. D'ailleurs, comme
assuré de toute l'autorité séculière et pour bien longtemps, sous un
prince aussi jeune et étranger qui lui devait tant, il ne pouvait
souffrir la puissance ecclésiastique dans un autre, et avait un désir
extrême de les réunir toutes deux en sa personne par la charge de grand
inquisiteur\,; tellement qu'encouragé par l'exil de la reine qu'il
venait d'emporter, il s'aventura d'exposer l'autorité naissante du roi
en lui demandant l'exil du grand inquisiteur. M. d'Harcourt, son ami, et
qui le connaissait bien, n'eut garde de s'opposer à un désir si ardent
et si causé\,; et quoique le roi eût déclaré qu'il ne disposerait
d'aucune chose, ni petite ni considérable, qu'après son arrivée à
Madrid, de l'avis de M. d'Harcourt, il envoya au cardinal l'ordre qu'il
demandait par son même courrier. Mendoze, qui sentit bien d'où le coup
lui venait, balança tout un jour entre demeurer et obéir. En demeurant,
il eût fort embarrassé par l'autorité et les ressorts de sa place, et le
nombre de gens considérables attachés à là reine. Mais il prit enfin le
partir d'obéir, et combla de joie la vanité et la vengeance du cardinal,
qui, enhardi par ces deux grands coups, en fit un troisième\,: ce fut un
ordre qu'il obtint du roi, qui approchait déjà de Madrid, au comte
d'Oropesa de demeurer dans son exil. Il était premier ministre et
président du conseil de Castille. Il y avait deux ans que Charles II l'y
avait envoyé sur une furieuse sédition que le manque de pain et de
vivres avait causée à Madrid qui fit grande peur à ce prince, et dont la
faute fut imputée au premier ministre. Puisque je me trouve ici en
pleine Espagne, et qu'il est curieux de la connaître un peu à cet
avènement de la branche de France, et qu'il sera souvent mention de ce
pays dans la suite, je m'y espacerai un peu à droite et à gauche en
parlant de ce qu'il s'y passa à l'arrivée du nouveau roi.

Oropesa était de la maison de Bragance\footnote{Ce passage, jusqu'à la
  p.~97 (\emph{L'Espagne est partagée tout entière}), a été supprimé
  dans les précédentes éditions des Mémoires de Saint-Simon. Il est
  cependant nécessaire, puisque Saint-Simon s'y réfère à l'époque de son
  ambassade en Espagne\,: «\,On a tâché d'expliquer, dit-il (t. XIX,
  p.~292 de l'édition Sautelet), les branches royales de Portugal,
  Oropesa, Lemos, Veragua, Cadaval, etc. Ainsi je n'en ferai point de
  redites.} et l'aîné des trois branches de cette maison établies et
restées en Espagne. Le grand-père du comte d'Oropesa était cousin
germain de Jean, duc de Bragance, que la fameuse révolution de Portugal
mit sur le trône en 1640, dont la quatrième génération y est
aujourd'hui. Ce même grand-père de, notre comte d'Oropesa était
petit-fils puîné de Jean Ier, duc de Bragance, et eut Oropesa par sa
mère Béatrix de Tolède. Le père de notre comte passa par les
vice-royautés de Navarre et de Valence, eut la présidence du conseil
d'Italie, fut fait grand d'Espagne et mourut en 1671. Cette branche
d'Oropesa, quoique si proche et si fraîchement sortie de celle de
Bragance, en était mortellement ennemie. Lorsque l'Espagne eut enfin
reconnu le roi de Portugal, il vint un ambassadeur de Portugal à Madrid.
Le jour de sa première audience, Oropesa fit lever son fils malade de la
fièvre, qui était dans les gardes espagnoles, et lui fit prendre la
pique devant le palais, afin, dit-il, que le roi de Portugal sût quelle
était la grandeur du roi d'Espagne, qui était gardé par ses plus proches
parents. Ce fils est notre comte d'Oropesa, qui fut capitaine général de
la Nouvelle-Castille, conseiller d'État, président du conseil d'Italie
comme son père, très bien avec Charles II, qui le fit président du
conseil de Castille et premier ministre, et qui deux ans avant sa mort
l'exila, comme je l'ai raconté.

Tout d'un temps achevons la fortune de ce seigneur et de cette
branche\,: lassé de son exil, auquel il ne voyait point de fin, il passa
du côté de l'archiduc en 1706, et mourut à Barcelone, en décembre de
l'année suivante, à soixante-cinq ans. Il avait mené ses deux fils avec
lui. Le marquis d'Alcaudete eut douze mille livres de pension de
l'empereur sur Naples et ne fit ni fortune ni alliance, l'aîné passa à
Vienne, fut chambellan de l'empereur, chevalier de la Toison d'or en
1712, puis garde-sceau de Flandre. Il était gendre et beau-frère des
ducs de Frias, connétables de Castille. La paix étant faite en 1725, en
avril, entre l'empereur et Philippe V, le comte d'Oropesa revint avec sa
femme en Espagne, où il mourut bientôt après. Son fils unique y épousa
fort jeune la fille du comte de San-Estevan de Gormaz, premier capitaine
des gardes du corps et qui devint peu après marquis de Villena et
majordome-major du roi à la mort de son père. Le comte d'Oropesa fut
fait chevalier de la Toison d'or, et mourut peu après sans postérité
masculine. Ainsi cette branche d'Oropesa est finie.

Celle de Lémos sort de Denis, fils puîné de Ferdinand II, duc de
Bragance, petit-fils d'Alphonse, bâtard du roi de Portugal Jean Ier. Ce
Denis, par conséquent, était frère puîné de Jacques, duc de Bragance,
grand-père de Jean, premier duc de Bragance, duquel est sortie la
branche d'Oropesa. Denis devint comte de Lémos en Castille avec une
fille héritière de Roderic, bâtard d'Alphonse, mort sans enfants avant
son père Pierre Alvarez de Castro Ossorio, seigneur de Cabrera et
Ribera, en faveur duquel Henri IV, roi de Castille, avait érigé Lémos en
comté. C'est de là que cette branche de Lémos a toujours ajouté le nom
de Castro à celui de Portugal, comme celle d'Oropesa y ajouta toujours
celui de Tolède. Par ce mariage, les enfants de Denis s'attachèrent plus
à l'Espagne qu'au Portugal. Ferdinand, l'aîné, fut fait grand d'Espagne
et fut ambassadeur de Charles V et de Philippe Il à Rome\,; et son fils,
Pierre-Ferdinand, servit Philippe II à la conquête de Portugal.

Les quatre générations suivantes ont eu les plus grands emplois
d'Espagne et les premières vice-royautés. La quatrième, qui est le père
du comte de Lémos vivant à l'avènement de Philippe V, était gendre du
duc de Gandie et vice-roi du Pérou\footnote{La phrase est reproduite
  textuellement d'après le manuscrit\,; elle semble incomplète. Il
  faudrait probablement\,: \emph{la quatrième a eu pour chef le père du
  comte de Lémos, etc}.}. Son fils, qui a épousé la sœur du duc de
l'Infantado, de la maison de Silva, n'en a point eu d'enfants. Il vit
encore et n'a jamais eu d'emploi. Pour le premier de cette branche, en
qui elle va finir, c'est un bon homme, mais un très pauvre homme, qui
est bien connu pour tel et qui passe sa vie à fumer. Sa femme et son
beau-frère l'entraînèrent du côté de l'archiduc pendant la guerre. Ils
furent arrêtés comme ils y passaient, et prisonniers quelque temps. Le
duc de l'Infantado a toujours été mal à la cour depuis. Sa sœur, qui a
de l'esprit et du manège, s'y sut raccommoder, et à la fin fut
camarera-mayor de M\textsuperscript{lle} de Beaujolais, lorsqu'elle fut
envoyée en Espagne pour épouser don Carlos, et c'était une des dames
d'Espagne des plus capables de cet emploi, mais qu'on fut surpris
qu'elle voulût bien accepter.

La troisième branche de la maison de Bragance ou de Portugal établie en
Espagne est celle de Veragua. Mais, pour l'expliquer, il faut remonter à
celle de Cadaval ou de Ferreira, dont elle est sortie, laquelle est
demeurée en Portugal. Alvare, marquis de Ferreira, était fils puîné de
Ferdinand Ier, duc de Bragance, lequel était fils d'Alphonse, bâtard du
roi de Portugal Jean Ier. Ainsi ce premier marquis de Ferreira était
frère puîné de Ferdinand II, duc de Bragance, duquel est sortie la
branche de Lémos, et qui était aussi quatrième aïeul du duc de Bragance
que la révolution de Portugal remit sur le trône, bisaïeul du roi de
Portugal d'aujourd'hui, par où on voit l'extrême éloignement de sa
parenté avec les durs de Cadaval et de Veragua, et combien leur branche
est cadette et éloignée de celle de Lémos, et encore plus de celle
d'Oropesa.

Alvare, premier marquis de Ferreira, eut deux fils\,: Roderic, marquis
de Ferreira, duquel les ducs de Cadaval sont sortis, et Georges, comte
de Gelves, de qui les ducs de Veragua sont venus. Georges, comte de
Gelves, épousa la fille héritière du fils de ce fameux Christophe
Colomb, qui était duc de Veragua, marquis de la Jamaïque, que les
Anglais ont usurpée, et amiral héréditaire et vice-roi des Indes après
son célèbre père. De ce mariage, un fils qui en eut deux et qui mourut
de bonne heure, et sa branche ne dura pas. Le second, Nuño de
Portugal-Colomb, dont cette branche ajouta toujours le nom au sien,
disputa les droits de son aïeule, héritière des Colomb, et gagna son
procès. Il devint ainsi duc de Veragua, grand d'Espagne et amiral
héréditaire des Indes. Son fils n'eut point d'emplois\,; mais son
petit-fils mourut gouverneur de la Nouvelle-Espagne, ayant la Toison
d'or. Lui et le comte de Lémos d'alors avaient été des seigneurs témoins
à l'acte fait à Fontarabie par l'infante Marie-Thérèse allant épouser le
roi. Celui-ci mourut en 1674. Pierre-Emmanuel Nuño, duc de Veragua, son
fils, fut vice-roi de Galice, de Valence et de Sicile, général des
galères d'Espagne, chevalier de la Toison d'or, enfin conseiller d'État
et président du conseil d'Italie. C'est le père de celui qui existait
lors de l'avènement de Philippe V. Cette branche est encore finie dans
le fils de ce dernier, dont sa sœur, la duchesse de Liria, a recueilli
toute la riche succession. J'aurai lieu ailleurs de parler d'elle et de
son frère, dernier duc de Veragua de la branche de Portugal, cadette de
celle de Cadaval, dont je dirai un mot par curiosité à cause des
alliances lorraines qu'elle a nouvellement prises en France.

On a vu que Georges, comte de Gelves, de qui descendent les ducs de
Veragua, était frère puîné de Roderic, marquis de Ferreira, d'où sont
sortis les ducs de Cadaval, tous deux fils d'Alvare, fils et frère puîné
de Ferdinand Ier et de Ferdinand II, ducs de Bragance. Alvare épousa
Philippe, fille héritière de Roderic de Mello, comte d'Olivença\,; ce
qui a fait ajouter le nom de Mello à celui de Portugal à toute cette
branche jusques à aujourd'hui. Roderic, chef de cette branche, Ferdinand
et Nuño Alvarez, fils et petits-fils de Roderic, portèrent le nom de
marquis de Ferreira, et tous demeurèrent en Portugal.

François, fils de Nuño Alvarez, aidé de Roderic son frère,
administrateur de l'évêché d'Evora, et de sa charge de général de la
cavalerie de Portugal, eut une part principale à la révolution de
Portugal, qui remit le duc de Bragance sur le trône. Il commandait la
cavalerie pour ce prince à la bataille de Badajoz, que les Espagnols
perdirent en 1644, après avoir été ambassadeur en France en 1641, et
mourut en 1645 à Lisbonne. Ses frères, qui n'eurent point d'enfants,
eurent de grands emplois en Espagne et en Portugal.

Le roi de Portugal étant mort, en 1656, après quinze ans depuis que la
révolution, l'avait porté sur le trône, Louise de Guzman, sa femme,
fille et sœur des ducs de Medina-Sidonia, dont l'esprit et le grand
courage l'avaient porté dans cette élévation, fut régente de ses fils en
bas âge et du royaume. Nuño Alvarez, marquis de Ferreira, fils de
François, dont je viens de parler, fut dans le premier crédit auprès
d'elle. Il avait eu la charge de son père de général de la cavalerie, et
il fut fait duc de Cadaval, n'y ayant plus aucun autre duc dans le
royaume, et n'y {[}en{]} ayant point eu depuis. À ce titre furent
attachés de grands honneurs et la charge héréditaire de grand maître de
la maison du roi. Mais, en 1662, le roi don Alphonse, gouverné par Louis
Vasconcellos Sousa, comte de Castelmelhor, se retira à Alcantara au mois
d'avril, d'où il manda à la reine sa mère qu'il voulait gouverner par
lui-même et relégua en même temps le duc de Cadaval. La reine se retira
dans un couvent près de Lisbonne, et y mourut en février 1666. En juin
suivant, ce roi épousa la sœur de la mère du premier roi de Sardaigne,
fille du duc de Nemours et d'une fille de César, duc de Vendôme, qui,
lasse de ses folies et de la cruauté qu'il faisait paraître, forma un
parti, l'accusa de faiblesse d'esprit et d'impuissance, se fit
juridiquement démarier (24 mars 1668), l'y fit consentir et abdiquer, et
la même année, le 2 avril, c'est-à-dire dix jours après la cassation de
son mariage, elle rappela le duc de Cadaval, qui fut premier
plénipotentiaire pour la paix avec l'Espagne, en 1667 et 1668, et ayant
pratiqué avec la duchesse de Savoie, sa sœur, le mariage de sa fille
unique avec le duc de Savoie, son fils, depuis premier roi de Sardaigne,
pour être roi de Portugal après Pierre, ce fut le duc de Cadaval qui
l'alla chercher à Nice avec la flotte qu'il commandait pour l'amener en
Portugal, où ce prince ne voulut jamais se laisser conduire ni achever
ce mariage. C'était en 1680.

M. de Cadaval se retira de la cour bientôt après la mort de la reine et
céda son titre et ses emplois à son fils aîné, qui mourut jeune, en
1700, sans enfants d'une bâtarde du roi don Pierre. Son frère lui
succéda. Le père, qui survécut son aîné, avait été marié trois fois\,:
la première sans enfants, la deuxième à une Lorraine, fille et sœur des
princes d'Harcourt, la troisième à une fille de M. le Grand. De la
deuxième il n'eut qu'une fille, et de la troisième, ses autres enfants
Nuño Alvarez, duc de Cadaval, par la mort de son aîné, né en décembre
1679, a joint à ses autres emplois héréditaires ceux de conseiller
d'État, de majordome-major de la reine, de président du
\emph{desembargo}\footnote{Conseil de finances qui accordait
  provisoirement la jouissance de certains revenus, en attendant le
  brevet signé de la main du roi.} du palais, et de mestre de camp du
palais et de l'Estrémadure. Il épousa la veuve de son frère, et, l'ayant
perdue, s'est remarié, en 1738, à une fille du prince de Lambesc,
c'est-à-dire du fils du frère de sa mère. Il y a en Portugal plusieurs
branches masculinement et légitimement sorties des ducs de Bragance, qui
n'ont aucune distinction particulière.

Achevons tout d'un temps les branches de Portugal établies en Espagne.

Jean II, roi de Portugal, était arrière-petit-fils du roi Jean Ier, qui,
comme on l'a vu, était bâtard du roi Pierre Ier, qui ne laissa point
d'enfants mâles ni légitimes, et ce bâtard fut élu roi par les états
généraux de Portugal à Coïmbre. Jean II était donc petit-fils du roi
Édouard, duquel Alphonse, tige de la maison de Bragance, était bâtard,
tellement que ce roi Jean II était cousin issu de germain par bâtardise
de Ferdinand II, duc de Bragance, frère de don Alvarez, duquel sont
sorties les branches de Cadaval et Veragua, et père de Denis, comte de
Lémos son puîné, de qui la branche de Lémos est sortie, et ce même
Ferdinand, duc de Bragance, était bisaïeul de Jean Ier, duc de Bragance,
duquel, par Édouard son puîné, la branche d'Oropesa est venue\,; lequel
Jean Ier, duc de Bragance, fut grand-père de l'autre Jean, duc de
Bragance, que la révolution de Portugal mit sur le trône à la fin de
1640.

Ce Jean II, roi de Portugal, ne laissa qu'un bâtard nommé Georges. La
couronne passa à Emmanuel, frère du cardinal Henri, qui succéda au roi
don Sébastien, tué, sans enfants, en Afrique, duquel Emmanuel était
bisaïeul\,; après la mort duquel Philippe II, roi d'Espagne, s'empara du
Portugal. Emmanuel et ce cardinal étaient fils du duc de Viseu, frère
d'Alphonse V, roi de Portugal, père du roi Jean II.

Georges, bâtard de ce roi Jean II, fut fait duc de Coïmbre par le roi
Emmanuel pour sa vie, et pour sa postérité seigneur d'Aveiro,
Torres-Nuevas et Montemor en 1500. Il épousa une fille d'Alvarez, tige
des branches de Cadaval et Veragua, et prit pour sa postérité le nom
d'Alencastro, c'est-à-dire de Lancastre, en mémoire de la reine Ph. de
Lancastre, femme du roi Jean Ier de Portugal, grand-père et grand'mère
du roi Jean II, dont il était bâtard.

Jean d'Alencastro, fils du bâtard Georges, fut fait duc d'Aveiro par le
même roi don Emmanuel en 1530. Son fils ne laissa qu'une fille qui
épousa Alvarez, son cousin germain, fils du frère de son père. De ce
mariage plusieurs enfants, l'aîné desquels continua la suite des ducs
d'Aveiro, et du puîné vinrent les ducs d'Abrantès. L'aîné des deux ne
laissa qu'un fils et une fille. Le fils mourut sans enfants en 1665, à
trente-huit ans, s'étant jeté tout jeune dans le parti d'Espagne, et y
passa, en 1661, sous prétexte d'y demander le duché de Maqueda de
l'héritage de sa mère. Il fut fait général de la flotte et grand
d'Espagne. Sa sœur unique hérita de sa grandesse et du duché d'Aveiro,
confisqué en Portugal, avec les autres biens qui y étaient, et de
Maqueda et des biens situés en Espagne. Elle eut après ordre de sortir
de Portugal, vint en Espagne où elle épousa Emmanuel-Ponce de Léon, duc
d'Arcos, grand d'Espagne. Elle plaida contre le prince Pierre, régent et
depuis roi de Portugal, et contre le duc d'Abrantès pour les biens qui
lui furent adjugés en 1679, à condition qu'elle irait demeurer en
Portugal.

Elle n'en tint pas grand compte et demeura veuve en 1693. C'était une
personne très vertueuse, mais très haute, et fort rare pour son esprit
et son érudition. Elle savait parfaitement l'histoire sacrée et profane,
le latin, le grec, l'hébreu et presque toutes les langues vivantes. Sa
maison à Madrid était le rendez-vous journalier de tout ce qu'il y avait
de plus considérable en esprit, en savoir et, en naissance, et c'était
un tribunal qui usurpait une grande autorité, et avec lequel la cour,
les ministres et les ministres étrangers même qui s'y rendaient assidus,
se ménageaient soigneusement. M. d'Harcourt eut grande attention à être
bien avec elle, et le roi d'Espagne la distingua fort en arrivant. Elle
était mère des ducs d'Arcos et de Baños, tous deux grands d'Espagne,
dont j'aurai cette année même occasion de parler, et du voyage qu'ils
firent en France. Ainsi la branche aînée d'Alencastro des ducs d'Aveiro
s'éteignit dans les Ponce de Léon, ducs d'Arcos.

Alphonse, puîné d'Alvarez, duc d'Aveiro fils du bâtard Georges, eut de
grands emplois et fut fait duc d'Abrantès et grand d'Espagne par
Philippe IV. Il se fit prêtre après la mort de sa femme, et il mourut en
1654. C'est le père du duc d'Abrantès qu'on a vu ci-devant, qui apprit
si cruellement et si plaisamment à l'ambassadeur, de l'empereur la
disposition du testament de Charles II qu'on venait d'ouvrir. C'est lui
aussi qui perdit contre la duchesse d'Arcos, dont je viens de parler,
ses prétentions sur les duchés d'Aveiro et de Torres-Nuevas. Il vécut
jusqu'en 1720, fort considéré et ménagé par les ministres. Il avait
infiniment d'esprit, des saillies plaisantes, d'adresse et surtout de
hardiesse et de hauteur, et se sut maintenir jusqu'à la fin dans la
privance et dans l'amitié du roi. Il mourut à quatre-vingt-trois ans, et
avait épousé Jeanne de Noroña, fille du premier duc de Liñarès, grand
d'Espagne, dont elle eut la succession et la grandesse. Il en eut deux
fils et plusieurs filles, et laissa un bâtard. Le fils {[}aîné{]} fut
duc de Liñarès et grand d'Espagne par la mort de sa mère, et mourut
vice-roi du Mexique du vivant de son père. Il fit un tour en France, où
je le vis à la cour, avant d'aller au Mexique. Il ne laissa point
d'enfants de Léonore de Silva, que j'ai vue à Bayonne camarera-mayor de
la reine, veuve de Charles II. Le frère cadet de ce duc de Liñarès était
évêque lorsqu'il mourut. Il recueillit la grandesse, et, après la mort
de son père, prit le nom de duc d'Abrantès et plus du tout celui
d'évêque de Cuença, quoiqu'il le fût. J'aurai dans les suites occasion
de parler de lui. Quelques années après la mort du père, son bâtard, par
le crédit de sa famille, fut duc de Liñarès et grand d'Espagne. Par ce
détail on voit que ces branches de Bragance ont toutes grandement figuré
en Espagne, mais qu'elles y sont maintenant toutes éteintes.

Après avoir parlé du comte d'Oropesa, président du conseil de Castille,
de son exil et à son occasion des quatre branches de la maison de
Portugal, établies et finies en Espagne, et de celle de Cadaval, qui a
pullulé en Portugal, il faut dire un mot du conseil de Castille et de
celui qui en est chef.

L'Espagne est partagée tout entière entre ce conseil, de qui dépend tout
ce qui est joint à la couronne de Castille, et le conseil d'Aragon, de
qui dépend tout ce qui est joint à la couronne d'Aragon. Ce dernier
avait un bien plus grand pouvoir que celui de Castille, et son chef, qui
portait le titre de grand justicier, et par corruption celui simplement
de Justice {[} \emph{Justiza} {]}, avait une morgue et une autorité qui
balançait celle du roi. Il se tenait à Saragosse, où le roi fut, peu
après son arrivée à Madrid, recevoir les hommages de l'Aragon et prêter
le serment accoutumé d'en maintenir les immenses privilèges, après quoi
le justicier lui prête serment au nom du royaume\,; en le prêtant il
débute par ces mots\,: \emph{Nous qui valons autant que vous}, puis le
serment fondé sur celui que le roi vient de prêter et qu'il y sera
fidèle, et finit par ceux-ci\,: \emph{sinon, non}. Tellement qu'il ne
laisse pals ignorer par les paroles mêmes du serment qu'il n'est que
conditionnel. Je n'en dirai pas davantage parce que la révolte de
l'Aragon et de la Catalogne en faveur de l'archiduc engagea Philippe V à
la fin de la guerre d'abroger pour jamais tous les privilèges de
l'Aragon et de la Catalogne qu'il a presque réduits à la condition de
province de Castille.

Le conseil de Castille se tient à Madrid. Il est composé d'une vingtaine
au plus de conseillers et d'un assez grand nombre de subalternes. Il n'y
a qu'un seul président qui y doit être fort assidu et qui, pour le
courant, lorsqu'il manque par maladie ou par quelque autre événement,
est suppléé par le doyen, mais uniquement pour l'intérieur du conseil.
Je n'en puis donner une idée plus approchante de ce qu'il est, suivant
les nôtres, que d'un tribunal qui rassemble en lui seul le ressort, la
connaissance et la juridiction qui sont ici partagées entre tous les
parlements et les chambres des comptes du royaume\,; ces derniers pour
les mouvances\footnote{La \emph{mouvance} désignait, dans le langage
  féodal, la suzeraineté d'un seigneur dominant sur ses vassaux. On
  disait dans ce sens qu'un fief avait beaucoup de terre dans sa
  \emph{mouvance}.}, le grand conseil, et le conseil privé\footnote{Il a
  été question, dans la note II placée à la fin du Ier volume de ces
  Mémoires, des différents conseils que l'on appelait \emph{conseils du
  roi}. Le conseil privé, dont parle ici Saint-Simon et auquel seul
  s'applique le dernier membré de phrase, est le même que le conseil des
  parties que le chancelier présidait. Le grand conseil était un
  tribunal particulier, qu'on ne doit pas confondre avec le conseil
  d'État de l'ancienne monarchie. Les attributions du grand conseil
  étaient très diverses\,: il jugeait la plupart des procès relatifs aux
  bénéfices ecclésiastiques, les causes évoquées des parlements, les
  conflits entre les parlements et les tribunaux inférieurs appelés
  présidiaux, etc. Le grand conseil fut institué par Charles VIII, en
  1497. Ce tribunal était primitivement présidé par le chancelier\,;
  mais il eut, dans la suite, un premier président et plusieurs
  présidents.}, c'est-à-dire celui où le chancelier de France préside
aux conseillers d'État et aux maîtres des requêtes. C'est là où toutes
les affaires domaniales et, particulières sont portées en dernier
ressort, où les érections et les grandesses sont enregistrées et où les
édits et les déclarations sont publiés, les traités de paix, les dons,
les grâces, en un mot où passe tout ce qui est public, et on juge tout
ce qui est litigieux. Tout s'y rapporte, rien ne s'y plaide\,; avec tout
ce pouvoir, ce conseil ne rend que des sentences. Il vient une fois la
semaine dans une pièce tout au bout en entrant dans l'appartement du roi
à jour et heure fixée le matin. Il est en corps, et il est reçu et
conduit au bas de l'escalier du palais par le majordome de semaine\,;
dans cette pièce le fauteuil du roi est sous un dais sur une estrade et
un tapis. Vis-à-vis et aux deux côtés trois bancs de bois nu où se place
le conseil. Le président a la première place à droite le plus près du
roi, et à côté du président celui qui ce jour-là est chargé de rapporter
les sentences de la semaine quoique rendues au conseil au rapport de
différents conseillers. Ce rapporteur est nommé pour chaque affaire par
le président, comme ici dans nos tribunaux, qui nomme aussi, tantôt
l'un, tantôt l'autre, pour rapporter les sentences de la semaine au roi.

Le conseil placé, le roi arrive\,: sa cour et son capitaine des gardes
même s'arrêtent à la porte en dehors de cette pièce. Dès que le roi y
entre, tout le conseil se met à genoux, chacun devant sa place. Le roi
s'assied dans son fauteuil et se couvre, et tout de suite ordonne au
conseil de se lever, de s'asseoir et de se couvrir. Alors la porte se
ferme, et le roi demeure seul avec ce conseil dont le président n'est
distingué en rien pour cette cérémonie. Les sentences de la semaine sont
là rapportées\,: le nom des parties, leurs prétentions, leurs raisons
respectives et principales, et les motifs du jugement. Tout cela le plus
courtement qu'il se peut, mais sans rien oublier d'important. Tout se
rapporte de suite, après quoi le président et le rapporteur présentent
au roi chaque sentence l'une après l'autre qui la signe avec un parafe
pour avoir plus tôt fait, et de ce moment ces sentences deviennent des
arrêts. Si le roi trouve quelque chose à redire à quelque sentence, et
que l'explication qu'on lui en donne ne le satisfasse pas, il la laisse
à un nouvel examen ou il la garde par-devers lui. Tout étant fini, et
cela dure une heure et souvent davantage, le roi se lève, le conseil se
met à genoux jusqu'à ce qu'il ait passé la porte, et s'en va comme il
est venu, excepté le président seul, qui, au lieu de se mettre à genoux,
suit le roi qui trouve sa cour dans une pièce voisine, y en ayant un
vide entre-deux, et avec ce cortège passe une partie de son
appartement\,; dans une des pièces vers la moitié, il trouve un
fauteuil, une table à côté, et vis-à-vis du fauteuil un tabouret. Là le
roi s'arrête, sa cour continue de passer, puis les portes d'entrée et de
sortie se ferment, et le roi dans son fauteuil reste seul avec le
président assis sur ce tabouret\,; là il revoit les sentences qu'il a
retenues et les signe si bon lui semble, ou il les garde pour les faire
examiner par qui il lui plaît, et le président lui rend un compte
sommaire du grand détail public et particulier dont il est chargé. Cela
dure moins d'une heure. Le roi ouvre lui-même la porte pour retrouver sa
cour qui l'attend et s'en aller chez lui, et le président retourne par
l'autre par où il est entré, trouve un majordome qui l'accompagne à son
carrosse et s'en va chez lui. Ces sentences retenues, ceux à qui le roi
les renvoie lui en rendent compte avec leur avis. Il les envoie au
président de Castille, et finalement l'arrêt se rend comme le roi le
veut. On voit par la qu'il est parfaitement absolu en toute affaire
publique et particulière, et que les rois d'Espagne ont retenu l'effet,
comme nos rois le droit, d'être les seuls juges de leurs sujets et de
leur royaume. Ce n'est pas qu'il n'arrive bien aussi que le conseil de
Castille, ou en corps ou le président seul, ne fasse des remontrances au
roi sur des affaires ou publiques ou particulières auxquelles il se
rend, mais s'il persiste, tout passe à l'instant sans passions ni toutes
les difficultés qu'on voit souvent en France.

Le corrégidor de Madrid et ceux de toutes les villes rendent un compte
immédiat de toute leur administration au président de Castille, et
reçoivent et exécutent ses ordres sur tout ce qui la regarde, comme
eux-mêmes font à l'égard des régidors et des alcades des moindres villes
et autres lieux de leur ressort\,; l'idée d'un corrégidor de Madrid
suivant les nôtres, et à proportion de ceux des autres grandes villes
non fortifiées, c'est tout à la fois l'intendant, le commandant, le
lieutenant civil, criminel et de police, et le maire ou prévôt des
marchands\footnote{Comme la définition de chacun de ces termes et
  l'indication des attributions de ces diverses magistratures exigent
  des développements assez étendus, nous les avons renvoyées aux notes
  de la fin du volume.}. Les gouverneurs des provinces d'Espagne n'ont
guère que l'autorité des armes, et s'ils se mêlent d'autre chose ce
n'est pas sans démêlé ni sans subordination du président et du conseil
de Castille.

On voit par ce court détail quel personnage c'est dans la monarchie.
Aussi en est-il le premier, le plus accrédité et le plus puissant,
tandis qu'il exerce cette grande charge, et dès que la personne du roi
n'est pas dans Madrid il y a seul la même autorité que lui, sans
exception aucune. Son rang aussi répond à un si vaste pouvoir. Il ne
rend jamais aucune visite à qui que ce soit, et ne donne chez lui la
main à personne. Les grands d'Espagne qui ont affaire à lui tous les
jours essuient cette hauteur, et ne sont ni reçus ni `conduits\,: la
vérité est qu'ils le font avertir, et qu'ils entrent et sortent par un
degré dérobé. Les cardinaux et les ambassadeurs de têtes couronnées
n'ont pas plus de privilège. Tout ce qu'ils ont, c'est qu'ils envoient
lui demander audience. Il répond toujours qu'il est indisposé, mais que
cela ne l'empêchera pas de les recevoir tel jour et à telle heure. Ils
s'y rendent, sont reçus et conduits par ses domestiques et ses
gentilshommes, et le trouvent au lit, quelque bien qu'il se porte. Quand
il sort (et ce ne peut être que pour aller chez le roi, à quelques
dévotions, mais dans une tribune séparée, ou prendre l'air), cardinaux,
ambassadeurs, grands d'Espagne, dames, en un mot tout ce qui le
rencontre par les rues, arrête tout court précisément comme on fait ici
pour le roi et pour les enfants de France\,; mais assez souvent il a la
politesse de tirer à demi ses rideaux, et alors cela veut dire que,
quoiqu'en livrées et ses armes à son carrosse, il veut bien n'être pas
connu. On n'arrête point et on passe son chemin. S'il va chez le roi,
comme il arrive assez souvent, hors du jour ordinaire du conseil de
Castille, ce n'est jamais que par audience. Le majordome de semaine le
reçoit et le conduit au carrosse. Dès qu'il paraît on lui présente
auprès de la porte du cabinet, où toute la cour attend, un des trois
tabourets qui sont les trois seuls siégea de tout ce vaste appartement,
par grandeur, qui d'ailleurs est superbement meublé. Le sien, qui est
pareil aux deux autres, est toujours caché et ne se tiré que pour lui\,;
les deux autres sont toujours en évidence, l'un pour je majordome-major,
l'autre pour le sommelier du corps ou grand chambellan. En leur absence,
le gentilhomme de la chambre de jour s'assait sur l'un, et quelque vieux
grand d'Espagne sur l'autre, mais il faut que ce soit un homme incommodé
et qui ait passé par les premiers emplois. Nul autre, ni grand d'Espagne
ni vieux, n'oserait le faire. J'ai pourtant vu les trois sièges remplis
et en apporter un quatrième au prince de Santo-Burno Caraccioli, et une
autre fois au marquis de Bedmar, tous deux alors grands d'Espagne, tous
deux conseillers d'État, et tous deux avant été dans les premiers
emplois, et le dernier y étant encore. C'était pendant mon ambassade en
Espagne, mais je ne l'ai vu faire que pour ces deux-là dont le premier
ne se pouvait soutenir sur ses jambes percluses de goutte, et l'autre
fort goutteux aussi.

Le président du conseil de Castille ne peut être qu'un grand d'Espagne,
et ne peut être destitué que pour crime qui emporte peine de mort. Mais
contre une telle puissance on a le même remède dont on se sert en France
contre le chancelier\,: on exile le président de Castille à volonté et
sans être obligé de dire pourquoi, et on crée un gouverneur du conseil
de Castille, qui on veut, pourvu qu'il ne soit pas grand d'Espagne. Ce
gouverneur a toutes les fonctions, l'autorité et le rang entier du
président et le supplée en tout et partout. Mais cette grande place,
bien supérieure à notre garde des sceaux, a le même revers à craindre et
pis encore que lui\,; car il peut être destitué à volonté et sans dire
pourquoi, même sans l'exiler\,; il perd tout son crédit et toute sa
puissance, il n'est et ne peut plus rien, et toutefois il conserve son
rang en entier pendant sa vie, qui n'est bon qu'à l'empoisonner\,;
puisqu'il ne peut faire aucune visite, et à le réduire en solitude,
parce que personne n'a plus d'affaire à lui, et ne prend la peine de
l'aller voir pour n'en recevoir ni réception, ni la main, ni conduite.
Plusieurs en sont morts d'ennui. Lorsque le président de Castille vient
à mourir, il est au choix du roi de faire un autre président ou un
gouverneur. Depuis la mort du comte d'Oropesa le roi d'Espagne n'a mis
que des gouverneurs\,; il en est de même des autres conseils dont le
président ne peut être ôté, et doit toujours être grand, au lieu duquel
on peut mettre un gouverneur\,; mais comme ces présidents n'ont de rang
que celui de grands\,; parce qu'ils le sont, et que leur autorité n'est
rien quoique les places en soient fort belles, très rarement y met-on
des gouverneurs.

On appelle en Espagne conseillers d'État précisément ce que nous
connaissons ici sous le nom de ministres d'État, et c'est là le but
auquel les plus grands seigneurs, les plus distingués, et qui ont passé
par les plus grands emplois, tendent de toutes leurs brigues. Ils ont
l'Excellence, et passent immédiatement après les grands quand ils ne le
sont point. Il y en a fort peu\,; ils ont une seule distinction que les
grands n'ont pas, qui est de pouvoir, comme les grandes dames, aller par
la ville en chaise à porteurs entourés de leur livrée à pied, suivis de
leur carrosse avec leurs gentilshommes dedans, et de monter en chaise le
degré du palais jusqu'à la porte de la première pièce extérieurement. Je
ne m'étends point sur le conseil d'État, parce qu'il tomba fort peu
après l'arrivée du roi, et qu'il est demeuré depuis en désuétude. Il a
fait rarement des conseillers d'État, mais toujours sans fonctions.

Je parlerai avec la même sobriété du secrétaire des dépêches
universelles par la même raison. Ubilla a été le dernier, et ne le
demeura pas longtemps. C'était presque nos quatre secrétaires d'État
ensemble pour le crédit et les fonctions\footnote{Voy., sur les
  secrétaires d'État, t. II. p.~43, note.}, mais non pas pour le reste.
Il était demeuré pour l'extérieur comme nos secrétaires d'État
d'autrefois et comme eux venu par les emplois de commis dans les
bureaux, ce qui peut faire juger de leur naissance et de leur état. Au
conseil d'État, ils étaient au bas bout de la table auprès de leur
écritoire, rapportant les affaires, lisant les dépêches, écrivant ce qui
leur était dicté, sans opiner et toujours à genoux sur un petit carreau
qui leur fut accordé à la fin, à cause de la longueur des conseils, et
tête à tête avec le roi de même. Ils étaient fort craints et considérés,
mais ils n'allaient point avec la noblesse même ordinaire. De six qu'ils
sont des débris de celui des dépêches universelles, j'en ai vu deux,
celui qui travaillait toujours avec le roi et celui de la guerre qui n'y
travaillait guère, et jamais ne le suivait en aucun voyage hors Madrid,
qui tâchaient de se mettre sur le pied de nos secrétaires d'État
d'aujourd'hui, surtout le premier, qui était Grimaldo, quoique venu des
bureaux comme les autres, et Castellar, qui est mort ici depuis
ambassadeur, frère de Patino, alors premier ministre.

Passons maintenant à la cour, et voyons-en les principaux emplois et
même quelques médiocres, pour l'intelligence de ce qui suivra et pour ne
plus interrompre un récit plus intéressant. Il y en a trois qui
répondent ici au grand maître, au grand chambellan et au grand écuyer,
qu'on appelle tout court les trois charges, parce qu'elles sont à peu
près égales entre elles, et sans proportion avec toutes les autres. Ce
sont toujours trois grands, à qui elles donnent une grande distinction
sur tous les autres et une considération principale par toute l'Espagne.
Il est pourtant arrivé, quoique extrêmement rarement, que quelqu'une de
ces charges, tantôt l'une, tantôt l'autre, ont été possédées par de très
grands seigneurs qui n'étaient pas grands, mais favoris ou fort
distingués, et qui sont bientôt devenus grands d'Espagne. Expliquons-les
pour les faire connaître.

Le majordome-major du roi est notre grand maître de France dans toute
l'étendue, qu'il avait autrefois. Tous les palais du roi, tous les
meubles, toutes les provisions, de quelque espèce qu'elles soient, la
bouche et toutes les tables, la réception, la conduite et le traitement
des ambassadeurs et des autres personnes distinguées à qui le roi en
fait, l'ordre, l'ordonnance, la disposition de toutes les fêtes que le
roi donne, de tous les spectacles, de tous les festins et
rafraîchissements, la distribution des places, l'autorité sur les
acteurs de récit, de machines, de musiques, les mascarades publiques et
particulières du palais, l'autorité, la disposition, les places de
toutes les cérémonies, la disposition de tous les logements pendant les
voyages et de toutes les voitures de la cour, l'autorité sur les
médecins, chirurgiens et apothicaires du roi, qui ne peuvent consulter
ni donner aucun remède au roi que de son approbation et en sa présence,
tout cela est de la charge du majordome-major qui a sous lui quatre
majordomes, tous quatre de la première qualité, qui de là passent
souvent aux premières charges et arrivent à la grandesse, mais ne
peuvent être grands tandis qu'ils sont majordomes. Ils font le détail
chacun par semaine de tout ce que je viens de remarquer, sous les ordres
du majordome-major qui fait et arrête les comptes des fournitures avec
tous quatre et les gens qui ont fourni qui sont payés sur ses
ordonnances. Le majordome de semaine ne sort presque point du palais, et
tous quatre rendent compte de tout au majordome-major, et ne peuvent
s'absenter qu'avec sa permission. Ils ont des maîtres d'hôtel et toutes
sortes d'autres officiers sous eux.

Le majordome-major a toutes les entrées chez le roi à toutes les heures.
Grand d'Espagne ou non, comme il est quelquefois arrivé, quoique fort
rarement, il est grand par sa charge, et le premier d'entre les grands
partout où ils se trouvent. À la chapelle il a un siège ployant à la
tête de leur banc, qui demeure vide quand il n'y vient pas\,; et je l'ai
vu arriver. Si les grands ont pour leur dignité quelque assemblée à
faire, c'est chez lui, et quelque représentation à porter au roi, c'est
par lui. Au bal et à la comédie nul homme ne s'assied, non pas même les
danseurs, excepté le majordome-major qui est assis sur un ployant à la
droite du fauteuil du roi, un demi-pied au plus en arrière, mais
joignant sa chaise. Je l'ai vu ainsi à l'un et à l'autre, et couvert si
le roi se couvre. Aux audiences qui se donnent sur le trône aux
ambassadeurs des princes hors l'Europe, le roi est assis dans un
fauteuil sur une estrade de plusieurs degrés couverte d'un tapis avec un
dais par-dessus. On met un ployant à la droite du fauteuil du roi en
même plain-pied sur l'estrade et en même ligne, mais hors du dais. Le
roi monte sur l'estrade seul avec le majordome-major qui s'assait sur ce
ployant en même temps que le roi se place dans son fauteuil, et il se
couvre en même temps que lui. Tous les grands couverts et tous autres
découverts sont au bas des marches et debout, et l'ambassadeur aussi, et
en tous actes de cérémonies, il est joignant le roi à sa droite. Il ne
va pourtant jamais dans les carrosses du roi, parce que c'est au grand
écuyer à y prendre la première place, ni dans ceux de la reine pour même
raison, ni aux audiences chez la reine où son majordome-major prendrait
aussi la première place. Comme celui du roi l'a sans difficulté partout
ailleurs, il s'abstient toujours des trois seuls endroits où il ne
l'aurait pas.

Il ne prête serment entre les mains de personne\,; les quatre
majordomes, l'introducteur des ambassadeurs, tous les officiers qui sont
sous eux (et il y en a un grand nombre), et toute la médecine, chirurgie
et apothicairerie du roi, prêtent serment entre ses mains. Outre ceux-là
qui sont sous sa charge, il reçoit de même le serment du grand
chambellan ou sommelier du corps, du grand écuyer et du patriarche des
Indes. Les chefs et les membres des conseils et des tribunaux, et les
secrétaires d'État, le prêtent entre les mains du président ou du
gouverneur du conseil de Castille, et le roi n'en reçoit aucun lui-même,
ce qui fait que le majordome-major n'en prête point. Pour en revenir à
nos idées, on voit que cette charge est en beaucoup plus grand ce
qu'était autrefois le grand maître de la maison du roi\footnote{Le grand
  maître de la maison du roi avait primitivement une partie des
  attributions qui avaient appartenu au grand sénéchal jusque vers la
  fin du XIIe siècle, entre autres, droit de juridiction sur tous les
  officiers de la maison du roi. Mais, au XVIIe siècle, il ne lui
  restait que la surveillance du service des maîtres d'hôtel. On
  trouvera, dans le \emph{Traité des Offices}, de Guyot (t. Ier,
  p.~464), un règlement du 7 janvier 1681 qui détermine les attributions
  du grand maître de la maison du roi.}, qui depuis les Guise n'ont plus
rien à la bouche\footnote{On appelait la \emph{bouche du roi}, ou
  simplement la \emph{bouche}, tons les officiers chargés du service de
  la table du roi, maîtres d'hôtel, contrôleurs, etc.}, dont le premier
maître d'hôtel est maître indépendant, et qu'il n'a plus que le serment
de cette charge, de celle de grand maréchal des logis, de grand maître
des cérémonies et d'introducteur des ambassadeurs, sans avoir conservé
rien du tout dans l'exercice de ces charges, qui avec tout leur détail
sont entièrement subordonnées, et en tout dépendantes en Espagne du
majordome-major, et toutes exercées sous lui par le majordome de
semaine. Le majordome-major les réprimande très bien, et change ce
qu'ils ont fait quand il le juge à propos.

Le grand chambellan ou sommelier du corps est en tout et partout à la
fois ce que sont ici le grand chambellan, les quatre premiers
gentilshommes de la chambre\footnote{Les attributions des quatre
  premiers gentilshommes de la chambre, qui servaient à tour de rôle,
  par année, consistaient à recevoir le serment de fidélité de tous les
  officiers de la chambre, à donner les ordres aux huissiers pour les
  personnes qu'ils pourraient admettre aux audiences du roi, à régler
  toutes les dépenses des menus plaisirs, etc. Dans la suite, les
  comédiens français et italiens furent placés sous la surveillance des
  premiers gentilshommes de la chambre, et on leur confia aussi la
  direction des réjouissances publiques.}, le grand maître\footnote{La
  charge de grand maître de la garde-robe fut créée en 1669 par Louis
  XIV. Les détails des fonctions de ce dignitaire caractérisent
  l'étiquette de cette époque\,: il avait la surveillance des vêtements
  du roi. Lorsque le roi s'habillait, il lui mettait la camisole, le
  cordon bleu et le justaucorps. Quand le roi se déshabillait, le grand
  maître lui présentait la camisole de nuit et lui demandait ses ordres
  pour le costume du lendemain. Les jours de cérémonie, il mettait au
  roi le manteau et le collier de l'ordre du Saint-Esprit. C'était le
  grand maître de la garde-robe qui faisait faire les vêtements
  ordinaires du roi\,; mais c'était aux premiers gentilshommes de la
  chambre qu'appartenait d'ordonner le premier vêtement de chaque deuil
  et les vêtements extraordinaires pour les bals, mascarades et autres
  divertissements.} et les deux maîtres de la garde-robe réunis en une
seule charge. Les mêmes fonctions, le même commandement, le même détail,
et ordonnateur des mêmes dépenses. Il a sous lui un nombre indéfini de
gentilshommes de la chambre, tant qu'il plaît au roi d'en faire, qui ont
son service en son absence, et qui sont grands d'Espagne presque tous et
la plupart aussi grands ou plus grands seigneurs que lui. Car c'est le
but de tous les seigneurs de la cour. La différence est que le sommelier
couche au palais, et qu'il entre chez le roi comme le majordome-major,
toutes heures, au lieu que le gentilhomme de la chambre de jour, qui a
tout son service et tout son commandement dans l'appartement du roi et
sur tous les officiers de sa chambre et de sa garde-robe, ne peut entrer
qu'aux temps des fonctions et se retire dès que le service est fait. Ces
gentilshommes de la chambre prêtent serment entre les mains du
sommelier, et lui sont tellement subordonnés qu'ils ne peuvent
s'absenter sans sa permission ni rien faire dans leurs charges sans ses
ordres. Ils sont obligés de lui rendre compte de tout en son absence, et
de l'envoyer avertir quand il le leur a dit, ou sans cela dès qu'il
arrive quelque chose d'extraordinaire, s'il trouve quelque chose qu'ils
aient mal fait ou mauvais, il le change ou les réprimande très bien sans
qu'ils aient un mot à dire que se taire avec respect, quels qu'ils
soient, et lui obéir. Il a sous lui, pour le détail des habits, un
officier qui tient plus du valet que du noble, mais qui est pourtant
considéré plus que les premiers valets de garde-robe d'ici.

Le grand écuyer est là comme ici le même, avec deux grandes
différences\,: l'une, que dès que le roi est dehors, il a toutes les
fonctions du sommelier, même en sa présence. Il le sert s'il mange dans
son carrosse ou sur l'herbe, et s'il a besoin d'un surtout ou de quelque
autre chose, il le lui présente\,; et si à la chasse, à la promenade, en
chemin, quelque seigneur ait à être présenté au roi, c'est le grand
écuyer et non le sommelier qui le présente. La deuxième, est qu'il a un
premier écuyer et point de petite écurie, le premier écuyer fait sous
lui, et dans une dépendance entière et journalière, le détail de
l'écurie, et s'il se trouve présent quand le grand écuyer monte à
cheval, c'est lui qui l'y met, et toujours un écuyer du roi qui lui
tient l'étrier à monter et à descendre. Le premier écuyer le conduit à
pied, la main au mors du cheval sur lequel il est monté, depuis l'écurie
jusqu'au palais tout du long de la place, et lorsqu'en suivant le roi,
il monte dans le carrosse qui le précède ou qu'il en descend, c'est au
premier écuyer à ouvrir et à fermer la portière, comme le grand écuyer
ouvre et ferme celle du roi. Dans ce carrosse du grand écuyer, il n'y
entre que les trois charges principales du roi, les deux de la reine, et
le capitaine des gardes en quartier. Quelquefois, par un hasard
extrêmement rare, il y entrera quelque vieux grand d'Espagne, mais fort
distingué et fort considérable.

Excepté la charge de premier écuyer, le grand écuyer dispose de toute
l'écurie du roi, chevaux, mules, voitures de toute espèce, valets,
officiers, écuyers, livrées, fournitures, et est seul ordonnateur de
toutes ces dépenses. Il est en même temps le chef de toutes les chasses
avec la même autorité et dispensation que de l'écurie. Les meutes et les
chasses à courre sont inconnues en Espagne par la chaleur, l'aridité et
la rudesse du pays\,; mais tirer, voler, et des battues aux grandes
bêtes, de mille et quinze cents paysans dont le grand écuyer ordonne,
sont les chasses ordinaires, et la dernière celle du roi Philippe V de
presque tous les jours. Avec cela il y quatre ou cinq petites maisons de
chasse, la vaste capitainerie de l'Escurial et quelques autres moindres
attachées à la charge du grand écuyer. C'est le seul seigneur sans
exception qui aille dans Madrid à six mules ou à six chevaux, et à huit
s'il veut, avec un postillon, parce que c'est un carrosse et un attelage
du roi. S'il mène quelqu'un avec lui, qui que ce pût être, il n'est pas
permis au grand écuyer de le faire monter devant lui ni de lui donner la
droite, et cela n'en retient personne ni ne fait aucune difficulté pour
aller avec lui faire des visites ou à la promenade. Le duc del Arco,
dont j'aurai lieu de parler\,; qui l'était pendant mon ambassade, fut le
parrain de mon second fils pour sa couverture de grand d'Espagne. Il
vint donc le prendre en grande cérémonie pour le mener au palais, mais
par politesse, et pour lui pouvoir donner la place et la main, il vint
avec son carrosse et ses livrées à lui, et rien de l'écurie. Il tient
une table où, comme partout ailleurs, il est servi par les pages du roi,
qui font à son égard et toujours tout ce que feraient les siens. Chez
lui encore ils servent tous ceux qui mangent à sa table comme s'ils
étaient à eux, mais aussi ceux qui servirent hier se mettent aujourd'hui
à table, et mangent de droit avec le grand écuyer et avec tous ceux qui
mangent chez lui, et ainsi de suite tous les jours. Le premier écuyer
tient la petite table quand il y en a une, et fait les honneurs chez le
grand écuyer. En son absence il a toutes ses fonctions, mais il n'ôte en
dehors le service qu'aux gentilshommes de la chambre et non au
sommelier\,; il ne va point à six chevaux ou mules par Madrid, ne monte
point à la suite du roi dans le carrosse marqué pour le grand écuyer, et
n'est point servi par les pages du roi qu'à table seulement chez le
grand écuyer comme tous ceux qui y mangent. Il suit le roi dans une
voiture à part ou à cheval.

Le capitaine des hallebardiers ne peut être mieux comparé lui et sa
compagnie en tout et pour tout qu'aux Cent-Suisses de la garde du
roi\footnote{Les Cent-Suisses étaient une ancienne compagnie des gardes
  du roi, qui étaient armés de hallebardes et choisis parmi les hommes
  de la plus haute taille.} et à leur capitaine\,; c'est une ancienne
garde des rois d'Espagne.

Je parlerai en son temps des capitaines des gardes du corps que Philippe
V a établis, et qui avec leurs compagnies étaient avant lui inconnus en
Espagne, ainsi que des deux colonels de ses régiments des gardes qui
sont aussi de son établissement.

Le patriarche des Indes n'a pas seulement la plus légère idée qui ait la
moindre conformité à ce grand titre. Il ne peut rien aux Indes, il n'en
touche rien, il n'en prétend même rien, il y est inconnu. C'est un
évêque sacré \emph{in partibus}, dont la fonction est d'être toujours à
la cour pour y suppléer à l'absence de l'archevêque de Saint-Jacques de
Compostelle qui n'y paraît jamais, non plus que tous les autres évêques
d'Espagne qui résident continuellement. Celui-là est grand aumônier né
par son siège, et cette place de grand aumônier enferme tout ce que nous
connaissons ici sous les noms de grand aumônier \footnote{La charge de
  grand aumônier était considérée comme une des plus importantes du
  royaume\,; il n'avait pas seulement la direction de la chapelle
  royale, mais la surveillance des hôpitaux, la nomination des
  professeurs du collège royal (plus tard Collège de France), la
  disposition d'une partie des bourses dans les collèges de
  Louis-le-Grand, de Navarre, de Sainte-Barbe, etc. Outre le grand
  aumônier et le premier aumônier, il y avait huit aumôniers qui
  servaient par quartier\,: ils devaient se trouver au lever et au
  coucher du roi et à tous les offices de l'église auxquels il
  assistait. Ils présentaient l'eau bénite au roi, et, pendant le
  service divin, tenaient ses gants et son chapeau\,; aux repas du roi,
  ils bénissaient les viandes et disaient les grâces. Le maître de
  chapelle avait sous sa direction la chapelle-musique, et le maître de
  l'oratoire les huit chapelains et le clergé inférieur. Voy., pour les
  détails, le \emph{Traité des Offices}, de Guyot.}, premier aumônier,
maître de la chapelle\,; et maître de l'oratoire. Ce prélat devient
presque toujours cardinal, s'il ne l'est déjà quand on lui donne la
charge. Si par un hasard qui est arrivé quelquefois, l'archevêque de
Compostelle venait à la cour, il effacerait le patriarche des Indes,
qui, même cardinal, ne serait plus rien en sa présence.

Comme il n'y vient jamais, le patriarche dispose de tout ce qui est de
la chapelle, et les sommeliers de cortine, qui sont les aumôniers du
roi, et fort souvent gens de la première qualité, sont sous lui et dans
son absolue dépendance. Il y a en Espagne la même dispute qu'ici sur
l'indépendance de la chapelle du roi du diocésain, qui empêche
l'archevêque de Tolède de se trouver à la chapelle, où il ne veut pas
aller sans y faire porter sa croix que le patriarche des Indes n'y veut
pas souffrir\,; et sur les autres prétentions d'exemption, ils se
chamaillent toujours, et, chacun en tire à soi quelque chose.

La reine d'Espagne, outre ses dames, a aussi deux grands officiers, son
majordome-major et son grand écuyer\,; mais elle n'a point de chapelle,
de chancelier, ni les autres officiers qu'ont ici nos reines. Son
majordome-major a dans la maison de la reine toutes les mêmes choses que
celui du roi a chez lui, et trois majordomes sous ses ordres, mais
ceux-là sont d'une condition et d'une considération fort inférieure à
ceux du roi qui ont les détails des fêtes, des spectacles, des
cérémonies de toutes les sortes\,; et des logements, tandis que ceux de
la reine sont bornés aux détails intérieurs de sa maison sous son
majordome-major. Celui-ci reçoit leur serment, ceux des autres officiers
inférieurs qui sont sous sa charge et ceux encore du grand écuyer de la
reine et de la camarera-mayor, et comme celui du roi n'en prête point.
Il partage en premier avec la camarera-mayor le commandement chez la
reine, même aux officiers extérieurs de sa chambre. Les meubles se font
et se tendent par ses ordres, et, hors les habits et l'écurie, il est
ordonnateur de toutes les dépenses qui se font chez elle. Il est placé
derrière elle partout, à la droite de la camarera-mayor, et à certains
services comme de présenter à la reine ses gants, son éventail, son
manchon, sa mantille, quand la camarera-mayor n'y est pas, et lui met
même sa mantille en présence de ses autres dames. Il ne laisse pas
d'être fort considéré, quoiqu'il n'ait rien hors de chez la reine, et
n'ait aucune distinction parmi les grands, comme à celui du roi
seulement. Il prend aux audiences de la reine la première place
au-dessus d'eux, comme fait celui du roi chez le roi à qui il ne la cède
pas chez la reine, et ne se trouve jamais aux audiences chez le roi,
comme celui du roi ne va jamais à celles de la reine\,; mais il va parmi
les grands à la chapelle et partout ailleurs avec eux. Il est au-dessus
de la camarera-mayor, même dans l'appartement de la reine, y a plus
d'autorité qu'elle, et entre chez la reine à toutes heures, même quand
elle est au lit ou qu'elle se lève ou se couche. Cet emploi n'est que
pour les grands ainsi que celui de grand écuyer de la reine qui a sous
lui un premier écuyer, dont il reçoit le serment, et il est chez elle
entièrement comme est le grand écuyer du roi chez lui, mais il n'ôte le
service à personne au dehors, comme fait celui du roi, et ne va point à
six chevaux ou à six mules dans Madrid, quoiqu'il se serve des équipages
de la reine. Il y a un carrosse de la reine où il n'entre que lui et son
majordome-major à sa suite, et très rarement quelquefois quelque grand
d'Espagne très distingué à qui le grand écuyer en fera l'honnêteté. Il y
prend, comme celui du roi, la première place.

La camarera-mayor rassemble les fonctions de notre surintendante, de
notre dame d'honneur et de notre dame d'atours. C'est toujours une
grande d'Espagne, veuve, et ordinairement vieille, et presque toujours
de la première distinction. Elle loge au palais, elle présente les
personnes de qualité à la reine, elle entre chez elle à toute heure, et
elle partage le commandement de la chambre avec le majordome-major. Sa
charge répond en tout à celle du sommelier du corps. Elle ordonne des
habits et des dépenses personnelles de la reine, qu'elle ne doit jamais
quitter, mais la suivre partout où elle va.

Elle entre presque toujours seule, mais de droit est la première dans le
carrosse où est la reine quand le roi n'y est pas, et ce n'est que par
grande faveur et distinction si, très rarement, quelque autre grande
d'Espagne y est appelée. Les bas officiers de la chambre la servent en
beaucoup de choses, même chez elle, et elle use de beaucoup de
provisions de sa maison. Son appartement au palais est aussi meublé
{[}que celui{]} de la reine. Le concert doit être entier entre elle et
le majordome-major, et y est presque toujours, sans quoi il y aurait
lieu à beaucoup de disputes et de prétentions l'un sur l'autre.

La reine, après la camarera-mayor, a deux sortes de dames au nom
desquelles il serait aisé de se méprendre lourdement selon nos idées.
Les premières sont précisément nos dames du palais, mais qui ont un
service\,; les autres sont appelées \emph{señoras de honor}, dames
d'honneur. Les dames du palais, et qui en ont le non comme les nôtres,
sont des femmes de grands d'Espagne, ou leurs belles-filles aînées, ou
des héritières de grands et qui mariées feront leurs maris grands, et de
plus choisies parmi tout ce qu'il y a de plus considérable. Les dames
d'honneur sont des dames d'un étage très inférieur, et cette place ne
convient pas aux personnes d'une qualité un peu distinguée. Les unes et
les autres servent par semaine, suivent la reine partout, sont de garde
à certains temps dans son appartement, et toutes également dans la même
dépendance de la camarera-mayor, pour ne rien répéter, que les
gentilshommes de la chambre sont du sommelier. En l'absence de la
camarera-mayor, la plus ancienne dame du palais en semaine la supplée en
tout. La camarera-mayor sert le roi et la reine quand ils mangent
ensemble chez elle, ou la reine seule, quand le roi n'y vient point, et
met un genou en terre pour leur donner à laver et à boire. Derrière elle
sont les dames du palais de semaine, et derrière celles-ci les señoras
d'honneur de semaine. Tout le service se fait par la camarera-mayor, et
lui est présenté par les dames du palais, qui le reçoivent des señoras
d'honneur\,; celles-ci le vont prendre, à la porte, des femmes de
chambre, à qui les officiers de la bouche le présentent, et cela tous
les jours. La camarera-mayor est ordonnatrice de toute la dépense de la
garde-robe de la reine.

Les femmes de chambre sont toutes personnes de condition, et au moins de
bonne noblesse. Filles toutes, elles deviennent quelquefois señoras
d'honneur en se mariant. Toutes logent au palais, ainsi que la première
femme de chambre qu'on appelle \emph{l'azafata}, laquelle est
d'ordinaire la nourrice du roi ou de la reine, et par conséquent
ordinairement très inférieure aux femmes de chambre, sur lesquelles elle
a pourtant les mêmes distinctions de services et d'honneurs, et le même
commandement que la camarera-mayor a sur les autres dames, à laquelle
l'azafata et les autres femmes de chambre sont totalement subordonnées,
et sous son autorité et commandement. Quand le roi et la reine vont en
cérémonie à Notre-Dame d'Atocha, qui est une dévotion célèbre à une
extrémité de Madrid, ou quelque autre part, marchent d'abord un ou deux
carrosses remplis de gentilshommes de la chambre, celui du grand écuyer
et du roi, celui où le roi et la reine sont seuls, celui du roi vide,
celui du grand écuyer de la reine, la camarera-mayor seule dans le sien
à elle environné de sa livrée à pied, et un écuyer à elle à cheval à sa
portière droite, un ou deux carrosses de la reine remplis de dames du
palais, magnifiques comme pour servir à la reine, un ou deux autres bien
inférieurs, mais aussi de la reine, remplis des señoras de honor, un
autre inférieur encore où est l'azafata seule, et deux autres pareils
pour les femmes de chambre. Ce crayon suffira pour donner une idée des
charges et du service de la cour d'Espagne, jusqu'à ce qu'il y ait lieu
de parler du changement que Philippe V y a fait, et des grands et des
cérémonies. J'ajouterai seulement qu'aucune charge n'est vénale dans
toute l'Espagne, et que tous les appointements en sont fort petits comme
ils étaient anciennement en France. Le majordome-major du roi, qui a,
plus du double de toutes les autres charges, n'a guère que vingt-cinq
mille livres il y en a très peu à mille pistoles, et beaucoup fort
au-dessous. Les deux majordomes-majors, les majordomes, et la
camarera-mayor, tirent, outre leurs appointements, force commodités de
leurs charges, ainsi que les deux grands écuyers et les deux premiers
écuyers. Le capitaine des hallebardiers tire aussi quelque chose de la
sienne au delà de ses appointements.

Il faut remarquer que le sommelier et les gentilshommes de la chambre
portent tous une grande clef, qui sort par le manche de la couture de la
patte de leur poche droite\,; le cercle de cette clef est ridiculement
large et oblong. Il est doré, et est encore rattaché à la boutonnière du
coin de la poche, avec un ruban qui voltige, de couleur indifférente.
Les valets intérieurs qui sont en très petit nombre, la portent de même,
à la différence que ce qui paraît de leur clef n'est point doré. Cette
clef ouvre toute s les portes des appartements du roi, de tous ses
palais en Espagne. Si un d'eux vient à perdre sa clef, il est obligé
d'en avertir le sommelier, qui sur-le-champ fait changer toutes les
serrures et toutes les clefs aux dépens de celui qui a perdu la sienne,
à qui il en coûte plus de dix mille écus. Cette clef se porte partout
comme je viens de l'expliquer, et tous les jours, même hors d'Espagne.
Mais parmi les gentilshommes de la chambre, il y en a de deux sortes\,:
de véritables clefs qui ouvrent et qui sont pour les gentilshommes de la
chambre en exercice\,; et des clefs qui n'en ont que la figure, qui
n'ouvrent rien, et qui s'appellent des clefs \emph{caponnes}, pour les
gentilshommes sans exercice et qui n'ont que le titre et l'extérieur de
cette distinction. Les plus grands seigneurs sont gentilshommes de la
chambre de ces deux sortes, et s'il en vaque une place en exercice, elle
est souvent donnée à un des gentilshommes de la chambre qui n'en a
point, quelquefois aussi à un seigneur qui n'est pas gentilhomme de la
chambre. Tous sont égaux, sans aucun premier entre eux, et ceux
d'exercice y entrent tour à tour suivant leur ancienneté d'exercice
entre eux.

J'ai oublié un emploi assez subalterne par la qualité de celui qui l'a
toujours successivement exercé, non pas héréditairement, mais qui est de
la plus grande confiance et importance. L'emploi, l'employé, et
l'instrument de son emploi, ont le même nom, qui ne se peut rendre en
français. Il s'appelle \emph{estampilla\,;} c'est un sceau d'acier, sur
lequel est gravée la signature du roi, mais semblable à ne la pouvoir
distinguer de la sienne. Avec une espèce d'encre d'imprimerie, ce sceau
imprime la signature du roi, et c'est l'estampilla lui-même qui y met
l'encre et qui imprime. Je l'ai vu faire à La Roche qui l'a eue en
arrivant avec le roi en Espagne, et cela se fait en un instant. Cette
invention a été trouvée pour soulager les rois d'Espagne, qui signent
une infinité de choses, et qui passeraient sans cela un quart de leurs
journées à signer. Les émoluments sont continuels, mais petits\,; et La
Roche, qui était un homme de bien, d'honneur, doux, modeste, bienfaisant
et désintéressé, l'a faite jusqu'à sa mort avec une grande fidélité et
une grande exactitude. Il était fort bien avec le roi, et généralement
aimé, estimé et considéré, et voyait chez lui les plus grands seigneurs.
Cet estampilla ne peut jamais s'absenter du lieu où est, le roi, et les
ministres le ménagent.

J'attendrai à parler des infants, infantes et de leur maison quand
l'occasion s'en présentera, parce qu'il y en a eu peu et encore moins de
maisons pour eux en Espagne, jusqu'aux enfants de Philippe V.

\hypertarget{chapitre-vii.}{%
\chapter{CHAPITRE VII.}\label{chapitre-vii.}}

1701

~

{\textsc{Changements à la cour d'Espagne à l'arrivée du roi.}}
{\textsc{- Singularité de suzeraineté et de signatures de quelques
grands d'Espagne.}} {\textsc{- Autres conseillers d'État.}} {\textsc{-
Mancera et son étrange régime.}} {\textsc{- Amirante de Castille.}}
{\textsc{- Frigilliane.}} {\textsc{- Monterey.}} {\textsc{- Tresmo.}}
{\textsc{- Fuesalida.}} {\textsc{- Montijo.}} {\textsc{- Patriarche des
Indes.}} {\textsc{- Vie du roi d'Espagne en arrivant.}} {\textsc{-
Louville en premier crédit.}} {\textsc{- Duc de Monteléon.}} {\textsc{-
Coutume en Espagne, dite la saccade du vicaire.}} {\textsc{- P.
Daubenton, jésuite, confesseur du roi d'Espagne.}} {\textsc{- Aversberg,
ambassadeur de l'empereur après Harrach, renvoyé avant l'arrivée du roi
à Madrid.}} {\textsc{- Continuation du voyage des princes.}} {\textsc{-
Folie du cardinal Le Camus sur sa dignité.}}

~

Aussitôt après que le roi d'Espagne fut arrivé à Madrid, il prit l'habit
espagnol et la golille\footnote{La golille était une espèce de collet en
  usage chez les Espagnols.}, et fit quelques changements et réformes.
D'une trentaine de gentilshommes de la chambre en exercice il les
réduisit à six, et ôta les appointements à ceux qui n'avaient jamais eu
d'exercice. Le comte de Palma, grand d'Espagne et neveu du cardinal
Portocarrero, eut la vice-royauté de Catalogne en la place du prince de
Darmstadt, qui sortit d'Espagne sans revenir à Madrid. Le duc
d'Escalona, qu'on appelait plus ordinairement le marquis de Villena,
alla relever en Sicile le duc de Veragua\,; il le fut bientôt lui-même
par le cardinal del Giudice qui vint exercer la vice-royauté par
intérim, de Rome où il était\,; et Villena s'en alla vice-roi à Naples,
d'où le duc de Medina-Celi revint à Madrid, où il fut fait président du
conseil des Indes, qu'il désirait extrêmement et qui est une place fort
lucrative. Il l'était du conseil des ordres qui fut donné au duc
d'Uzeda, quoique absent, et qui remplissait l'ambassade de Rome depuis
que Medina-Celi l'avait quittée pour aller à Naples.

Le plus grand changement fut la disgrâce du connétable de Castille. Hors
les présidences des conseils et la plupart des places dans les conseils,
rien n'est à vie en Espagne, et à la mort du roi, toutes les charges se
perdent, et le successeur confirme ou changé, comme il lui plaît, ceux
qui les ont. Le connétable était grand écuyer et gentilhomme de la
chambre en exercice. L'exercice lui fut ôté et sa charge de grand
écuyer, que le duc, de Medina-Sidonia préféra à la sienne de
majordome-major, je ne sais par quelle fantaisie, sinon que, ayant
désormais affaire à un jeune roi, il la trouva plus brillante et crut
qu'il serait souvent dehors, en voyage, à la chasse, à la guerre, où le
grand écuyer a plus beau jeu que le majordome-major. Le marquis de
Villafranca le fut en sa place\,; et par ce qu'il avait fait sur le
testament, et par son vote fameux il avait bien mérité cette grande
récompense. La duchesse d'Ossone, dont j'aurai lieu de parler, disait de
lui et de don Martin de Tolède, depuis duc d'Albe et mort ambassadeur en
France, qu'ils étaient tous deux Espagnols en chausses et en pourpoint,
l'un en vieux, l'autre en jeune. Villafranca ainsi que Villena avaient
beaucoup du caractère du duc de Montausier, mais ce dernier n'était
point espagnol pour l'habit, de sa vie il n'avait porté golille ni
l'habit espagnol. Il le disait insupportable, et partout fut toute sa
vie vêtu à la française. Cela s'appelait en Espagne à la flamande ou à
la guerrière, et presque personne ne s'habillait ainsi. Le comte de
Benavente fut conservé sommelier du corps. Il se prit d'une telle
affection pour le roi, qu'il pleurait souvent de tendresse en le
regardant.

Puisque j'y suis\footnote{Nouveau passage omis dans les précédentes
  éditions jusqu'à \emph{Comme il n'y connaissait personne} (p.~127).
  C'est le complément indispensable de ce que Saint-Simon a déjà dit du
  conseil d'État d'Espagne à l'époque de l'avènement de Philippe V.}, je
ne veux pas oublier une singularité de ces deux seigneurs et de quelques
autres d'Espagne\,: le duché de Bragance en Portugal relève du comte de
Benavente, duquel les armes sont sur la porte du château de Bragance à
la droite de celles du roi de Portugal\,: toutes deux saluées une fois
l'an en cérémonie\,; le premier salut est aux armes du comte et le
second à celles du roi.

Le duc de Medina-Celi, qui lors était sept fois grand d'Espagne, et dont
les grandesses se sont depuis plus que doublées, mais qui n'en a pas
plus de rang ni de préférence parmi les autres grands que s'il n'en
avait qu'une, ne signe jamais que \emph{El Duque Duque}, pour faire
entendre sa grandeur par ce redoublement de titre sans ajouter de nom.
Le marquis de Villena, qui est aussi duc d'Escalona, signe \emph{El
Marqués}, sans y rien ajouter\,; mais le marquis d'Astorga, qui est
Guzman et grand d'Espagne aussi, signe de même, de manière qu'il faut
connaître leur écriture pour savoir lequel c'est. Il est pourtant vrai
que le droit passe en Espagne pour être du côté de Villena, et qu'il est
comme le premier marquis d'Espagne. Le duc de Veragua signe tout court
\emph{El Almirante Duque}, à cause de son titre héréditaire d'amiral des
Indes, donné aux Colomb\footnote{On a vu plus haut, p.~91, que le duc de
  Veragua descendait par les femmes de Christophe Colomb.}.

Il faut maintenant achever les conseillers d'État. Je n'ai fait
connaître que ceux qui ont eu part au testament d'une manière ou d'une
autre. Ce caractère est le bout de l'ambition\,: il ne faut donc pas
oublier ceux qui en étaient revêtus à l'avènement de Philippe V. J'ai
déjà parlé\footnote{Voy., plus haut, p.~3, 4 et 5.} du cardinal
Portocarrero, du comte d'Oropesa, de don Manuel Arias, l'un président
exilé, l'autre gouverneur du conseil de Castille, de Mendoze, évêque de
Ségovie, exilé et grand inquisiteur, du duc de Medina-Sidonia, du
marquis de Villafranca, du comte de San-Estevan del Puerto et d'Ubilla,
secrétaire des dépêches universelles. J'ai parlé aussi du comte de
Benavente qui devint conseiller d'État pour avoir été mis comme grand
dans la junte par le testament. Reste à dire un mot de Mancera, de
l'amirante, Aguilar, Monterey, del Tresno, Fuensalida et Montijo, sur
lesquels je ne me suis pas étendu, quoique j'aie déjà dit quelque chose
de quelques-uns de ces sept derniers.

Pour retoucher le marquis de Mancera\footnote{Voy., plus haut, p.~6, 7,
  ce que nous appellerons, avec Saint-Simon, \emph{la première touche du
  portrait du marquis de Mancera}.} de la maison de Tolède, grand de
première classe et fort riche, président du conseil d'Italie, à
quatre-vingt-six ans qu'il avait lors de l'arrivée du roi, {[}il{]}
avait l'esprit aussi sain et aussi net qu'à quarante ans et la
conversation charmante, doux, sage, un peu timide, parlant cinq ou six
langues, bien et sans confusion, et la politesse et la galanterie d'un
jeune homme sensé. De ses emplois et de ses vertus j'en ai parlé
ci-devant. Mais voici une singularité bien étrange à notre genre de vie
et qui n'est pas sans exemple en Espagne\,: il y avait cinquante ans
qu'il n'avait mangé de pain, à l'arrivée du roi d'Espagne. Sa nourriture
était un verre d'eau à la glace en se levant, avec un peu de conserve de
roses et quelque temps après du chocolat. À souper, des cerises ou
d'autres fruits, ou une salade, et encore de l'eau rougie, et sans
sentir mauvais ni être incommodé d'un si étonnant régime\,; et sa femme,
fille du duc de Caminha, dont une seule fille, vivait à peu près de même
à quatre-vingts ans.

L'amirante de Castille, qui s'appelait J. Thomas Enriquez de Cabrera,
duc de Rioseco et comte de Melgar, était grand d'Espagne de la première
classe, un des plus riches et des plus grands seigneurs et le premier
d'Espagne pour la naissance, quoique bâtarde. Alphonse XI, roi de
Castille et de Léon, eut de Marie, fille d'Alphonse V, roi de Portugal,
un fils unique qui lui succéda, qui fut don Pierre le Cruel, si fameux
par ses crimes, qui révoltèrent enfin tout contre lui, qui n'eut point
de fils de la sœur du duc de Bourbon, qu'il tua et qui fut tué lui-même
en {[}1368{]} par Henri, comte de Transtamare, son frère bâtard, qui lui
succéda, et dont la couronne passa à sa postérité, Henri II, Henri III,
Jean II, père d'Isabelle, qui épousa Ferdinand d'Aragon son cousin issu
de germain paternel. Il était petit-fils de Ferdinand le Juste, second
fils de Henri, comte de Transtamare, qui fut roi après avoir tué Pierre
le Cruel, dont il était frère bâtard, comme je viens de le dire.

Ce Ferdinand, père du Catholique, fut appelé le Juste, pour avoir
opiniâtrement refusé la couronne de Castille, qui lui fut plus
qu'offerte à la mort du roi Henri III son frère, qui ne laissa qu'un
fils en très bas âge, dont son oncle fut le défenseur et le tuteur, et
qui fut père de la reine Isabelle. Il fut dès ce monde récompensé de sa
vertu par l'élection qui fut faite de lui en 1390 par Martin, roi
d'Aragon et de Valence et prince de Catalogne, frère de sa mère, mourant
sans enfants, confirmée par les états de tous ces pays, pour lui
succéder Alphonse V et Jean II, ses deux fils, l'aîné sans enfants,
régnèrent l'un après l'autre, et, Ferdinand II, fils de Jean II, lui
succéda, et réunit toutes les Espagnes, excepté le Portugal, par son
mariage avec Isabelle, reine de Castille, si connus sous le nom de rois
catholiques, dont la fille, héritière de leurs couronnes, fut mère de
l'empereur Charles V et de l'empereur Ferdinand Ier, desquels sont
sorties les branches d'Espagne et impériale de la maison d'Autriche.

Alphonse II, roi de Castille, père de Pierre le Cruel, eut d'Éléonore de
Guzman, sa maîtresse, deux bâtards jumeaux. L'un fut ce comte de
Transtamare qui vainquit, tua et succéda à Pierre le Cruel, et fut de
père en fils bisaïeul d'Isabelle, reine de Castille et de Ferdinand le
Catholique, roi d'Aragon, son mari\,; l'autre jumeau fut la tige d'où
est sortie légitimement et masculinement cette suite d'amirantes de
Castille. Il s'appelait Frédéric. Son fils Pierre, comte de Transtamare
comme lui, fut connétable de Castille, dont les enfants n'en eurent
point. Mais Alphonse, son frère, leur succéda\,; il fut le premier
amirante de Castille de sa maison, à laquelle il donna pour lui et pour
sa postérité le nom de Enriquez en mémoire du roi de Castille, Henri II,
frère de son père, laquelle est directe (dont l'amirante, qui fait le
sujet de cette dissertation, est la dixième génération), n'a presque été
connue que par le nom d'amirante, parce qu'ils l'ont tous été et que
cette charge, dont je parlerai ailleurs, leur était devenue héréditaire.
Le deuxième amirante fut premier comte de Melgar. Il maria sa fille à
Jean, roi d'Aragon, fils du Juste, et elle fut mère du roi Ferdinand le
Catholique, mari d'Isabelle, reine de Castille. Le quatrième amirante
{[}était{]} fils du frère de cette reine d'Aragon et cousin germain de
Ferdinand le Catholique, outre qu'ils étaient de même maison et issus de
germain, de mâle en mâle, des rois de Castille, père d'Isabelle, et
d'Aragon, père de Ferdinand le Catholique, lesquels deux rois étaient
fils des deux frères, cousins germains de `son père\,; et cette parenté
ainsi rapprochée était d'autant plus illustre que les Enriquez n'avaient
que la même bâtardise du comte de Transtamare devenu roi de Castille,
père de ces rois, et frère jumeau de Frédéric, tige dés Enriquez.

Le cinquième amirante, cousin issu de germain de Charles-Quint, fut fait
par lui duc de Rioseco et grand d'Espagne. Celui-ci, que je compte le
cinquième, parce qu'il eut un frère aîné amirante, qui n'eut point
d'enfants et fut chevalier de la Toison d'or. Le sixième épousa A. de
Cabrera, et la postérité de ce mariage joignit depuis le nom de Cabrera
à celui d'Enriquez. Le septième et le huitième eurent la Toison d'or. Le
neuvième fut vice-roi de Sicile, et le dixième eut d'une Ponce de Léon
l'amirante dont je vais parler, qui est l'onzième amirante, lé sixième
duc de Rioseco, grand d'Espagne de la première classe, et la dixième
génération de Frédéric, frère jumeau du comte de Transtamare, qui
détrôna et tua Pierre le Cruel, dont il était frère bâtard, fut roi de
Castille en sa place et en transmit la couronne à sa postérité. Le père
de notre amirante mourut en 1680.

Notre amirante de Castille avait, en premières noces, épousé la fille du
duc de Medina-Celi, ambassadeur à Rome, puis vice-roi de Naples, où nous
avons dit qu'il fut relevé par le marquis de Villena, pour revenir à
Madrid, où Philippe V le fit président du conseil des Indes. Il n'eut
point d'enfants d'aucune\,; mais le comte d'Alcanizès, son frère, eut un
fils.

Cet amirante, homme de cinquante-cinq ans à l'avènement du roi
d'Espagne, était un composé fort extraordinaire\,: de l'esprit
infiniment, de la politesse, l'air et les manières aimables, obligeant,
insinuant, caressant, curieux, prenant toutes sortes de formes pour
plaire, haut, libre, ambitieux à l'excès, et très dangereux sans son
extrême paresse de corps qui n'influait point sur l'esprit. Pour donner
un trait de sa hauteur, le cardinal Portocarrero, qui le haïssait fort,
eut le crédit de le faire exiler à Grenade, quoique intimement attaché à
la reine qui dominait alors, et que lui-même fût en grande autorité
auprès de Charles II pendant beaucoup d'années. Il avait eu une affaire
avec le comte de Cifuentès, dont il s'était mal tiré, et s'était perdu
d'honneur\,; ce qui fut l'occasion de son exil. En y allant, il s'arrêta
à Tolède, dont le cardinal était archevêque et y donna une superbe fête
de taureaux. À Grenade, il se logea dans l'Alhambra, qui est le palais
des rois, où, après avoir demeuré assez longtemps, il se mit dans la
ville pour être plus commodément. Déshonoré sur le courage, il ne
l'était pas moins sur la probité. Personne ne se fiait à lui, et il en
riait le premier, et avec cela fort haï du peuple. Il ne se souciait ni
de sa maison ni d'avoir des enfants, mais avait la rage de gouverner et
une haine mortelle contre tous les gens qui gouvernaient, et par cette
seule raison. Ami intime du prince de Vaudemont, extrêmement faits l'un
pour l'autre, ennemi déclaré du duc de Medina-Sidonia et de tous les
Guzman, et passionné pour les jésuites, dont il avait toujours quatre
chez lui, sans lesquels il ne mangeait point, ni ne faisait aucune
chose. Il avait dans Madrid quatre palais, tous quatre superbes et
superbement meublés, d'une étendue très vaste, que par grandeur il ne
louait point, et logeait dans chacun par saison trois mois de l'année.
Ce sont presque les seuls de Madrid où j'aie vu cour et jardin, et les
plus grands qu'il y ait. C'était un personnage, malgré de tels défauts,
très considérable, le plus grand seigneur d'Espagne, et, quoique fort
laid, {[}il{]} avait le plus grand air. Il fut pourtant la dupe du
testament, et, avec tout son attachement à la reine et à la maison
d'Autriche, il n'osa proférer un seul mot. Nous le reverrons bientôt sur
la scène.

Le comte de Frigilliane (don Roderic Manrique de Lara), devenu grand
d'Espagne par son mariage avec Marie d'Avellano, comtesse d'Aguilar,
héritière, s'appelait le comte d'Aguilar, et quoique veuf et que le
comte d'Aguilar son fils fût grand d'Espagne, il continuait d'en avoir
le rang et les honneurs, qui ne se perdent point en Espagne quand on les
a eus, et de porter ainsi que son fils le nom de comte d'Aguilar,
quoique le plus souvent on l'appelât Frigilliane. C'était un grand
seigneur, haut, fier, ardent, libre, à mots cruels, dangereux,
extrêmement méchant avec infiniment d'esprit et de capacité. Il était
accusé d'avoir empoisonné dans une tabatière le père du duc d'Ossone. Il
était fort autrichien et fort attaché à la reine. Le cardinal
Portocarrero et lui se haïssaient à mort. Aussi le testament fut-il pour
lui un mystère impénétrable. Il plaisantait le premier de sa laideur,
qui était extrême, et de sa méchanceté, et disait que son fils était
dans l'aime ce que lui portait sur le visage, et avouait que sans son
fils il serait le plus méchant homme d'Espagne, et je pense qu'il avait
raison.

Le comte de Monterey, grand d'Espagne par sa mère, second fils du
célébré don Louis de Haro, avec lequel le cardinal Mazarin conclut la
paix des Pyrénées et le mariage du roi en 1660 dans l'île des Faisans de
la petite rivière de Bidassoa. Il avait été gouverneur des Pays-Bas, et
était lors président du conseil de Flandre. C'était un génie supérieur
en tout\,; mais haut, méchant et dangereux. Quoiqu'on lui eût caché le
testament, il parut s'attacher au roi, quoique grand ennemi du cardinal
Portocarrero. Qu'eût dit son père s'il eût vu ce que {[}son fils{]}
voyait, avec toutes ses précautions pour les renonciations de notre
reine Marie-Thérèse\,? Monterey n'avait point d'enfants.

Le marquis del Tresno, grand d'Espagne de la maison de Velasco, comme
connétable de Castille, était homme de beaucoup de probité et de
capacité. Le comte de Fuensalida et le comte de Montijo, aussi grands
d'Espagne et conseillers d'État. Ce dernier n'a eu qu'une fille qui a
épousé un Acuña Pacheco, qui a joint le nom de Portocarrero de sa mère,
dont il a eu la grandesse. Il a fait fortune par l'ambassade
d'Angleterre et les grands emplois. Le conseiller d'État, qui était
comme le cardinal Portocarrero, Boccanegra, était frère du patriarche
des Indes qui ne mangeait pas plus de pain que le marquis de Mancera,
mais qui était méchant, hargneux, haineux, malintentionné et pestant
toujours contre le gouvernement. Il ne savait mot de latin, quoiqu'il ne
manquât ni d'esprit ni de lecture. Sa parenté et l'amour du cardinal
Portocarrero le firent\,; malgré tout cela, confirmer dans sa charge de
patriarche des Indes. Voilà la plupart des personnages qui figuraient en
Espagne, lorsque le roi y arriva.

Comme il n'y connaissait personne, il se laissa conduire au duc
d'Harcourt et à ceux qui avaient eu la principale part au testament, qui
étaient fort liés entre eux, et avec les principaux desquels il passait
sa vie par les fonctions intimes de leurs emplois, comme le cardinal
Portocarrero qui était l'âme de tout, et les marquis de Villafranca, duc
de Medina-Sidonia, et comte de Benavente qui avaient les trois charges.
Mais comme tous ceux-là mêmes lui étaient étrangers et M. d'Harcourt
lui-même, il se dérobait volontiers pour être seul avec le peu de
François qui l'avaient suivi, entre lesquels il n'était bien accoutumé
qu'avec Valouse, son écuyer en France, et Louville qui, depuis l'âge de
sept ans, était gentilhomme de sa manche. C'était celui-là beaucoup plus
qu'aucun qui était le dépositaire de son âme. M. de Beauvilliers, qui
l'éprouvait depuis tout le temps de cette éducation, le lui avait
recommandé comme un homme sage, instruit, plein de sens, d'esprit et de
ressource, uniquement attaché à lui et digne de toute sa confiance.
Louville avait en effet tout cela, et une gaieté et des plaisanteries
salées, mais avec jugement, dont les saillies réveillaient le froid et
le sérieux naturel du roi, et lui étaient d'une grande ressource dans
les premiers temps d'arrivée en cette terre étrangère. Louville était
intimement attaché à M. de Beauvilliers, et extrêmement bien avec Torcy.
Il était leur intime et unique correspondant, et sûr de ses lettres et
de ses chiffres, parce que Torcy avait les postes. Il connaissait à fond
le roi d'Espagne, il agissait de concert avec Harcourt, Portocarrero,
Ubilla, Arias et les trois charges, et ménageait les autres seigneurs
dont il eut bientôt une cour. On voyait bien la prédilection et la
confiance du roi pour lui. Mais Harcourt étant, peu de jours après
l'arrivée, tombé dans une griève et longue maladie, tout le poids des
affaires tomba sur Louville à découvert, et pour en parler au vrai, il
gouverna le roi et l'Espagne. C'était lui qui voyait et faisait toutes
ses lettres particulières à notre cour, et par qui tout passait
directement. Il commençait à peine à connaître à demi son monde qu'il
lui tomba sur les bras la plus cruelle affaire du monde\,; pour
l'entendre il faut reprendre les intéressés de plus haut.

Le comte de San-Estevan del Puerto, grand écuyer de la reine, et qui
malgré cet attachement de charge avait tant eu de part au testament, ne
devait pas être surpris qu'elle eût préféré le connétable de Castille,
de temps attaché à elle et à la maison d'Autriche, et qu'elle avait
attaché à Harcourt pour négocier avec lui, ni que la junte qui
d'ailleurs la comptait si peu n'eût pu lui refuser l'ambassade passagère
de France pour un seigneur si distingué. Néanmoins le dépit qu'il en
conçut fut tel qu'il la quitta, et lui fit en partie déserter sa maison,
dont le connétable porta en France ces lettres de plaintes si
romanesques et si inutiles. Le duc de Monteléon, de la maison Pignatelli
comme Innocent XII, dont tous les biens étaient en Italie, fin et adroit
Napolitain, et qui voulait se tenir en panne en attendant qu'il vit d'où
viendrait le vent, saisit l'occasion, se donna à la reine, qui fut trop
heureuse d'avoir un seigneur si marqué. Il fut donc son grand écuyer,
et, faute d'autre, en même temps son majordome-major, son conseil et son
tout, et sa femme sa camarera-mayor. Ce fut ce duc que la reine envoya,
de Tolède complimenter le roi d'Espagne. Le cardinal voyait avec dépit
un homme si considérable chez la reine, tout exilée qu'elle était, et,
n'oublia rien de direct, ni d'indirect pour engager Monteléon de la
quitter\,; mais il avait affaire à un homme plus délié que lui, et qui
répondait toujours qu'il ne quitterait pas pour rien des emplois aussi
bons à user que ceux qui le retenaient à Tolède\,; mais, qu'il était
prêt à revenir si on lui donnait une récompense raisonnable. Ce n'était
pas le compte du cardinal. Il voulait isoler entièrement la reine, et
qu'elle ne trouvât au plus que des valets\,; et c'était lui procurer
quelque autre seigneur en la place de Monteléon, si on achetait
l'abandon de celui-ci, qui serait une espérance et un exemple pour le
successeur. Quelques mois se passèrent de la sorte qui allumèrent de
plus eu plus le dépit du cardinal, qui, outré de colère, résolut enfin
de se porter aux dernières extrémités contre le duc de Monteléon, et de
faire en même temps le plus sanglant outrage à la reine.

Pour entendre l'occasion qu'il en saisit, il faut savoir une coutume
d'Espagne que l'usage a tournée en loi, et qui est également folle et
terrible pour toutes les familles. Lorsqu'une fille par caprice, par
amour, ou par quelque raison que ce soit, s'est mise en tête d'épouser
un homme, quelque disproportionné qu'il soit d'elle, fut-ce le
palefrenier de son père, elle et le galant le font savoir au vicaire de
la paroisse de la fille, pourvu qu'elle ait seize ans accomplis. Le
vicaire se rend chez elle, fait venir son père, et en sa présence et de
la mère, demande à leur fille si elle persiste à vouloir épouser un tel.
Si elle répond que oui, à l'instant il l'emmène chez lui, et il y fait
venir le galant\,; là il réitère la même question à la fille devant cet
homme qu'elle veut épouser, et si elle persiste dans la même volonté, et
que lui aussi déclare la vouloir épouser, le vicaire les marie
sur-le-champ sans autre formalité, et, de plus, sans que la fille puisse
être déshéritée. C'est ce qui se peut traduire du terme espagnol la
\emph{saccade du vicaire}, qui, pour dire la vérité, n'arrive comme
jamais.

Monteléon avait sa fille dame du palais de la reine, qui voulait épouser
le marquis de Mortare, homme d'une grande naissance mais fort pauvre, à
qui le duc de Monteléon ne la voulut point donner. Mortare l'enleva et
en fut exilé. Là-dessus arriva la mort de Charles II. Cette aventure
parut au cardinal Portocarrero toute propre à satisfaire sa haine. Il se
mit donc à presser Monteléon de faire le mariage de Mortare avec sa
fille, ou de lui laisser souffrir la saccade du vicaire. Le duc tira de
longue, mais enfin serré de près avec une autorité aiguisée de
vengeance, appuyée de la force de l'usage tourné en loi et du pouvoir
alors tout-puissant du cardinal, il eut recours à Montriel, puis à
Louville à qui il exposa son embarras et sa douleur. Ce dernier n'y
trouva de remède que de lui obtenir une permission tacite de faire
enlever sa fille par d'Urse, gentilhomme des Pays-Bas, qui s'attachait
fort à Louville, et qui en eut depuis la compagnie des mousquetaires
flamands, formée sur le modèle de nos deux compagnies de mousquetaires.
Monteléon avait arrêté le mariage avec le marquis de Westerloo, riche
seigneur flamand de la maison de Mérode et chevalier de la Toison d'or,
qui s'était avancé à Bayonne, et qui sur l'incident fait par le cardinal
Portocarrero n'avait osé aller plus loin. D'Urse y conduisit la fille du
duc de Monteléon qui, en arrivant à Bayonne, y épousa le marquis de
Westerloo, et s'en alla tout de suite avec lui à Bruxelles, et le comte
d'Urse s'en revint à Madrid. Le cardinal, qui de plus en plus serrait la
mesure tant que la fuite fut arrêtée et exécutée, la sut quand le secret
en fut devenu inutile, et que Monteléon compta n'avoir plus rien à
craindre depuis que sa fille était mariée en France, et avec son mari en
chemin des Pays-Bas.

Mais il ignorait encore jusqu'à quel excès se peut porter la passion
d'un prêtre tout-puissant, qui se voit échapper d'entre les mains une
proie qu'il s'était dès longtemps ménagée. Portocarrero en furie ne se
ménagea plus, alla trouver le roi, lui rendit compte de cette affaire,
et lui demanda la permission de la poursuivre. Le roi, tout jeune et
arrivant presque, et tout neuf encore aux coutumes d'Espagne, ne pensa
jamais que cette poursuite fût autre qu'ecclésiastique, comme diocésain
de Madrid, et, sans s'informer, n'en put refuser le cardinal, qui au
partir de là sans perdre un instant, fait assembler le conseil de
Castille, de concert avec Arias, gouverneur de ce conseil et son ami, et
avec Monterey, qui s'y livra par je ne sais quel motif\,; et là, dans la
même séance, en trois heures de temps, un arrêt par lequel Monteléon fut
condamné à perdre six cent mille livres de rente en Sicile, applicables
aux dépenses de la guerre, à être lui appréhendé au corps jusque dans le
palais de la reine à Tolède, mis et lié sur un cheval, conduit ainsi
dans les prisons de l'Alhambra à Grenade, où il y avait plus de cent
lieues, et par les plus grandes chaleurs, d'y demeurer prisonnier gardé
à vue le reste de sa vie, et de plus, de représenter sa fille, et la
marier au marquis de Mortare, à faute de quoi à avoir la tête coupée et
à perdre le reste de ses biens.

D'Urse fut le premier qui eut avis de cet arrêt épouvantable. La peur
qu'il eut pour lui-même le fit courir à l'instant chez Louville. Lui qui
ne s'écartait jamais s'était ce jour-là avisé d'aller à la promenade, et
ce contretemps pensa tout perdre, parce qu'on ne le trouva que fort
tard. Louville, instruit de cet énorme arrêt, alla d'abord au roi, qui
entendait une musique, et ce fut un autre contretemps où les moments
étaient chers. Dès qu'elle fut finie, il passa avec le roi dans son
cabinet, où avec émotion il lui demanda ce qu'il venait de faire. Le roi
répondit qu'il voyait bien ce qu'il lui voulait dire, mais qu'il ne
voyait pas quel mal pouvait faire la permission qu'il avait donnée au
cardinal. Là-dessus, Louville lui apprit tout ce de quoi cette
permission venait d'être suivie, et lui représenta avec la liberté d'un
véritable serviteur combien sa jeunesse avait été surprise, et combien
cette affaire le déshonorait après la permission qu'il avait donnée de
l'enlèvement et du mariage de la fille\,; que sa bouche avait, sans le
savoir, soufflé le froid et le chaud\,; et qu'elle était cause du plus
grand des malheurs, dont il lui fit aisément sentir toutes les suites.
Le roi, ému et touché, lui demanda quel remède à un si grand mal, et
qu'il avait si peu prévu\,; et Louville, ayant fait à l'instant apporter
une écritoire, dicta au roi deux ordres bien précis\,: l'un à un
officier de partir au moment même, de courir en diligence à Tolède, pour
empêcher l'enlèvement de Monteléon, et en cas qu'il fût déjà fait, de
pousser après jusqu'à ce qu'il l'eût joint, le tirer des mains de ses
satellites, et de le ramener à Tolède chez lui\,; l'autre au cardinal,
d'aller lui-même à l'instant au lieu où se tient le conseil de Castille,
d'arracher de ses registres la feuille de cet arrêt et de la jeter au
feu, en sorte que la mémoire en fût à jamais éteinte et abolie.

L'officier courut si bien, qu'il arriva à la porte de Tolède au moment
même que l'exécuteur de l'arrêt y entrait. Il lui montra l'ordre de la
main du roi, et le renvoya de la sorte, sans passer outre. Celui qui fut
porter l'autre ordre du roi au cardinal le trouva déjà couché, et
quoique personne n'entrât jamais chez lui dès qu'il était retiré, au nom
du roi toutes les portes tombèrent. Le cardinal lut l'ordre de la main
du roi, se leva et s'habilla, et fut tout de suite l'exécuter, sans
jamais proférer une parole. Il n'y a au monde qu'un Espagnol capable de
ce flegme apparent, dans l'extrême fureur ou ce contrecoup le devait
faire entrer. Avec la même gravité et la même tranquillité, il parut le
lendemain matin à son ordinaire chez le roi, qui, dès qu'il l'aperçut,
lui demanda s'il avait exécuté son ordre. \emph{Sí, señor}, répondit le
cardinal, et ce monosyllabe fut le seul qu'on ait ouï sortir de sa
bouche, sur une affaire pour lui si mortellement piquante, et qui lui
dérobait sa vengeance et la montre de son pouvoir. Arias et lui en
boudèrent huit jours Louville, mais {[}ils ne{]} s'en sont jamais parlé
en sorte du monde. Lui avec eux, quoiqu'un peu retenu, ne fit pas
semblant de rien, puis se rapprochèrent à l'ordinaire\,: ces deux
puissants Espagnols ne voulaient pas demeurer brouillés avec lui, ni lui
aussi sortir avec eux du respect, de la modestie, et de la privance qui
était nécessaire qu'il se conservât avec eux, et qu'ils avaient pour le
moins autant de désir de ne pas altérer.

Harcourt, qui avait été à l'extrémité à plusieurs reprises, était lors
encore fort mal à la Sarzuela, petite maison de plaisance des rois
d'Espagne dans le voisinage de Madrid, et entièrement hors d'état d'ouïr
parler d'aucune affaire. Celle-ci néanmoins parut à Louville si
importante, qu'il alla dès le lendemain lui en rendre compte. Harcourt
approuva non seulement la conduite de Louville, mais il trouva qu'il
avait rendu au roi le plus important service. Il dépêcha là-dessus un
courrier qui rapporta les mêmes louanges à Louville. Monteléon cependant
accourut se jeter aux pieds du roi, et remercier son libérateur de lui
avoir sauvé l'honneur, les biens et la vie\,; mais Louville se défendit
toujours prudemment d'une chose dont il voulait que le roi eût tout
l'honneur, et dont l'aveu l'eût trop exposé au cardinal\,; mais toute la
cour, et bientôt toute l'Espagne, ne s'y méprit pas, et ne l'en aima et
estima que davantage.

Avant de sortir d'Espagne, il faut dire un mot du P. Daubenton, jésuite
Français, qui y suivit le roi pour être son confesseur. Ce fut au grand
regret des dominicains, en possession de tout temps du confessionnal des
rois d'Espagne, appuyés de l'inquisition, chez lesquels, comme partout
ailleurs où elle est établie, ils tenaient le haut bout, et soutenus de
toute la maison de Guzman, une des plus grandes d'Espagne, de laquelle
étaient plusieurs grands, et plusieurs grands seigneurs, qui, tous se
faisaient un grand honneur de porter le même nom que saint Dominique. Le
crédit des jésuites fit que le roi ne balança pas d'en donner un pour
confesseur au roi son petit-fils, bien que persuadé que ce choix n'était
pas politique. On se figurait l'autorité des dominicains tout autre
qu'elle était en Espagne. Il se trouva qu'avec tout ce qui la leur
devait donner principale, ils y avaient moins de crédit, de
considération et d'amis puissants et nombreux que les jésuites, qui
avaient su les miner et s'établir à leurs dépens. L'Espagne fourmillait
de leurs collèges, de leurs noviciats, de leurs maisons professes\,; et
comme ils héritent en ce pays-là comme s'ils n'étaient pas religieux,
toutes ces maisons, vastes, nombreuses, magnifiques, en tout, sont
extrêmement riches. Ce changement d'ordre du confesseur ne fit donc pas
la moindre peine, sinon à des intéressés tout à fait hors de moyens de
s'en ressentir.

Ce P. Daubenton fut admirablement bien choisi. C'était un petit homme
grasset, d'un visage ouvert et avenant, poli, respectueux avec tous ceux
dont il démêla qu'il y avait à craindre ou à espérer, attentif à tout,
de beaucoup d'esprit, et encore plus de sens, de jugement et de
conduite, appliqué surtout à bien connaître l'intrinsèque de chacun, et
à mettre tout à profit, et cachant sous des dehors retirés,
désintéressés, éloignés d'affaires et du monde, et surtout simples et
même ignorants, une finesse la plus déliée, un esprit le plus dangereux
en intrigues, une fausseté la plus innée, et une ambition démesurée
d'attirer tout à soi et de tout, gouverner. Il débuta par faire semblant
de ne vouloir se mêler de rien, de se soumettre comme sous un joug
pénible à entrer dans les sortes d'affaires qui en Espagne se renvoient
au confesseur, de ne faire que s'y prêter avec modestie et avec dégoût,
d'écarter d'abord beaucoup de choses qu'il sut bien par où reprendre, de
ne recommander ni choses ni personnes, et de refuser même son général
là-dessus. Avec cette conduite qui se pourrait mieux appeler manège, et
une ouverture et un liant jusqu'avec les moindres, qui le faisait passer
pour aimer à obliger, et qui faisait regretter qu'il ne se voulût pas
mêler, il fit une foule de dupes, il gagna beaucoup d'amis, et quoique
ses progrès fussent bientôt aperçus auprès du roi d'Espagne et dans la
part aux affaires, il eut l'art de se maintenir longtemps dans cette
première réputation qu'il avait su s'établir. C'est un personnage avec
qui il fallut compter, et en France à la fin comme en Espagne. Nous le
retrouverons plus d'une fois.

Des autres Français, Valouse ne se mêla que de faire sa fortune, qu'il
fixa en Espagne\,; Montriel de rien, et qui revint comme il était
allé\,; La Roche de presque rien au delà de son estampilla\,: Hersent de
peu de choses, et encore de cour\,; ceux de la Faculté de rien, ni
quelques valets intérieurs ou, gens de la bouche française que
d'amasser\,; et Louville de tout et fort à découvert. Mais son règne,
très utile aux deux rois et à l'Espagne, fut trop brillant et trop court
pour leur bien.

Le comte d'Harrach, ambassadeur de l'empereur, était sur le point d'être
relevé lorsque Charles II mourut. Il partit bientôt après d'un pays qui
ne pouvait plus que lui être très désagréable, et le comte d'Aversberg
lui succéda. Mais la junte, qui dans ces circonstances le prit moins
pour un ambassadeur que pour un espion, lui conseilla doucement de se
retirer, jusqu'à ce qu'on sût à quoi l'empereur s'en tiendrait. Il
résista jusqu'à proposer de demeurer en attendant, comme particulier,
sans caractère\,; à la fin, il fut prié de ne pas attendre l'arrivée du
roi d'Espagne, et il partit\,; mais il passa par Paris, où il s'arrêta
en voyageur pour y voir les choses de plus près, et en rendre compte de
bouche plus commodément encore que Zinzendorf, envoyé ici de l'empereur,
ne pouvait faire par ses amples dépêches.

Cependant les deux princes, frères du roi d'Espagne, continuaient leur
voyage par la France, où, malgré la fâcheuse saison de l'hiver, les
provinces qu'ils parcoururent n'oublièrent rien pour les recevoir avec
les plus grands honneurs et les fêtes les plus galantes. Le Languedoc
s'y distingua, le Dauphiné fit de son mieux. Ils logèrent à Grenoble
dans l'évêché, et ils y séjournèrent quelques jours dans l'espérance de
pouvoir aller de là voir la grande Chartreuse. Mais les neiges furent
impitoyables, et quoi qu'on pût faire, elles leur en fermèrent tous les
chemins. Le cardinal Le Camus, avec tout son esprit et cette
connaissance du monde que tant d'années de résidence, sans sortir de son
diocèse que pour un conclave, n'avaient pu effacer, se surpassa dans la
réception qu'il leur fit, sans toutefois sortir de ce caractère d'évêque
pénitent et tout appliqué à ses devoirs qu'il soutenait depuis si
longtemps. Mais sa pourpre l'avait enivré au point de lui faire perdre
la tête dans tout ce qui la regardait, jusque-là qu'un homme qui avait
passé ses premières années à la cour aumônier du roi, et dans les
meilleures compagnies, avait oublié comment les cardinaux y vivaient, si
bien qu'il fut longtemps en peine, sur le point de l'arrivée des princes
chez lui, si dans sa maison même il devait leur donner la main. Ils
passèrent en Provence où Aix, Arles, et surtout Marseille et Toulon leur
donnèrent des spectacles, dont la nouveauté releva pour eux la
magnificence et la galanterie par tout ce que la marine exécuta. Avignon
se piqua de surpasser les villes du royaume par la réception qu'elle
leur fit, et Lyon couronna tous ces superbes plaisirs par où ils
finirent avec leur voyage. C'est où je les laisserai pour reprendre ce
que la digression d'Espagne m'a fait interrompre.

\hypertarget{chapitre-viii.}{%
\chapter{CHAPITRE VIII.}\label{chapitre-viii.}}

1701

~

{\textsc{M\textsuperscript{lle} de Laigle, fille d'honneur de
M\textsuperscript{me} la duchesse, à Marly\,; et mange avec
M\textsuperscript{me} la duchesse de Bourgogne.}} {\textsc{- Violente
indigestion de Monseigneur.}} {\textsc{- Capitulation.}} {\textsc{-
Grande augmentation de troupes.}} {\textsc{- Force milice.}} {\textsc{-
Électeur de Bavière à Munich\,; Ricous l'y suit.}} {\textsc{- Bedmar,
commandant général des Pays-Bas espagnols par intérim.}} {\textsc{-
Traités et fautes.}} {\textsc{- Succession à la couronne d'Angleterre
établie dans la ligne protestante.}} {\textsc{- Plaintes et droits de M.
de Savoie.}} {\textsc{- Vénitiens neutres.}} {\textsc{- Catinat général
en Italie.}} {\textsc{- Dépit et vues de Tessé\,; sa liaison avec
Vaudémont.}} {\textsc{- Boufflers général en Flandre et Villeroy en
Allemagne.}} {\textsc{- M. de Chartres refusé de servir\,; grand
mécontentement de Monsieur, qui ne s'en contraint pas avec le roi.}}
{\textsc{- Nyert revient d'Espagne.}} {\textsc{- Retours des Princes.}}
{\textsc{- La Suède reconnaît le roi d'Espagne.}} {\textsc{- Archevêques
d'Aix et de Sens nommés à l'ordre.}} {\textsc{- Traits du premier.}}
{\textsc{- Refus illustre de l'archevêque de Sens.}} {\textsc{- M. de
Metz commandeur de l'ordre.}} {\textsc{- Tallard chevalier de l'ordre,
etc.}} {\textsc{- Mort de M\textsuperscript{me} de Tallard, de la
duchesse d'Arpajon, de M\textsuperscript{me} d'Hauterive, de
M\textsuperscript{me} de Bournouville, de Segrais, du maréchal de
Tourville.}} {\textsc{- Châteaurenauld vice-amiral.}} {\textsc{- Mort du
comte de Staremberg.}} {\textsc{- L'Angleterre reconnaît le roi
d'Espagne.}} {\textsc{- Duc de Beauvilliers grand d'Espagne.}}
{\textsc{- Mariage déclaré du roi d'Espagne avec la fille du duc de
Savoie.}} {\textsc{- Égalité réglée en France et en Espagne entre les
ducs et les grands.}} {\textsc{- Abbé de Polignac rappelé.}} {\textsc{-
Duc de Popoli salue le roi, qui lui promet l'ordre.}} {\textsc{-
Banqueroute des trésoriers de l'extraordinaire des guerres.}}

~

On a vu en plusieurs endroits de ces Mémoires les distinctions que le
roi se plaisait à donner à ses filles par-dessus les autres princesses
du sang, à la différence desquelles entre autres il fit manger avec
M\textsuperscript{me} la duchesse de Bourgogne, M\textsuperscript{lle}s
de Sanzay et de Viantais, filles d'honneur de M\textsuperscript{me} la
princesse de Conti. M\textsuperscript{me} la Duchesse n'en avait plus il
y avait longtemps\,; elle en prit une cette année qui fut la fille de
M\textsuperscript{me} de Laigle, sa dame d'honneur, laquelle tout de
suite eut le même honneur que celles de M\textsuperscript{me} la
princesse de Conti sa sœur, et, comme elles, fut de tous les voyages de
Marly.

Le samedi 19 mars, veille des Rameaux, au soir, le roi étant à son
prie-Dieu, pour se déshabiller tout de suite à son ordinaire, entendit
crier dans sa chambre pleine de courtisans, et appeler Fagon et Félix
avec un grand trouble. C'était Monseigneur, qui se trouvait extrêmement
mal. Il avait passé la journée à Meudon, où il n'avait fait que
collation, et au souper du roi s'était crevé de poisson. Il était grand
mangeur, comme le roi et comme les reines ses mère et grand'mère. Il n'y
avait pas paru après le souper. Il venait de descendre chez lui du
cabinet du roi, et à son ordinaire aussi s'était mis à son prie-Dieu en
arrivant, pour se déshabiller tout de suite. Sortant de son prie-Dieu et
se mettant dans sa chaise pour se déshabiller, il perdit tout d'un coup
connaissance. Ses valets éperdus et quelques-uns des courtisans qui
étaient à son coucher accoururent chez le roi chercher le premier
médecin et le premier chirurgien du roi avec le vacarme que je viens de
dire. Le roi, tout déboutonné, se leva de son prie-Dieu à l'instant et
descendit chez Monseigneur par un petit degré noir, étroit et difficile,
qui, du fond de l'antichambre qui joignait sa chambre, descendait tout
droit dans ce qu'on appelait le Caveau, qui était un cabinet assez
obscur sur la petite cour, qui avait une porte dans la ruelle du lit de
Monseigneur et une autre qui entrait dans son premier grand cabinet sur
le jardin. Ce caveau avait un lit dans une alcôve, où il couchait
souvent l'hiver\,; mais comme c'était un fort petit lieu, il se
déshabillait et s'habillait toujours dans sa chambre.
M\textsuperscript{me} la duchesse de Bourgogne, qui ne faisait aussi que
passer chez elle, arriva en même temps que le roi, et dans un instant la
chambre de Monseigneur, qui était vaste, se trouva pleine.

Ils trouvèrent Monseigneur à demi nu que ses gens promenaient ou plutôt
traînaient par la chambre. Il ne connut ni le roi qui lui parla, ni
personne, et se défendit tant qu'il put contre Félix qui, dans cette
nécessité pressante, se hasarda de le saigner en l'air, et y réussit. La
connaissance revint\,; il demanda un confesseur\,; le roi avait déjà
envoyé chercher le curé. On lui donna force émétique, qui fut longtemps
à opérer, et qui sur les deux heures fit une évacuation prodigieuse haut
et bas. À deux heures et demie, n'y paraissant plus de danger, le roi,
qui avait répandu des larmes, s'alla coucher, laissant ordre de
l'éveiller, s'il survenait quelque accident. À cinq heures, tout l'effet
étant passé, les médecins le laissèrent reposer et firent sortir tout le
monde de sa chambre. Tout y accourut toute la nuit de Paris. Il en fut
quitte pour garder sa chambre huit ou dix jours, où le roi l'allait voir
deux fois par jour, et où, quand il fut tout à fait bien, il jouait ou
voyait jouer toute la journée. Depuis, il fut bien plus attentif à sa
santé et prit fort garde à ne se pas trop charger de nourriture. Si cet
accident l'eût pris un quart d'heure plus tard, le premier valet de
chambre qui couchait dans sa chambre l'aurait trouvé mort dans son lit.

Paris aimait Monseigneur, peut-être parce qu'il y allait souvent à
l'Opéra. Les harengères des halles imaginèrent de se signaler. Elles en
députèrent quatre de leurs plus maîtresses commères pour aller savoir
des nouvelles de Monseigneur. Il les fit entrer. Il y en eut une qui lui
sauta au collet et qui l'embrassa des deux côtés\,; les autres lui
baisèrent la main. Elles furent très bien reçues. Bontems les promena
par les appartements, et leur donna à dîner. Monseigneur leur donna de
l'argent, le roi aussi leur en envoya. Elles se piquèrent d'honneur,
elles en firent chanter un beau \emph{Te Deum} à Saint-Eustache, puis se
régalèrent.

Le roi, voyant que l'alliance unie contre lui à la dernière guerre se
rejoignait et se préparait à y rentrer contre lui, en même temps que ces
puissances essayaient de l'amuser pour se donner le temps de mettre
ordre à leurs affaires, songea aussi à s'y préparer. Il augmenta son
infanterie de cinquante mille hommes\,; il forma soixante-dix bataillons
de milice, et augmenta sa, cavalerie de seize mille et ses dragons à
proportion. Ces dépenses renouvelèrent la capitation dont l'invention
est due à Bâville, intendant ou plutôt roi de Languedoc. Elle eut lieu
pour la première fois à la fin de la dernière guerre. Pontchartrain y
avait résisté tant qu'il avait pu, comme au plus pernicieux impôt par la
facilité de l'augmenter à. volonté d'un trait de plume, l'injustice
inévitable de son imposition, à proportion des facultés de chacun
toujours ignorées, et nécessairement livrée à la volonté des intendants
des provinces, et l'appât de la rendre ordinaire, comme il est enfin
arrivé malgré les édits et les déclarations remplies des plus fortes
promesses de la faire cesser à la paix. Mais à la fin il eut la main
forcée par la nécessité des dépenses, par les persécutions de Bâville,
et par les mouvements des financiers. Celle-ci fut beaucoup plus forte
que n'avait été la première, comme sont toujours les impôts, qui vont
toujours en augmentant.

Il y avait plusieurs années que l'électeur de Bavière n'avait été chez
lui. Bruxelles lui plaisait plus que le séjour de Munich, et après avoir
passé toute la dernière guerre aux Pays-Bas dont il était gouverneur, il
y demeura encore pendant la paix. À la fin, ses affaires d'Allemagne le
pressèrent d'y retourner. Il le fit trouver bon au roi, et le pria en
même temps de lui donner quelqu'un qui fût homme de guerre pour être
témoin de ses actions, et à qui il pût communiquer les propositions de
traités qui ne manqueraient pas de lui être faites, parce qu'il voulait
que le roi et le roi d'Espagne fussent informés de tout ce qui le
regarderait, et ne rien faire que de concert avec eux. On lui envoya
Ricous. C'était un homme de beaucoup d'esprit, qui avait servi avec
valeur, ami particulier de M. et de M\textsuperscript{me} de Castries,
qui était de Languedoc et qui avait déjà eu quelques commissions en
Allemagne. Castries, fort ami de Torcy, le lui avait fait connaître, et
par lui à Croissy. Depuis que Ricous était revenu, il s'était toujours
entretenu fort bien avec Torcy, s'était fait des amis de considération,
et il était souvent à Versailles dans les bonnes maisons, où on était
bien aise de le voir. L'électeur partit donc et se fit suivre par toutes
ses troupes, et laissa le marquis de Bedmar, commandant général des
Pays-Bas espagnols, en son absence.

On fit en même temps imprimer les propositions que les Hollandais et les
Anglais avaient faites à d'Avaux dans les conférences de la Haye. Les
premiers demandaient d'avoir leurs garnisons dans une douzaine de
places, parmi lesquelles Luxembourg, Namur, Charleroi et Mons\,; et les
Anglais dans Ostende et Nieuport. Cela montrait qu'ils ne cherchaient
qu'à rompre, et la faute si lourde de leur avoir renvoyé leurs
vingt-deux bataillons. Ce n'était pas tout\,: ils ajoutaient qu'on
donnât satisfaction à l'empereur, et cela n'était pas facile à un prince
qui prétendait tout, et qu'il entrât dans leur traité. Aussi ces
conférences ne durèrent-elles pas longtemps après des propositions si
sauvages. Les Hollandais, pour gagner temps, n'oublièrent rien pour
amuser toujours\,; mais à la fin, Briord convalescent revint et d'Avaux
peu après, qui ne laissa qu'un secrétaire à la Haye, lequel même n'y
demeura pas longtemps.

Tallard aussi quitta Londres et y laissa Poussin, espèce de secrétaire
qui dans la suite fut subalternement employé et fit bien partout.
Presque en même temps, Molès, ambassadeur d'Espagne à Vienne, fut
congédié. Sous prétexte de pourvoir à ses dettes, il s'arrêta dans les
faubourgs, et fit si bien qu'il y fut arrêté contre le droit des gens,
quoiqu'il eût pris congé et dépouillé le caractère. Je dis qu'il fit si
bien qu'il y fut arrêté, parce que la suite fit juger que ç'avait été un
jeu, qui finit en tournant casaque et se donnant à l'empereur.

En même temps le roi eut nouvelle de la signature de trois traités
avantageux. Par l'un le Portugal faisait avec lui une alliance offensive
et défensive, interdisait ses ports aux Anglais et aux Hollandais, et
défendait tout commerce avec eux à ses sujets. C'était un coup de partie
que de fermer cette porte d'Espagne. Mais, faute d'argent et de troupes
à temps pour joindre à celles que le Portugal fournissait et qu'il
réclama en vain, il fut forcé, le pied sur la gorge, à recevoir les
vaisseaux et les troupes de ces deux nations, de se joindre à elles
contre l'Espagne malgré lui, et de la prendre ainsi par le seul endroit
en prise, et qui fit sentir tout le danger et toute la dépense de ce que
nous avions manqué.

Cette faute et celle du renvoi des garnisons hollandaises furent
capitales et influèrent sur tout. Celle encore d'espérer toujours contre
toute, espérance, et cette délicatesse de ne vouloir pas paraître
agresseur, et de s'opiniâtrer à se laisser attaquer après tous les
amusements et tous les délais qu'ils voulurent employer, fut une autre
cause de ruine. Avec un parti pris et le courage et la célérité du début
des précédentes guerres, on les aurait déconcertés et réduits à
l'impossible avant qu'ils se fussent arrangés, et on les eût réduits à
cette paix qu'on désirait tant par la posture ou on se serait mis de
leur faire tout craindre pour eux-mêmes. Mais nos ministres n'étaient
plus les mêmes\,; et on ne s'aperçut que trop après que c'était aussi
d'autres généraux. L'autre traité fut celui par lequel M. de Mantoue
livra au roi ses places et ses États. Rien n'était plus important que
Mantoue, ni rien de si pressé de s'en assurer. Enfin, par celui de M. de
Savoie, il fut déclaré généralissime des forces des deux couronnes en
Italie, et s'engagea à fournir dix mille hommes de ses troupes, outre
tous les passages et toutes les facilités pour les nôtres, et il se
flatta en même temps du mariage de sa seconde fille avec le roi
d'Espagne.

M. de Savoie fut fort blessé de la loi que le parlement d'Angleterre
venait de faire pour régler l'ordre de la succession à la couronne de la
Grande-Bretagne et la fixer en même temps dans la ligne protestante, en
faveur de Sophie, femme du nouvel électeur d'Hanovre, et mère de
l'électeur roi d'Angleterre, et fille de l'électeur palatin roi de
Bohème déposé et chassé de tous ses États, et d'une fille de Jacques
Ier, roi de la Grande-Bretagne et sœur du roi Charles Ier à qui ses
sujets coupèrent la tête. Or, Charles était père de la première femme de
Monsieur, dont la fille était épouse de M. de Savoie, et par conséquent
excluait de droit sa tante paternelle et les Hanovre ses enfants. M. de
Savoie porta ses plaintes en forme en Angleterre, qui ne furent pas
écoutées. On n'y voulait plus ouïr parler d'un roi catholique après
avoir chassé et proscrit le roi Jacques II et sa postérité.

Les Vénitiens aussi déclarèrent qu'ils se tiendraient neutres, et qu'ils
appelleraient à leur secours l'ennemi de celui qui se voudrait saisir de
quelqu'une de leurs places malgré eux. C'est tout ce que le cardinal
d'Estrées en put obtenir, qui de Venise se mêla aussi du traité de
Savoie avec Phélypeaux, notre ambassadeur là Turin, et avec Tessé de
celui du duc de Mantoue. Le bonhomme La Haye, notre ambassadeur à
Venise, voulut finir sa longue ambassade à ce période. Il avait été
longtemps ambassadeur à Constantinople avec grande réputation, et bien
servi encore ailleurs. Charmant, nouveau secrétaire du cabinet, lui
succéda à Venise.

Catinat fut choisi pour commander en Italie. Il venait de, perdre
Croisille, son frère, qui avait servi avec grande réputation, mais que
sa mauvaise santé avait empêché de continuer. C'était un homme fort
sage, fort instruit, fort judicieux, qui avait beaucoup d'amis
considérables, quoique fort retiré et grand homme de bien. C'était le
conseil et l'ami, du cœur de son frère, qui partit dans cette
affliction. Tessé fut outré d'avoir un général. Le brillant et le solide
qu'il avait tiré de la fin de la dernière guerre d'Italie, les avantages
qu'il avait taché d'en prendre à la cour depuis que la paix et sa charge
l'y avaient attaché, la familiarité qu'il avait acquise à la cour de
Turin et la part qu'il venait d'avoir au traité de Mantoue lui avaient
fait espérer de commander en chef les troupes du roi sous M. de Savoie.
Il était gâté, mais M. de Vaudemont avait achevé de lui tourner la tête.
Ce favori de la fortune, qui ne négligeait rien pour s'en tenir les
chaînes assurées, et qui était l'homme le mieux informé de l'intérieur
des cours dont il avait affaire, avait tout prodigué pour s'attacher
Tessé, que le roi lui avait envoyé pour concerter avec lui tout ce qui
regardait le militaire. Fêtes, galanteries, confiance, déférences,
honneurs partout et civils et militaires, en tout pareils à ceux qui lui
étaient rendus à lui-même, rien ne fut épargné. Il parut donc bien dur à
Tessé, qui avait eu la sotte vanité de recevoir des honneurs de
gouverneur et de capitaine général du Milanais, d'en tomber tout à coup,
et dans le Milanais même, dans l'état commun de simple lieutenant
général roulant avec tous les autres. Il tâcha au moins de tirer ce
parti de leur commander sous Catinat, comme autrefois on avait fait
quelques capitaines généraux, mais il en fut refusé, et se vit par là
loin encore du bâton de maréchal de France qu'il croyait déjà tenir,
quoiqu'il n'eût jamais vu d'action ni peut-être brûler une amorce par le
hasard d'absence, de détachement ou de commissions, mais on ne se rend
pas justice et on se prend à qui on peut. Il attendit donc Catinat qui
l'avait proposé à fa fin de la dernière guerre pour traiter avec la cour
de Turin, et qui par là avait fait sa fortune. Il l'attendit, dis-je,
avec ferme dessein de lui faire du pis qu'il pourrait, afin d'essayer de
le chasser de cette armée, dans l'espérance de lui succéder, et
qu'appuyé comme il comptait de l'être de M. de Savoie et de Vaudemont,
elle ne lui échapperait pas, et qu'à ce coup on ne pourrait lui différer
le bâton de maréchal de France.

En même temps les, armées furent réglées en Flandre sous le maréchal de
Boufflers, et en Allemagne sous le maréchal de Villeroy. Monseigneur le
duc de Bourgogne fut destiné un moment à commander celle de ce dernier,
mais cela fut changé sur le dépit que témoigna Monsieur de ce que M. de
Chartres fut refusé de servir.

Le roi y avait consenti dans l'espérance que Monsieur, piqué de ce qu'on
ne lui donnait point d'armée, n'y consentirait pas, et y mit la
condition que ce serait avec l'agrément de Monsieur. Monsieur, et M. le
duc de Chartres, qui comprirent que servant toujours, il n'était plus
possible à son âge de lui refuser le commandement d'une armée l'année
suivante, s'ils ne le pouvaient obtenir celle-ci, aimèrent mieux sauter
le bâton du service subalterne encore cette campagne. Le roi, qui pour
cette même raison ne voulait pas que son neveu servît, fut surpris de
trouver Monsieur dans la même volonté que M. son fils, et, si cela s'ose
dire, fut pris pour dupe\,; mais il ne la fut pas, et montra la corde
par le refus chagrin qu'il fit tout net pour qu'on ne lui en parlât
plus. Il s'y trompa encore. M. de Chartres fit des escapades peu
mesurées, mais de son âge, qui fâchèrent le roi et l'embarrassèrent
encore davantage. Il ne savait que faire à son neveu qu'il avait forcé à
être son gendre, et {[}à{]} qui, excepté les conditions écrites,
{[}il{]} n'avait rien tenu, tant de ce qu'il avait laissé espérer que de
ce qu'il avait promis. Ce refus de servir qui éloignait sans fin, pour
ne pas dire qui anéantissait, toute espérance de commandement d'armée,
rouvrit la plaie du gouvernement de Bretagne, et donnait beau jeu à.
Madame d'insulter à la faiblesse que Monsieur avait eue, qui n'en était
pas aux premiers repentirs. Il laissait donc faire son fils en jeune
homme, qui, avec d'autres jeunes têtes, se proposait de faire un trou à
la lune, tantôt pour l'Espagne et tantôt pour l'Angleterre\,; et
Monsieur, qui le connaissait bien et qui n'était pas en peine qu'il
exécutât ces folies, ne disait mot, bien aise que le roi en prit de
l'inquiétude, comme à la fin il arriva.

Le roi en parla à Monsieur, et, sur ce qu'il le vit froid, lui reprocha
sa faiblesse de ne savoir pas prendre autorité sur son fils. Monsieur
alors se fâcha, et bien autant de résolution prise que de colère, il
demanda au roi à son tour ce qu'il voulait faire de son fils à son
âge\,; qu'il s'ennuyait de battre les galeries de Versailles et le pavé
de la cour, d'être marié comme il l'était, et de demeurer tout nu
vis-à-vis ses beaux-frères comblés de charges, de gouvernements,
d'établissements et de rangs sans raison, sans politique et sans
exemple\,; que son fils était de pire condition que tout ce qu'il y
avait de gens en France de son âge qui servaient et à qui on donnait des
grades bien loin de les en empêcher\,; que l'oisiveté était la mère de
tout vice\,; qu'il lui était bien douloureux de voir son fils unique
s'abandonner à la débauche, à la mauvaise compagnie et aux folies, mais
qu'il lui était cruel de ne s'en pouvoir prendre à une jeune cervelle
justement dépitée, et de n'en pouvoir accuser que celui qui l'y
précipitait par ses refus. Qui fut bien étonné de ce langage si clair\,?
ce fut le roi. Jamais il n'était arrivé à Monsieur de s'échapper avec
lui à mille lieues près de ce ton, qui était d'autant plus fâcheux qu'il
était appuyé de raisons sans réplique, auxquelles toutefois le roi ne
voulait pas céder. Dans la surprise de cet embarras, il fut assez maître
de soi pour répondre, non en roi, mais en frère. Il dit à Monsieur qu'il
pardonnait tout à la tendresse paternelle. Il le caressa, il fit tout ce
qu'il put pour le ramener par la douceur et l'amitié. Mais le point
fatal était ce service pour le but du commandement en chef que Monsieur
voulait, et que le roi par cette raison même ne voulait pas\,; raison
qu'ils ne se disaient point l'un à l'autre, mais que tous deux
comprenaient trop bien l'un de l'autre. Cette forte conversation fut
longue et poussée, Monsieur toujours sur le haut ton et le roi toujours
au rabais. Ils se séparèrent de la sorte, Monsieur outré, mais n'osant
éclater, et le roi très piqué, mais ne voulant pas étranger Monsieur, et
moins encore que leur brouillerie pût être aperçue.

Saint-Cloud, ou Monsieur passait les étés en grande partie, et où il
alla plus tôt qu'à son ordinaire, les mit à l'aise en attendant un
raccommodement, et Monsieur, qui vint depuis voir le roi et quelquefois
dîner avec lui, y vint plus rarement qu'il n'avait accoutumé, et leurs
moments de tête-à-tête se passaient toujours en aigreurs du côté de
Monsieur\,; mais en public il n'y paraissait rien ou bien peu de chose,
sinon que les gens familiers avec eux remarquaient des agaceries et des
attentions du roi, et une froideur de Monsieur à y répondre, qui
n'étaient dans l'habitude ni de l'un ni de l'autre. Cependant Monsieur
qui vit bien que de tout cela il n'en résulterait rien de ce qu'il
désirait, et que la fermeté du roi là-dessus ne se laisserait point
affaiblir, jugea sagement par l'avis du maréchal de Villeroy, qui
s'entremit fort dans tout cela, et surtout par ceux du chevalier de
Lorraine et du marquis d'Effiat, qu'il ne fallait pas pousser le roi à
bout et qu'il était temps d'arrêter les saillies de la conduite de M.
son fils. Il le fit donc peu à peu, mais le cœur restant ulcéré, et
toujours avec le roi de la même manière.

Les princes du sang ne servirent point non plus. Ce fut M. le Prince
encore à qui le roi s'adressa pour faire entendre ce qu'il appelait
raison à M. le Duc et à M. le prince de Conti\,; mais M. du Maine et M.
le comte de Toulouse allèrent comme lieutenants généraux en Flandre sous
le maréchal de Boufflers.

Nyert, premier valet de chambre du roi, qui, sous prétexte de curiosité
à son âge et dans son emploi, avait suivi le roi d'Espagne à Madrid, et
qui y était demeuré pour y être spectateur des premiers temps de son
arrivée, revint au bout de cinq mois, et entretint le roi fort
longtemps, à plusieurs reprises, tête-à-tête. Mgr le duc de Bourgogne
arriva aussi le mercredi 20 avril\,; il avait pris la poste à Lyon. Le
roi l'attendit dans son cabinet\,; et en sortit au-devant de lui pour
l'embrasser, puis lui fit embrasser M\textsuperscript{me} la duchesse de
Bourgogne\,: c'était à trois heures après midi\,; il avait couché à
Sens. M. le duc de Berry, qui n'avait pas pris la poste si loin, arriva
quatre jours après.

Le roi eut presque en même temps la joie que la Suède, qui tenait de
fort près les Moscovites et le roi de Pologne unis contre lui, et qui
les avait battus en plusieurs rencontres et obtenu de grands avantages,
reconnut le roi d'Espagne.

Ce même mois d'avril vit un exemple bien rare et bien respectable,
auquel on ne devrait jamais donner lieu, et qui a été mal imité, et en
mêmes cas et choses, depuis par plusieurs qui l'auraient dû. Le roi
voulut remplir les deux places vacantes par la mort de M. de Noyon et
par la promotion du cardinal de Coislin à la charge de grand aumônier de
France et de l'ordre\,; et sans qu'aucun des deux prélats choisis le
sussent ni personne, il nomma M. de Cosnac archevêque d'Aix, et M.
Fortin de La Goguette archevêque de Sens.

Cosnac était un homme de qualité de Guyenne, qui avait fait grand bruit
par son esprit et par ses intrigues autrefois, étant évêque de Valence
et premier aumônier de Monsieur. Il s'était entièrement attaché à feu
Madame, pour laquelle il a fait des choses tout à fait singulières. Il
était son conseil et son ami de cœur, et le roi lui en savait gré. Il ne
put pourtant refuser à Monsieur de le faire chercher et arrêter, sur ce
qu'il avait disparu avec soupçon qu'il était allé se saisir de papiers
qui inquiétaient la jalousie de Monsieur, pour les rendre à Madame, et
que Monsieur voulait avoir. Madame, avertie par le roi, en donna
aussitôt avis à M. de Valence, qui se cacha dans une auberge obscure à
un coin de Paris. Mais Monsieur, secondé de ceux qui le gouvernaient,
mit de telles gens en campagne qu'il fut découvert, et qu'un matin la
maison fut investie. À ce bruit, l'évêque ne perdit point le jugement\,;
il se mit tout aussitôt à crier la colique\,; et l'officier qui entra
pour l'arrêter le trouva dans des contorsions étranges. L'évêque, sans
disputer, comme un homme qui n'est occupé que de son mal, dit qu'il va
mourir s'il ne prend un lavement sur l'heure\,; et qu'après qu'il l'aura
rendu il obéira, et continue à crier de toute sa force. L'officier, qui
n'eut pas la cruauté de l'emmener en cet état, se hâta d'envoyer quérir
un lavement pour achever plus tôt sa capture, mais il déclara qu'il ne
sortirait point de la chambre qu'avec le prélat. Le lavement vint, il le
prit, et quand il fut question de le rendre, il se mit sur un large pot
dans son lit sans en sortir. Il avait ses raisons pour un si bizarre
manège. Les papiers qu'on lui voulait prendre étaient avec lui dans son
lit, parce que depuis qu'il les avait il ne les quittait point. En
rendant son lavement, il les mit adroitement par-dessous sa couverture
au fond du pot, et opéra par-dessus, de façon à n'en être plus en peine.
S'en étant défait de cette façon, il dit qu'il se trouvait fort soulagé,
et se mit à rire comme un homme qui se sent revenir de la mort à la vie
après de cruelles douleurs, mais en effet de son tour de souplesse, et
de ce que cet officier si vigilant n'aurait que la puanteur de sa selle,
avec laquelle les papiers furent jetés au privé. Le prélat, qui était
travesti et qui n'avait point là d'autres habits à prendre, fut conduit
au Châtelet, et là écroué sous le faux nom qu'il avait pris\,; mais
comme on ne trouva rien et qu'on n'en eut que la honte, il fut délivré
deux jours après, avec beaucoup d'excuses et quelques réprimandes de son
travestissement, qui, se disait-on, l'avait fait méconnaître. Madame se
trouva plus délivrée que lui, et comme le roi en fut fort aise, le
prélat ne fit que secouer les oreilles, et fut le premier à rire de son
aventure\footnote{Voy., \emph{les Mémoires de Daniel de Cosnac}, publiés
  par la Société de l'Histoire de France (2 vol.~in-8, Paris, 1852).}.

Une autre fois, quelque diable fit une satire cruelle sur Madame, le
comte de Guiche, etc., et la fit imprimer en Hollande. Le roi
d'Angleterre, qui en eut promptement avis, en avertit Madame, qui s'en
ouvrit aussitôt à M. de Valence. «\,Laissez-moi faire, lui dit-il, et ne
vous mettez en peine de rien\,;» et s'en va. Madame après qui lui
demande ce qu'il pense faire, il ne répond point et disparaît. De
plusieurs jours on n'en entend point parler. Voilà Madame bien en peine.
En moins de quinze jours Madame le voit entrer dans son cabinet\,; elle
s'écrie et lui demande ce qu'il est devenu et d'où il vient. «\,De
Hollande, répond-il, où j'ai porté de l'argent, acheté tous les
exemplaires et l'original de la satire, fait rompre les planches devant
moi, et rapporté tous les exemplaires, pour vous mettre hors de toute
inquiétude et vous donner le plaisir de les brûler.\,» Madame fut ravie,
et en effet tout fut fidèlement brûlé, et il n'en est pas demeuré la
moindre trace. Il y en aurait mille à raconter.

Personne n'avait plus d'esprit ni plus présent ni plus d'activité,
d'expédients et de ressources, et sur-le-champ. Sa vivacité était
prodigieuse\,; avec cela très sensé, très plaisant en tout ce qu'il
disait sans penser à l'être, et d'excellente compagnie. Nul homme si
propre à l'intrigue, ni qui eût le coup d'œill plus juste\,; au reste
peu scrupuleux, extrêmement ambitieux, mais avec cela haut, hardi,
libre\,; et qui se faisait craindre et compter par les ministres. Cet
ancien commerce intime de Madame dans beaucoup de choses, dans lequel le
roi était entré, lui avait acquis une liberté et une familiarité avec
lui qu'il sut conserver et s'en avantager toute sa vie. Il se brouilla
bientôt avec Monsieur après la mort de Madame, pour laquelle il avait eu
force prises avec lui et avec ses favoris. Il vendit sa charge à
Tressan, évêque du Mans, autre ambitieux, intrigant de beaucoup
d'esprit, mais dans un plus bas genre, et n'en fut que mieux avec le
roi, qui lui donna des abbayes et enfin l'archevêché d'Aix, où il était
maître de la Provence.

L'autre prélat était tout différent\,: c'était un homme sage, grave,
pieux, tout appliqué à ses devoirs et à son diocèse, dont tout était
réglé, rien d'outré, que son mérite avait sans lui fait passer de
Poitiers à Sens, aimé et respecté dans le clergé et dans le monde, et
fort considéré à la cour. Il était fort attaché à mon père, était
demeuré extrêmement de mes amis, et n'avait pas oublié que mon père
avait fait le sien major de Blaire, qui fut le commencement de leur
fortune, qui avait poussé La Hoguette, petit-fils de celui-là et fils du
frère de l'archevêque, à être premier sous-lieutenant des mousquetaires
noirs et lieutenant général fort distingué. Il fut tué aux dernières
campagnes de la dernière guerre d'Italie, avait épousé une femme fort
riche, fort dévote, fort glorieuse, fort dure, sèche et avare, dont une
seule fille, qui devait être et fut en effet un grand parti. C'était
donc de quoi le rehausser que ce cordon bleu à son grand-oncle paternel,
et le tenter de ne pas faire à cette nièce à marier la honte et le
dommage d'un refus. Mais la vérité fut plus forte en lui\,; il répondit
avec modestie qu'il n'était pas en état de faire des preuves, et refusa
avec beaucoup de respect et de reconnaissance. Ces Fortin en effet
n'étaient rien du tout, et c'est au plus si ce major de Blaye avait été
anobli. Ce n'est pas que M. de Sens ne sentit le poids de ce refus.
Quoique savant, appliqué, à la tête des affaires temporelles et
ecclésiastiques du clergé, il était aussi homme du monde, voyait chez
lui, à Fontainebleau qui est du diocèse de Sens, la meilleure compagnie
de la cour. Il y donnait à dîner tous les jours\,; grands seigneurs,
ministres, tout y allait hors les femmes\,; et très souvent les soirs,
qu'il ne soupait jamais, compagnie distinguée et choisie à causer avec
lui, et à Paris, quelques mois d'hiver, toujours dans les meilleures
maisons\,; mais il ne voulait point dérober les grâces ni se donner pour
autre qu'il était.

Ce refus embarrassa le roi, qui l'avait déclaré en plein chapitre\,; il
l'aimait, et ce trait ne le lui fit qu'estimer davantage. Il lui fit
donc l'honneur de lui écrire lui-même, et après l'avoir loué, il lui
manda qu'étant publiquement nommé, il faudrait en trouver un autre à sa
place, ce qui ne se pouvait sans alléguer la cause de son refus\,; qu'il
acceptât donc hardiment sur sa parole\,; que les commissaires de ses
preuves ne lui en demanderaient jamais\,; qu'au prochain chapitre il
ordonnerait de passer outre à l'admission en attendant les preuves\,;
qu'il serait reçu tout de suite, et que de preuves après il ne s'en
parlerait jamais. Le roi eut la bonté de lui représenter l'intérêt de sa
famille, aux dépens de laquelle il ne devait pas faire une action, belle
pour lui, mais qui la noterait pour toujours, et d'ajouter qu'il
désirait qu'il acceptât et qu'il prenait tout sur lui. Si quelque chose
peut flatter et tenter au delà des forces, il faut convenir que c'est
une lettre aussi complète\,; mais rien ne put ébranler l'humble
attachement de ce prélat aux règles et à la vérité. Après s'être répandu
comme il devait en actions de grâces, il répondit qu'il ne pouvait
mentir, ni par conséquent fournir de preuves\,; qu'il ne pouvait aussi
se résoudre à être cause que, par un excès de bonté, le roi manquât au
serment qu'il avait fait à son sacre de maintenir l'ordre et ses
statuts\,; que celui qui obligeait aux preuves était de ceux dont le
souverain, grand maître, ne pouvait dispenser, et que ce serait lui
faire violer son serment que d'être reçu sans preuves préalables, sur la
certitude de les faire après, quand il savait que sa condition lui en
ôtait le moyen\,; et il finit une lettre d'autant plus belle qu'il n'y
avait ni fleurs ni tours, mais de la vérité, de l'humilité et beaucoup
de sentiment, par supplier le roi d'en nommer un autre, et de ne point
craindre d'en dire la raison, puisqu'il le fallait. Cette grande action
fut universellement admirée, et ajouta encore à la considération du roi
et au respect de tout le monde.

Son refus commençait à transpirer lorsque le roi assembla un autre
chapitre pour nommer M. de Metz à sa place, par amitié pour le cardinal
de Coislin son oncle, qui ne s'y attendaient ni l'un ni l'autre. Le roi
déclara le refus de M. de Sens, voulut bien parler de ce qu'il lui avait
offert, et fit son éloge. Il n'y eut personne dans le chapitre qui ne le
louât extrêmement\,; mais, sans louanges, M. de Marsan fit mieux que pas
un, et tint là le meilleur propos de toute sa vie\,: «\,Sire, dit-il au
roi tout haut, cela mériterait bien que Votre Majesté changeât le bleu
en rouge.\,» Tout y applaudit comme par acclamation, et à la fin du
chapitre, tous louèrent et remercièrent M. de Marsan.

Tallart, qui ne faisait qu'arriver d'Angleterre, eut le gouvernement du
pays de Foix, et d'autres petites charges à vendre, et fut déclaré
chevalier de l'ordre, pour être reçu à la Pentecôte avec les deux
prélats. Il parut fort content, mais le duché d'Harcourt émoussait fort
la joie de ces faveurs. À un mois de là il perdit sa femme, du nom de
Groslée, fille de Virville, qui avait été longtemps capitaine de
gendarmerie. C'était une femme fort d'un certain monde à Paris, dont la
réputation était médiocre, et qui ne partageait en rien avec son mari\,:
elle n'allait jamais à la cour et ils ne vivaient comme point ensemble.

La duchesse d'Arpajon, sœur de Beuvron, et M\textsuperscript{me}
d'Hauterive, ci-devant duchesse de Chaulnes, et sœur du maréchal de
Villeroy, moururent en même temps. J'ai tant parlé d'elles que je n'ai
rien à y ajouter.

M\textsuperscript{me} de Bournonville qui, faute de tabouret, très mal à
propos prétendu, n'allait point à la cour, et s'en dépiquait à Paris par
ses charmes, mourut fort jeune aussi. Elle était sœur du second lit de
M. de Chevreuse, et son mari cousin germain de la maréchale de Noailles.
Elle laissa un fils et une fille forts enfants. Le père de
M\textsuperscript{me} de Noailles, frère du sien, avait été duc à brevet
après son père. Le père de M. de Bournonville était l'aîné, et eut de
grands emplois en Espagne, où il mourut. Le cadet, père de
M\textsuperscript{me} de Noailles, s'attacha à la France, et y eut des
charges considérables. Le brevet du duc lui fut renouvelé. Ils ne sont
point héréditaires\,; ainsi M. de Bournonville, dont il s'agit ici, n'y
avait pas ombre de droit.

Segrais, poète français illustre, élevé chez Mademoiselle, fille de
Gaston, et retiré à Caen dans le sein des belles-lettres, était mort
fort vieux auparavant.

La France perdit le plus grand homme de mer, de l'aveu des Anglais et
des Hollandais, qui eût été depuis un siècle, et en même temps le plus
modeste. Ce fut le maréchal de Tourville, qui n'avait pas encore
soixante ans. Il ne laissa qu'un fils, qui promettait, et qui fut tué
dès sa première campagne, et une fille fort jeune. Tourville possédait
en perfection toutes les parties de la marine, depuis celle du
charpentier jusqu'à celles d'un excellent amiral. Son équité, sa
douceur, son flegme, sa politesse, la netteté de ses ordres, les signaux
et beaucoup d'autres détails particuliers très utiles qu'il avait
imaginés, son arrangement, sa justesse, sa prévoyance, une grande
sagesse aiguisée de la plus naturelle et de la plus tranquille valeur,
tout contribuait à faire désirer de servir sous lui, et d'y apprendre.
Sa charge de vice-amiral fut donnée à Châteaurenauld, qui était lors en
Amérique pour en ramener les galions.

L'Allemagne à son tour perdit un homme moins nécessaire et plus vieux,
mais qui s'était immortalisé par la défense de Vienne, dont il était
gouverneur, assiégée par les Turcs, le célèbre comte de Staremberg, qui
était président du conseil de guerre, la plus belle et la plus
importante charge de la cour de l'empereur.

Le roi d'Angleterre, qui n'oubliait rien pour redresser promptement son
ancienne grande alliance et la bien organiser contre nous, avait peine à
rajuster ensemble tant de pièces une fois désunies et à trouver les
fonds nécessaires à ses projets, dans la disette d'argent où l'empereur
se trouvait. Il tâchait donc d'amuser toujours le roi des flatteuses
espérances d'une tranquillité que tout démentait. Pour tenir toujours
tout en suspens en attendant que ses machines fussent tout à fait
prêtes, il avait engagé les Hollandais, qu'il gouvernait pleinement à
reconnaître le roi d'Espagne, et à la fin, il le reconnut aussi,
tellement que ce prince le fut de toute l'Europe, excepté de l'empereur.
Quoique le roi goûtât extrêmement des démarches si précises en faveur de
la paix, il ne laissait pas de se préparer puissamment\,; et comme il
disposait de l'Espagne comme de la France, elle ne perdait pas de temps
aussi à se mettre en état de bien soutenir la guerre. Le comte d'Estrées
était dans la Méditerranée. Le roi d'Espagne le fit capitaine général de
la mer, qui répond à la charge qu'il avait ici, tellement qu'il commanda
également aux forces navales des deux couronnes. Ce prince, en même
temps excité par Louville, dépêcha un courrier au duc de Beauvilliers,
avec la patente d'une grandesse de la première classe pour lui et pour
les siens, mâles et femelles. Le duc, qui n'y avait pas songé, et qui,
comme ministre d'État et comme ayant été gouverneur du roi d'Espagne,
ouvrait librement les lettres qu'il recevait de ce prince, trouvant
cette patente et une lettre convenable au sujet qui lui en donnait la
nouvelle, les porta au roi l'une et l'autre, qui approuva fort cette
marque de sentiment du roi son petit-fils, et qui ordonna à M. de
Beauvilliers de l'accepter.

Presque en même temps le mariage du roi d'Espagne fut déclaré avec la
seconde fille de M. de Savoie, sœur cadette de M\textsuperscript{me} la
duchesse de Bourgogne, pour qui ce fut une grande joie comme un grand
honneur et un grand avantage à M. son père, d'avoir pour gendres les
deux premiers et plus puissants rois de l'Europe. Le roi crut fixer ce
prince dans ses intérêts par de si hautes alliances redoublées et par la
confiance du commandement général en Italie.

Le roi aussi, pour mieux cimenter l'union des deux couronnes et des deux
nations, convint avec le roi d'Espagne que les grands d'Espagne auraient
désormais en France le rang, les honneurs, le traitement et les
distinctions des ducs\,; et que réciproquement les ducs de France
auraient en Espagne le rang, les honneurs, le traitement, et les
distinctions qu'y ont les grands. Rien de mieux ni de plus convenable,
si on s'en était tenu là. On verra en son lieu ce que quelques grands
d'Espagne en pensèrent, et l'abus étrange d'une si sage convention.

L'abbé de Polignac qui, depuis son arrivée de Pologne, était demeuré
exilé en son abbaye de Bonport, près le Pont-de-l'Arche, eut permission
de revenir à Paris et à la cour. Torcy son ami, et bien des gens qui
s'intéressaient en lui avaient travaillé en sa faveur.

Le duc de Popoli, frère du cardinal Canteloni archevêque de Naples, y
retournant d'Espagne, fut présenté au roi par l'ambassadeur d'Espagne.
C'est une maison ancienne et illustre qui est puissante à Naples, et le
cardinal Canteloni avait très bien fait pour le roi d'Espagne. Le roi
traita donc fort bien le duc de Popoli, et si bien, que ce seigneur, qui
désirait fort l'ordre et qui avait pris ses précautions sur cela avant
de quitter Madrid, se crut en état de le pouvoir demander. Le roi le lui
promit, et lui dit qu'il lui en coûterait un voyage, parce qu'il serait
bien aise de le revoir\,; et qu'il voulait le recevoir lui-même. Nous
lui verrons faire une grande fortune en Espagne, et il donnera lieu d'en
parler plus d'une fois. Il fut très peu ici et s'en alla à Naples.

La Touane et Saurion, trésoriers de l'extraordinaire des guerres,
culbutèrent et firent banqueroute. Ils en avertirent Chamillart, qui par
l'examen de leurs affaires, la trouva de quatre millions. On les mit à
la Bastille\,; le roi prit ce qu'il leur restait, et se chargea de payer
les dettes pour conserver son crédit à l'entrée d'une grosse guerre,
pour laquelle cette faillite ne fut pas de bon augure. On en fut fort
surpris par le soin avec lequel ils avaient soutenu et caché leur
désordre jusqu'à rien plus sous la sérénité et le luxe des financiers.

\hypertarget{chapitre-ix.}{%
\chapter{CHAPITRE IX.}\label{chapitre-ix.}}

1701

~

{\textsc{L'empereur fait arrêter Ragotzi.}} {\textsc{- Retour des eaux
du roi Jacques.}} {\textsc{- Peines de Monsieur.}} {\textsc{- Forte
prise du roi et de Monsieur.}} {\textsc{- Mort de Monsieur.}} {\textsc{-
Spectacle de Saint-Cloud.}} {\textsc{- Spectacle de Marly.}} {\textsc{-
Diverses sortes d'afflictions et de sentiments.}} {\textsc{- Caractère
de Monsieur.}} {\textsc{- Trait de hauteur de Monsieur à M. le Duc.}}
{\textsc{- Visite curieuse de M\textsuperscript{me} de Maintenon à
Madame.}} {\textsc{- Traitement prodigieux de M. le duc de Chartres, qui
prend le nom de duc d'Orléans.}} {\textsc{- M. le Prince fait pour sa
vie premier prince du sang.}} {\textsc{- Veuvage étrange de Madame\,;
son traitement.}} {\textsc{- Obsèques de Monsieur.}} {\textsc{- Ducs à
l'eau bénite, non les duchesses ni les princesses.}} {\textsc{-
Désordres des carrosses.}} {\textsc{- Curieuse anecdote sur la mort de
Madame, première femme de Monsieur.}}

~

Le royaume de Hongrie n'avait jamais tari de mécontents, et en avait
souvent des marques qui leur avaient été funestes depuis que la maison
d'Autriche avait dépouillé les états du droit d'élection des rois de
Hongrie. Cela intéressait extrêmement la noblesse, surtout les grands
seigneurs. Les peuples aussi se prétendaient vexés et foulés\,; et les
griefs de religion, ou la grecque et la protestante ont un grand nombre
de sectateurs, étoient une autre semence de soulèvement. Mais les
garnisons allemandes, et presque toutes les grandes places occupées par
des Allemands, indisposaient toute la nation en général. Il en coûta la
tête en 1671 aux comtes Serini du nom d'Esdrin, gouverneur de Croatie, à
Frangipani et à sa femme, sœur de Serini, et à Nadasti, président du
conseil souverain de Hongrie, et la prison perpétuelle au fils du comte
Serini, où il est mort plus de trente ans après. Sa sœur, fille du comte
Serini exécuté, avait épousé le prince Ragotzi, dont elle eut le prince
Ragotzi dont je vais parler, et qui me donnera lieu d'en parler plus
d'une fois. Elle se remaria en 1681, au fameux comte Tekeli, chef des
mécontents, qui a tant fait de bruit dans le monde, et n'en eut point
d'enfants. Ragotzi, son premier mari, vécut particulier, et ne fut rien.
Il avait été de la conspiration de son beau-père, mais la peur qu'il eut
quand il le vit arrêté fit qu'il en usa si mal avec lui qu'il se sauva
du naufrage\,; mais il ne fut rien toute sa vie. Il avait de grands
biens. Son père, son grand-père qui fut fait prince de l'empire, et son
bisaïeul, avaient été princes de Transylvanie, ce dernier élu en 1606,
après la mort de Botzkay, Le Ragotzi dont je parle avait été bien élevé,
et n'avait encore guère pu faire parler de lui, observé de près comme il
l'était, lorsque, devenu par tant d'endroits si proches suspect à
l'empereur qui découvrit de nouveaux remuements en Hongrie, il le fit
arrêter et enfermer à Neustadt, au mois d'avril de cette année. On
prétendit qu'il y était entré innocent\,; nous verrons bientôt que s'il
n'en sortit pas coupable, il le devint bientôt après. Il était dès lors
marié à une princesse de Hesse-Rhinfeltz.

Le roi d'Angleterre était revenu de Bourbon avec peu ou point de
soulagement, et Monsieur était toujours à Saint-Cloud, dans la même
situation de cœur et d'esprit, et gardant avec le roi la même conduite
que j'ai expliquée. C'était pour lui être hors de son centre, à la
faiblesse dont il était, et à l'habitude de toute sa vie d'une grande
soumission et d'un grand attachement pour le roi, et de vivre avec lui,
dans le particulier, dans une liberté de frère, et d'en être traité en
frère aussi avec toutes, sortes de soins, d'amitié et d'égards, dans
tout ce qui n'allait point à faire de Monsieur un personnage. Lui ni
Madame n'avaient pas mal au bout du doigt que le roi n'y allât dans
l'instant, et souvent après, pour peu que le mal durât. Il y avait six
semaines que Madame avait la fièvre double tierce, à laquelle elle ne
voulait rien faire, parce qu'elle se traitait à sa mode allemande, et ne
faisait pas cas des remèdes ni des médecins. Le roi qui, outre l'affaire
de M. le duc de Chartres, était secrètement outré contre elle, comme on
le verra bientôt, n'avait point été la voir, quoique Monsieur l'en eût
pressé dans ces tours légers qu'il venait faire sans coucher. Cela était
pris par Monsieur, qui ignorait le fait particulier de Madame au roi,
pour une marque publique d'une inconsidération extrême, et comme il
était glorieux et sensible, il en était piqué au dernier point.

D'autres peines d'esprit le tourmentaient encore. Il avait depuis
quelque temps un confesseur qui, bien que jésuite, le tenait de plus
court qu'il pouvait\,; c'était un gentilhomme de bon lieu et de
Bretagne, qui s'appelait le P. du Trévoux. Il lui retrancha, non
seulement d'étranges plaisirs, mais beaucoup de ceux qu'il se croyait
permis, pour pénitence de sa vie passée. Il lui représentait fort
souvent qu'il ne se voulait pas damner pour lui, et que, si sa conduite
lui paraissait trop dure, il n'aurait nul déplaisir de lui voir prendre
un autre confesseur. À cela il ajoutait qu'il prît bien garde à lui,
qu'il était vieux, usé de débauche, gras, court de cou, et que, selon
toute apparence, il mourrait d'apoplexie, et bientôt. C'étaient là
d'épouvantables paroles pour un prince le plus voluptueux et le plus
attaché à la vie qu'on eût vu de longtemps, qui l'avait toujours passée
dans la plus molle oisiveté, et qui était le plus incapable par nature
d'aucune application, d'aucune lecture sérieuse, ni de rentrer en
lui-même. Il craignait le diable, il se souvenait que son précédent
confesseur n'avait pas voulu mourir dans cet emploi, et qu'avant sa mort
il lui avait tenu les mêmes discours. L'impression qu'ils lui firent le
forcèrent de rentrer un peu en lui-même, et de vivre d'une manière qui
depuis quelque temps pouvait passer pour serrée à son égard. Il faisait
à reprises beaucoup de prières, obéissait, à son confesseur, lui rendait
compte de la conduite qu'il lui avait prescrite sur son jeu, sur ses
autres dépenses, et sur bien d'autres choses, souffrait avec patience
ses fréquents entretiens, et y réfléchissait beaucoup. Il en devint
triste, abattu, et parla moins qu'à l'ordinaire, c'est-à-dire encore
comme trois ou quatre femmes, en sorte que tout le monde s'aperçut
bientôt de ce grand changement. C'en était bien à la fois que ces peines
intérieures, et les extérieures du côté du roi, pour un homme aussi
faible que Monsieur, et aussi nouveau à se contraindre, à être fâché et
à le soutenir\,; et il était difficile que cela ne fit bientôt une
grande révolution dans un corps aussi plein et aussi grand mangeur, non
seulement à ses repas, mais presque toute la journée.

Le mercredi 8 juin, Monsieur vint de Saint-Cloud dîner avec le roi à
Marly, et, à son ordinaire, entra dans son cabinet lorsque le conseil
d'État en sortit. Il trouva le roi chagrin de ceux que M. de Chartres
donnait exprès à sa fille, ne pouvant se prendre à lui directement. Il
était amoureux de M\textsuperscript{lle} de Sery, fille d'honneur de
Madame, et menait cela tambour battant. Le roi prit son thème là-dessus,
et fit sèchement des reproches à Monsieur de la conduite de son fils.
Monsieur qui, dans la disposition où il était, n'avait pas besoin de ce
début pour se fâcher, répondit avec aigreur que les pères qui avaient
mené de certaines vies avaient peu de grâce et d'autorité à reprendre
leurs enfants. Le roi, qui sentit le poids de la réponse, se rabattit
sur la patience de sa fille, et qu'au moins devait-on éloigner de tels
objets de ses yeux. Monsieur, dont la gourmette était rompue, le fit
souvenir, d'une manière piquante, des façons qu'il avait eues pour la
reine avec ses maîtresses, jusqu'à leur faire faire les voyages dans son
carrosse avec elle. Le roi outré renchérit, de sorte qu'ils se mirent
tous deux à se parler à pleine tête.

À Marly, les quatre grands appartements en bas étaient pareils et
seulement de trois pièces. La chambre du roi tenait au petit salon, et
était pleine de courtisans à ces heures-là pour voir passer le roi
s'allant mettre à table\,; et par de ces usages propres aux différents
lieux, sans qu'on en puisse dire la cause, la porte du cabinet qui,
partout ailleurs, était toujours fermée, demeurait en tout temps ouverte
à Marly hors le temps du conseil, et il n'y avait dessus qu'une portière
tirée que l'huissier ne faisait que lever pour y laisser entrer. À ce
bruit il entra, et dit au roi qu'on l'entendait distinctement de sa
chambre et Monsieur aussi, puis ressortit. L'autre cabinet du roi
joignant le premier ne se fermait ni de porte ni de portière, il sortait
dans l'autre petit salon, et il était retranché dans sa largeur pour la
chaise percée du roi. Les valets intérieurs se tenaient toujours dans ce
second cabinet, qui avaient entendu d'un bout à l'autre tout le dialogue
que je viens de rapporter.

L'avis de l'huissier fit baisser le ton, mais n'arrêta pas les
reproches, tellement que Monsieur, hors des gonds, dit au roi qu'en
mariant son fils il lui avait promis monts et merveilles, que cependant
il n'en avait pu arracher encore un gouvernement\,; qu'il avait
passionnément désiré de faire servir son fils pour l'éloigner de ces
amourettes, et que son fils l'avait aussi fort souhaité, comme il le
savait de reste, et lui en avait demandé la grâce avec instance\,; que
puisqu'il ne le voulait pas, il ne s'entendait point à l'empêcher de
s'amuser pour se consoler. Il ajouta qu'il ne voyait que trop la vérité
de ce qu'on lui avait prédit, qu'il n'aurait que le déshonneur et la
honte de ce mariage sans en tirer jamais aucun profit. Le roi, de plus
en plus outré de colère, lui repartit que la guerre l'obligerait bientôt
à faire plusieurs retranchements\,; et que, puisqu'il se montrait si peu
complaisant à ses volontés, il commencerait par ceux de ses pensions
avant que retrancher sur soi-même.

Là-dessus le roi fut averti que sa viande était portée. Ils sortirent un
moment après pour se venir mettre à table, Monsieur d'un rouge enflammé,
avec les yeux étincelants de colère. Son visage ainsi allumé fit dire à
quelqu'une des dames qui étaient à table et à quelques courtisans
derrière, pour chercher à parler, que Monsieur, à le voir, avait grand
besoin d'être saigné. On le disait soit de même à Saint-Cloud il y avait
quelque temps, il en crevait de besoin, il l'avouait même, le roi l'en
avait même pressé plus d'une fois malgré leurs piques. Tancrède, son
premier chirurgien, était vieux, saignait mal et l'avait manqué. Il ne
voulait pas se faire saigner par lui, et pour ne point lui faire de
peine il eut la bonté de ne vouloir pas être saigné par un autre et d'en
mourir. À ces propos de saignée, le roi lui en parla encore, et ajouta
qu'il ne savait à quoi il tenait qu'il ne le menât dans sa chambre et
qu'il ne le fît saigner tout à l'heure. Le dîner se passa à l'ordinaire,
et Monsieur y mangea extrêmement, comme il faisait à tous ses deux
repas, sans parler du chocolat abondant du matin, et de tout ce qu'il
avalait de fruits, de pâtisserie, de confitures et de toutes sortes de
friandises toute la journée, dont les tables de ses cabinets et ses
poches étaient toujours remplies. Au sortir de table, le roi seul,
Monseigneur avec M\textsuperscript{me} la princesse de Conti, Mgr le duc
de Bourgogne seul, M\textsuperscript{me} la duchesse de Bourgogne avec
beaucoup de dames, allèrent séparément à Saint-Germain voir le roi et la
reine d'Angleterre. Monsieur, qui avait amené M\textsuperscript{me} la
duchesse de Chartres de Saint-Cloud dîner avec le roi, la mena aussi à
Saint-Germain, d'où il partit pour retourner à Saint-Cloud avec elle,
lorsque le roi arriva à Saint-Germain.

Le soir après le souper, comme le roi était encore dans son cabinet avec
Monseigneur et les princesses comme à Versailles, Saint-Pierre arriva de
Saint-Cloud qui demanda à parler au roi de la part de M. le duc de
Chartres. On le fit entrer dans le cabinet, où il dit au roi que
Monsieur avait eu une grande faiblesse en soupant, qu'il avait été
saigné, qu'il était mieux, mais qu'on lui avait donné de l'émétique. Le
fait était qu'il soupa à son ordinaire avec les dames qui étaient à
Saint-Cloud. Vers l'entremets, comme il versait d'un vin de liqueur à
M\textsuperscript{me} de Bouillon, on s'aperçut qu'il balbutiait et
qu'il montrait quelque chose de la main.

Comme il lui arrivait quelquefois de leur parler espagnol, quelques
dames lui demandèrent ce qu'il disait, d'autres s'écrièrent\,; tout cela
en un instant, et il tomba en apoplexie sur M. le duc de Chartres qui le
retint. On l'emporta dans le fond de son appartement, on le secoua, on
le promena, on le saigna beaucoup, on lui donna force émétique, sans en
tirer presque aucun signe de vie.

À cette nouvelle, le roi, qui pour des riens accourait chez Monsieur,
passa chez M\textsuperscript{me} de Maintenon qu'il fit éveiller\,; il
fut un quart d'heure avec elle, puis sur le minuit rentrant chez lui, il
commanda ses carrosses tout prêts, et ordonna au marquis de Gesvres
d'aller à Saint-Cloud et, si Monsieur était plus mal, de revenir
l'éveiller pour y aller, et se coucha. Outre la situation en laquelle
ils se trouvaient ensemble, je pense que le roi soupçonna quelque
artifice pour sortir de ce qui s'était passé entre eux, qu'il alla en
consulter M\textsuperscript{me} de Maintenon, et qu'il aima mieux
manquer à toute bienséance que de hasarder d'en être la dupe.
M\textsuperscript{me} de Maintenon n'aimait pas Monsieur\,; elle le
craignait. Il lui rendait peu de devoirs, et avec toute sa timidité et
sa plus que déférence, il lui était échappé des traits sur elle plus
d'une fois avec le roi, qui marquaient son mépris, et la honte qu'il
avait de l'opinion publique. Elle n'était donc pas pressée de porter le
roi à lui rendre, et moins encore de lui conseiller de voyager la nuit,
de ne se point coucher, et d'être témoin d'un aussi triste spectacle et
si propre à toucher et à faire rentrer en soi-même\,; et qu'elle espéra
que, si la chose allait vite, le roi se l'épargnerait ainsi.

Un moment après que le roi fut au lit, arriva un page de Monsieur. Il
dit au roi que Monsieur était mieux, et qu'il venait demander à M. le
prince de Conti de l'eau de Schaffouse, qui est excellente pour les
apoplexies. Une heure et demie après que le roi fut couché, Longueville
arriva de la part de M. le duc de Chartres, qui éveilla le roi, et qui
lui dit que l'émétique ne faisait aucun effet, et que Monsieur était
fort mal. Le roi se leva, partit et trouva le marquis de Gesvres en
chemin qui l'allait avertir\,; il l'arrêta et lui dit les mêmes
nouvelles. On peut juger quelle rumeur et quel désordre cette nuit à
Marly, et quelle horreur à Saint-Cloud, ce palais des délices. Tout ce
qui était à Marly courut comme il put à Saint-Cloud\,; on s'embarquait
avec les plus tôt prêts\,; et chacun, hommes et femmes, se jetaient et
s'entassaient dans les carrosses sans choix et sans façon. Monseigneur
alla avec M\textsuperscript{me} la Duchesse. Il fut si frappé, par
rapport à l'état duquel il ne faisait que sortir, que ce fut tout ce que
put faire un écuyer de M\textsuperscript{me} la Duchesse, qui se trouva
là, de le traîner et le porter presque et tout tremblant dans le
carrosse. Le roi arriva à Saint-Cloud avant trois heures du matin.
Monsieur n'avait pas eu un moment de connaissance depuis qu'il s'était
trouvé mal. Il n'en eut qu'un rayon d'un instant, tandis que sur le
matin le P. du Trévoux était allé dire la messe, et ce rayon même ne
revint plus.

Les spectacles les plus horribles ont souvent des instants de contrastes
ridicules. Le P. du Trévoux revint et criait à Monsieur\,: «\,Monsieur,
ne connaissez-vous pas votre confesseur\,? Ne connaissez-vous pas le bon
petit père du Trévoux qui vous parle\,?» et fit rire assez indécemment
les moins affligés.

Le roi le parut beaucoup\,; naturellement il pleurait aisément, il était
donc tout en larmes. Il n'avait jamais eu lieu que d'aimer Monsieur
tendrement\,; quoique mal ensemble depuis deux mois, ces tristes moments
rappellent toute la tendresse\,; peut-être se reprochait-il d'avoir
précipité sa mort par la scène du matin\,; enfin il était son cadet de
deux ans, et s'était toute sa vie aussi bien porté que lui et mieux. Le
roi entendit la messe à Saint-Cloud, et sur les huit heures du matin,
Monsieur étant sans aucune espérance, M\textsuperscript{me} de Maintenon
et M\textsuperscript{me} la duchesse de Bourgogne l'engagèrent de n'y
pas demeurer davantage, et revinrent avec lui dans son carrosse. Comme
il allait partir et qu'il faisait quelques amitiés à M. de Chartres, en
pleurant fort tous deux, ce jeune prince sut profiter du moment.
«\,Eh\,! sire, que deviendrai-je\,? lui dit-il en lui embrassant les
cuisses\,; je perds Monsieur, et je sais que vous ne m'aimez point.\,»
Le roi surpris et fort touché l'embrassa, et lui dit tout ce qu'il put
de tendre. En arrivant à Marly, il entra avec M\textsuperscript{me} la
duchesse de Bourgogne chez M\textsuperscript{me} de Maintenon. Trois
heures après, M. Fagon, à qui le roi avait ordonné de ne point quitter
Monsieur qu'il ne fût mort ou mieux, ce qui ne pouvait arriver que par
miracle, lui dit dès qu'il l'aperçut\,: «\,Eh bien\,! monsieur Fagon,
mon frère est mort\,? --- Oui, sire, répondit-il, nul remède n'a pu,
agir.\,» Le roi pleura beaucoup. On le pressa de manger un morceau chez
M\textsuperscript{me} de Maintenon, mais il voulut dîner à l'ordinaire
avec les dames, et les larmes lui coulèrent souvent pendant le repas,
qui fut court, après lequel il se renferma chez M\textsuperscript{me} de
Maintenon jusqu'à sept heures, qu'il alla faire un tour dans ses
jardins. Il travailla avec Chamillart, puis avec Pontchartrain pour le
cérémonial de la mort de Monsieur, et donna là-dessus ses ordres à
Desgranges, maître des cérémonies, Dreux, grand maître, étant à l'armée
d'Italie. Il soupa une heure plus tôt qu'à l'ordinaire, et se coucha
fort tôt après. Il avait eu sur les cinq heures la visite du roi et de
la reine d'Angleterre, qui ne dura qu'un moment.

Au départ du roi la foule s'écoula de Saint-Cloud peu à peu, en sorte
que Monsieur mourant, jeté sur un lit de repos dans son cabinet, demeura
exposé aux marmitons et aux bas officiers, qui la plupart, par affection
ou par intérêt, étaient fort affligés. Les premiers officiers et autres,
qui perdaient charges et pensions faisaient retentir l'air de leurs
cris, tandis que toutes ces femmes qui étaient à Saint-Cloud, et qui
perdaient leur considération et tout leur amusement, couraient çà et là,
criant échevelées comme des bacchantes. La duchesse de La Ferté, de la
seconde fille de qui on a vu plus haut l'étrange mariage, entra dans ce
cabinet, où considérant attentivement ce pauvre prince qui palpitait
encore\,: «\,Pardi, s'écria-t-elle dans la profondeur de ses réflexions,
voilà une fille bien mariée\,! --- Voilà qui est bien important
aujourd'hui, lui répondit Châtillon qui perdait tout lui-même, que votre
fille soit bien ou mal mariée\,!»

Madame était cependant dans son cabinet qui n'avait jamais eu ni grande
affection ni grande estime pour Monsieur, mais qui sentait sa perte et
sa chute, et qui s'écriait dans sa douleur de toute sa force\,: «\,Point
de couvent\,! qu'on ne me parle point de couvent\,! je ne veux point de
couvent.\,» La bonne princesse n'avait pas perdu le jugement\,; elle
savait que, par son contrat de mariage, elle devait opter, devenant
veuve, un couvent, ou l'habitation du château de Montargis. Soit qu'elle
crût sortir plus aisément de l'un que de l'autre, soit que sentant
combien elle avait à craindre du roi, quoiqu'elle ne sût pas encore
tout, et qu'il lui eût fait les amitiés ordinaires en pareille occasion,
elle eut encore plus de peur du couvent. Monsieur étant expiré, elle
monta en carrosse avec ses dames, et s'en alla à Versailles suivie de M.
et de M\textsuperscript{me} la duchesse de Chartres, et de toutes les
personnes qui étaient à eux.

Le lendemain matin, vendredi, M. de Chartres vint chez le roi, qui était
encore au lit et qui lui parla avec beaucoup d'amitié. Il lui dit qu'il
fallait désormais qu'il le regardât comme son père\,; qu'il aurait soin
de sa grandeur et de ses intérêts\,; qu'il oubliait tous les petits
sujets de chagrin qu'il avait eus contre lui\,; qu'il espérait que de
son côté il les oublierait aussi\,; qu'il le priait que les avances
d'amitié qu'il lui faisait servissent à l'attacher plus à lui, et à lui
redonner son cœur comme il lui redonnait le sien. On peut juger si M. de
Chartres sut bien répondre.

Après un si affreux spectacle, tant de larmes et tant de tendresse,
personne ne douta que les trois jours qui restaient du voyage de Marly
ne fussent extrêmement tristes\,; lorsque ce même lendemain de la mort
de Monsieur, des dames du palais entrant chez M\textsuperscript{me} de
Maintenon où était le roi avec elle, et M\textsuperscript{me} la
duchesse de Bourgogne sur le midi, elles l'entendirent de la pièce où
elles se tenaient, joignant la sienne, chantant des prologues d'opéra.
Un peu après le roi, voyant M\textsuperscript{me} la duchesse de
Bourgogne fort triste en un coin de la chambre, demanda avec surprise à
M\textsuperscript{me} de Maintenon ce qu'elle avait pour être si
mélancolique, et se mit à la réveiller, puis à jouer avec elle et
quelques dames du palais qu'il fit entrer pour les amuser tous deux. Ce
ne fut pas tout que ce particulier. Au sortir du dîner ordinaire,
c'est-à-dire un peu après deux heures, et vingt-six heures après la mort
de Monsieur, Mgr le duc de Bourgogne demanda au duc de Montfort s'il
voulait jouer au brelan. «\,Au brelan ! s'écria Montfort dans un
étonnement extrême, vous n'y songez donc pas, Monsieur est encore tout
chaud. --- Pardonnez-moi, répondit le prince, j'y songe fort bien, mais
le roi ne veut pas qu'on s'ennuie à Marly, m'a ordonné de faire jouer
tout le monde, et de peur que personne ne l'osât faire le premier, d'en
donner moi l'exemple.\,» De sorte qu'ils se mirent à faire un brelan, et
que le salon fut bientôt rempli de tables de jeu.

Telle fut l'affliction du roi, telle celle de M\textsuperscript{me} de
Maintenon. Elle sentait la perte de Monsieur comme une délivrance\,;
elle avait peine à retenir sa joie\,: elle en eût eu bien davantage à
paraître affligée. Elle voyait déjà le roi tout consolé, rien ne lui
seyait mieux que de chercher à le dissiper, et ne lui était plus commode
que de hâter la vie ordinaire pour qu'il ne fût plus question de
Monsieur ni d'affliction. Pour des bienséances, elle ne s'en peina
point. La chose toutefois ne laissa pas d'être scandaleuse, et tout bas
d'être fort trouvée telle. Monseigneur semblait aimer Monsieur, qui lui
donnait des bals et des amusements avec toutes sortes d'attention et de
complaisance\,; dès le lendemain de sa mort, il alla courre le loup, et
au retour trouva le salon plein de joueurs, tellement qu'il ne se
contraignit pas plus que les autres. Mgr le duc de Bourgogne et M. le
duc de Berry ne voyaient Monsieur qu'en représentation, et ne pouvaient
être fort sensibles à sa perte. M\textsuperscript{me} la duchesse de
Bourgogne la fut extrêmement. C'était son grand-père, elle aimait
tendrement M\textsuperscript{me} sa mère, qui aimait fort Monsieur, et
Monsieur marquait toutes sortes de soins, d'amitié et d'attentions à
M\textsuperscript{me} la duchesse de Bourgogne, et l'amusait de toutes
sortes de divertissements. Quoiqu'elle n'aimât pas grand'chose, elle
aimait Monsieur, et elle souffrit fort de contraindre sa douleur, qui
dura assez longtemps dans son particulier. On a vu ci-dessus en deux
mots quelle fut la douleur de Madame.

Pour M. de Chartres la sienne fut extrême\,; le père et le fils
s'aimaient tendrement. Monsieur était doux, le meilleur homme du monde,
qui n'avait jamais contraint ni retenu M. son fils. Avec le cœur,
l'esprit était aussi fort touché. Outre la grande parure dont lui était
un père frère du roi, il lui était une barrière derrière laquelle il se
mettait à couvert du roi, sous la coupe duquel il retombait en plein. Sa
grandeur, sa considération, l'aisance de sa maison et de sa vie en
allaient dépendre sans milieu. L'assiduité, les bienséances, une
certaine règle, et pis que tout cela pour lui, une conduite toute
différente avec M\textsuperscript{me} sa femme, allaient devenir la
mesure de tout ce qu'il pouvait attendre du roi. M\textsuperscript{me}
la duchesse de Chartres, quoique bien traitée de Monsieur, fut ravie
d'être délivrée d'une barrière entre le roi et elle qui laissait à M.
son mari toute liberté d'en user avec elle comme il lui plaisait, et des
devoirs qui la tiraient plus souvent qu'elle ne voulait de la cour pour
suivre Monsieur à Paris ou à Saint-Cloud, où elle se trouvait tout
empruntée comme en pays inconnu, avec tous visages qu'elle ne voyait
jamais que là, qui tous étaient pour la plupart fort sur le pied gauche
avec elle, et sous les mépris et les humeurs de Madame qui ne les lui
épargnait pas. Elle compta donc ne plus quitter la cour, n'avoir plus
affaire à la cour de Monsieur, et que Madame et M. le duc de Chartres
seraient obligés à l'avenir d'avoir pour elle des manières et des égards
qu'elle n'avait pas encore éprouvés.

Le gros de la cour perdit en Monsieur\,: c'était lui qui y jetait les
amusements, l'âme, les plaisirs, et quand il la quittait tout y semblait
sans vie et sans action. À son entêtement près pour les princes, il
aimait l'ordre des rangs, des préférences, des distinctions\,; il les
faisait garder tant qu'il pouvait, et il en donnait l'exemple\,; il
aimait le grand monde, il avait une affabilité et une honnêteté qui lui
en attiraient foule, et la différence qu'il savait faire, et qu'il ne
manquait jamais de faire, des gens suivant ce qu'ils étaient, y
contribuait beaucoup. À sa réception, à son attention plus ou moins
grande ou négligée, à ses propos, il faisait continuellement toute la
différence qui flattait de la naissance et de la dignité, de l'âge et du
mérite, et de l'état des gens, et cela avec une dignité naturellement en
lui, et une facilité de tous les moments qu'il s'était formée. Sa
familiarité obligeait, et se conservait sa grandeur naturelle sans
repousser, mais aussi sans tenter les étourdis d'en abuser. Il visitait
et envoyait où il le devait faire, et il donnait chez lui une entière
liberté sans que le respect et le plus grand air de cour en souffrît
aucune diminution. Il avait appris et bien retenu de la reine sa mère
l'art de la tenir. Aussi la voulait-il pleine, et y réussissait. Par ce
maintien la foule était toujours au Palais-Royal.

À Saint-Cloud où toute sa nombreuse maison se rassemblait, il avait
beaucoup de dames qui à la vérité n'auraient guère été reçues ailleurs,
mais beaucoup de celles-là du haut parage, et force joueurs. Les
plaisirs de toutes sortes de jeux, de la beauté singulière du lieu que
mille calèches rendaient aisé aux plus paresseuses pour les promenades,
des musiques, de la bonne chère, en faisaient une maison de délices,
avec beaucoup de grandeur et de magnificence, et tout cela sans aucun
secours de Madame, qui dînait et soupait avec les dames de Monsieur, se
promenait quelquefois en calèche avec quelques-unes, boudait souvent la
compagnie, s'en faisait craindre par son humeur dure et farouche, et
quelquefois par ses propos, et passait toute la journée dans un cabinet
qu'elle s'était choisi, où les fenêtres étaient à plus de dix pieds de
terre, à considérer les portraits des palatins et d'autres princes
allemands dont elle l'avait tapissé, et à écrire des volumes de lettres
tous les jours de sa vie et de sa main, dont elle faisait elle-même les
copies qu'elle gardait. Monsieur n'avait pu la ployer à une vie plus
humaine et la laissait faire, et vivait honnêtement avec elle, sans se
soucier de sa personne avec qui il n'était presque point en particulier.
Il recevait à Saint-Cloud beaucoup de gens qui de Paris et de Versailles
lui allaient faire leur cour les après-dînées. Princes du sang, grands
seigneurs, ministres, hommes et femmes n'y manquaient point de temps en
temps encore ne fallait-il pas que ce fût en passant, c'est-à-dire en
allant de Paris à Versailles, ou de Versailles à Paris. Il le demandait
presque toujours, et montrait si bien qu'il ne comptait pas ces visites
en passant, que peu de gens l'avouaient.

Du reste Monsieur, qui avec beaucoup de valeur avait gagné la bataille
de Cassel, et qui en avait toujours montré une fort naturelle en tous
les sièges où il s'était trouvé, n'avait d'ailleurs que les mauvaises
qualités des femmes. Avec plus de monde que d'esprit, et nulle lecture,
quoique avec une connaissance étendue et juste des maisons, des
naissances et des alliances, il n'était capable de rien. Personne de si
mou de corps et d'esprit, de plus faible, de plus timide, de plus
trompé, de plus gouverné, ni de plus méprisé par ses favoris, et très
souvent de plus malmené par eux. Tracassier et incapable de garder aucun
secret, soupçonneux, défiant, semant des noises dans sa cour pour
brouiller, pour savoir, souvent aussi pour s'amuser, et redisant des uns
aux autres. Avec tant de défauts destitués de toutes vertus, un goût
abominable que ses dons et les fortunes qu'il fit à ceux qu'il avait
pris en fantaisie avaient rendu public avec le plus grand scandale, et
qui n'avait point de bornes pour le nombre ni pour les temps. Ceux-là
avaient tout de lui, le traitaient souvent avec beaucoup d'insolence, et
lui donnaient souvent aussi de fâcheuses occupations pour arrêter les
brouilleries de jalousies horribles\,; et tous ces gens-là ayant leurs
partisans rendaient cette petite cour très orageuse, sans compter les
querelles de cette troupe de femmes décidées de la cour de Monsieur, la
plupart fort méchantes, et presque toutes plus que méchantes, dont
Monsieur se divertissait, et entrait dans toutes ces misères-là.

Le chevalier de Lorraine et Châtillon y avaient fait une grande fortune
par leur figure, dont Monsieur s'était entêté plus que de pas une autre.
Le dernier, qui n'avait ni pain, ni sens, ni esprit, s'y releva, et y
acquit du bien. L'autre prit la chose en guisard qui ne rougit de rien
pourvu qu'il arrive, et mena Monsieur le bâton haut toute sa vie, fut
comblé d'argent et de bénéfices, fit pour sa maison ce qu'il voulut,
demeura toujours publiquement le maître chez Monsieur, et comme il avait
avec la hauteur des Guise leur art et leur esprit, il sut se mettre
entre le roi et Monsieur, et se faire ménager, pour ne pas dire craindre
de l'un et de l'autre, et jouir d'une considération, d'une distinction
et d'un crédit presque aussi marqué de la part du roi que de celle de
Monsieur. Aussi fut-il bien touché, moins de sa perte que de celle de
cet instrument qu'il avait su si grandement faire valoir pour lui. Outre
les bénéfices que Monsieur lui avait donnés, l'argent manuel qu'il en
tirait tant qu'il voulait, les pots-de-vin qu'il taxait et qu'il prenait
avec autorité sur tous les marchés qui se faisaient chez Monsieur, il en
avait une pension de dix mille écus, et le plus beau logement du
Palais-Royal et de Saint-Cloud. Les logements, il les garda à prière de
M. le duc de Chartres, mais il ne voulut pas accepter la continuation de
la pension par grandeur, comme par grandeur elle lui fut offerte.

Quoiqu'il fût difficile d'être plus timide et plus soumis qu'était
Monsieur avec le roi, jusqu'à flatter ses ministres et auparavant ses
maîtresses, il ne laissait pas de conserver avec un grand air de
respect, l'air de frère et des façons libres et dégagées. En particulier
il se licenciait bien davantage, il se mettait toujours dans un
fauteuil, et n'attendait pas que le roi lui dît de s'asseoir. Au
cabinet, après le souper du roi, il n'y avait aucun prince assis que
lui, non pas même Monseigneur\,; mais pour le service, et pour
s'approcher du roi ou le quitter, aucun particulier ne le faisait avec
plus de respect, et il mettait naturellement de la grâce et de la
dignité en toutes ses actions les plus ordinaires. Il ne laissait pas de
faire au roi par-ci par-là des pointes, mais cela ne durait pas\,; et
comme son jeu, Saint-Cloud et ses favoris lui coûtaient beaucoup, avec
de l'argent que le roi lui donnait il n'y paraissait plus. Jamais
pourtant il n'a pu se ployer à M\textsuperscript{me} de Maintenon, ni se
passer d'en lâcher de temps en temps quelques bagatelles au roi, et
quelques brocards au monde. Ce n'était pas sa faveur qui le blessait,
mais d'imaginer que la Scarron était devenue sa belle-sœur\,: cette
pensée lui était insupportable.

Il était extrêmement glorieux, mais sans hauteur, fort sensible et fort
attaché à tout ce qui lui était dû. Les princes du sang avaient fort
haussé dans leurs manières à l'appui de tout ce qui avait été accordé
aux bâtards, non pas trop M. le prince de Conti qui se contentait de
profiter sans entreprendre, mais M. le Prince, et surtout M. le Duc, qui
de proche en proche évita les occasions de présenter le service à
Monsieur, ce qui n'était pas difficile, et qui eut l'indiscrétion de se
vanter qu'il ne le servirait point. Le monde est plein de gens qui
aiment à faire leur cour aux dépens des autres, Monsieur en fut bientôt
averti\,; il s'en plaignit au roi fort en colère, qui lui répondit que
cela ne valait pas la peine de se fâcher, mais bien celle de trouver
occasion de s'en faire servir, et s'il le refusait de lui faire un
affront. Monsieur, assuré du roi, épia l'occasion. Un matin qu'il se
levait à Marly, où il logeait dans un des quatre appartements bas, il
vit par sa fenêtre M. le Duc dans le jardin, il l'ouvre vite et
l'appelle. M. le Duc vient, Monsieur se recule, lui demande où il va,
l'oblige toujours reculant d'entrer et d'avancer pour lui répondre, et
de propos en propos dont l'un n'attendait pas l'autre, tire sa robe de
chambre. À l'instant le premier valet de chambre présente la chemise à
M. le Duc, à qui le premier gentilhomme de la chambre de Monsieur fit
signe de le faire. Monsieur cependant défaisant la sienne, et M. le Duc,
pris ainsi au trébuchet, n'osa faire la moindre difficulté de la donner
à Monsieur. Dès que Monsieur l'eut reçue, il se mit à rire, et à dire\,:
«\,Adieu, mon cousin, allez-vous-en, je ne veux pas vous retarder
davantage.\,» M. le Duc sentit toute la malice et s'en alla fort fâché,
et le fut après encore davantage par les propos de hauteur que Monsieur
en tint.

C'était un petit homme ventru, monté sur des échasses tant ses souliers
étaient hauts, toujours paré comme une femme, plein de bagues, de
bracelets, de pierreries partout avec une longue perruque, tout étalée
en devant, noire et poudrée, et des rubans partout où il en pouvait
mettre, plein de toutes sortes de parfums, et en toutes choses la
propreté même. On l'accusait de mettre imperceptiblement du rouge. Le
nez fort long, la bouche et les yeux beaux, le visage plein mais fort
long. Tous ses portraits lui ressemblent. J'étais piqué à le voir qu'il
fit souvenir qu'il était fils de Louis VIII à ceux de ce grand prince,
duquel, à la valeur près, il était si complètement dissemblable.

Le samedi 11 juin, la cour retourna à Versailles, où, en arrivant, le
roi alla voir Madame, M. et M\textsuperscript{me} de Chartres, chacun
dans leur appartement. Elle, fort en peine de la situation où elle se
trouvait avec le roi dans une occasion où il y allait du tout pour elle,
et avait engagé la duchesse de Ventadour de voir M\textsuperscript{me}
de Maintenon. Elle le fit\,; M\textsuperscript{me} de Maintenon ne
s'expliqua qu'en général, et dit seulement qu'elle irait chez Madame au
sortir de son dîner, et voulut que M\textsuperscript{me} de Ventadour se
trouvât chez Madame et fût en tiers pendant sa visite. C'était le
dimanche, le lendemain du retour de Marly. Après les premiers
compliments ce qui était là sortit, excepté M\textsuperscript{me} de
Ventadour. Alors Madame fit asseoir M\textsuperscript{me} de Maintenon,
et il fallait pour cela qu'elle en sentît tout le besoin. Elle entra en
matière sur l'indifférence avec laquelle le roi l'avait traitée pendant
toute sa maladie, et M\textsuperscript{me} de Maintenon la laissa dire
tout ce qu'elle voulut\,; puis lui répondit que le roi lui avait ordonné
de lui dire que leur perte commune effaçait tout dans son cœur, pourvu
que dans la suite il eût lieu d'être plus content d'elle qu'il n'avait
eu depuis quelque temps, non seulement sur ce qui regardait ce qui
s'était passé à l'égard de M. le duc de Chartres, mais sur d'autres
choses encore plus intéressantes dont il n'avait pas voulu parler, et
qui étaient la vraie cause de l'indifférence qu'il avait voulu lui
témoigner pendant qu'elle avait été malade. À ce mot, Madame, qui se
croyait bien assurée, se récrie, proteste, qu'excepté le fait de son
fils elle n'a jamais rien dit ni fait qui pût déplaire, et enfile des
plaintes et des justifications. Comme elle y insistait le plus,
M\textsuperscript{me} de Maintenon tire une lettre de sa poche et là lui
montre, en lui demandant si elle en connaissait l'écriture. C'était une
lettre de sa main à sa tante la duchesse d'Hanovre, à qui elle écrivait
tous les ordinaires, où après des nouvelles de cour elle lui disait en
propres termes\,: qu'on ne savait plus que dire du commerce du roi et de
M\textsuperscript{me} de Maintenon, si c'était mariage ou concubinage\,;
et de là tombait sur les affaires du dehors et sur celles du dedans, et
s'étendait sur la misère du royaume qu'elle disait ne s'en pouvoir
relever. La poste l'avait ouverte, comme elle les ouvrait et les ouvre
encore presque toutes, et l'avait trouvée trop forte pour se contenter à
l'ordinaire d'en donner un extrait, et l'avait envoyée au roi en
original. On peut penser si, à cet aspect et à cette lecture, Madame
pensa mourir sur l'heure. La voilà à pleurer, et M\textsuperscript{me}
de Maintenon à lui représenter modestement l'énormité de toutes les
parties de cette lettre, et en pays étranger\,; enfin
M\textsuperscript{me} de Ventadour à verbiager pour laisser à Madame le
temps de respirer et de se remettre assez pour dire quelque chose. Sa
meilleure excuse fut l'aveu de ce qu'elle ne pouvait nier, des pardons,
des repentirs, des prières, des promesses.

Quand tout cela fut épuisé, M\textsuperscript{me} de Maintenon la
supplia de trouver bon qu'après s'être acquittée de la commission que le
roi lui avait donnée, elle pût aussi lui dire un mot d'elle-même, et lui
faire ses plaintes de ce que, après l'honneur qu'elle lui avait fait
autrefois de vouloir bien désirer son amitié et de lui jurer la sienne,
elle avait entièrement changé depuis plusieurs années. Madame crut avoir
beau champ. Elle répondit qu'elle était d'autant plus aise de cet
éclaircissement, que c'était à elle à se plaindre du changement de
M\textsuperscript{me} de Maintenon, qui tout d'un coup l'avait laissée
et abandonnée et forcée de l'abandonner à la fin aussi, après avoir
longtemps essayé de la faire vivre avec elle comme elles avaient vécu
auparavant. À cette seconde reprise, M\textsuperscript{me} de Maintenon
se donna le plaisir de la laisser enfiler comme à l'autre les plaintes
et de plus les regrets et les reproches, après quoi elle avoua à Madame
qu'il était vrai que c'était elle qui la première s'était retirée
d'elle, et qui n'avait osé s'en rapprocher, que ses raisons étaient
telles qu'elle n'avait pu moins que d'avoir cette conduite\,; et par ce
propos fit redoubler les plaintes de Madame, et son empressement de
savoir quelles pouvaient être ses raisons. Alors M\textsuperscript{me}
de Maintenon lui dit que c'était un secret qui jusqu'alors n'était
jamais sorti de sa bouche\,; quoiqu'elle en fût en liberté depuis dix
ans qu'était morte celle qui le lui avait confié sur sa parole de n'en
parler à personne, et de là raconte à Madame mille choses plus
offensantes les unes que les autres qu'elle avait dites d'elle à
M\textsuperscript{me} la Dauphine, lorsqu'elle était mal avec cette
dernière, qui dans leur raccommodement les lui avait redites de mot à
mot. À ce second coup de foudre Madame demeura comme une statue. Il y
eut quelques moments de silence. M\textsuperscript{me} de Ventadour fit
son même personnage pour laisser reprendre les esprits à Madame, qui ne
sut faire que comme l'autre fois, c'est-à-dire qu'elle pleura, cria, et
pour fin demanda pardon, avoua, puis repentirs et supplications.
M\textsuperscript{me} de Maintenon triompha froidement d'elle assez
longtemps, la laissant s'engouer de parler, de pleurer et lui prendre
les mains. C'était une terrible humiliation pour une si rogue et fière
Allemande. À la fin, M\textsuperscript{me} de Maintenon se laissa
toucher, comme elle avait bien résolu, après avoir pris toute sa
vengeance. Elles s'embrassèrent, elles se promirent oubli parfait et
amitié nouvelle. M\textsuperscript{me} de Ventadour se mit à en pleurer
de joie, et le sceau de la réconciliation fut la promesse de celle du
roi, et qu'il ne lui dirait pas un mot des deux matières qu'elles
venaient de traiter, ce qui plus que tout soulagea Madame. Tout se sait
enfin dans les cours, et si je me suis peut-être un peu étendu sur ces
anecdotes, c'est que je les ai sues d'original, et qu'elles m'ont paru
très curieuses.

Le roi qui n'ignorait ni la visite de M\textsuperscript{me} de Maintenon
à Madame, ni ce qu'il s'y devait traiter, donna quelque temps à cette
dernière de se remettre, puis alla le même jour chez elle ouvrir en sa
présence, et de M. le duc de Chartres, le testament de Monsieur, où se
trouvèrent le chancelier et son fils comme secrétaires d'État de la
maison du roi, et Terat, chancelier de Monsieur. Ce testament était de
1690, simple et sage, et nommait pour exécuteur celui qui se trouverait
premier président du parlement de Paris le jour de son ouverture. Le roi
tint la parole de M\textsuperscript{me} de Maintenon, il ne parla de
rien, et fit beaucoup d'amitiés à M\textsuperscript{me} et à M. le duc
de Chartres qui fut, et le terme n'est pas trop fort, prodigieusement
bien traité.

Le roi lui donna, outre les pensions qu'il avait et qu'il conserva,
toutes celles qu'avait Monsieur, ce qui fit six cent cinquante mille
livres\,; en sorte qu'avec son apanage et ses autres biens, Madame payée
de son douaire et de toutes ses reprises, il lui restait un million huit
cent mille livres de rente avec le Palais-Royal, en sus Saint-Cloud et
ses autres maisons. Il eut, ce qui ne s'était jamais vu qu'aux fils de
France, des gardes et des Suisses, les mêmes qu'avait Monsieur, sa salle
des gardes dans le corps du château de Versailles où était celle de
Monsieur, un chancelier, un procureur général, au nom duquel il
plaiderait et non au sien propre, et la nomination de tous les bénéfices
de son apanage excepté les évêchés, c'est-à-dire que tout ce qu'avait
Monsieur lui fut conservé en entier. En gardant ses régiments de
cavalerie et d'infanterie, il eut aussi ceux qu'avait Monsieur, et ses
compagnies de gens d'armes et de chevau-légers, et il prit le nom de duc
d'Orléans. Des honneurs si grands et si inouïs, et plus de cent mille
écus de pension au delà de celles de Monsieur, furent uniquement dus à
la considération de son mariage, aux reproches de Monsieur si récents
qu'il n'en aurait que la honte et rien de plus, et à la peine que
ressentit le roi de la situation où lui et Monsieur étaient ensemble,
qui avait pu avancer sa mort.

On s'accoutume à tout\,; mais d'abord ce prodigieux traitement surprit
infiniment. Les princes du sang en furent extrêmement mortifiés. Pour
les consoler, le roi incontinent après donna à M. le Prince tous les
avantages pour lui et pour sa maison, sa vie durant, de premier prince
du sang, comme M. son père les avait, et augmenta de dix mille écus sa
pension, qui était de quarante, pour qu'il en eût cinquante, qui est
celle de premier prince du sang.

M. de Chartres avait tout cela du vivant de Monsieur, quoique petit-fils
de France, mais devenu fort au-dessus par tout ce qui lui fut donné à la
mort de Monsieur, M. le Prince en profita. Les pensions de Madame et de
la nouvelle duchesse d'Orléans furent augmentées. Après qu'elles eurent
reçu les visites et les ambassadeurs, et que les quarante jours furent
passés, pendant lesquels le roi visita souvent Madame, elle alla chez
lui, chez les fils de France, chez M\textsuperscript{me} la duchesse de
Bourgogne, qui l'avaient, excepté le roi, été tous voir en grand manteau
et en mante, et à Saint-Germain en grand habit de veuve, après quoi
elle, eut permission de souper tous les soirs en public avec le roi à
l'ordinaire, d'être de tous les Marlys et de paraître partout sans
maille, sans voile, sans bandeau, qui, à ce qu'elle disait, lui faisait
mal à la tête. Pour le reste de cet équipage lugubre, le roi le supprima
pour ne point voir tous les jours des objets si tristes. Il ne laissa
pas de paraître fort étrange de voir Madame en public, et même à la
messe de Monseigneur en musique, à côté de lui, où était toute la cour,
enfin partout en tourière de filles de Sainte-Marie à leur croix près,
sous prétexte qu'étant avec le roi et chez lui elle était en famille.
Ainsi il ne fut pas question un instant de couvent ni de Montargis, et
elle garda à Versailles l'appartement de Monsieur avec le sien. Il n'y
eut donc que la chasse de retranchée pour un temps et les spectacles\,;
encore le roi la fit-il venir souvent chez M\textsuperscript{me} de
Maintenon l'hiver suivant, où on jouait devant lui des comédies avec de
la musique, et toujours sous prétexte de famille, et là de particulier.
Le roi lui permit d'ajouter à ses dames, mais sans nom, pour être
seulement de sa suite, la maréchale de Clérembault et la comtesse de
Beuvron, qu'elle aimait fort. Monsieur avait chassé l'une et l'autre du
Palais-Royal\,; la première étant gouvernante de ses filles, à la place
de laquelle il mit la maréchale de Grancey, et M\textsuperscript{me} de
Maré, sa fille, dans la suite. L'autre était veuve d'un capitaine de ses
gardes, frère du marquis de Beuvron et de la duchesse d'Arpajon. Madame
leur donna quatre mille livres de pension à chacune, et le roi deux
logements à Versailles auprès de celui de Madame, et les mena toujours
depuis toutes deux à Marly, ce qui fut réglé une fois pour toutes. Avant
cela, elle voyait peu la maréchale de Clérembault, que Monsieur
haïssait, et point du tout la comtesse de Beuvron, qu'il haïssait encore
davantage pour des tracasseries et des intrigues du Palais-Royal. Très
rarement elle la voyait dans quelque intérieur de couvent à Paris en
cachette\,; mais à découvert elle lui écrivait tous les jours de sa vie
par un page qu'elle lui envoyait de quelque lieu où elle fût.

Le roi drapa six mois et fit tous les frais de la superbe pompe funèbre.
Le lundi, 13 juin, toute la cour parut en long manteau devant le roi.
Monseigneur, qui était venu le matin de Meudon, quitta le sien seulement
pour le conseil, au sortir duquel il alla à Saint-Cloud en long manteau
donner l'eau bénite avec tous les princes du sang, et M. de Vendôme, et
force ducs, tous en rang d'ancienneté, et fut reçu au carrosse par M. le
duc d'Orléans et la maison de Monsieur. L'abbé de Grancey, premier
aumônier de Monsieur, lui présenta le goupillon et aux deux fils de
France ses fils\,; un autre aumônier à tous les autres.

L'après-dînée du même jour, toutes les dames vinrent en mante chez
M\textsuperscript{me} la duchesse de Bourgogne, qui y était aussi avec
toutes les princesses du sang. Le cercle assis il ne dura qu'un moment,
et M\textsuperscript{me} la duchesse de Bourgogne, suivie de toute cette
cour, alla chez le roi, chez Madame, chez M. et chez
M\textsuperscript{me} la duchesse d'Orléans, puis monta en carrosse au
derrière avec M\textsuperscript{me} la grande-duchesse, trois princesses
du sang au devant, M\textsuperscript{me} la Duchesse à une portière et
la duchesse du Lude à l'autre, suivie de cinquante dames dans ses
carrosses ou dans des carrosses du roi. Tout y fut en confusion. Il plut
aux princesses du sang, dont chacune devait avoir un des carrosses, de
se mettre toutes dans celui de M\textsuperscript{me} la duchesse de
Bourgogne. On ne pouvait s'y attendre, parce que c'était la première
fois que cela était arrivé, et je ne sais quel avantage elles crurent y
trouver. Cela dérangea l'ordre des autres carrosses qui étaient réglés à
l'avantage des duchesses sur les princesses, dont M\textsuperscript{me}
d'Elbœuf se jeta de dépit dans le dernier carrosse. La princesse
d'Harcourt avait fait tant de vacarme à M\textsuperscript{me} de
Maintenon que, pour la première fois encore, le roi ordonna que, s'il y
avait des princesses, personne ne donnerait d'eau bénite que les
princesses du sang\,; et cela fut exécuté. Les cris furent horribles, et
M\textsuperscript{me} la duchesse de Bourgogne, qui huit jours
auparavant avait été à Saint-Cloud, où Monsieur lui avait donné une
grande collation et une espèce de fête, fut si affligée qu'elle s'en
trouva mal, et fut longtemps dans l'appartement de M. le duc d'Orléans
avant de pouvoir aller donner l'eau bénite. M. le Duc, qui devait mener
le corps pour prince du sang avec M. de La Trémoille pour duc, aima
mieux conduire le cœur au Val-de-Grâce pour en être plus tôt quitte, et
laissa mener le corps à M. le prince de Conti et à M. de Luxembourg. Le
service fut superbe, où les cours assistèrent\,; et où Mgr le duc de
Bourgogne, M. le duc de Berry, et M. le duc d'Orléans furent les princes
du deuil, parce que Monseigneur, peu éloigné encore de l'accident qu'il
avait eu, ne voulut pas s'exposer à la longueur et à la chaleur de la
cérémonie. M. de Langres fit l'oraison funèbre, et s'en acquitta assez
bien. Cela lui convenait. Le comte de Tonnerre, son frère, avait passé
presque toute sa vie dans la charge de premier gentilhomme de la chambre
de Monsieur.

Je ne puis finir sur ce prince sans raconter une anecdote, qui a été sue
de bien peu de gens, sur la mort de Madame\footnote{Henriette
  d'Angleterre. Voy., notes à la fiai du volume.} que personne n'a douté
qui n'eût été empoisonnée, et même grossièrement. Ses galanteries
donnaient de la jalousie à Monsieur. Le goût opposé de Monsieur
indignait Madame. Les favoris qu'elle haïssait semaient tant qu'ils
pouvaient la division entre eux, pour disposer de Monsieur tout à leur
aise. Le chevalier de Lorraine, dans le fort de sa jeunesse et de ses
agréments, étant né en 1643, possédait Monsieur avec empire, et le
faisait sentir à Madame comme à toute la maison. Madame, qui n'avait
qu'un an moins que lui, et qui était charmante, ne pouvait à plus d'un
titre souffrir cette domination\,; elle était au comble de faveur et de
considération auprès du roi, dont elle obtint enfin l'exil du chevalier
de Lorraine. À cette nouvelle Monsieur s'évanouit, puis fondit en larmes
et s'alla jeter aux pieds du roi pour faire révoquer un ordre qui le
mettait au dernier désespoir. Il ne put y réussir\,; il entra en fureur,
et s'en alla à Villers-Cotterêts. Après avoir bien jeté feu et flammes
contre le roi et contre Madame qui protestait toujours qu'elle n'y avait
point de part, il ne put soutenir longtemps le personnage de mécontent
pour une chose si publiquement honteuse. Le roi se prêta à le contenter
d'ailleurs, il eut de l'argent, des compliments, des amitiés, il revint
le cœur fort gros, et peu à peu vécut à l'ordinaire avec le roi et
Madame.

D'Effiat, homme d'un esprit hardi, premier écuyer de Monsieur, et le
comte de Beuvron, homme liant et doux, mais qui voulait figurer chez
Monsieur, dont il était capitaine des gardes, et surtout tirer de
l'argent pour se faire riche en cadet de Normandie fort pauvre, étaient
étroitement liés avec le chevalier de Lorraine dont l'absence nuisait
fort à leurs affaires, et leur faisait appréhender que quelque autre
mignon ne prît sa place, duquel ils ne s'aideraient pas si bien. Pas un
des trois n'espérait la fin de cet exil, à la faveur où ils voyaient
Madame, qui commençait même à entrer dans les affaires et à qui le roi
venait de faire faire un voyage mystérieux en Angleterre, où elle avait
parfaitement réussi, et en venait de revenir plus triomphante que
jamais. Elle était de juin 1644, et d'une très bonne santé\footnote{Voy.,
  à la fin du volume la note sur la mort de Madame.}, qui achevait de
leur faire perdre de vue le retour du chevalier de Lorraine. Celui-ci
étroit allé promener son dépit en Italie et à Rome. Je ne sais lequel
des trois y pensa le premier, niais le chevalier de Lorraine envoya à
ses deux amis un poison sûr et prompt par un exprès qui ne savait
peut-être pas lui-même ce qu'il portait.

Madame était à Saint-Cloud, qui, pour se rafraîchir, prenait depuis
quelque temps, sur les sept heures du soir, un verre d'eau de chicorée.
Un garçon de sa chambre avait soin de la faire. Il la mettait dans une
armoire d'une des antichambres de Madame, avec son verre, etc. Cette eau
de chicorée était dans un pot de faïence ou de porcelaine, et il y avait
toujours auprès d'autre eau commune, en cas que Madame trouvât celle de
chicorée trop amère, pour la mêler. Cette antichambre était le passage
public pour aller chez Madame, où il ne se tenait jamais personne, parce
qu'il y en avait plusieurs. Le marquis d'Effiat avait épié tout cela. Le
29 juin 1670, passant par cette antichambre, il trouva le moment qu'il
cherchait, personne dedans, et il avait remarqué qu'il n'était suivi de
personne qui allât aussi chez Madame\,; il se détourne, va à l'armoire,
l'ouvre, jette son boucon, puis entendant quelqu'un, s'arme de l'autre
pot d'eau commune, et comme il le remettait, le garçon de la chambre,
qui avait le soin de cette eau de chicorée, s'écrie, court à lui, et lui
demande brusquement ce qu'il va faire à cette armoire. D'Effiat, sans
s'embarrasser le moins du monde, lui dit qu'il lui demande pardon, mais
qu'il crevait de soif, et que sachant qu'il y avait de l'eau là dedans,
lui montrant le pot d'eau commune, il n'a pu résister à en aller boire.
Le garçon grommelait toujours, et l'autre toujours l'apaisant et
s'excusant, entre chez Madame, et va causer comme les autres courtisans,
sans la plus légère émotion. Ce qui suivit, une heure après, n'est pas
de mon sujet, et n'a que trop fait de bruit par toute l'Europe.

Madame étant morte le lendemain 30 juin, à trois heures du matin, le roi
fut pénétré de la plus grande douleur. Apparemment que dans la journée
il eut des indices, et que ce garçon de chambre ne se tut pas, et qu'il
y eut notion que Purnon, premier maître d'hôtel de Madame, était dans le
secret, par la confidence intime où, dans son bas étage, il était avec
d'Effiat. Le roi couché, il se relève, envoie chercher Brissac, qui dès
lors était dans ses gardes et fort sous sa main, lui commande de choisir
six gardes du corps bien sûrs et secrets, d'aller enlever le compagnon,
et de le lui amener dans ses cabinets par les derrières. Cela fut
exécuté avant le matin. Dès que le roi l'aperçut, il fit retirer Brissac
et son premier valet de chambre, et prenant un visage et un ton à faire
la plus grande terreur\,: «\,Mon ami, lui dit-il en le regardant depuis
les pieds jusqu'à la tête, écoutez-moi bien\,: si vous m'avouez tout, et
que vous me répondiez vérité sur ce que je veux savoir de vous, quoi que
vous ayez fait, je vous pardonne, et il n'en sera jamais mention. Mais
prenez garde à ne me pas déguiser la moindre chose, car si vous le
faites, vous êtes mort avant de sortir d'ici. Madame n'a-t-elle pas été
empoisonnée\,? --- Oui, sire, lui répondit-il. --- Et qui l'a
empoisonnée, dit le roi, et comment l'a-t-on fait\,?» Il répondit que
c'était le chevalier de Lorraine qui avait envoyé le poison à Beuvron et
à d'Effiat, et lui conta ce que je viens d'écrire. Alors, le roi
redoublant d'assurance de grâce et de menace de mort\,: «\,Et mon frère,
dit le roi, le savait-il\,? --- Non, sire, aucun de nous trois n'était
assez sot pour le lui dire\,: il n'a point de secret\,; il nous aurait
perdus.\,» À cette réponse, le roi fit un grand ha\,! comme un homme
oppressé, et qui tout d'un coup respire. «\,Voilà, dit-il, tout ce que
je voulais savoir. Mais m'en assurez-vous bien\,?» Il rappela Brissac et
lui commanda de remener cet homme quelque part, où tout de suite il le
laissât aller en liberté. C'est cet homme lui-même qui l'a conté,
longues années depuis, à M. Joly de Fleury, procureur général du
parlement, duquel je tiens cette anecdote.

Ce même magistrat, à qui j'en ai reparlé depuis, m'apprit ce qu'il ne
m'avait pas dit la première fois, et le voici\,: Peu de jours après le
second mariage de Monsieur, le roi prit Madame en particulier, lui conta
ce fait, et ajouta qu'il la voulait rassurer sur Monsieur et sur
lui-même, trop honnête homme pour lui faire épouser son frère s'il était
capable d'un tel crime. Madame en fit son profit. Purnon, le même Cl.
Bonneau, était demeuré son premier maître d'hôtel. Peu à peu elle fit
semblant de vouloir entrer dans la dépense de sa maison, le fit trouver
bon à Monsieur, et tracassa si bien Purnon, qu'elle le fit quitter, et
qu'il vendit sa charge, sur la fin de 1674, au sieur Michel Viel de
Suranne.

\hypertarget{chapitre-x.}{%
\chapter{CHAPITRE X.}\label{chapitre-x.}}

1701

~

{\textsc{Guerre de fait en Italie.}} {\textsc{- Ségur gouverneur du pays
de Foix\,; son aventure et celle de l'abbesse de la Joye.}} {\textsc{-
Ses enfants.}} {\textsc{- Maréchal d'Estrées gouverneur de Nantes, et
lieutenant général et commandant en Bretagne.}} {\textsc{- Chamilly
commandant à la Rochelle et pays voisins.}} {\textsc{- Briord conseiller
d'État d'épée.}} {\textsc{- Abbé de Soubise sacré.}} {\textsc{- Mariage
de Vassé avec M\textsuperscript{lle} de Beringhen.}} {\textsc{- Mariage
de Renel avec une sœur de Torcy.}} {\textsc{- Mort du président Le
Bailleul.}} {\textsc{- Mort de Bartillat.}} {\textsc{- Mort du marquis
de Rochefort.}} {\textsc{- Mort de la duchesse douairière de
Ventadour.}} {\textsc{- Armenonville et Rouillé directeurs des
finances.}} {\textsc{- Le roi d'Espagne reçoit le collier de la Toison
et l'envoie aux ducs de Berry et d'Orléans, à qui le roi le donne.}}
{\textsc{- Marsin ambassadeur en Espagne\,; son caractère et son
extraction.}} {\textsc{- Raison du duc d'Orléans de désirer la Toison.}}
{\textsc{- Menées domestiques en Italie.}} {\textsc{- Situation de
Chamillart.}} {\textsc{- M\textsuperscript{lle} de Lislebonne et
M\textsuperscript{me} d'Espinoy, et leur éclat solide.}} {\textsc{-
Position de Vaudémont.}} {\textsc{- Tessé et ses vues.}} {\textsc{-
Combat de Carpi.}} {\textsc{- Maréchal de Villeroy va en Italie\,; mot à
lui du maréchal de Duras.}} {\textsc{- Le pape refuse l'hommage de
Naples, et y reconnaît et fait reconnaître Philippe V, où une révolte
est étouffée dès sa naissance.}}

~

Après s'être tant tâtés et regardés par toute l'Europe, la guerre enfin
se déclara de fait par les Impériaux en Italie par quelques coups de
fusil qu'ils tirèrent sur une vingtaine de soldats, à qui Pracontal
avait fait passer l'Adige au-dessous de Vicence, près d'Albaredo, où ils
étaient, pour amener un bac de notre côté. Ils tuèrent un Espagnol, et
prirent presque tous les autres, et ne les voulurent pas rendre,
quoiqu'on les eût envoyé répéter, et dirent qu'ils ne les rendraient
point que le cartel ne fût fait.

Le roi fit donc partir les officiers généraux. Tallard, qui en fut un,
avait fait de l'argent des petites charges que le roi lui avait données
à vendre en revenant d'Angleterre, entre autres le gouvernement du pays
de Foix, que la mort de Mirepoix avait fait vaquer, à Ségur, capitaine
de gendarmerie, bon gentilhomme de ce pays-là, et fort galant homme, qui
avait perdu une jambe à la bataille de la Marsaille.

Il avait été beau en sa jeunesse, et parfaitement bien fait, comme on le
voyait encore, doux, poli et galant. Il était mousquetaire noir, et
cette compagnie avait toujours son quartier à Nemours pendant que la
cour était à Fontainebleau. Ségur jouait très bien du luth\,; il
s'ennuyait à Nemours, il fit connaissance avec l'abbesse de la Joye, qui
est tout contre, et la charma si bien par les oreilles et par les yeux,
qu'il lui fit un enfant. Au neuvième mois de la grossesse, madame fut
bien en peine que devenir, et ses religieuses la croyaient fort malade.
Pour son malheur, elle ne prit pas assez tôt ses mesures, ou se trompa à
la justesse de son calcul. Elle partit, dit-elle, pour les eaux, et
comme les départs sont toujours difficiles, ce ne put être que tard, et
n'alla coucher qu'à Fontainebleau, dans un mauvais cabaret plein de
monde, parce que la cour y était alors. Cette couchée lui fut perfide,
le mal d'enfant la prit la nuit, elle accoucha. Tout ce qui était dans
l'hôtellerie entendit ses cris\,; on accourut à son secours, beaucoup
plus qu'elle n'aurait voulu, chirurgien, sage-femme, en un mot, elle en
but le calice en entier, et le matin ce fut la nouvelle.

Les gens du duc de Saint-Aignan la lui contèrent en l'habillant, et il
en trouva l'aventure si plaisante, qu'il en fit une gorge chaude au
lever du roi, qui était fort gaillard en ce temps-là, et qui rit
beaucoup de M\textsuperscript{me} l'abbesse et de son poupon, que, pour
se mieux cacher, elle était venue pondre en pleine hôtellerie au milieu
de la cour, et ce qu'on ne savait pas, parce qu'on ignorait d'où elle
était abbesse, à quatre lieues de son abbaye, ce qui fut bientôt mis au
net.

M. de Saint-Aignan, revenu chez lui, y trouva la mine de ses gens fort
allongée\,; ils se faisaient signe les uns aux autres, personne ne
disait mot\,; à la fin il s'en aperçut, et leur demanda à qui ils en
avaient\,; l'embarras redoubla, et enfin M. de Saint-Aignan voulut
savoir de quoi il s'agissait. Un valet de chambre se hasarda de lui dire
que cette abbesse dont on lui avait fait un si bon conte était sa fille,
et que depuis qu'il était allé chez le roi, elle avait envoyé chez lui
au secours pour la tirer du lieu où elle était. Qui fut bien penaud\,?
ce fut le duc qui venait d'apprendre cette histoire au roi et à toute la
cour, et qui, après en avoir bien fait rire tout le monde\,; en allait
devenir lui-même le divertissement. Il soutint l'affaire comme il put,
fit emporter l'abbesse et son bagage\,; et, comme le scandale en était
public, elle donna sa démission, et a vécu plus de quarante ans depuis,
cachée dans un autre couvent. Aussi n'ai-je presque jamais vu Ségur chez
M. de Beauvilliers, qui pourtant lui faisait politesse comme à tout le
monde.

C'est le père de Ségur qui était à M. le duc d'Orléans, et qui, pendant
la régence, épousa une de ses bâtardes, qui a servi avec distinction et
est devenu lieutenant général, et d'un aumônier du roi, qui fut fait et
sacré évêque de Saint-Papoul, et qui le quitta en 1739, pour un
mandement qui a tant fait de bruit dans le monde, et dont la vérité et
l'humilité l'ont couvert d'honneur et de gloire, comme la vie pénitente,
dépouillée et cachée qu'il mène depuis, en fera vraisemblablement un de
ces saints rares, et dont le sublime exemple sera un terrible jugement
pour bien des prélats.

Le gouvernement de Nantes et la lieutenance générale de cette partie de
Bretagne fut donnée au maréchal d'Estrées, pour commander en chef dans
la province. Il y avait longtemps qu'il vaquait par la mort de Rosmadec.
Beaucoup de gens l'avaient demandé, et M. le comte de Toulouse fortement
pour d'O, qui, avec son importance, se donnait pour être à portée de
tout. Chamillart, dont la femme était parente et amie de
M\textsuperscript{me} de Chamilly, fit donner le commandement de la
Rochelle, Aunis, Poitou, etc., que le maréchal d'Estrées quittait, à
Chamilly, et remit ainsi à flot cet ancien lieutenant général, illustré
par bien des sièges, et surtout par la célèbre défense de Grave, mais
noyé par Louvois et par Barbezieux, son fils. Briord qui avait fort bien
fait eu son ambassade de Hollande, où il avait pensé mourir, eut une des
trois places vacantes depuis fort longtemps de conseiller d'État d'épée,
qui fut une belle fortune pour un écuyer de M. le Prince.

Enfin les bulles et tout ce qu'il fallait pour l'abbé de Soubise étant
arrivées, il fut sacré le dimanche 26 juin, à vingt-sept ans tout juste,
par le cardinal de Fürstemberg, dans Saint-Germain des Prés, assisté des
évêques-ducs de Laon et de Langres, tous deux Clermont, en présence de
la plus grande et de la plus illustre compagnie. Il n'y avait point de
plus beaux visages, chacun pour leur âge, que ceux du consécrateur et du
consacré\,; ceux des deux assistants y répondaient\,; les plus belles
dames et les mieux parées y firent cortège à l'Amour, qui ordonnait la
fête avec les Grâces, les Jeux et les Ris\,; ce qui la fit la plus
noble, la plus superbe, la plus brillante et la plus galante qu'il fût
possible de voir.

Avant de quitter les particuliers, il faut dire que le premier écuyer
avait marié depuis peu sa fille à Vassé, dont la mère, seconde fille du
maréchal d'Humières, s'était remariée à Surville, cadet d'Hautefort, et
en fut longtemps sans que sa famille la voulût voir\,; et Torcy maria
aussi sa seconde sœur à Renel, dont le père avait été tué mestre de camp
général de la cavalerie, et qui était Clermont-Gallerande\,; il y avait
longtemps que l'aînée de celle-ci avait épousé Bouzols.

Deux hommes de singulière vertu moururent en même temps\,: Le Bailleul,
retiré depuis longtemps à Saint-Victor dans une grande piété, étant
l'ancien des présidents à mortier, il avait cédé sa charge à son fils,
qu'il avait longuement exercée avec grande probité. Il était fils du
surintendant des finances, et frère de la mère du marquis d'Huxelles et
de celle de Saint-Germain-Beaupré. C'était un homme rien moins que
président à mortier\,; car il était doux, modeste et tout à fait à sa
place. D'ailleurs, obligeant et gracieux autant que la justice le lui
pouvait permettre. Aussi était-il aimé et estimé, au point que personne
n'ayant plus besoin de lui, et n'y ayant chez lui ni jeu ni table, il
était extrêmement visité à Saint-Victor, et de quantité de gens
considérables, quoiqu'il ne sortît guère de cette retraite. Il fut aussi
fort regretté\,; je l'allais voir assez souvent, parce qu'il avait
toujours été fort des amis de mon père. L'autre fut le bonhomme
Bartillat, homme de peu, et qui, dans sa charge de garde du trésor
royal, s'était illustré par sa fidélité, son exactitude, son
désintéressement, sa frugalité et sa bonté. Aussi était-il demeuré
pauvre. Le roi qui l'aimait le voulait voir de temps en temps et lui
faisait toujours amitié. Il avait été trésorier de la reine mère, et je
l'ai toujours vu fort accueilli de ce qu'il y avait de principal à la
cour. Il avait près de quatre-vingt-dix ans, et laissa un fils qu'il eut
la joie de voir aussi applaudi dans le métier de la guerre, où il devint
lieutenant général avec un gouvernement, qu'il l'avait été dans celui
des finances.

La maréchale de Rochefort perdit aussi son fils unique qui n'était point
marié, et qui à force de débauches avait, à la fleur de son âge,
quatre-vingts ans. Il était menin de Monseigneur\,; on a vu comment en
son temps ce n'était rien du tout.

La maréchale de Duras perdit sa mère la vieille duchesse de Ventadour-La
Guiche qu'on ne voyait plus guère à l'hôtel de Duras, où elle logeait,
et qui depuis longtemps vivait chez elle en basse Normandie en très
grande dame qu'elle était et qu'elle savait bien faire.

Chamillart ne put enfin suffire au travail des finances et à celui de la
guerre à la fois, que celle où on allait entrer augmentait très
considérablement l'un et l'autre\,; mais il avait peine à réduire le roi
qui n'aimait pas les visages nouveaux. Pour réussir à se faire soulager,
il en fit une affaire de finance qui valut au roi un million cinquante
mille livres d'argent comptant. Pour cela on fit deux charges nouvelles
qu'on appela directeurs des finances, qui payèrent huit cent mille
livres chacune, et eurent quatre-vingt mille livres de rente, qui furent
données à deux personnages fort dissemblables, Armenonville et Rouillé.

Le premier, qui ne donna que quatre cent mille livres, parce qu'on
supprima sa charge d'intendant des finances qui lui avait coûté autant,
était un homme léger, gracieux, respectueux quoique familier, toujours
ouvert, toujours accessible, qu'on voyait peiné d'être obligé de
refuser, et ravi de pouvoir accorder, aimant le monde, la dépense et
surtout la bonne compagnie, qui était toujours nombreuse chez lui. Il
était frère très disproportionné d'âge de la femme de Pelletier le
ministre d'État, qui l'avait fait intendant des finances pendant qu'il
était contrôleur général. Outre cet accès et là faveur publique,
Saint-Sulpice le portait auprès de M\textsuperscript{me} de Maintenon à
cause du supérieur de tous ses séminaires, qui était fils de Pelletier,
le ministre, et il avait auprès du roi le crédit des jésuites à cause du
P. Fleuriau son frère qui l'était.

Rouillé, procureur général de la chambre des comptes, dont il accommoda
son beau-frère, Bouvard de Fourqueux, petit-fils du premier médecin de
Louis XIII, était un rustre brutal, bourru, plein d'humeur, qui, sans
vouloir être insolent, en usait comme font les insolents, dur, d'accès
insupportable, à qui les plus secs refus ne coûtaient rien, et qu'on ne
savait comment voir ni prendre\,; au reste, bon esprit, travailleur,
savant et capable, mais qui ne se déridait qu'avec des filles et entre
les pots, où il n'admettait qu'un petit nombre de familiers obscurs. M.
de Noailles qui tout dévotement était sournaisement dans le même goût
sous cent clefs, était son ami intime, et la débauche avait fait cette
liaison. Il cultivait fort tout ce qui sentait le ministère, surtout
celui de la finance et lui, ou plutôt sa femme qui avait plus d'esprit
et de vrai manège que lui, avaient toujours affaire à ceux qui s'en
mêlaient. Ils n'étaient pas encore riches\,; leur fille de Guiche
mourait de faim\,; ils avaient si bien fait auprès de
M\textsuperscript{me} de Maintenon, que le roi avait ordonné à
Pontchartrain, puis à Chamillart, quand il lui succéda aux finances, de
faire en faveur de la mère et de la fille toutes les affaires qu'elles
présenteraient, et de lui en procurer tant qu'ils pourraient, et il est
incroyable ce qu'elles en ont tiré. Ce fut donc pour M. de Noailles un
coup de partie et d'intérêt et d'amitié, de porter Rouillé en cette
place, et c'est ce qui lui donna la protection de M\textsuperscript{me}
de Maintenon. La fonction des deux directeurs fut de faire au conseil
des finances tous les rapports dont le contrôleur général était chargé,
après le lui avoir fait en particulier, tellement que cela le déchargea
de l'examen et du rapport d'une infinité d'affaires, et de travailler
avec lui. La charge d'intendant des finances, qu'avait eue pour rien
Breteuil, conseiller d'État, fut supprimée en lui donnant pourtant
cinquante mille écus\,; il ne laissa pas d'en être bien fâché. Ainsi il
n'en demeura que quatre, qui de garçons du contrôleur général qu'ils
étaient le devinrent des directeurs chez qui il leur fallut aller porter
le portefeuille, dont Caumartin pensa enrager, lui qui avait espéré
d'être contrôleur général après Pontchartrain, et qui sous lui était le
seul maître des finances\,; mais à force de bonne chère, de bonne
compagnie et de faire le grand seigneur, il s'était mis hors d'état de
se passer de sa charge, de sorte qu'il fallut en boire le calice.
Pelletier de Sousy eut le choix d'une des deux places de directeur en
supprimant sa charge d'intendant des finances, mais en homme sage, qui
était conseiller d'État, et qui était devenu une manière de tiercelet de
ministre par son emploi de directeur général des fortifications qui le
faisait travailler seul avec le roi une fois toutes les semaines, et qui
lui donnait un logement à Versailles et à Marly tous les voyages, avec
la distinction de n'avoir plus de manteau, mais seulement le rabat et la
canne, il aima mieux quitter sa charge d'intendant des finances, et la
donner à son fils qui, par ce début à l'âge de vingt-cinq ans, fut en
chemin d'aller à tout, comme il lui est arrivé dans la suite.

Le roi d'Espagne qui se préparait au voyage d'Aragon et de Catalogne
pour y prêter et y recevoir les serments accoutumés aux avènements à la
couronne d'Espagne, reçut en cérémonie le collier de l'ordre de la
Toison des mains du duc de Monteléon, le plus ancien chevalier de cet
ordre qui se trouvât lors en Espagne, et tout de suite y nomma M. le duc
de Berry et M. le duc d'Orléans, à qui quelque temps après le roi le
donna par commission du roi son petit-fils. La cérémonie s'en fit à la
messe, en la même façon et en même temps que les évêques nouvellement
sacrés y prêtent au roi leur serment de fidélité. Torcy y fit la
fonction de chancelier de la Toison. Comme il n'y avait ici aucun
chevalier de cet ordre, il n'y eut point de parrains, et les grands
habits de cérémonie qui appartiennent à l'ordre et non aux chevaliers,
étant demeurés en Flandre, ils ne se portaient point en Espagne, où on
recevait, et puis on portait le collier sur ses habits ordinaires, ce
qui fit que ces deux princes le reçurent de même de la main du roi.

M. d'Harcourt un peu rétabli, mais hors d'état de supporter aucune
fatigue ni aucun travail, obtint son rappel. Marsin\footnote{Saint-Simon
  écrit tantôt Marchin, tantôt Marsin\,; nous avons suivi, pour ce nom,
  la forme ordinairement adoptée.}, qui servait sous le maréchal Catinat
et qui était en Italie, fut choisi pour l'aller relever en la même
qualité. C'était un très petit homme, vif, sémillant, ambitieux, bas
complimenteur sans fin, babillard de même, dévot pourtant, et qui par là
avait plu à Charost avec qui il avait fort servi en Flandre, s'était
fait son ami, et par lui s'était fait goûter à M. de Cambrai et aux ducs
de Chevreuse et de Beauvilliers. Il ne manquait ni d'esprit ni de
manège, ne laissait pas, malgré ce flux de bouche, d'être de bonne
compagnie et d'être mêlé à l'armée avec la meilleure, et toujours bien
avec le général sous qui il servait. Tout cela le fit choisir pour cette
ambassade fort au-dessus de sa capacité et de son maintien. Il était
pauvre et fils de ce Marsin qui a tant fait parler de lui dans le parti
de M. le Prince, et à qui son mérite militaire et son manège entre les
diverses factions valurent enfin la Jarretière de Charles II au scandale
universel, parce que c'était un Liégeois de très peu de chose. C'était
en 1658 qu'il commandait l'armée d'Espagne aux Pays-Bas, et que
l'empereur le fit aussi comte de l'empire. Il eut des gouvernements et
des établissements qui lui firent épouser une Balzac-Entragues, cousine
germaine de la marquise de Verneuil qui devint héritière, mais dont le
fils, qui est celui dont je parle, n'en fut pas plus riche aussi
était-ce un panier percé. Il rendit compte au roi assez au long des
affaires militaires d'Italie. Il eut les mêmes appointements et
traitements pécuniaires qu'Harcourt\,; le roi voulut même qu'il eût en
tout un équipage et une maison pareille, lui dit de les commander, et
paya tout. Aussi Marsin n'était-il pas en état d'y fournir. Je l'avais
fort connu à l'armée et à la cour, et il venait souvent chez moi\,;
Charost aussi, qui était intimement de mes amis, avait fait cette
liaison entre nous, et Marsin l'avait fort désirée et la cultivait
soigneusement à cause de la mienne, si intime avec les ducs de
Beauvilliers et de Chevreuse, laquelle n'était plus ignorée de personne,
mais non encore sue au point d'intimité où elle était déjà, et de
confiance qui, de leur part, commençait à poindre.

Dès que le bagage de Marsin fut prêt, et il le fut bientôt, parce que le
roi payait, on le fit partir d'autant plus vite que le, Portugal se
joignit alors à l'Espagne, et que M. de Savoie signa le traité du
mariage de sa fille avec le roi d'Espagne, et celui de la jonction de
ses troupes avec les nôtres et celles d'Espagne en Italie qu'il devait
commander en chef, avec Catinat sous lui pour les nôtres, et Vaudemont
pour les espagnoles.

Je m'aperçois qu'en parlant de la Toison de M. le duc de Berry et de M.
le duc d'Orléans, j'ai oublié une chose importante. Le testament du roi
d'Espagne en faveur de la postérité de la reine sa sœur, épouse du roi,
n'avait point, à son défaut, rappelé celle de la reine sa tante, mère du
roi, mais au contraire M. de Savoie et sa postérité, plus éloignée que
celle de la reine mère. Monsieur et M. le duc d'Orléans firent donc
leurs protestations contre cette disposition seconde, et Louville vers
ce temps-ci les fit enregistrer au conseil de Castille. C'est ce qui fit
désirer à M. le duc d'Orléans d'avoir la Toison en même temps que M. le
duc de Berry, comme étant de droit appelé par sa ligne, du chef de la
reine sa grand'mère, à la couronne d'Espagne au défaut de toute celle de
la feue reine, épouse du roi. Retournons maintenant en Italie.

Pour bien entendre ce qui s'y passait dès lors et tout ce qui arriva
depuis, il en faut expliquer les ressorts et les manèges qui de l'un à
l'autre s'étendirent bien au delà dans la suite, et mirent l'État à
cieux doigts de sa perte. Il faut se souvenir de ce qui a été dit de la
fortune et du caractère de Chamillart, et ajouter que jamais ministre
n'a été si avant, non dans l'esprit du roi par l'estime de sa capacité,
mais dans son cour par un goût que, dès les premiers temps du billard,
il avait pris pour lui, qu'il lui avait continuellement marqué depuis
par toutes les distinctions, les avancements et les privances qu'il lui
pouvait donner, qu'il combla par les deux emplois des finances et de la
guerre dont il l'accabla, et qui s'augmentait tous les jours par les
aveux de Chamillart au roi de son ignorance sur bien des choses, et par
le petit et l'orgueilleux plaisir dans lequel le roi se baignait de
former, d'instruire et de conduire son ministre en deux fonctions si
principales. M\textsuperscript{me} de Maintenon n'avait pas moins de
tendresse pour lui, car c'est de ce nom que cette affection doit
s'appeler. Sa dépendance parfaite d'elle la charmait, et son amitié pour
lui plaisait extrêmement au roi. Un ministre dans cette position est
tout-puissant\,: cette position était visible\,; il n'y avait personne
qui ne se jetât bassement à lui. Ses lumières, des plus courtes, étaient
abandonnées à elles-mêmes par sa famille telle que je l'ai représentée,
et se trouvaient incapables d'un bon discernement. Il se livra à ses
anciens amis, à ceux qui l\,»avaient produit à la cour, et aux personnes
qu'il estima avoir une considération et un éclat qui méritait d'être
ménagé.

Matignon était des premiers\,: il avait vu son père intendant de Caen et
lui de Rouen\,; il avait été leur ami et, tout Normand très intéressé
qu'il était, il avait fait l'amitié à celui-ci de lui céder la mouvance
d'une terre qui relevait de Torigny. Cela avait tellement gagné le cœur
à Chamillart qu'il ne l'oublia jamais, que Matignon eut tout pouvoir sur
lui dans tout le cours de son ministère, et qu'il en tira des millions,
lui et Marsan son beau-frère et son ami intime, qu'il lui produisit, et
qui par ses bassesses se le dévoua. Aussi M. le Grand, son frère, qui
aimait fort Chamillart, qui était un de ceux qui l'avaient produit au
billard, et pour qui Chamillart avait la plus grande et la plus
respectueuse déférence, appelait publiquement son frère de Marsan le
chevalier de La Proustière, et lui tombait rudement dessus pour la cour
indigne niais très utile qu'il faisait à Chamillart.

Des seconds étaient le même M. le Grand et le maréchal de Villeroy, dont
le grand air de faveur et celui d'autorité qu'ils prirent aisément sur
lui, et ces manières de supériorité qu'ils usurpaient à la cour, lui
imposaient et l'étourdissaient\,; et il leur était d'autant plus soumis
que ce n'était pas pour de l'argent comme les deux autres. Par ceux-là
il se trouva peu à peu lié avec la duchesse de Ventadour, amie intime et
de tout temps quelque chose de plus du maréchal de Villeroy, et très
unie aussi par là avec M. le Grand. De là résulta une autre liaison qui
devint bientôt après directe et la plus intime\,; ce fut celle de
M\textsuperscript{lle} de Lislebonne et de sa sœur M\textsuperscript{me}
d'Espinoy, qui n'étaient ensemble qu'un cœur, qu'une âme et qu'un
esprit. La dernière était une personne douce, belle, qui n'avait
d'esprit que ce qu'il lui en fallait pour aller à ses fins, mais qui
l'avait au dernier point, et qui jamais ne faisait rien qua par vues\,;
d'ailleurs naturellement bonne, obligeante et polie. L'autre avait tout
l'esprit, tout le sens et toutes les sortes de vues qu'il est
possible\,; élevée à cela par sa mère, et conduite par le chevalier de
Lorraine, avec lequel elle était si anciennement et si étroitement unie
qu'on les croyait secrètement mariés. On a vu en plus d'un endroit de
ces Mémoires quel homme c'était que ce Lorrain, qui, du temps des Guise,
eût tenu un grand coin parmi eux. M\textsuperscript{lle} de Lislebonne
ne lui était pas inférieure, et sous un extérieur froid, indolent,
paresseux, négligé, intérieurement dédaigneux, brûlait de la plus vaste
ambition avec une hauteur démesurée, mais qu'elle cachait sous une
politesse distinguée, et qu'elle ne laissait se déployer qu'à propos.

Sur ces deux sœurs étaient les yeux de toute la cour. Le désordre des
affaires et de la conduite de leur père, frère du feu duc d'Elbœuf,
avait tellement renversé leur marmite, que très souvent elles n'avaient
pas à dîner chez elles. M. de Louvois leur donnait noblement de l'argent
que la nécessité leur faisait accepter. Cette même nécessité les mit à
faire leur cour à M\textsuperscript{me} la princesse de Conti, d'avec
qui Monseigneur ne bougeait alors\,; elle s'en trouva honorée, elle les
attira fort chez elle, les logea, les nourrit à la cour, les combla de
présents, leur procura tous les agréments qu'elle put, que toutes trois
surent bien suivre et faire valoir. Monseigneur les prit toutes trois en
affection, puis en confiance\,; elles ne bougèrent plus de la cour, et
comme compagnie de Monseigneur, furent de tous les Marlys, et eurent
toutes sortes de distinctions. La mère, âgée et retirée de tout cela
avec bienséance, ne laissait pas de tenir le timon de loin, et rarement
venait voir Monseigneur, pour qui c'était une fête. Tous les matins il
allait prendre du chocolat chez M\textsuperscript{lle} de Lislebonne. Là
se ruaient les bons coups c'était à cette heure-là un sanctuaire où il
ne pénétrait personne que lime d'Espinoy. Toutes deux étaient les
dépositaires de son âme, et les confidentes de son affection pour
M\textsuperscript{lle} Choin, qu'elles n'avaient eu garde d'abandonner,
lorsqu'elle fut chassée de la cour, et sur qui elles pouvaient tout.

À Meudon elles étaient les reines\,: tout ce qui était la cour de
Monseigneur la leur faisait presque avec le même respect qu'à lui\,; ses
équipages et son domestique particulier étaient à leurs ordres. Jamais
M\textsuperscript{lle} de Lislebonne n'a appelé du Mont \emph{monsieur},
qui était l'écuyer confident de Monseigneur et pour ses plaisirs et pour
ses dépenses et pour ses équipages, et l'appelait d'un bout à l'autre
d'une chambre à Meudon, où Monseigneur et toute sa cour était, pour lui
donner ses ordres, comme elle eût fait à son écuyer à elle\,; et
l'autre, avec qui tout le monde jusqu'aux princes du sang comptait à
Meudon, accourait et obéissait avec un air de respect, plus qu'il ne
faisait à Monseigneur, avec lequel il avait des manières plus libres\,;
et les particuliers, longtemps si secrets de Monseigneur et de
M\textsuperscript{lle} Choin, n'eurent dans ces premiers temps pour
tiers que ces deux sœurs. Personne ne doutait donc qu'elles ne
gouvernassent après la mort du roi, qui lui-même les traitait avec une
distinction et une considération la plus marquée, et
M\textsuperscript{me} de Maintenon les ménageait fort.

Un plus habile homme que Chamillart eût été ébloui de cet éclat. Le
maréchal de Villeroy, si lié avec M. le Grand, et encore plus
intimement, s'il se pouvait, avec le chevalier de Lorraine, l'était
extrêmement avec elles. Par lui, elles furent bien aises de ranger
Chamillart sous leur empire, et lui désira fort de pouvoir compter sur
elles, d'autant qu'elles étaient sûres. Ils avaient tous leurs
raisons\,: celles de Chamillart se voient par les choses mêmes qui
viennent d'être expliquées\,; celles des deux sœurs, outre la faveur de
Chamillart, étaient de servir par lui Vaudemont, frère de leur mère,
dans les rapports continuels que la guerre d'Italie allait leur donner.
Le maréchal de Villeroy donc, tout à elles, fit cette union avec
Chamillart, et ce qui n'était que la même chose, par une suite
nécessaire, celle de Vaudemont que Villeroy avait vu autrefois à la
cour, qui s'était fait un honneur de bel air et de galanterie de se
piquer d'être de ses amis, qui, malgré leur éloignement d'attachement et
de lieux, s'en était toujours piqué, et qui était entretenu dans cette
fantaisie par ses nièces qui, dans la faveur et les emplois où était
Villeroy, le regardaient avec raison comme pouvant être fort utile à
leur oncle. De M. de Vendôme qui tint un si grand coin dans cette
cabale, j'en parlerai en son temps, et cabale d'autant plus dangereuse,
que jamais le maréchal ni Chamillart, presque aussi courts l'un que
l'autre, ne s'en aperçurent. Ces liaisons étaient déjà faites avant la
mort du roi d'Espagne\,; cette époque ne fit que les resserrer et y
faire entrer Vaudemont de l'éloignement où il était, qui, dans la place
qu'il occupait, sut bientôt seconder ses nièces, et sous leur direction
y entrer directement par le commerce nécessaire de lettres et d'affaires
avec le ministre de France, qui disposait, avec toute la confiance et le
goût du roi, de tout ce qui regardait la guerre et les finances. Voilà
pour la cour\,; voici pour l'Italie\,:

Vaudemont, fils bâtard de ce Charles IV, duc de Lorraine, si connu par
ce tissu de perfidies qui le rendirent odieux à toutes les puissances,
qui lui fit passer une vie si misérable et si errante, qui le
dépouillèrent, et lui coûtèrent la prison en Espagne, était, avec plus
de conduite, de prudence et de jugement, le très digne fils d'un tel
père. J'ai assez parlé de lui plus haut pour l'avoir fait connaître\,;
il ne s'agit plus ici que de le suivre dans ce grand emploi de
gouverneur et de capitaine général du Milanais, qu'il devait à l'amitié
intime du roi Guillaume, et par lui à la poursuite ardente que
l'empereur en avait faite en Espagne. Avec un tel engagement de toute sa
vie acquis par les propos les plus indécents sur le roi, qui le firent
chasser de Rome, comme je l'ai raconté, et fils et frère bâtard de deux
souverains toute leur vie dépouillés par la France, il était difficile
qu'il changeât d'inclination. Pour se conserver dans ce grand emploi et
si lucratif, lui fils de la fortune, sans biens, sans être, sans
établissement que ce qu'elle lui donnait, il s'était soumis aux ordres
d'Espagne, en faisant proclamer Philippe V duc de Milan, avec toutes les
grâces qu'il y sut mettre pour en tirer le gré qui lui était nécessaire
pour sa conservation et sa considération dans son emploi\,; en quoi il
fut merveilleusement secondé par l'art et les amis de ses nièces, les
Lorrains, Villeroy, les dames, Monseigneur et Chamillart, qui en
engouèrent tellement le roi, qu'il ne se souvint plus de rien de ce qui
s'était passé jusque-là, et qu'il se coiffa de cette pensée que le roi
son petit-fils devait le Milanais à Vaudemont.

Ancré de la sorte, il n'oublia rien, comme je l'ai déjà remarqué, pour
s'attacher Tessé comme l'homme de confiance que notre cour lui envoyait
pour concerter avec lui tout ce qui regardait le militaire, et à qui, à
force d'honneurs et d'apparente confiance, il tourna la tête. Tessé,
court de génie, de vues, d'esprit, non pas d'ambition, et qui, en bon
courtisan, n'ignorait pas les appuis de Vaudemont en notre cour, et
prévenu par lui au point qu'il le fut en tout, ne chercha qu'à lui
plaire et à le servir pour s'accréditer en Italie, et y faire un grand
saut de fortune par les amis de Vaudemont à la cour, qui, sûr de lui,
l'aurait mieux aimé que tout autre pour commander notre armée. C'eût
bien été en effet la rapide fortune de l'un, et toute l'aisance de
l'autre, qui l'aurait mené comme un enfant avec un bandeau sur les yeux.
Louvois, dont il avait été fort accusé d'être un des rapporteurs, et
auquel il s'était servilement attaché, l'avait mené vite et fait faire
chevalier de l'ordre en 1688, quoique jeune et seulement maréchal de
camp. Il savait ce que valait la protection des ministres et des gens en
grand crédit, et s'y savait ployer avec une basse souplesse. Il avait
donc fort courtisé Chamillart, qui par sa décoration de la paix de
Savoie et du mariage de M\textsuperscript{me} la duchesse de Bourgogne,
et les accès de sa charge, y avait assez répondu pour faire tout espérer
à Tessé.

Ce ne fut donc pas merveille s'il vit avec désespoir arriver un maître
en Italie, quelque obligation qu'il lui eût du traité de Turin, de sa
charge qui en fut une suite, et de tout ce qui en résulta pour lui
d'avantageux\,; et s'il résolut de s'en défaire pour tâcher à lui
succéder, en lui faisant toutes les niches possibles pour le décréditer
et faire avorter toutes ses entreprises. Il y fut d'autant plus
encouragé qu'il sentait avoir affaire à un homme qui n'avait d'appui ni
d'industrie que sa capacité, et dont la vertu et la simplicité étaient
entièrement éloignées de toute intrigue et de tout manège pour se
soutenir\,; homme de peu, d'une robe toute nouvelle, qui, avec beaucoup
d'esprit, de sagesse, de lumière et de savoir, était peu agréable dans
le commandement, parce qu'il était sec, sévère, laconique, qu'il était
exact sur la discipline, qu'il se communiquait peu, et que, désintéressé
pour lui, il tenait la main au bon ordre sans craindre personne,
d'ailleurs, ni filles, ni vin, ni jeu, et, partant, fort difficile à
prendre. Vaudemont ne fut pas longtemps à s'apercevoir du chagrin de
Tessé, qu'il flatta tant qu'il put sans se commettre avec Catinat, qu'il
reçut avec tous les honneurs et toutes les grâces imaginables, mais qui
en savait trop pour lui, et dont, pour d'autres raisons que Tessé, il
n'avait pas moins d'envie que lui de se défaire.

Le prince Eugène commandait l'armée de l'empereur en Italie, et les deux
premiers généraux après lui, par leur rang de guerre, étaient le fils
unique de Vaudemont et Commercy, fils de sa sœur de Lislebonne. La
moindre réflexion aurait engagé à tenir les yeux bien ouverts sur la
conduite du père, et la moindre suite d'application aurait bientôt
découvert quelle elle était, et combien plus que suspecte. Catinat la
démêla bientôt. Il ne put jamais rien résoudre avec lui que les ennemis
n'en fussent incontinent avertis, en sorte qu'il ne sortit jamais aucun
parti qu'il ne fût rencontré par un des ennemis plus fort du double,
jusque-là même que cela était grossier.

Catinat s'en plaignait souvent\,; il le mandait à la cour, mais sans
oser conclure. Il n'y était soutenu de personne, et Vaudemont y avait
tout pour lui. Il captait nos officiers généraux par une politesse, une
magnificence, et surtout par d'abondantes subsistances\,; tout l'utile,
tout l'agréable venait de son côté\,; tout le sec, toute l'exactitude
venait du maréchal. Il ne faut pas demander qui des deux avait les
volontés et les cœurs. L'état de Vaudemont, qui ne pouvait se soutenir,
ni guère se tenir à cheval, et les prétextes d'être à Milan ou ailleurs
à donner des ordres, le délivraient de beaucoup de cas embarrassants
vis-à-vis d'un général aussi éclairé que Catinat, et par des subalternes
affidés de ses troupes les avis mouchaient à Commercy et à son fils.
Avec de si cruelles entraves, Tessé, qui, bien qu'à son grand regret
roulant avec les lieutenants généraux, était pourtant dans l'armée avec
une distinction fort soutenue, et qui avait dès l'arrivée de Catinat
rompu lance contre lui, excitait les plaintes de tous les contretemps
qui ne cessaient point, et finement appuyé de Vaudemont bandait tout
contre lui, et mandait à la cour tout ce qu'il croyait pouvoir lui nuire
davantage. Vaudemont, de concert, écrivait des demi-mots en homme
modeste qui tâte le pavé, qui ménage un général qu'il voudrait qui n'eût
point de tort, et qui en fait penser cent fois davantage, et il se
ménageait là-dessus avec tant de sobriété et d'adresse qu'il s'en
attirait les reproches qu'il désirait pour s'expliquer davantage et
avoir plus de confiance. Avec tant et de telles contradictions tout
était impossible à Catinat, qui voyait de reste ce qu'il y avait à
faire, et qui ne pouvait venir à bout de rien.

Avec ces beaux manèges ils donnèrent le temps aux Impériaux, d'abord
fort faibles et fort reculés, de grossir, d'avancer peu à peu, et de
passer toutes les rivières sans obstacle, de nous approcher, et, avertis
de tout comme ils l'étaient de point en point, de venir le 9 juillet
attaquer Saint-Frémont logé à Carpi, entre l'Adige et le Pô, avec cinq
régiments de cavalerie et de dragons. Le prince Eugène y amena de
l'infanterie, du canon et le triple de cavalerie, sans qu'on en eût le
moindre avis, et tomba brusquement sur ce quartier. Tessé, qui n'en
était pas éloigné, avec quelques dragons, accourut au bruit. Le prince
Eugène, qui comptait enlever cela d'emblée, y trouva une résistance sur
laquelle il ne comptait pas, et qui fut belle et longue\,; mais il
fallut enfin céder au nombre et se retirer. Ce fut en si bon ordre que
la retraite ne fut pas inquiétée. On y perdit beaucoup de monde, et de
gens de marque\,: le dernier fils du duc de Chevreuse, colonel de
dragons, et du Cambout, brigadier de dragons, parent du duc de Coislin,
bon officier et fort galant homme. Tel fut notre début en Italie, dont
toute la faute fut imputée à Catinat, en quoi Vaudemont, en pinçant
seulement la matière, et Tessé, à pleine écritoire, ne s'épargnèrent
pas\footnote{On trouvera des entraits des lettres de Tessé à Chamillart
  contre Catinat, dans l'ouvrage intitulé\,: \emph{Mémoires militaires
  relatifs à la} succes \emph{sion d'Espagne}, t. Ier, p.~591 et
  suivantes. Cet ouvrage fait partie de la collection des
  \emph{Documents inédits relatifs à l'histoire de France}.}.

Le roi, piqué de ces désavantageuses prémices, et continuellement
prévenu contre un général modeste et sans défenseurs, manda au maréchal
de Villeroy, qui était sur la Moselle, de partir sans dire mot, aussitôt
son courrier reçu, et de venir recevoir ses ordres, tellement qu'il
arriva à Marly, où tout le monde se frotta les yeux en le voyant et ne
se pouvait persuader que ce fût lui. Il fut quelque temps chez
M\textsuperscript{me} de Maintenon avec le roi, Chamillart y vint
ensuite, et comme le roi sortit suivi du maréchal de Villeroy pour se
mettre à table, on sut qu'il allait commander l'armée d'Italie. Jamais
on ne l'eût pris pour le réparateur des fautes de Catinat. La surprise
fut donc complète, et, quoique ce choix fût peu approuvé, le génie
courtisan se déborda en compliments et en louanges. À la fin du souper,
M. de Duras, qui était en quartier, vint à l'ordinaire se mettre
derrière le roi. Un instant après un brouhaha qui se fit dans le salon
annonça le maréchal de Villeroy, qui avait été manger un morceau et
revenait voir le roi sortir de table. Il arriva donc auprès de M. de
Duras avec cette pompe dans laquelle on le voyait baigné. Le maréchal de
Duras qui ne l'aimait point et ne l'estimait guère, et qui ne se
contraignait pas même pour le roi, écoute un instant le bourdon des
applaudissements, puis se tournant brusquement au maréchal de Villeroy
et lui prenant le bras\,: «\,Monsieur le maréchal, lui dit-il tout haut,
tout le monde vous fait des compliments d'aller en Italie, moi j'attends
à votre retour à vous faire les miens\,;» se met à rire et regarde la
compagnie. Villeroy demeura confondu sans proférer un seul mot, et tout
le monde sourit et baissa les yeux. Le roi ne sourcilla pas.

Le pape, fort en brassière par les troupes impériales en Italie, n'osa
recevoir l'hommage annuel du royaume de Naples, que le connétable
Colonne se préparait à lui rendre à l'accoutumée comme ambassadeur
extraordinaire d'Espagne pour cette fonction\,; mais, sur les plaintes
qui lui en furent faites, il fit dire à Naples et par tout le royaume
que, encore qu'il eût des raisons de différer à recevoir cet hommage, il
reconnaissait réellement Philippe V pour roi de Naples, qu'il enjoignait
à tous les sujets du royaume, et particulièrement aux ecclésiastiques,
de lui obéir et de lui être fidèles\,; et il expédia sans difficulté,
sur les nominations du roi d'Espagne, les bénéfices du royaume de
Naples, au grand mécontentement de l'empereur, qui eut encore la douleur
d'y voir avorter une révolte dès sa première naissance, qui avait été
assez bien ménagée.

\hypertarget{chapitre-xi.}{%
\chapter{CHAPITRE XI.}\label{chapitre-xi.}}

1701

~

{\textsc{Dangereuse maladie de M\textsuperscript{me} la duchesse de
Bourgogne.}} {\textsc{- Malice du roi à M. de Lauzun.}} {\textsc{-
Spectacle singulier chez M\textsuperscript{me} la duchesse de Bourgogne
convalescente.}} {\textsc{- Mort de Saint-Herem\,; singularité de sa
femme.}} {\textsc{- Mort de la maréchale de Luxembourg.}} {\textsc{-
Mort de M\textsuperscript{me} d'Épernon, carmélite.}} {\textsc{- Mort du
marquis de Lavardin.}} {\textsc{- Villars de retour de Vienne, et
d'Avaux de Hollande.}} {\textsc{- Matignon gagne un grand procès contre
un faussaire.}} {\textsc{- Villeroy en Italie.}} {\textsc{- M. de Savoie
à l'armée.}} {\textsc{- Combat de Chiari.}} {\textsc{- Étrange
mortification du maréchal de Villeroy par M. de Savoie.}} {\textsc{-
Villeroy et Phélypeaux fort brouillés.}} {\textsc{- Frauduleuse inaction
en Flandre.}} {\textsc{- Castel Rodrigo ambassadeur à Turin pour le
mariage, et grand écuyer de la reine.}} {\textsc{- San-Estevan del
Puerto majordome-major de la reine.}} {\textsc{- Choix, fortune et
caractère de la princesse des Ursins, camarera-mayor de la reine.}}
{\textsc{- M\textsuperscript{me} des Ursins évite Turin.}} {\textsc{-
Légat a latere à Nice vers la reine d'Espagne.}} {\textsc{- Philippe V
proclamé aux Indes, va en Aragon et à Barcelone.}} {\textsc{- Louville
chef de la maison française du roi d'Espagne et gentilhomme de sa
chambre.}} {\textsc{- La reine d'Espagne, charmante, va par terre en
Catalogne.}} {\textsc{- Épouse de nouveau le roi à Figuères.}}
{\textsc{- Scène fâcheuse.}} {\textsc{- Ducs d'Arcos et de Baños à
Paris, puis en Flandre.}}

~

M\textsuperscript{me} la duchesse de Bourgogne, qui, par ses caresses,
son enjouement, sa soumission, ses attentions continuelles à plaire au
roi et à M\textsuperscript{me} de Maintenon, qu'elle appelait toujours
sa tante, leur avait entièrement gagné le cœur, et usurpé une
familiarité qui les amusait, pour s'être baignée imprudemment dans la
rivière après avoir mangé beaucoup de fruit, tomba dans une grande
fièvre vers les premiers jours d'août, comme on était sur le point
d'aller à Marly. Le roi, dont l'amitié n'allait pas jusqu'à la
contrainte, ne voulut ni retarder son voyage ni la laisser à Versailles.
Le mal augmenta à tel point qu'elle fut à l'extrémité. Elle se confessa
deux fois, car en huit jours elle eut une dangereuse rechute. Le roi,
M\textsuperscript{me} de Maintenon, Mgr le duc de Bourgogne étaient au
désespoir et sans cesse auprès d'elle. Enfin elle revint à la vie à
force d'émétique, de saignées et d'autres remèdes. Le roi voulut
retourner à Versailles au temps qu'il l'avait résolu et ce fut avec
toutes les peines du monde que les médecins de M\textsuperscript{me} de
Maintenon l'arrêtèrent encore huit jours, au bout desquels il fallut
partir. M\textsuperscript{me} la duchesse de Bourgogne fut longtemps si
faible qu'elle se couchait les après-dînées, où ses dames et quelques
privilégiées faisaient un jeu pour l'amuser. Bientôt il s'y en glissa
d'autres, et incontinent après toutes celles qui avaient de l'argent
pour grossir le jeu. Mais pas un homme n'y entra que les grandes
entrées\footnote{On appelait les \emph{grandes entrées} les seigneurs
  qui avaient droit d'entrer chez le roi dès qu'il était éveillé et
  d'assister à sa toilette. Le grand chambellan, les premiers
  gentilshommes de la chambre du roi, et, en général les officiers
  attachés à la chambre et à la garde-robe du roi avaient de droit les
  grandes entrées. Pour les autres seigneurs, il fallait un brevet
  spécial.} avec le roi, qui y allait le matin et les après-dînées
pendant ce jeu, en sortant ou rentrant de la chasse ou de la promenade.

M. de Lauzun, à qui, à son retour en ramenant la reine d'Angleterre, les
grandes entrées avaient été rendues, et qui alors les avait seul sans
charge qui les donne, suivit un jour le roi chez M\textsuperscript{me}
la duchesse de Bourgogne. Un huissier ignorant et fort étourdi le fut
tirer par la manche et lui dit de sortir. Le feu lui monta au visage,
mais, peu sur du roi, il ne répondit rien et s'en alla. Le duc de
Noailles, qui par hasard avait le bâton ce jour-là, s'en aperçut le
premier et le dit au roi, qui malignement ne fit qu'en rire et eut
encore le temps de se divertir à voir Lauzun passer la porte. Le roi se
permettait rarement les malices, mais il y avait des gens pour lesquels
il y succombait, et M. de Lauzun, qu'il avait toujours craint et jamais
aimé depuis son retour, en était un. La duchesse du Lude, qui en fut
avertie, entra en grand émoi. Elle craignait fort Lauzun, ainsi que tout
le monde, mais elle craignait encore plus les valets, tellement qu'au
lieu d'interdire l'huissier elle se contenta de l'envoyer le lendemain
matin demander pardon de sa sottise à Lauzun, qui ne fut que plus en
colère d'une si légère satisfaction. Cependant le roi, content de s'être
diverti un moment à ses dépens, lui fit une honnêteté le lendemain à son
petit lever sur son aventure, et l'après-dînée l'envoya chercher pour
qu'il le suivît chez M\textsuperscript{me} la duchesse de Bourgogne.

Le spectacle y était particulier pour un lieu de pleine cour, puisque
toutes les dames y entraient et y étaient en grand'nombre, et qu'il n'y
avait que les hommes d'exclus. À une ruelle était le jeu et tout ce
qu'il y avait de daines\,; à l'autre, au chevet du lit,
M\textsuperscript{me} de Maintenon dans un grand fauteuil\,; à la
quenouille du pied du lit, du même côté, vis-à-vis de
M\textsuperscript{me} de Maintenon, le roi sur un ployant\,; autour
d'eux les dames familières et privilégiées, à les entretenir, assises ou
debout selon leur rang, excepté M\textsuperscript{me} d'Heudicourt, qui
était auprès du roi sur un petit siège tout bas et presque à ras de
terre, parce qu'elle ne pouvait se tenir sur ses hautes et vieilles
jambes\,; et tous les jours cet arrangement était pareil, qui ne laissa
pas de surprendre et de scandaliser assez pour qu'on ne pût s'accoutumer
à ce fauteuil public de M\textsuperscript{me} de Maintenon.

Le bonhomme Saint-Herem mourut à plus de quatre-vingts ans, chez lui, en
Auvergne, où il s'était avisé d'aller. Il avait été grand louvetier, et
avait vendu à Heudicourt pour le recrépir, lorsque le maréchal d'Albret
lui fit en 1666 épouser sa belle et chère nièce de Pons, et il en avait
acheté la capitainerie, etc., de Fontainebleau. Tout le monde l'aimait,
et M. de La Rochefoucauld reprocha au roi en 1688 de ne l'avoir pas fait
chevalier de l'ordre. Il était Montmorin, et le roi le croyait un pied
plat, parce qu'il était beau-frère de Courtin, conseiller d'État, avec
qui le roi l'avait confondu. Ils avaient épousé les deux sœurs. Le roi,
quoique avisé sur sa naissance, ne l'a pourtant point fait chevalier de
l'ordre, quoiqu'il en ait fait plusieurs depuis. Cette
M\textsuperscript{me} de Saint-Herem était la créature du monde la plus
étrange dans sa figure et la plus singulière dans ses façons. Elle se
grilla une fois une cuisse au milieu de la rivière de Seine, auprès de
Fontainebleau, où elle se baignait\,; elle trouva l'eau trop froide,
elle voulut la chauffer, et pour cela elle en fit bouillir quantité au
bord de l'eau qu'elle fit verser tout auprès d'elle et au-dessus,
tellement qu'elle en fut brûlée à en garder le lit, avant que cette eau
pût être refroidie dans celle de la rivière. Quand il tonnait, elle se
fourrait à quatre pattes sous un lit de repos, puis faisait coucher tous
ses gens dessus, l'un sur l'autre en pile, afin que si le tonnerre
tombait il eût fait son effet sur eux avant de pénétrer jusqu'à elle.
Elle s'était ruinée elle et son mari qui étaient riches, par
imbécillité, et il n'est pas croyable ce qu'elle dépensait à se faire
dire des évangiles sur la tête.

La meilleure aventure, entre mille, fut celle d'un fou, qui, une
après-dînée que tous ses gens dînaient, entra chez elle à la place
Royale, et, la trouvant seule dans sa chambre, la serra de fort près. La
bonne femme, hideuse à dix-huit ans, mais qui était veuve et en avait
plus de quatre-vingts, se mit à crier tant qu'elle put. Ses gens à la
fin l'entendirent, et la trouvèrent, ses cottes troussées, entre les
mains de cet enragé, qui se débattait tant qu'elle pouvait. Ils
l'arrêtèrent et le mirent en justice, pour qui ce fut une bonne gorge
chaude, et pour tout le monde qui le sut et qui s'en divertit beaucoup.
Le fou fut trouvé l'être, et il n'en fut autre chose que le ridicule
d'avoir donné cette histoire au public. Son fils avait la survivance de
Fontainebleau. Le roi leur donna quelque pension, car ils étaient fort
mal dans leurs affaires. Ce fils était un très galant homme et fort de
mes amis. Parlant de Fontainebleau, ce fut cette année qu'on doubla la
galerie de Diane, ce qui donna de beaux appartements, et, au-dessus,
quantité de petits.

La maréchale de Luxembourg finit sa triste et ténébreuse vie dans son
château de Ligny, où M. de Luxembourg l'avait tenue presque toute sa vie
sans autre cause que d'être importuné d'elle, après en avoir tiré sa
fortune, en grands biens et en dignité, comme je l'ai expliqué en son
temps, et qui elle était. Elle n'avait presque jamais demeuré à Paris,
où pourtant j'eus une fois en ma vie la fortune de me rencontrer auprès
d'elle à un sermon. On me dit qui elle était et à elle qui j'étais, et
tout aussitôt elle m'entreprit sur notre procès de préséance en
attendant le prédicateur. Je me défendis d'abord avec le respect et la
modestie qu'on doit à une femme, puis voyant le toupet s'échauffer, je
me tus et me laissai quereller, mais fortement, sans dire une parole. Il
est vrai que je trouvai le temps long en attendant le prédicateur, et
que je me sentis bien soulagé lorsque je le vis paraître.
M\textsuperscript{me} de Luxembourg ressemblait d'air, de visage et de
maintien à ces grosses vilaines harengères qui sont dans un tonneau avec
leur chaufferette sous elles. Elle avait été fort maltraitée, fort
méprisée, et avait passé sa vie dans une triste solitude à Ligny, où son
mari lui donnait peu de ses nouvelles.

M\textsuperscript{me} d'Épernon mourut aussi aux Carmélites du faubourg
Saint-Jacques, dans une éminente sainteté. Elle était petite-fille et le
seul reste de ce fameux duc d'Épernon, et fille du second et dernier duc
d'Épernon, colonel général de l'infanterie après son père et gouverneur
de Guyenne, et de sa première femme, bâtarde d'Henri IV et de la
marquise de Verneuil, sœur du duc de Verneuil. M\textsuperscript{me}
d'Épernon, par la mort de ce galant duc de Candale, son frère qui mourut
à la fleur de son âge colonel général de l'infanterie, en survivance de
son père, et général de l'armée de Catalogne, hérita de son père de la
dignité de duchesse d'Épernon, mais renonça à l'éclat de ce grand
héritage, et aux plus grands, partis qui la voulurent épouser, pour
faire profession aux Carmélites, dans un âge où elle avait vu et connu
le monde et tout ce qu'il avait d'attrayant pour elle. La reine,
M\textsuperscript{me} la Dauphine et M\textsuperscript{me} la duchesse
de Bourgogne, allant de temps en temps aux Carmélites, étaient toujours
averties par le roi de la demander et de la faire asseoir. Elle
répondait modestement qu'elle n'était plus que carmélite, et qu'en se la
faisant elle avait renoncé à tout, et il ne fallait pas moins que
l'autorité de ces princesses pour la faire asseoir, elle et
M\textsuperscript{me} de La Vallière, à leur grand regret.

M. de Lavardin, lieutenant général de Bretagne, si connu par l'étrange
ambassade où il se fit excommunier par Innocent XI, sans avoir jamais pu
obtenir audience de lui, mourut à cinquante-cinq ans. Il était chevalier
de l'ordre. C'était un gros homme extrêmement laid, de beaucoup d'esprit
et fort orné, et d'une médiocre conduite. Il avait épousé en premières
noces une sœur du duc de Chevreuse, dont il n'eut que
M\textsuperscript{me} de La Châtre. Il s'était remarié à la sœur du duc
et du cardinal de Noailles, dont il était veuf. Il en laissa une fille
et un fils jeunes, auquel il défendit au lit de la mort, sous peine de
sa malédiction, d'épouser jamais une Noailles, et le recommanda ainsi au
cardinal de Noailles son beau-frère. Nous verrons dans la suite qu'il
fut mal obéi, mais que sa malédiction n'eut que trop son effet. On
l'accusait d'être fort avare, difficile à vivre, et d'avoir hérité de la
lèpre des Rostaing, dont était sa mère. Il disait que de sa vie il
n'était sorti de table sans appétit, et assez pour bien manger encore.
Sa goutte, sa gravelle, et l'âge où il mourut, ne persuadèrent personne
d'imiter son régime.

Villars, envoyé du roi à Vienne, parut à Versailles, le 20 août, qui
rendit compte de tous les efforts que l'empereur faisait pour la guerre.
Il avait laissé président du conseil de guerre, à la place du fameux
comte de Staremberg, qui avait défendu Vienne et qui est la plus grande
charge et la plus puissante de la cour de Vienne, ce même comte de
Mansfeld qui, pendant son ambassade en Espagne, s'était servi de la
comtesse de Soissons, mère du prince Eugène, pour empoisonner la reine
d'Espagne, fille de Monsieur, et qui s'enfuit aussitôt après sa mort.
D'Avaux, notre ambassadeur en Hollande, lassé de toutes les amusettes
avec lesquelles on le menait, salua le roi le lendemain. Le roi
Guillaume était arrivé à la Haye, après avoir tiré de son parlement tout
ce qu'il avait voulu pour nous faire la guerre, et rien de tout ce qu'il
en désirait d'ailleurs\,; il ne tint pas à lui, malgré sa harangue à ce
parlement, de retenir encore d'Anaux à la Haye, à qui il dit lorsqu'il
en prit congé, qu'en l'état où il le voyait il était aisé de juger qu'il
ne souhaitait point la guerre, mais que, si le roi la lui commençait, il
emploierait le peu de vie qui lui restait à défendre ses sujets et ses
alliés. Pouvait-on, pour un habile homme, pousser la dissimulation plus
loin et plus gratuitement, lui qui était l'âme, le boute-feu, et le
constructeur de cette guerre\,? Il avait alors les jambes ouvertes, il
ne pouvait marcher sans le secours de deux écuyers, et il fallait le
mettre entièrement à cheval, et prendre ses pieds pour les mettre dans
les étriers. Aussi ne comptait-il pas apparemment de commander d'armée,
mais bien de tout diriger de son cabinet. Le lendemain, 22 août,
Zinzendorf, envoyé de l'empereur, prit congé du roi, et s'en retourna à
Vienne. C'est le même qui y a fait depuis une si grande fortune,
chancelier de la cour, c'est-à-dire ministre des affaires étrangères,
conseiller de conférences, c'est-à-dire ministre d'État, et il n'y en a
que trois, au plus quatre, chevalier de la Toison d'or, et des millions,
et voir son fils cardinal tout jeune et évêque d'Olmütz.

Matignon avait alors une très fâcheuse affaire. Un va-nu-pieds lui fit
un procès au parlement de Rouen, et y produisit des pièces qui mirent
Matignon au moment d'être condamné à lui payer un million deux cent
mille livres, malgré tout son crédit dans la province, soutenu de celui
de Chamillart. Ce procès dura longtemps, et ce va-nu-pieds avait tant
d'argent et de recommandations qu'il voulait de tous les dévots et
dévotes, à force de crier à l'oppression\,; à la fin, les pièces furent
reconnues fausses, il avoua tout et fut pendu.

Vaudemont, fut satisfait d'avoir le maréchal de Villeroy en Italie, ce
fut un nouveau crève-cœur pour Tessé, d'autant plus grand qu'il n'espéra
plus de bricoles pour arriver au commandement de l'armée, et qu'il n'y
avait pas moyen de se jouer à ce nouveau général comme avec Catinat,
avec lequel ses démêlés devinrent scandaleux à l'armée, et firent ici
beaucoup de bruit. Il n'y eut souplesses qu'il ne fit à Villeroy pour le
mettre de son côté. Catinat reçut cette mortification en philosophe, et
fit admirer sa modération et sa vertu. La tranquillité avec laquelle il
remit le commandement au maréchal de Villeroy, et la conduite qu'il tint
après à l'armée la lui ramena. On s'y souvint enfin des lauriers qu'il
avait cueillis en Italie. On n'en trouvait aucuns chez Villeroy. Les
manèges, l'ingratitude, le succès de Tessé révoltèrent. Mais ce fut
tout. Tessé, venu seul avec son fils et un aide de camp au secours de
Saint-Frémont, à Carpi, au lieu de se faire suivre par tout son
quartier, ou du moins de l'envoyer chercher après avoir vu de quoi il
était question, fut fort accusé d'avoir voulu laisser rompre le cou à
Saint-Frémont, et donne lieu à un passage des Impériaux au milieu de
tous les postes de l'armée, qui, pour garder inutilement un trop grand
pays, étaient trop nombreux, se pouvaient trop peu entre-secourir\,; et
dispersaient trop l'armée. C'est ce dont Tessé se plaignait aux dépens
de Catinat, comme si Vaudemont n'en eût pas été de moitié\,; mais ces
plaintes et les souterrains de Tessé firent tant d'effet à Paris et à la
cour, que personne n'osait défendre Catinat, et que ses parents du
parlement cessèrent quelque temps d'y aller pour éviter les discours
trop désagréables dont ils étaient assaillis. Catinat offrit sa maison
et ses équipages à Villeroy, en attendant les siens, mais il fut
descendre chez son ami Vaudemont, qui le reçut avec les grâces et la
magnificence d'un homme qui sent le besoin qu'il a d'un autre, et qui
connaît les moyens de l'aveugler. En effet, il en fit tout ce qu'il
voulut, et eut de plus en lui un favori du roi, et un ami du ministre
tout occupé à le faire valoir.

Tessé ne pouvant abattre Villeroy, espéra une part principale dans sa
confiance, et par lui, aidé de Vaudemont et appuyé du généralissime, se
donner un crédit et une autorité principale dans l'armée. Mais son
débordement sur Catinat donna des soupçons, puis de la jalousie à
Villeroy, qui le traita plus sèchement, et M. de Savoie même ne put
s'empêcher d'en parler publiquement à Tessé d'une manière assez forte,
qui lui rabattit fort le ton. On disputa sur la conduite de Catinat sans
femme ni enfants, et libre par conséquent de se retirer pour n'entendre
jamais parler de cour ni de guerre, ou de demeurer, comme il fit, à
l'armée, ne se mêlant presque de rien avec une rare modestie.

M. de Savoie enfin la joignit avec ses troupes après de longs délais, et
très suspects. Son arrivée ne changea rien à l'exactitude avec laquelle
les ennemis étaient avertis de tous les desseins, de toutes les mesures,
et des moindres mouvements qui se faisaient dans notre armée.
L'intelligence entre lui et Vaudemont fut parfaite. Le besoin d'un
gouverneur du Milanais aussi soutenu que l'était, Vaudemont du temps du
feu roi d'Espagne, l'avait commencée par les plus grandes avances,
jusque-là que M. de Savoie l'alla rencontrer en chemin lorsqu'il arriva
dans le Milanais, et qu'il lui donna l'Altesse\,: au fond, quoique
français de parti en apparence, leurs liaisons fondamentales étaient les
mêmes à l'un et à l'autre. M. de Savoie, quoique peu content de
l'empereur, qui ne lui avait pas tenu tout ce qu'il lui avait promis, ni
du roi Guillaume, qui l'avait fort maltraité, pour s'être détaché d'eux
par le traité de Turin, ne voyait qu'avec un extrême regret la monarchie
d'Espagne devenue française, et lui enfermé entre le grand-père et le
petit-fils, par le Milanais et la France. Il ne se prêtait donc que pour
tirer parti de ce qu'il ne pouvait empêcher, et il désirait avec ardeur
le rétablissement de l'empereur en Italie\,; comme il ne parut que trop
tôt. En attendant, il parut faire avec soin toutes les fonctions de
généralissime.

Les armées cependant s'approchaient, celle des Impériaux gagnant
toujours du terrain, et elles en vinrent au point que ce fut à qui
s'emparerait les premiers du poste de Chiari. Le prince Eugène fut le
plus diligent. C'était un gros lieu fermé de murailles, sur un tertre
imperceptible, mais qui dérobait la vue de ce qui était derrière, au bas
d'un ruisseau qui coulait tout auprès. M. de Savoie, trop bon général
pour tomber dans la même faute que le maréchal d'Humières avait faite à
Valcourt, l'imita pourtant de point en point, et avec un plus fâcheux
succès, parce qu'il s'y opiniâtra davantage. Il fit attaquer ce poste le
1er septembre, par huit brigades d'infanterie. Il augmenta toujours, et
s'exposa extrêmement lui-même pour gagner estime et confiance, et
montrait qu'il y allait avec franchise\,; mais il attaquait des
murailles et une armée entière qui rafraîchissait toujours, tellement
qu'après avoir bien fait tuer du monde il fallut se retirer
honteusement. Cette folie dans un prince qui savait le métier de la
guerre, et à qui le péril personnel ne coûtait rien, fut dès lors très
suspecte. Villeroy s'y montra fort partout, et Catinat, sans se mêler de
rien, sembla y chercher la mort, qui n'osa l'atteindre. Nous y perdîmes
cinq ou six colonels peu marqués, et quantité de monde, et eûmes force
blessés. Cette action, où la valeur française parut beaucoup, étonna
fort notre armée, et encouragea beaucoup celle des ennemis, qui firent à
peu près tout ce qu'ils voulurent le reste de la campagne. Nos troupes
étaient si accoutumées, dès qu'on en envoyait dehors, à rencontrer
toujours le double d'Impériaux bien avertis qui les attendaient, que la
timidité s'y mit, et que les troupes de M. de Vaudemont surent bien dire
plus d'une fois qu'elles ne savaient encore qui de l'archiduc ou du duc
d'Anjou était leur maître, et qu'il en fallut enfermer entre les nôtres.

Dans la fin de cette campagne, les grands airs de familiarité que le
maréchal de Villeroy se donnait avec M. de Savoie lui attirèrent un
cruel dégoût, pour ne pas dire un affront. M. de Savoie, étant au milieu
de tous les généraux et de la fleur de l'armée, ouvrit sa tabatière en
causant et allant prendre une pincée de tabac, le maréchal de Villeroy
qui se trouva auprès de lui allonge la main et prend dans la tabatière
sans mot dire. M. de Savoie rougit\,; et à l'instant renverse sa
tabatière par terre, puis la donne à un de ses gens à qui il dit de lui
rapporter du tabac. Le maréchal ne sut que devenir, et but sa honte sans
oser proférer une parole, M. de Savoie continuant toujours la
conversation qu'il n'interrompit même que par ce seul mot pour avoir
d'autre tabac.

La vanité du maréchal de Villeroy eut à souffrir de la présence de
Phélypeaux, ambassadeur auprès de M. de Savoie, qui le suivit à l'armée.
Par ce caractère il avait la même garde, les mêmes saluts et tous les
mêmes honneurs militaires que le général de l'armée du roi, et il avait
de plus la préférence du logement et de la marche de ses équipages,
comme il avait aussi le pas sur lui partout. Cela était insupportable au
maréchal dans un homme comme Phélypeaux, qui, était à peine lieutenant
général, et Phélypeaux, qui avait de l'esprit comme cent diables, et
autant de malice qu'eux, se plaisait à désespérer le maréchal en prenant
partout sur lui ses avantages. Cela mit une telle pique entre eux qu'il,
en résulta beaucoup de mal. Phélypeaux, qui en tout voyait clair, se
lassa d'aviser un homme qui de dépit n'en faisait aucun usage, et qui se
plaisait à mander à la cour tout le contraire de Phélypeaux, qui
s'aperçut bientôt de la perfidie de M. de Savoie, et dont les avis
furent détruits par les lettres du maréchal de Villeroy, dont la faveur
prévalut à toutes les lumières de l'autre.

Ainsi s'écoula la campagne, nous toujours reculant, et les Impériaux
avançant avec tant de facilité et d'audace, et leurs troupes
grossissant, tandis que les nôtres diminuaient tous les jours par un
détail journalier de petites pertes et par les maladies, qu'on en vint à
craindre le siège de Milan, c'est-à-dire du château, auquel néanmoins le
prince Eugène ne songea jamais sérieusement. Lui et le maréchal de
Villeroy prirent leur quartier d'hiver chacun de leur côté, et le
passèrent sur la frontière. M. de Savoie se retira à Turin, et Catinat
s'en alla à Paris. Le roi le reçut honnêtement, mais il ne lui parla que
des chemins et de son voyage, et ne le vit point en particulier\,; lui
aussi ne se mit en aucun soin d'en obtenir une audience.

En Flandre on ne fit que se regarder sans hostilités, qui fut une grande
faute, sortie toujours de ce même principe de ne vouloir pas être
l'agresseur, c'est-à-dire de laisser bien arranger, dresser et organiser
ses ennemis, et attendre leur bon point, et aisément, et leur signal
pour entrer en guerre qu'on ne doutait plus qu'ils ne nous voulussent
faire. Si, au lieu de cette fausse et pernicieuse politique, l'armée du
roi eût agi, elle aurait pénétré les Pays-Bas où rien n'était prêt ni en
état de résistance, eût fait crier miséricorde aux Hollandais au milieu
de leur pays, les eût mis hors d'état de soutenir la guerre, déconcerté
cette grande alliance dont leur bourse fut l'âme et le soutien, mis
l'empereur hors d'état de pousser la guerre faute d'argent, et avec les
princes du Rhin et M. de Bavière alliés avec la Souabe et ces cercles
leurs voisins pour leur tranquillité et leur neutralité, l'empire
n'aurait pas pris forcément, comme il fit, parti pour l'empereur, et,
malgré la faute d'avoir rendu les vingt-deux bataillons hollandais, on
aurait eu encore la paix par les armes d'une campagne, avec la totalité
de la monarchie d'Espagne assurée à Philippe V.

Ce prince avait envoyé un ambassadeur extraordinaire à Turin pour signer
son contrat de mariage, et porter au prince de Carignan, ce fameux muet
si sage et si capable, sa procuration pour épouser en son nom la
princesse de Savoie. Cet ambassadeur était un homme de beaucoup
d'esprit, de sens et de conduite, et fort propre dans les cours. Il
était Homodeï, frère du cardinal de ce nom, et avait porté celui de
marquis d'Almonacid jusqu'à son mariage avec Éléonore de Moura, fille
aîné du marquis de Castel Rodrigo, gouverneur des Pays-Bas. Son père
l'avait été aussi, et son grand-père, qui était Portugais et qui avait
fort bien servi Philippe II, en avait été fait comte. Il fut le premier
vice-roi de Portugal pour l'Espagne, et Philippe III le fit grand
d'Espagne. Almonacid le fut donc en 1671 par la mort de son beau-père
sans enfants moles, et prit le nom de Castel Rodrigo.

Il fut en même temps chargé de la conduite de la nouvelle reine en
Espagne, de laquelle il fut aussi grand écuyer. Et le, comte de
San-Estevan del Puerto dont j'ai fort parlé à propos du testament de
Charles II, et qui avait quitté la reine sa veuve dont il était
majordome-major, le fut de la nouvelle reine.

Rien n'était meilleur que ces deux choix pour ces deux grandes charges,
mais il y en avait un troisième à faire bien plus important, et par
lequel il fallait élever et former la jeune reine. C'était celui de sa
camarera-mayor. Une dame de notre cour ne pouvait y convenir\,; une
Espagnole n'était pas sûre et eût aisément rebuté la reine\,; on chercha
un milieu et on ne trouva que la princesse des Ursins. Elle était
Française, elle avait été en Espagne, elle avait passé la plus grande
partie de sa vie à Rome et en Italie, elle était veuve sans enfants,
elle était de la maison de La Trémoille\,; son mari était chef de la
maison des Ursins, grand d'Espagne et prince du Soglio, et, par son âge
plus avancé que celui du connétable Colonne, il était reconnu le premier
laïque de Rome avec de grandes distinctions. M\textsuperscript{me} des
Ursins n'était pas riche depuis la mort de son mari\,; elle avait passé
des temps assez longs en France pour être fort connue à la cour et y
avoir des amis. Elle était liée d'un grand commerce d'amitié avec les
deux duchesses de Savoie, et avec la reine de Portugal sœur de la
douairière. C'était le cardinal d'Estrées, leur parent proche et leur
conseil, qui avait formé cette union\,; que les passages à Turin avaient
fort entretenue, avec M\textsuperscript{me}s de Savoie\,; enfin ce
cardinal qui avait fait sa fortune en la mariant aussi grandement à Rome
où elle était veuve de Chalais, sans bien, sans enfants et comme sans
être, était demeuré depuis ce temps-là son ami intime après lui avoir
été quelque chose de plus en leur jeunesse, conseilla fort ce choix, et
ce qui y détermina peut-être tout à fait, c'est qu'on fut informé par
lui que le cardinal Portocarrero en avait été fort amoureux à Rome, et
qu'il en était demeuré depuis une grande liaison d'amitié entre eux.
C'était avec lui qu'il fallait tout gouverner, et ce concert si
heureusement trouvé entre lui et elle emporta son choix pour une place
si importante, et d'un rapport si nécessaire et si continuel avec lui.

Elle était fille du marquis de Noirmoutiers qui fit tant d'intrigues
dans les troubles de la minorité de Louis XIV, et qui en tira un brevet
de duc et le gouvernement de Charleville et du Mont-Olympe. Sa mère
était une Aubry, d'une famille riche de Paris. Elle épousa en 1659
Adrien-Blaise de Talleyrand, qui se faisait appeler le prince de
Chalais, mais sans rang ni prétention quelconque. Son fameux duel avec
un cadet de Noirmoutiers, Flamarens et le frère aîné de M. de Montespan
contre Argenlieu, les deux La Frette, et le chevalier de Saint-Aignan,
frère du duc de Beauvilliers, obligea Chalais aussitôt après, et c'était
en 1663, de sortir du royaume\,; et sa femme le suivit en Espagne et de
là par mer en Italie, où il mourut sans enfants en février 1670 auprès
de Venise, en allant trouver sa femme, qui l'attendait à Rome. Dans ce
désastre, les cardinaux de Bouillon et d'Estrées, prirent soin d'elle\,;
le reste on l'a vu épars dans ces Mémoires.

L'âge et la santé convenaient, et la figure aussi. C'était une femme
plutôt grande que petite, brune avec des yeux bleus qui disaient sans
cesse tout ce qui lui plaisait, avec une taille parfaite, une belle
gorge, et un visage qui, sans beauté, était charmant\,; l'air
extrêmement noble, quelque chose de majestueux en tout son maintien, et
des grâces si naturelles et si continuelles en tout, jusque dans les
choses les plus petites et les plus indifférentes, que je n'ai jamais vu
personne en approcher, soit dans le corps, soit dans l'esprit, dont elle
avait infiniment et de toutes les sortes\,; flatteuse, caressante,
insinuante, mesurée, voulant plaire pour plaire, et avec des charales
dont il n'était pas possible de se défendre, quand elle voulait gagner
et séduire\,; avec cela un air qui avec de la grandeur attirait au lieu
d'effaroucher, une conversation délicieuse, intarissable et d'ailleurs
fort amusante par tout ce qu'elle avait vu et connu de pays et de
personnes, une voix et un parler extrêmement agréables, avec un air de
douceur\,; elle avait aussi beaucoup lu, et elle était personne à
beaucoup de réflexion. Un grand choix des meilleures compagnies, un
grand usage de les tenir, et même une cour, une grande politesse, mais
avec une grande distinction, et surtout une grande attention à ne
s'avancer qu'avec dignité et discrétion. D'ailleurs la personne du monde
la plus propre à l'intrigue, et qui y avait passé sa vie à Rome par son
goût\,; beaucoup d'ambition, mais de ces ambitions vastes, fort
au-dessus de son sexe, et de l'ambition ordinaire des hommes, et un
désir pareil d'être et de gouverner. C'était encore la personne du monde
qui avait le plus de finesse dans l'esprit, sans que cela parût jamais,
et de combinaisons dans la tête, et qui avait le plus de talents pour
connaître son monde et savoir par où le prendre et le mener. La
galanterie et l'entêtement de sa personne fut en elle, la faiblesse
dominante et surnageante à tout jusque, dans sa dernière vieillesse\,;
par conséquent, des parures qui ne lui allaient plus et que d'âge en âge
elle poussa toujours fort au delà du sien\,; dans le fond haute, fière,
allant à ses fins sans trop s'embarrasser des moyens, mais tant qu'elle
pouvait sous une écorce honnête\,; naturellement assez bonne et
obligeante en général, mais qui ne voulait rien à demi, et que ses amis
fussent à elle sans réserve\,; aussi était-elle ardente et excellente
amie, et d'une amitié que les temps ni les absences n'affaiblissaient
point, et conséquemment cruelle et implacable ennemie, et suivant sa
haine jusqu'aux enfers\,; enfin, un tour unique dans sa grâce, son art
et sa justesse, et une éloquence simple et naturelle en tout ce qu'elle
disait, qui gagnait au lieu de rebuter par son arrangement, tellement
qu'elle disait tout ce qu'elle voulait et comme elle le voulait dire, et
jamais mot ni signe le plus léger de ce qu'elle ne voulait pas\,; fort
secrète pour elle et fort sûre pour ses amis, avec une agréable gaieté
qui n'avait rien que de convenable, une extrême décence en tout
l'extérieur, et jusque dans les intérieures même qui en comportent le
moins, avec une égalité d'humeur qui en tout temps et en toute affaire
la laissait toujours maîtresse d'elle-même. Telle était cette femme
célèbre qui a si longtemps et si publiquement gouverné la cour et toute
la monarchie d'Espagne, et qui a fait tant de bruit dans le monde par
son règne et par sa chute, que j'ai cru nie devoir étendre pour la faire
connaître et en donner l'idée qu'on en doit avoir pour s'en former une
qui soit véritable.

Une personne de ce caractère fut fort sensible à un choix qui, lui
ouvrait une carrière si fort à son gré\,; mais elle eut le bon esprit de
sentir qu'on ne venait à elle que faute de pouvoir trouver un autre
sujet qui rassemblât en soi tant de parties si manifestement convenables
à la place qu'on lui offrait, et qu'une fois offerte, on ne la lui
laisserait pas refuser. Elle se fit donc prier assez pour augmenter le
désir qu'on avait d'elle, et non assez pour dégoûter ni rien faire de
mauvaise grâce, mais pour qu'on lui sût gré de son acceptation. Quoique
désirée par la Savoie encore plus s'il se pouvait que par la France, et
si étroitement bien et en commerce de lettres avec les deux duchesses,
elle évita Turin, parce que le cérémonial l'avait toujours empêchée de
les voir autrement qu'incognito (qu'elle pouvait garder aisément dans
ses voyages en passant à Turin), ce qui ne pouvait plus se faire dans
l'occasion qui la menait, tellement que tout se traita par lettres entre
elles, et qu'elle alla droit de Rome à Gènes, et de Gènes à
Villefranche, y attendre la nouvelle reine.

Son mariage se fit à Turin, le 11 septembre, avec assez peu d'appareil.
Elle en partit le 13 pour venir en huit jours à Nice s'y embarquer sur
les galères d'Espagne, commandées par le comte de Lémos, qui la devait
porter à Barcelone. Elle reçut à Nice le cardinal Archinto, légat
\emph{a latere} exprès pour la fonction de lui faire les compliments du
pape sur son mariage. Cette démarche du pape fâcha extrêmement

L'empereur, et la cour de Savoie demeura fort piquée de ce que, passant
par ses États, elle n'en avait reçu aucun compliment. M. de Savoie,
justement ennuyé du cérémonial des cardinaux, n'en voyait aucun depuis
fort longtemps. Ceux qui ont le caractère de légats \emph{a latere} ont
des prétentions immenses apparemment que le cardinal fut mécontent et
qu'il les paya de cette incivilité.

Le roi d'Espagne eut nouvelle des Indes qu'il avait été proclamé au
Pérou et au Mexique avec beaucoup d'unanimité et de tranquillité, et
avec beaucoup de cérémonies et de fêtes. Il partit le 5 septembre de
Madrid pour son voyage d'Aragon et de Catalogne, et aller attendre la
reine sa femme à Barcelone. Il laissa le cardinal Portocarrero
gouverneur de la monarchie d'Espagne, avec ordre à tous les conseils, à
tous ses officiers de tous États, et à tous ses ambassadeurs et
ministres dans les cours étrangères, de recevoir ses ordres et leur
obéir comme aux siens mêmes. En partant il donna à Louville une clef de
gentilhomme de la chambre en service, et le titre de chef de sa maison
française, c'est-à-dire l'autorité sur tous les officiers français de sa
bouche, pour en être mieux servi. Il fit force grâces sur sa route.
Saragosse lui fit une magnifique entrée. Il confirma tous les privilèges
de l'Aragon et de la Catalogne. Quelques réjouissances que fissent les
provinces dépendantes de l'Aragon, et surtout la Catalogne, il n'y parut
pas la même franchise et la même affection que dans celles qui dépendent
de la couronne de Castille, quoique le roi, qui ne fit pas semblant de
le remarquer, se les attirât par toutes sortes de bienfaits.

La reine d'Espagne, que les galères de France avaient amené à Nice, se
trouva si fatiguée de la mer, qu'elle voulut achever son voyage par
terre à travers la Provence et le Languedoc. Ses grâces, sa présence
d'esprit, la justesse et la politesse de ses courtes réponses, sa
judicieuse curiosité surprit dans une princesse de son âge, et donna de
grandes espérances à la princesse des Ursins.

Sur les premières frontières du Roussillon, Louville vint lui faire les
compliments, et lui apporter les présents du roi, qui vint au-devant
d'elle jusqu'à Figuères, à deux journées de Barcelone. On avait envoyé
au-devant d'elle toute sa maison au delà, d'où Louville la joignit, et
on avait renvoyé toute sa maison piémontaise. Elle parut plus sensible à
cette séparation que M\textsuperscript{me} la duchesse de Bourgogne.
Elle pleura beaucoup, et se trouva fort étonnée au milieu de tous
visages dont le moins inconnu lui était celui de M\textsuperscript{me}
des Ursins, avec qui la connaissance ne pouvait pas être encore bien
faite depuis le bord de la mer où elle l'avait rencontrée. En arrivant à
Figuères, le roi, impatient de la voir, alla à cheval au-devant d'elle
et revint de même à sa portière, où, dans ce premier embarras,
M\textsuperscript{me} des Ursins leur fut d'un grand secours, quoique
tout à fait inconnue au roi, et fort peu connue encore de la reine.

En arrivant à Figuères, l'évêque diocésain les maria de nouveau avec peu
de cérémonie, et bientôt après ils se mirent à table pour souper, servis
par la princesse des Ursins et par les dames du palais, moitié de mets à
l'espagnole, moitié à la française. Ce manège déplut à ces dames et à
plusieurs seigneurs espagnols, avec qui elles avaient comploté de le
marquer avec éclat\,; en effet, il fut scandaleux. Sous un prétexte ou
un autre, de la pesanteur ou de la chaleur des plats, ou du peu
d'adresse avec laquelle ils étaient présentés aux dames, aucun plat
Français ne put arriver sur la table, et tous furent renversés, au
contraire des mets espagnols, qui y furent tous servis sans malencontre.
L'affectation et l'air chagrin, pour ne rien dire de plus, des dames du
palais étaient trop visibles pour n'être pas aperçus. Le roi et la reine
eurent la sagesse de n'en faire aucun semblant, et M\textsuperscript{me}
des Ursins, fort étonnée, ne dit pas un mot.

Après un long et fâcheux repas, le roi et la reine se retirèrent. Alors
ce qui avait été retenu pendant le souper débanda. La reine se mit à
pleurer ses Piémontaises. Comme un enfant qu'elle était, elle se crut
perdue entre les mains de dames si insolentes, et quand il fut question
de se coucher, elle dit tout net qu'elle n'en ferait rien et qu'elle
voulait s'en retourner. On lui dit ce qu'on put pour la remettre, mais
l'étonnement et l'embarras furent grands quand on vit qu'on n'en pouvait
venir à bout. Le roi déshabillé attendait toujours. Enfin la princesse
des Ursins, à bout de raisons et d'éloquence, fut obligée d'aller avouer
au roi et à Marsin tout ce qui se passait. Le roi en fut piqué et encore
plus fâché. Il avait jusque-là vécu dans la plus entière retenue, cela
même avait aidé à lui faire trouver la princesse plus à son gré\,; il
fut donc sensible à cette fantaisie, et par même raison aisément
persuadé qu'elle ne se pousserait pas au delà de cette première nuit.
Ils ne se virent donc que le lendemain, et après qu'ils furent habillés.
Ce fut un bonheur que la coutume d'Espagne ne permette pas d'assister au
coucher d'aucuns mariés, non pas même les plus proches, en sorte que ce
qui aurait fait un très fâcheux éclat demeura étouffé entre les deux
époux, M\textsuperscript{me} des Ursins, une ou deux caméristes, et deux
ou trois domestiques Français intérieurs, Louville et Marsin.

Ces deux-ci cependant se mirent à consulter avec M\textsuperscript{me}
des Ursins comment on pourrait s'y prendre pour venir à bout d'un enfant
dont les résolutions s'exprimaient avec tant de force et de tenue. La
nuit se passa en exhortations et en promesses aussi sur ce qui était
arrivé au souper, et la reine enfin consentit à demeurer reine. Le duc
de Medina-Sidonia et le comte de San-Estevan furent consultés le
lendemain. Ils furent d'avis qu'à son tour le roi ne couchât point avec
elle la nuit suivante pour la mortifier et la réduire. Cela fut exécuté.
Ils ne se virent point en particulier de tout le jour. Le soir, la reine
fut affligée. Sa gloire et sa petite vanité furent blessées, peut-être
aussi avait-elle trouvé le roi à son gré. On parla ferme aux dames du
palais, et plus encore aux seigneurs qu'on soupçonna d'intelligence avec
elles, et à ceux de leurs parents qui se trouvèrent là. Excuses,
pardons, craintes, promesses, tout fut mis en règle et en respect, et le
troisième jour fut tranquille, et la troisième nuit encore plus agréable
aux jeunes époux. Le quatrième, comme tout se trouva dans l'ordre où il
devait être, ils retournèrent tous à Barcelone, où il ne fut question
que d'entrées, de fêtes et de plaisirs.

Avant de partir de Madrid, le roi d'Espagne avait ordonné aux ducs
d'Arcos et de Baños frères, dont j'ai expliqué la naissance ci-dessus,
d'aller servir en Flandre pour les punir. Ils avaient été les seuls
d'entre les grands d'Espagne qui à voient trouvé mauvais l'égalité,
convenue entre le roi et le roi son petit-fils, entre les ducs et les
grands pour les rangs, honneurs, distinctions et traitements des uns et
des autres en France et en Espagne. Au moins tous en avaient témoigné
leur approbation et leur joie, qu'ils le pensassent ou, non, et ces deux
jeunes gens seuls, non contents de marquer tout le contraire,
présentèrent au roi d'Espagne un écrit de leurs raisons. Ce mémoire
était bien fait, respectueux pour le roi, mesuré même sur la chose, mais
il ne fit d'autre effet que de leur attirer cette punition, et le blâme
de leurs confrères, dont quelques-uns en eussent peut-être fait autant
s'ils en eussent espéré un autre succès. Ils obéirent, ils virent le roi
dans son cabinet qui les traita fort bien, furent peu à Paris et à la
cour où on les festoya fort, et où ils furent les premiers grands
d'Espagne qui baisèrent M\textsuperscript{me} la duchesse de Bourgogne,
et qui jouirent de tout ce dont jouissent les ducs.

\hypertarget{chapitre-xii-38-.}{%
\chapter{CHAPITRE XII {[}38{]} .}\label{chapitre-xii-38-.}}

1701

~

{\textsc{Digression sur la dignité de grands d'Espagne et sa comparaison
avec celle de nos ducs.}} {\textsc{- Son origine.}} {\textsc{-
Ricos-hombres, et leur multiplication.}} {\textsc{- Idée dès lors de
trois sortes de classes.}} {\textsc{- Leur part aux affaires et
comment.}} {\textsc{- Parlent couverts au roi.}} {\textsc{- Ferdinand et
Isabelle dits les rois catholiques.}} {\textsc{- Philippe Ier ou le
Beau.}} {\textsc{- Flatterie des ricos-hombres sur leur couverture.}}
{\textsc{- Affaiblissement de ce droit et de leur nombre.}} {\textsc{-
Première gradation.}} {\textsc{- Charles-Quint.}} {\textsc{- Deuxième
gradation\,: ricos-hombres abolis en tout.}} {\textsc{- Grands d'Espagne
commencent et leur sont substitués.}} {\textsc{- Grandeur de la
grandesse au dehors des États de Charles-Quint.}} {\textsc{- Troisième
gradation\,: couverture et seconde classe de grands par Philippe II.}}
{\textsc{- Trois espèces de grands et deux classes jusqu'alors.}}
{\textsc{- Quatrième gradation\,: patentes d'érection et leur
enregistrement de Philippe III.}} {\textsc{- Nulle ancienneté observée
entre les grands, et leur jalousie sur ce point et sa cause.}}
{\textsc{- Troisième classe de grands.}} {\textsc{- Grands à vie de
première classe.}} {\textsc{- Nul autre rang séculier en Espagne en la
moindre compétence avec ceux du pays.}} {\textsc{- Seigneurs couverts en
une seule occasion sans être grands.}} {\textsc{- Cinquième gradation\,:
certificat de couverture.}} {\textsc{- Suspension de grandesse en la
main du roi.}} {\textsc{- Exemples entre autres du duc de Médina
Sidonia.}} {\textsc{- Sixième gradation\,: grandesses devenues amovibles
et pour les deux dernières classes en besoin de confirmation à chaque
mutation.}} {\textsc{- Grandesse ôtée au marquis de Vasconcellos et à sa
postérité.}} {\textsc{- Septième gradation\,: tributs pécuniaires pour
la grandesse.}} {\textsc{- Mystères affectés des trois différentes
classes}}

~

L'occasion de parler un peu de la dignité de grand d'Espagne et de la
comparer avec celle de nos ducs est ici trop naturelle pour n'y pas
succomber. Ce n'est pas un traité que je prétends donner ici de ces
dignités, mais, à l'occasion du mécontentement et du mémoire des ducs
d'Arcos et de Baños, donner une idée des grands d'Espagne, d'autant plus
juste que je me suis particulièrement appliqué à m'en instruire par
eux-mêmes en Espagne, et que je n'ai pas vu qu'on se l'ait formée telle
qu'elle est. Quoique les digressions soient d'ordinaire importunes,
celle-ci s'excusera elle-même par sa curiosité.

La dignité des grands d'Espagne tire son origine des grands fiefs
relevant immédiatement de la couronne, et comme la totalité de ce que
nous appelons aujourd'hui l'Espagne était divisée en plusieurs royaumes,
tantôt indépendants, tantôt tributaires, tantôt membres les uns des
autres, selon le sort des armes ou celui du partage des familles des
rois, chaque royaume avait ses grands ou premiers vassaux relevant
immédiatement du grand fief qui était le royaume même, et qui de tout
temps avaient le droit de bannière et de chaudière. Le premier est trop
connu dans nos histoires et dans notre France pour avoir besoin d'être
expliqué. Celui de chaudière marquait les richesses suffisantes pour
fournir à l'entretien de ceux qui étaient sous la bannière levée par le
seigneur banneret. Ces seigneurs étaient plus ou moins considérables,
non seulement par leur puissance particulière, mais encore par celle des
royaumes dont ils étaient vassaux immédiats. C'est ce qui a fait que la
couronne de Castille ayant toujours tenu le premier lieu dans les
Espagnes depuis que de comté dépendant du royaume de Navarre elle devint
royaume elle-même, et bientôt supérieure à tous les autres, même à celui
dont elle était sortie, et encore à celui de Léon, ses premiers vassaux
furent aussi les plus considérés parmi les premiers vassaux des autres
royaumes, et par la même raison ceux d'Aragon après eux.

Les fréquentes révolutions arrivées dans les Espagnes par les
différentes divisions et réunions qui se firent sous tant de rois
séparés, et qui furent encore augmentées par l'espèce de chaos que
l'invasion des Maures y apporta, par la rapidité de leurs conquêtes, et
les événements divers de l'étendue de leur puissance, altéra l'économie
des fiefs immédiats à proportion de celle des dynasties, trop souvent
plus occupées à s'agrandir aux dépens les unes des autres que de se
défendre ensemble de l'ennemi commun de leur religion et de leur État,
tandis que cet ennemi en profitait avec autant d'adresse que de force.
Cette confusion, qui dura jusque bien près du temps des rois qui ont
usurpé le nom de \emph{catholiques} par excellence, qu'ils ont transmis
à leurs successeurs, ne laisse voir rien de bien clair ni de bien réglé
parmi ces premiers vassaux des divers royaumes des Espagnes, sinon la
part qu'ils avaient aux affaires, plus par l'autorité de leur personne,
soit mérite, soit grandes alliances, soit grands biens, que par la
dignité de ces biens mêmes. Le nom de \emph{grand} était inconnu dans
les Espagnes, celui de \emph{ricos-hombres} passait pour la seule grande
distinction, comme qui dirait \emph{puissants hommes}, et ce nom, devenu
commun à tous ceux des familles des ricos-hombres, s'était peu à peu
extrêmement multiplié. La faiblesse et le besoin des rois les obligeait
à souffrir cet abus dans les cadets subdivisés de ces ricos-hombres, ou
dans des sujets dont le mérite ou les services ne permettaient pas de
leur refuser un titre que l'exemple de ces cadets avait détaché de la
possession des fiefs immédiats, enfin aux premières charges de leur
maison\,; ce qui a peut-être donné la première idée, dans la suite, de
la distinction des trois classes des grands que nous y voyons
aujourd'hui.

Soit que l'usage de parler couvert aux rois pour les gens d'une certaine
qualité fût de tout temps établi dans les Espagnes, comme il l'était
constamment dans notre France d'être couvert devant eux jusqu'au milieu
pour le moins des règnes de la branche des Valois\,; soit que cet
honneur, d'abord réservé aux premiers vassaux pères de famille, eût peu
à peu été communiqué à leurs cadets et aux enfants des cadets avec leurs
armes si souvent chargées de bannières et de chaudières en Espagne, pour
marque de leur ancien droit, et qui ont passé avec les filles dans des
familles étrangères à ces premiers ricos-hombres à l'infini, qui
écartelèrent {[}39{]} ces armes, et souvent les prirent pleines\,; il
est certain qu'il y avait un grand nombre de ces ricos-hombres dans les
Espagnes, et qui, avec le nom, jouissaient de cet honneur de parler
couverts au roi, par droit, par abus, ou par la nécessité de s'attacher
les familles puissantes et d'éviter les mécontentements, lorsqu'y
parurent les rois catholiques.

Les deux principales couronnes des Espagnes, la Castille et l'Aragon,
qui peu à peu s'étaient réuni les autres, s'unirent entre elles par le
mariage de Ferdinand et d'Isabelle, et se confondirent dans leur
successeur pour n'être plus séparées que par certaines lois, usages et
privilèges propres à chacune d'elles. Ce sont ces deux époux qui,
apportant chacun leur couronne, en conservèrent le domaine et toute
l'administration indépendamment l'un de l'autre, et qui de là furent
indistinctement appelés \emph{les rois}, nonobstant la différence de
sexe, ce qui a passé depuis eux jusqu'à nous dans l'usage espagnol pour
dire ensemble le roi et la reine régnants, et qui enfin ne sont guère
plus connus dans les histoires par leurs propres noms, et même dans le
langage ordinaire, que par celui de \emph{rois catholiques}, que
Ferdinand obtint à bon marché des papes, et transmit à ses successeurs
jusqu'à aujourd'hui, moins par la conquête de tout ce qu'il restait aux
Maures dans le continent des Espagnes, que par la proscription des
juifs, la réception de l'inquisition, le don des papes, qu'il reconnut,
des Indes et des royaumes de Naples et de Navarre, avec aussi peu de
droit à eux de les conférer, qu'à lui de les occuper par adresse et par
force.

Devenu veuf d'Isabelle, il eut besoin de toute son industrie pour éluder
l'effet du peu d'affection qu'il s'était concilié. L'Aragon et tout ce
qui y était annexé avait des lois qui tempéraient beaucoup la puissance
monarchique et en voulait reprendre tous les usages, que l'union du
sceptre de Castille avec le sien avait affaiblis en beaucoup de façons.
La Castille avec ses dépendances ne reconnaissait plus guère Ferdinand
que par cérémonie et par vénération pour son Isabelle qui l'avait fait
régent par son testament, et tous ne respiraient qu'après l'arrivée de
Philippe Ier, dit le Beau, fils de l'empereur Maximilien lei et mari de
la fille aînée des rois catholiques, à qui la tête avait commencé à
tourner d'amour et de jalousie de ce prince, et à laquelle la Castille
était déjà dévolue du chef d'Isabelle, en attendant que l'Aragon lui
tombât aussi par la mort de Ferdinand, qui n'eut point d'enfants de
Germaine de Grailly, dite de Foix, sa seconde femme, sœur de ce fameux
Gaston de Foix, duc de Nemours, tué victorieux à la bataille de Ravenne,
sans alliance, à la fleur de son âge, tous deux enfants de la sœur de
Louis XII.

Tout rit donc à Philippe, à ce soleil levant, dès qu'il parut dans les
Espagnes, et presque tous les seigneurs abandonnèrent le soleil
couchant\,; lorsque le beau-père et le gendre allèrent se rencontrer.
Dans le dessein de plaire à Philippe, les ricos-hombres ne voulurent
point user à la rigueur du droit ou de l'usage de se couvrir devant lui,
et il en profita pour le diminuer, ou du moins pour éclaircir le nombre
de ceux qui en prétendaient la possession.

Tel fut le premier pas qui commença à limiter, et tout d'un temps à
réduire en quelque forme, ce qui bientôt après devint une dignité réglée
par différents degrés, sous le nom de \emph{grands d'Espagne}. Philippe
le Beau introduisit sans peine, par la facilité des ricos-hombres,
qu'ils ne se couvrissent plus qu'il ne le leur commandât, et il affecta
de ne le commander qu'aux grands seigneurs d'entre eux par les fiefs ou
par le mérite, c'est-à-dire à ceux dont il ne pouvait aisément se
passer. La douceur de son gouvernement, le mérite de sa vertu, les
charmes de sa personne, sa qualité de gendre et d'héritier d'Isabelle,
si chère aux Castillans, leur haine pour Ferdinand, sous l'empire duquel
ils ne voulaient pas retomber, les rendit flexibles à cette nouveauté,
qui prévalut sans obstacle. Mais Ferdinand, ne pouvant supporter sa
propre éclipse, y mit bientôt fin. Il fut accusé d'avoir empoisonné son
gendre, qui ne la fit pas longue après ce brillant voyage de prise de
possession de la couronne de Castille. Jeanne son épouse acheva d'en
perdre l'esprit de douleur. Leurs enfants étaient en bas âge, et
Ferdinand reprit les rênes du gouvernement de la Castille, avec la
qualité de régent. Sa mort les remit au grand cardinal Ximénès, dont le
nom est immortel dans tout genre de vertus et de qualités éminentes, et
que les Espagnols ne connaissent que sous le nom de Cisneros. On sait
avec quelle justice et quelle capacité il gouverna en chef après les
avoir tant montrées sous les rois catholiques, et avec quelle force et
quelle autorité il sut contenir et réprimer les plus puissants seigneurs
des Espagnes, dont toutes les couronnes, excepté celle de Portugal,
étaient réunies sur la tête de Charles, fils aîné de Philippe le, le
Beau et Jeanne la folle et enfermée, lequel devint si célèbre sous le
nom de Charles-Quint.

Ximénès mourut comme il se préparait à remettre le gouvernement entre
les mains de ce jeune prince, qui était déjà abordé en Espagne, mais
qu'il ne vit jamais. On prétendit que sa mort n'avait pas été naturelle,
et que le mérite prodigieux et la fermeté d'âme de ce grand homme
épouvantèrent les Flamands, qui à la suite et à l'abri d'un jeune roi
élevé chez eux et par eux-mêmes, venaient partager les dépouilles de
l'Espagne. C'est à cette époque que disparurent les noms de Castille et
d'Aragon, comme les leurs avaient absorbé ceux des autres royaumes des
Espagnes. Charles fut le premier qui se nomma roi d'Espagne, dont il ne
porta pas le titre depuis un an qu'il y eut débarqué. Le court espace
qu'il y demeura ne fut rempli que de troubles, d'où naquit une guerre
civile, pendant laquelle il perdit son aïeul paternel, l'empereur
Maximilien Ier. Cette mort l'obligea de repasser la mer pour recueillir
la couronne impériale qu'il emporta sur notre François Ier.

Voici la seconde gradation de la dignité de grand d'Espagne\,: plusieurs
ricos-hombres qui s'étaient introduits à la cour de Charles-Quint en
Espagne, le suivirent quand il en partit. D'autres furent invités à
l'accompagner d'une manière à ne s'en pouvoir défendre, par honneur en
apparence, en effet pour la tranquillité de l'Espagne, laissée à des
lieutenants. Les ricos-hombres qui avaient suivi Charles-Quint
prétendirent se couvrir à son couronnement impérial. Les principaux
princes d'Allemagne en firent difficulté, et Charles-Quint, déjà habile,
sut en profiter contre des gens éloignés de leur patrie, et qui, par ce
comble de grandeur de toute la succession de Maximilien Ier arrivée à
leur jeune monarque, se crurent hors d'état de lui résister. C'est ici
qu'a disparu le nom de \emph{ricos-hombres}, et que s'éleva en son lieu
celui de \emph{grands}, nom pompeux dont Charles-Quint voulut éblouir
les Espagnols, dans le dessein d'abattre en eux une grandeur innée, pour
en substituer une autre qui ne pût être qu'un présent de sa main. La
facilité que les ricos-hombres avaient eue pour Philippe le Beau fraya
le chemin de leur destruction à son fils, qui dès lors en effaça les
droits et jusqu'au nom, et qui rendit le titre de grand aux plus
distingués d'entre eux, mais en petit nombre et avec grand choix, tant
de ceux qui l'avaient suivi, que de ceux qui étaient, demeurés en
Espagne, et qui conservèrent l'usage de se couvrir, le traitement de
cousin et d'autres prérogatives.

Charles-Quint n'osa pourtant faire expédier de patentes à aucun. Il se
contenta d'avoir changé le nom, l'usage et restreint infiniment le grand
nombre de ces seigneurs privilégiés, mis leur dignité dans sa main, et
exécuté cette hardie mutation comme par une transition insensible pour
ceux qui étaient conservés dans leurs distinctions, tandis qu'il les
laissa se repaître du vain nom, qui, sous une idée trop vaste, ne
renfermait rien de propre, et de l'imagination de se trouver d'autant
plus relevés qu'ils étaient en plus petit nombre. Soit surprise, soit
nécessité, comme il y a lieu de le croire, du moins de ceux qui, cessant
d'être ricos-hombres, virent des grands sans l'être eux-mêmes, soit
appât et flatterie, ce grand changement se fit sans obstacle et sans
trouble\,; à peine en fut-il parlé, même en Espagne, où les lieutenants
de l'empereur avaient conquis ou soumis toutes les places et toutes les
provinces, et réduit tous les seigneurs.

Charles-Quint fit dans la suite de nouveaux grands en Espagne et dans
les autres pays de sa domination, tant pour s'attacher de grands
seigneurs et donner de l'émulation, que pour anéantir toute idée de
ricos-hombres, et pour marquer en effet et que la dignité de grand
d'Espagne était la seule de la monarchie, et que cette dignité unique
était uniquement en ses mains.

Mais par une politique qui allait à flatter toute la nation, et qui, à
l'exemple de celle des papes sur les cardinaux, tournait toute à sa
propre grandeur, il l'établit dans un rang, des honneurs et des
distinctions les plus grandes qu'il lui fut possible, et en même temps
{[}qu'il lui fut{]} facile de faire admettre en Italie et en Allemagne,
dictateur comme il était de celle-ci, et presque roi de celle-là, par
les exemples éclatants que son bonheur et sa puissance surent faire des
princes, des électeurs et des papes même, et plus encore des princes
d'Italie qui ne respiraient qu'à l'ombre de sa protection, l'empire,
l'Allemagne et l'Italie étant demeurés jusqu'à nos jours, depuis
Charles-Quint, comme entre les mains de la maison d'Autriche, suivant le
partage qu'il en fit lui-même en abdiquant\,; et cette maison toujours
restée parfaitement unie, le même esprit a toujours conservé dans tous
ces pays-là la même protection à la dignité de grand d'Espagne, et la
même autorité au moins à cet égard, et pour des choses déjà établies, a
maintenu les grands dans tout ce dont Charles-Quint les avait mis en
possession partout, dont l'enflure a semblé, même aux Espagnols, les
dédommager de ce qui leur a été ôté de plus réel.

Philippe II, sous prétexte d'honneur, porta une atteinte à cette dignité
pour se l'approprier davantage. Ce fut lui qui introduisit la cérémonie
de la couverture, comme ils parlent en Espagne, ou de l'honneur de se
couvrir. J'en remets la description et de ses différences pour ne pas
interrompre le gros de cette matière. S'il n'osa tenter de donner des
patentes, il exécuta pis\,: c'est que, laissant les grands qu'il trouva
dans la possession de l'honneur qu'ils avaient de se couvrir avant de
commencer à lui parler, il voulut que ceux qu'il fit commençassent
découverts à lui parler, et n'en créa aucun que de cette sorte. Ce fut
ainsi qu'il donna l'être à la seconde classe des grands, et par même
moyen qu'il forma la première classe de ceux de Charles-Quint, qui
jusqu'alors avait été l'unique.

Pour résumer un moment avant de passer outre, jusqu'ici trois espèces et
deux classes de grands. Trois espèces\,: la première, ceux qui au
couronnement impérial de Charles-Quint passèrent par insensible manière
de l'état de ricos-hombres à celui de grands, en conservant sous un
autre nom, le rang et les usages dont ils étaient en possession, et
continuant à se couvrir devant Charles-Quint sans qu'il leur dit le
\emph{cobrios}{[}40{]}**, ni qu'il parût de sa part aucune marque de
concession, tandis que le reste des ricos-hombres demeura anéanti quant
à ce litre, et à tout le rang, honneurs et usages qu'ils y prétendaient
être attachés.

La seconde espèce, ceux tant Espagnols qu'étrangers, sujets de
Charles-Quint, qu'il fit grands par ce seul mot \emph{cobrios}, qu'il
leur dit une fois pour toutes, sans cérémonie s'ils étaient présents, ou
s'ils étaient absents par une simple lettre missive d'avis, par quoi
ceux-là redevinrent ce qu'ils n'étaient plus, s'ils avaient été
ricos-hombres, ou s'ils ne l'avaient pas été, ils devinrent ce qu'ils
n'avaient jamais été\,: ces deux espèces, aussi sans concession en
forme, ce qui vient d'être expliqué pour la deuxième n'en étant pas une,
et la première encore moins, puisque ce ne fut que par une simple
tolérance d'usage qu'elle continua de jouir des prérogatives dont elle
se trouvait en possession. La troisième espèce se trouvera ci-dessous.

Deux classes donc de grands\,: la première, tous ceux de
Charles-Quint\,; la seconde, ceux de Philippe II, lesquels forment notre
troisième espèce, et la troisième gradation de la dignité de grand
d'Espagne.

Philippe III alla plus loin, et fit la quatrième gradation en donnant le
premier des patentes. Il prit le prétexte que, trouvant deux classes de
grands établies, et voulant se réserver d'en faire de l'une et de
l'autre, il était nécessaire de pouvoir les discerner par un instrument
public. Il fit en effet des grands des deux classes, mais aucun sans
patentes, et il n'y en a point eu depuis sans leur en expédier. Elles
déclarent la classe, et contiennent l'érection en grandesse d'une terre
de l'impétrant\,; à quoi le plus petit fief suffit, pourvu qu'il soit
nûment mouvant {[}41{]} du roi, ou si l'impétrant l'aime mieux,
déclarent la grandesse sans terre, sous le simple nom dudit impétrant,
après quoi il les fait enregistrer au conseil de Castille, de quelque
pays qu'il soit et en quelque lieu que sa grandesse soit située.

C'est de l'établissement de ces patentes qu'est venue, je ne dirai pas
simplement l'incurie qui pouvait avoir quelque usage antérieur fondé sur
le mélange de politesse et d'indolence de la nation ou du dépit secret
de la destruction de la \emph{rico-hombrerie}, mais l'aversion si
marquée des grands d'Espagne à observer entre eux, en quelque occasion
que ce puisse être, aucun rang d'ancienneté. Ils n'en pourraient garder
qu'à titre de dates. Ceux de Charles-Quint et de Philippe II n'ont point
de patentes, par conséquent point de date écrite qui les puisse régler.
Ceux des règnes postérieurs, qui ont tous des patentes ne veulent point
montrer cette diversité qu'ils ne s'estiment pas avantageuse, et croient
se trouver mieux de la confusion\,; tous veulent faire croire l'origine
de leur dignité obscure par une antiquité reculée, et disent qu'étant
une pour tous, même de différentes classes, tous ceux qui en sont
revêtus sont égaux entre eux, et ne se peuvent entre-précéder ni suivre
que par l'ordre qu'y met le hasard.

Ils sont en effet si jaloux de n'y point observer d'autre ordre, que, y
ayant eu chapelle au sortir de la couverture de mon second fils, il
voulut laisser des places au-dessus de lui sur le banc des grands, et y
faire passer ceux qui arrivèrent après lui, sans qu'aucun le voulût
faire. Il prit garde, par mon avis, à n'arriver que des derniers, et le
dernier même aux chapelles suivantes. On s'en aperçut\,; plusieurs
grands de ceux avec qui j'avais le plus de familiarité me dirent
franchement qu'ils sentaient bien que c'était politesse, mais qu'elle ne
les accommodait point, m'en expliquèrent la raison, et me prièrent que
mon fils ne prît plus du tout garde à la manière de se placer, et qu'il
se mit désormais parmi eux au hasard, comme ils le pratiquaient tous, ce
qu'il fit aussi après que j'eus connu leur désir. Il arriva même qu'à la
cérémonie de la Chandeleur, où les ambassadeurs ne se trouvent point,
comme je l'expliquerai ailleurs, et où j'assistai comme grand d'Espagne,
le hasard fit que mon fils me précéda à recevoir le cierge et à marcher
à la procession, singularité dont les grands parurent assez aises.

La troisième classe, fort différente des deux premières en certaines
choses essentielles, et surtout à la couverture, mais qui leur est
pareille dans tout ce qui se présente le plus souvent dans les fonctions
et dans l'ordinaire du courant de la vie, est d'une date que je n'ai pu
découvrir. S'il était permis de donner des conjectures en ce genre, je
l'attribuerais à Philippe III, sur l'exemple de Philippe II son père,
qui inventa la seconde. Ce qui me la persuaderait est l'inclination
galante et facile de Philippe III, qui eut beaucoup de maîtresses et de
favoris, et qui ne pouvant refuser ses grâces aux sollicitations des
unes et aux empressements des autres, aura inventé cette classe, qui les
satisfit pour l'extérieur sans mécontenter les autres grands, par la
disproportion effective qu'il mit entre les deux premières et cette
dernière, qui souvent n'est qu'à vie, et ne va au plus qu'à deux
générations de l'impétrant. Les autres différences entre les trois
classes se trouveront en leur lieu.

Les rois d'Espagne ont fait aussi des grands de première classe à vie en
quelques occasions particulières, et le plus souvent pour se débarrasser
des difficultés de rang en faveur des princes étrangers, auxquels, comme
tels, on n'en accorde aucun en Espagne, et qui s'y trouvent au-dessus de
toutes prétentions quand ils peuvent obtenir celui de grands, et parmi
eux et mêlés, sans nulle idée, qui n'en serait pas soufferte, de se
distinguer d'eux en quoi que ce soit. Sans en aller chercher des
exemples bien loin, le prince Alexandre Farnèse, le duc Joachim-Ernest
d'Holstein et en dernier lieu le landgrave Georges de Hesse-Darmstadt,
tué à Barcelone, général de l'armée de Charles II, furent ainsi faits
grands de la première classe pour leur personne seulement.

Il est arrivé aussi des occasions singulières qui ont engagé les rois
d'Espagne de permettre à un seigneur de se couvrir en cette occasion-là
seulement sans le faire grand d'Espagne, et c'est, entre autres
exemples, mais ceux-là fort rares, ce qui arriva lors du passage de
l'archiduchesse Marie-Anne d'Autriche par le Milanais, allant en Espagne
épouser Philippe IV. Elle était accompagnée de sa part des ducs de
Najara et de Terranova, grands d'Espagne, qui se couvraient devant elle.
Le marquis de Caracène était pour lors gouverneur du Milanais et point
grand. Philippe IV lui envoya ordre de se couvrir, mais pour cette seule
occasion, à cause de la dignité du grand emploi qu'il remplissait, et
sans le faire grand.

La distinction des classes des grands, qui fut le prétexte de leur
expédier des lettres patentes pour l'érection de leurs différentes
sortes de grandesses, en servit encore pour une autre sorte d'expédition
aussi favorable à l'autorité royale que funeste à la dignité de grandi
qui y trouva une cinquième gradation par les suites qu'elle eut, et pour
lesquelles elle fut établie, sans rien paraître d'abord de ce qui arriva
de cette expédition.

Cette autre sorte d'expédition est un certificat que le secrétaire de
l'estampille expédie à chaque grand de la date de sa couverture, et
suivant quelle classe il a été admis, qui marque le parrain qui l'y a
présenté, et la plupart des grands qui y ont assisté, de sorte que cette
expédition se donne nécessairement à tous les grands, non seulement
nouvellement faits, mais devenus tels par succession directe ou
indirecte, parce que tous indistinctement ont une fois en leur vie à
faire leur couverture.

C'est de cette couverture que dépendent tellement le rang et toute
espèce de prérogatives de la grandesse de toute classe, que le grand de
succession, même de père à fils, et non contestée, ne peut jouir
d'aucune des distinctions attachées à cette dignité qu'il n'ait fait sa
couverture, par quoi il devient vrai par l'usage que les héritiers des
grands de toute classe, même leurs fils, ne le deviennent en effet que
par la volonté du roi qui, à la vérité, accorde presque toujours cette
couverture dans la même semaine qu'elle lui est demandée, mais qui peut
si bien la refuser, et par conséquent suspendre tout effet de la dignité
dans celui qui a cette cérémonie à faire, que le refus n'en est pas sans
exemple, et pour confirmer cette étrange vérité, j'en choisirai le plus
récent et peut-être en tout le plus marqué.

J'ai suffisamment parlé ci-dessus du duc de Medina-Sidonia à propos du
testament de Charles II, pour n'avoir rien à y ajouter. Il mourut grand
écuyer, chevalier du Saint-Esprit et conseiller d'État, dans la faveur,
l'estime et la considération qu'il méritait\,; et d'une sœur du comte de
Benavente ne laissa qu'un fils unique, gendre du duc de l'Infantado. Ce
fils avait des amis, de l'esprit, de la lecture et du savoir, avec le
défaut de la retraite et la folie d'aller dans les boucheries faire le
métier de boucher, et d'un attachement à son sens et à ses coutumes, que
rien ne pouvait vaincre\,; il conserva donc la golille et l'habit
espagnol, quoiqu'on fit sa cour au roi d'être vêtu à la française. La
plupart des seigneurs s'y étant accoutumés, le roi vint à défendre tout
autre habit, excepté à la magistrature et à la bourgeoisie chez qui la
golille et l'habit espagnol furent relégués, et interdit à tous autres
de paraître devant lui vêtus autrement qu'à la française. C'était avant
la mort du duc de Medina-Sidonia, grand écuyer, qui, aidé de l'exemple
général, ne put jamais obtenir cette complaisance de son fils, lequel
s'abstint d'aller au palais. C'était au fort de la guerre\,; il y suivit
constamment le roi et son père, campant à distance, ne le rencontrant
jamais, et servant comme volontaire, se trouvant et se distinguant
partout. Son père mort, et lui devenu duc de Medina-Sidonia, il fut
question de sa couverture. De s'y présenter en golille, il n'y avait pas
d'apparence\,; vêtu à la française, il ne le voulut jamais. Conclusion,
qu'il a vécu douze ou quinze ans de la sorte, et est mort peu avant que
j'allasse en Espagne, ayant autour de cinquante ans, sans avoir jamais
joui d'aucune prérogative de la grandesse, qui, à la cour et hors de la
cour, sont également suspendues sans difficulté à quiconque n'a pas fait
sa couverture. C'est son fils qui a épousé la fille du comte de
San-Estevan de Gormaz, qui n'a pas eu la folie de son père, et qui a été
fait chevalier de la Toison d'or avec son beau-père, en la promotion que
fit Philippe V en abdiquant.

On va aisément de l'un à l'autre\,; telle est la nature des progrès
quand ils ne trouvent point de barrières. Sixième gradation de la
grandesse pour arriver au point où elle se trouve aujourd'hui. De cette
puissance de suspendre tout effet de la grandesse, les rois ont prétendu
les grandesses même amovibles à leur volonté, encore que rien
d'approchant ne se trouve dans pas une de leurs patentes. De cette
prétention s'est introduite une coutume qui l'établit puissamment, et
qui est une des différences de la première classe d'avec les autres. Le
temps précis de son commencement, je ne l'établirai pas, mais s'il n'est
pas de Philippe II, auquel il ressemble fort, et qui a établi les deux
classes en inventant la seconde, il ne passe point Philippe III. C'est
que toutes les fois que l'on succède à une grandesse qui n'est pas de la
première classe, fût-ce de père à fils, l'héritier donne part au roi par
une lettre, même de Madrid à Madrid, de la mort du grand auquel il
succède, et la signe sans prendre d'autre nom que le sien accoutumé, et
point celui de grand qu'il doit prendre, ni faire sentir, en quoi que ce
soit de la lettre, qu'il se répute déjà grand. Le roi lui fait réponse,
et dans cette réponse, le nomme non de son nom accoutumé, mais de celui
de la grandesse qui lui est échue, et le traite de cousin et avec toutes
les distinctions qui appartiennent aux grands. Après cette réponse, et
non plus tôt, l'héritier prend le nom de sa grandesse et les manières
des grands\,; mais il attend pour le rang et toutes les prérogatives la
cérémonie de sa couverture. Ainsi le roi est non seulement le maître de
suspendre tant qu'il lui plaît l'effet de la grandesse de toute classe,
en suspendant ou refusant la couverture (comme il vient d'être montré
par l'exemple du dernier duc de Medina-Sidonia, grand de première classe
et de Charles-Quint), mais encore le nom et le titre, dont les héritiers
les plus incontestables, même de père à fils, pour les grandesses qui ne
sont pas de première classe, font nécessairement un acte si authentique
de reconnaître qu'il ne leur appartient pas de le prendre, jusqu'à ce
qu'il ait plu au roi par sa réponse de le leur donner, quoique sans
concession nouvelle. De ce que ceux de la première classe n'y sont point
assujettis, je me persuade encore davantage que cet usage est né sous
Philippe II, avec la distinction des classes, et que Philippe III, qui,
pour faire passer les patentes, se servit du prétexte de faire des
grands des deux classes, n'osa envelopper dans cet usage les grands
qu'il fit de la première à l'instar de ceux de Charles-Quint, qui
n'avait connu ni cet usage ni plus d'une classe de grands.

Voilà pour du possible\,; mais du possible à l'effet il n'y a qu'un pas
pour les rois, et cet effet s'est vu sous la dernière régence. Les
histoires sont pleines des orages qui agitèrent le gouvernement de la
reine mère de Charles II pendant sa minorité, et de ses démêlés avec don
Juan d'Autriche, bâtard du roi son mari et d'une comédienne, qui,
soutenu d'un puissant parti, la força de se défaire du jésuite Nitard
qui, sous le nom de son confesseur, s'était fait l'arbitre de l'État, et
qui, par un nouveau prodige, de proscrit, de chassé qu'il était à Rome,
y devint ambassadeur extraordinaire d'Espagne, et en fit publiquement
toutes les fonctions avec son habit de jésuite jusqu'à ce qu'il le
changea en celui de cardinal. À sa faveur en Espagne succéda le célèbre
Vasconcellos, fameux par son élévation et par sa chute, plus fameux par
sa modération dans sa fortune et par son courage dans sa disgrâce, qui
le fit plaindre même par ses ennemis. Don Juan, qui voulait être le
maître, et ne pouvait souffrir de confidents serviteurs ni de ministres
accrédités auprès de la reine, s'irrita contre celui-ci, comme il avait
fait contre le confesseur, et il en vint pareillement à bout.
Vasconcellos, qui venait d'être fait grand, et dont la naissance, sans
être fort illustre, n'était pourtant pas inférieure à celle de quelques
autres grands, fut dépouillé de sa dignité, sans crimes, et fut relégué
aux Philippines, où il dépensa tout ce qu'il avait en fondations utiles
et en charité, y vécut longtemps et content, et y mourut saintement,
sans que, depuis tant de temps et tant de différents gouvernements en
Espagne, il ait été question de grandesse pour sa postérité à qui elle
devait passer, qui dure encore, et qui vit obscure dans sa province.

Telles ont été les différentes gradations de la grandesse, qui ne sont
pas encore épuisées, sur lesquelles il faut remarquer que les étrangers,
je veux dire les grands d'Espagne qui sont en Flandre et en Italie, y
jouissent de toute leur dignité sans être obligés d'en aller prendre
possession en Espagne\,; mais s'ils y font un voyage, alors ils sont
soumis à la cérémonie de la couverture, et en attendant suspendus de
tout rang. Cette triste aventure arriva sous Philippe V au dernier comte
d'Egmont, en qui cette illustre maison s'est éteinte, lequel, pour avoir
perdu son certificat de couverture du secrétaire de l'estampille, fut
obligé de la réitérer.

Mais ce n'est pas encore tout ce que l'autorité des rois s'est peu à peu
acquise sur les grands d'Espagne. En voici une septième gradation. Ils y
ont ajouté un tribut d'autant plus humiliant, que c'est celui de leur
dignité même\,; cela s'appelle l'\emph{annate} et la \emph{mediannate}.
Celle-ci se paye à l'érection d'une grandesse, et va toujours à plus de
douze mille écus argent fort. Quelquefois le roi la remet, et c'est une
véritable grâce qui s'insère dans les patentes, en sorte que l'honneur
de la dignité et la honte du tribut qui y est attaché se rencontrent
dans le même instrument, dont mes patentes de grand d'Espagne de la
première classe sont un exemple récent. Mais rien de plus ordinaire que
le refus de cette grâce, et du temps que j'étais en Espagne, le duc de
Saint-Michel de la maison de Gravina, l'une des plus grandes de Sicile,
qui y avait perdu ses biens lorsque l'empereur s'empara de ce royaume,
et qui venait d'être fait grand pour les services qu'il y avait rendus,
postulait cette remise, et ne fit point sa couverture tant que je fus en
Espagne, parce qu'elle ne lui fut point accordée, et qu'il ne se
trouvait point en pouvoir de payer. Je ne parle point encore des autres
frais qui se font à l'occasion d'une érection de grandesse, qui ne vont,
guère loin moins en salaires et en gratifications indispensables, mais
dont la remise de la mediannate, quand le roi la fait, supprime de droit
les deux tiers.

L'annate est un tribut qui se doit tous les ans à cause de la grandesse,
et si le revenu en est trop petit, parce qu'un simple fief mouvant
nûment du roi suffit pour l'établissement d'une grandesse, ou nul comme
celles qui sont seulement attachées au nom et point à une terre, comme
récemment celle du duc de Bournonville\,; alors cela s'abonne à tant par
an. Quelquefois encore celui qui est fait grand en est exempté pour sa
vie, et alors cette grâce s'insère aussi dans les patentes, et les
miennes en sont encore un exemple, mais jamais aucun des successeurs,
dont l'annate est toujours plus forte que celle de l'impétrant, et il
est arrivé à plusieurs d'être saisis, faute de payement d'années
accumulées, et d'être encore suspendus de tout rang jusqu'à parfait
payement. Outre ces deux sortes de droits, il y en a un troisième\,;
faute duquel saisie et suspension de rang se font aussi. C'est un droit
plus fort que l'annate ordinaire à chaque mutation de grand. De l'époque
précise de ces usages, je n'en suis pas instruit, mais il y a toute
apparence que, si elle n'est pas la même que celle de l'établissement
des patentes, pour le moins se sont-ils suivis de près.

Il ne faut pas oublier que la diversité des classes est une espèce de
mystère parmi les grands, qu'ils n'aiment pas à révéler, ou par vanité
d'intérêt ou par politesse pour les autres, et d'autant plus difficile à
démêler que la différence ne s'en développe qu'aux couvertures, qui
s'oublient bientôt après\,; car pour les distinctions qu'y fait le style
de chancellerie, c'est un intérieur qui demeure dans leurs papiers.

De prétendre maintenant que le nom et la dignité de grand fit connue
avant Charles-Quint, c'est ce que je crois sans aucun fondement,
d'autant qu'il ne paraît rien qui distinguât le grand du rico-hombre,
pu, si l'on veut, les ricos-hombres entre eux, du côté des prérogatives.
J'ai donc lieu de me persuader que c'est une idée de vanité, destituée
de toute réalité, pour donner plus d'antiquité à la dignité de grand, en
faire perdre de vue l'origine, et la relever au-dessus de celle des
ricos-hombres, lesquels étaient les plus grands seigneurs en naissance
et en puissance, relevant immédiatement de la couronne, et avec droit de
bannière et de chaudière, qu'ils mirent souvent dans leurs armes, d'où
on en trouve tant dans celles des maisons d'Espagne\,; or, comme le
titre de ricos-hombres, leurs armes et ces marques passèrent peu à peu à
leurs cadets, et ensuite dans d'autres maisons par les filles
héritières, c'est de là, comme je l'ai remarqué, que les ricos-hombres
étaient devenus si multipliés par succession de temps lorsqu'ils
disparurent jusqu'au nom même à l'invention de celui de grand par
l'adresse et la puissance de Charles-Quint.

Comme ce prince ne donna point de patentes pour cette dignité, il est
très difficile de distinguer, parmi les premiers grands espagnols, ceux
qui, pour ainsi dire, le demeurèrent, c'est-à-dire, qui de ricos-hombres
devinrent insensiblement grands, conservant simplement sous ce titre les
prérogatives que leur donnait celui qu'ils avaient eu jusque-là, d'avec
ceux qui, n'étant point du nombre des ricos-hombres, furent néanmoins
faits grands dans la suite par le même Charles-Quint. J'aurais du
penchant à croire que ce prince eut le ménagement de n'élever à la
grandesse que ceux de ce rang parmi les Espagnols, pour les flatter
davantage dans ce grand changement, quoique je n'aie aucun autre motif
de cette opinion que celui de la convenance. Si elle était vraie, cette
distinction à faire serait peu importante, puisqu'il ne s'agirait entre
eux que de n'avoir point cessé de jouir de leurs prérogatives, par un
passage comme insensible d'un titre ancien à un nouveau, ou d'avoir
cessé d'en jouir un temps, et d'y avoir été rétablis après par ce mot
\emph{cobrios}, dit sans cérémonie, ou par une lettre missive sans forme
de patentes, ni de vraie nouvelle concession. Quoi qu'il en soit, la
commune opinion en Espagne, et qui usurpe l'autorité de la notoriété
publique, admet en ce premier ordre de grands, devenus insensiblement
tels de ricos-hombres qu'ils étaient lors de l'établissement du titre de
grand, les ducs de Medina-Celi, d'Escalona, del Infantado,
d'Albuquerque, d'Albe, de Bejar et d'Arcos, les marquis de Villena et
d'Astorga, les comtes de Benavente et de Lémos, pour la couronne de
Castille\,; et pour celle d'Aragon les ducs de Segorbe et de Montalte,
et le marquis d'Ayetone. Plusieurs y ajoutent pour la Castille, les
ducs, de Medina-Sidonia et de Najara, les ducs de Frias et de Rioseco,
l'un connétable, l'autre amirante héréditaire de Castille, et le marquis
d'Aguilar, tous à la vérité si anciennement et si fort en tout des plus
grands et des plus distingués seigneurs, surtout Medina-Celi, qu'on a
peine à leur disputer cette même origine. On verra dans les états des
grands d'Espagne quelles maisons portaient ces titres, et de celles-là
où ils ont passé.

\hypertarget{chapitre-xiii.}{%
\chapter{CHAPITRE XIII.}\label{chapitre-xiii.}}

1701

~

{\textsc{Indifférence pour les grands des titres de duc, marquis, ou
comte.}} {\textsc{- Titre de prince encore plus indifférent.}}
{\textsc{- Succession aux grandesses.}} {\textsc{- Majorasques.}}
{\textsc{- Étrange chaos de noms et d'armes en Espagne et sa cause.}}
{\textsc{- Bâtards\,; leurs avantages et leurs différences en Espagne.}}
{\textsc{- Récapitulation sur la grandesse.}} {\textsc{- Étrange coutume
en faveur des juifs et des Maures baptisés.}} {\textsc{- Nulle marque de
dignité aux armes, aux carrosses, aux maisons, que le dais.}} {\textsc{-
Honneurs dits en France du Louvre.}} {\textsc{- Distinction de quelques
personnes par-dessus les grands.}} {\textsc{- Démission de grandesses
inconnues en Espagne.}} {\textsc{- Exemples récents de grands étrangers
expliqués.}} {\textsc{- Successeurs à grandesses ont rang et honneurs.}}

~

Il y a maintenant deux choses à expliquer\,: l'indifférence des titres
de duc, marquis et comte\,; la succession à la dignité.

Pour la première, il faut encore en revenir aux ricos-hombres, tige,
pour ainsi dire, de la dignité des grands. On a vu que ce titre de
ricos-hombres, avec toutes les distinctions qui y étaient attachées, ne
fut d'abord que pour les grands vassaux immédiats à bannière et à
chaudière, et que dans la suite de leur multiplication, usurpée ou
concédée à la nécessité du temps ou à la confusion des affaires des
divers royaumes qui ont si longtemps composé les Espagnes, les cadets de
ces ricos-hombres, leurs gendres et la postérité des uns et des autres
se maintint peu à peu dans la possession de ce titre, sans posséder ces
premiers grands fiefs, qui dans leurs auteurs en avaient été le
fondement. Lorsque les titres de duc, de marquis et de comte
commencèrent à s'introduire dans les Espagnes, ce ne fut que pour les
grands vassaux effectifs, qui étaient ces ricos-hombres premiers, dont
le titre s'étant multiplié dans la suite par la voie qui vient d'être
expliquée, elle servit de même pour la multiplication des titres de duc,
de marquis et de comte\,; et ces derniers-ci, comme bien plus modernes,
et comme n'ayant en soi dans les Espagnes aucune distinction de
prérogative attachée, n'était qu'un accompagnement indifférent au titre
de rico-hombre\,; il fut aussi dès lors indifférent d'être duc, marquis
ou comte, parce que l'unique distinction éclatante et supérieure à toute
autre, n'était attachée, qu'au titre de rico-hombre. Bien est vrai que
le duché manquait et fut effectivement une terre plus noble et plus
grande que le marquisat et le comté, et c'est ce qui fit que tous les
ducs espagnols d'alors, se trouvant les plus distingués seigneurs et les
plus riches d'entre les ricos-hombres, passèrent tous de ce titre à
celui de grand, sous Charles-Quint, sans concession et comme
insensiblement\,; or comme il n'y eut plus alors que la grandesse à qui
le rang et les prérogatives fussent attachés comme ils l'étaient
uniquement auparavant à la rico-hombrerie, à laquelle les titres de duc,
marquis et comte étaient indifférents parce qu'ils ne lui donnaient
rien, ces mêmes titres, ne donnant rien aussi à la grandesse, lui furent
également indifférents. Il est pourtant vrai que, dans les Espagnols
naturels, duc et grand sont synonymes\,; non pas seulement en tant
seulement que duc ait aucune prérogative au-dessus du marquis et du
comte comme tels, mais bien parce que depuis Charles-Quint tous les ducs
espagnols passèrent de la rico-hombrerie à la grandesse\,; et ce prince
et ses successeurs ont si peu érigé de duchés en Espagne sans y joindre
en même temps la grandesse, que de ce peu-là même il n'y en a plus aucun
qui ne soit devenu grandesse ou qui ne soit tombé à des grands.

Le titre de prince est si peu connu en Espagne, et en même temps si peu
goûté, qu'aucun Espagnol ne l'a jamais porté, jusqu'aux enfants des
rois, si on en excepte quelques-uns des héritiers présomptifs de la
couronne, à qui le titre de prince des Asturies est affecté en
reconnaissance de l'attachement de cette province à ses rois du temps
des Maures, et par laquelle ils recommencèrent à régner, et à s'opposer
à ces infidèles. Encore fort peu d'aînés l'ont-ils porté, la singularité
du nom d'infant et d'infante, qui ne signifie pourtant que l'enfant,
jointe à l'usage, ayant toujours prévalu pour ceux des rois. Les
étrangers sujets d'Espagne, qui dans leur pays portent le titre de
prince, l'ont apporté avec eux en Espagne, sans rang aucun pour les
sujets ou non sujets, s'ils ne sont grands, et sans donner aux Espagnols
naturels la moindre envie de s'accoutumer pour eux-mêmes à ce titre,
quelques droits qu'ils y puissent prétendre, suivant d'autres manières
qui ont prévalu chez leurs voisins à bien meilleur marché.

La manière de succéder à la dignité de grand n'a rien de distinct de la
manière de succéder aux biens\,; et comme ils passent tous sans
distinction en quenouille et de femelle en femelle à l'infini, aussi
font les grandesses, avec la confusion de noms et d'armes qu'entraîne ce
même usage, établi parmi les Espagnols, de joindre à son nom tous les
autres noms de ceux des biens desquels on devient héritier, surtout avec
les grandesses, qui se substituent ainsi à l'infini, à la proximité du
sang, sans distinction de mâle et de femelle, sinon du frère à la sœur,
ou en quelques maisons ou occasions peu communes, de l'oncle paternel à
la nièce.

Ce sont, pour le dire en passant, ces substitutions de terres érigées ou
non en grandesses qu'ils appellent \emph{majorasques}, et qui ne peuvent
jamais être vendues pour dettes ni pour aucun cas que ce soit, mais qui
se saisissent par les créanciers pour les revenus seulement, et jusqu'à
une certaine concurrence, dont une partie plus ou moins légère, selon la
dignité des terres et leurs revenus, demeure au propriétaire pour
aliment avec les casuels. C'est ce qu'ils croient être le salut des
maisons, et c'est par cette raison que presque toutes les terres sont
substituées en Espagne\,; de là vient que, n'y ayant point de fin à ces
substitutions, il y a si peu de terres dans le commerce, et que ce peu
qui y pourraient être n'y sont plus en effet, parce qu'elles deviennent
le seul gage des créanciers, et qu'elles ne se peuvent acheter en
sûreté. J'eus la permission du roi et du roi d'Espagne d'en acheter une
en Espagne et d'y établir ma grandesse. Je me bornai même au plus petit
fief relevant nûment du roi. Je me retranchai après à l'acheter cher
sans aucun revenu. En deux années de recherches il me fut impossible
d'en trouver, quoique plusieurs personnes de considération et du conseil
même s'y soient soigneusement employées. Je ne dis pas que cela ne se
puisse trouver, mais je dis que cela est extrêmement difficile. Il ne
faut pas oublier que les héritiers de ces substitutions héritent aussi
de tous les domestiqués, femmes et enfants de ceux dont ils héritent,
qui se trouvent chez eux ou entretenus par eux\,; de manière que, par
eux-mêmes ou par ces successions, ils s'en trouvent infiniment chargés.
Outre leur logement, chez eux ou ailleurs, ils leur donnent à chacun une
ration par jour, suivant l'état et le degré de chaque domestique, et à
tout ce qui en peut loger chez eux deux tasses de chocolat à chacun tous
les jours. Du temps que j'étais en Espagne, le duc de Medina-Celi, qui,
à force de substitutions accumulées dont il avait hérité, était onze
fois grand, et qui depuis a hérité encore de plusieurs autres
grandesses, avait sept cents de ces rations à payer par jour. C'est
aussi ce qui les consume.

Mais pour revenir à ces héritages, il arrive souvent que les héritiers
par femmes des grandes maisons et par plusieurs degrés femelles laissent
tout à fait leurs propres noms et armes, que dans la suite un cadet
reprend quelquefois, tellement que, dans la multitude des noms et des
armes, qui souvent ne se suivent pas, quelquefois même dans l'unicité,
ce n'est pas une petite difficulté parmi les Espagnols, même entre eux,
de démêler le vrai nom d'avec ceux qui ont été ajoutés, ou de savoir si
tel nom qui se porte seul est le véritable. Ainsi des armes de celles-ci
je n'en ai pu avoir le temps que fort en gros. Pour les noms, c'est ce
qui m'a donné le plus de peine à bien éclaircir sur les lieux avec ceux
qui passaient pour être les plus instruits sur ces matières et sur
celles de la grandesse, d'aucun desquels je n'ai été plus satisfait ni
plus pleinement que du profond savoir du duc de Veragua, fils de celui
dont j'ai fait mention en parlant du testament de Charles II, qui m'a
fait la grâce de vouloir bien m'en instruire avec une bonté, une
simplicité, une patience et une exactitude peu communes. Je dois encore
à la vérité cette justice aux \emph{Recherches historiques et
généalogiques}, d'Imhof\footnote{\emph{Recherches historiques et
  généalogiques des grands d'Espagne}, par Imhof. Amsterdam, 1707,
  in-12.}, \emph{des grands d'Espagne}, que j'y portai exprès, qu'elles
y sont estimées des connaisseurs, et qu'elles m'ont infiniment aplani de
difficultés, soit en m'apprenant un grand nombre de choses que j'ai
trouvées vraies par l'information la plus scrupuleuse et la plus
multipliée que j'en ai pu prendre, soit par m'avoir donné lieu à des
questions nombreuses qui m'ont beaucoup instruit dans le peu que je le
suis\,; soit encore en m'apprenant à me défier des meilleurs livres par
trouver des fautes en celui-ci, en recherchant exactement en mes
conversations la vérité ou la fausseté, et le mélange de toutes les deux
de plusieurs choses qu'il avance, mais non bien importantes. Avec un
plus long séjour, moins de fonctions et d'occupations, et le
\emph{Tizon, d'Espagne} à discuter comme j'ai fait les \emph{Recherches}
d'Imhof, j'aurais pu rapporter de bonnes choses\,; mais ce livre, jamais
je ne l'ai pu recouvrer. Ils l'ont bien quelques-uns, en Espagne, et
sourient quand on leur en parle, sans s'en expliquer jamais. Ils l'ont
fait supprimer tant qu'ils ont pu partout à force de soins, d'autorité
où elle a eu lieu, et même d'argent, parce qu'il prétend prouver que
presque toutes les maisons considérables et les plus distinguées
d'Espagne sont bâtardes, et souvent plus d'une fois, en quoi presque
tous les grands et les plus hauts seigneurs d'Espagne sont enveloppés.
Quoique leur bâtardise cachée, s'ils en ont, m'ait échappé, et ce s'ils
en ont n'est pas douteux en général, il faut néanmoins dire un mot de
leurs sentiments et de leurs usages pour la grandesse et pour les
successions par rapport aux bâtards.

Convenons de bonne foi qu'à cet égard l'Espagne se sent encore d'avoir
été pendant plusieurs siècles sous la domination des Maures, et du
commerce de mélange qu'elle eut depuis avec eux presque jusqu'au règne
des rois catholiques. Car il est très vrai qu'elle ne sent pas assez
toute la différence d'une naissance légitime d'avec une naturelle
provenue de deux personnes libres. Ces sortes de bâtards héritent sang
difficulté presque comme les légitimes, et sont grands par succession,
s'il ne survient un légitime par le mariage du père\,; en ce cas, le
bâtard a sa part de droit qui peut même être grossie jusqu'à un certain
point par la volonté du père. De ceux-là sont sorties des maisons
puissantes et très difficiles à démêler d'avec les légitimes. Ils
deviennent grands, non seulement par succession directe, à faute de
légitime, mais encore par succession féminine et collatérale\,; et si
cette sorte de bâtard est fils d'un fort grand seigneur, et aimé de lui,
il trouve à se marier très souvent aussi bien que s'il était légitime.
Lui passé, il n'y a plus de différence.

Les bâtards d'une fille et d'un homme marié ont aussi leur part, mais
très légère\,; s'il y a un légitime, ils sont tout à fait sous sa main,
le père alors ayant les siennes bien plus liées à l'égard du bâtard.
Ceux-ci n'ont pas la même part aux successions femelles et collatérales
que ceux de deux libres, lesquels, à faute de frères et de sœurs
légitimes, les recueillent entièrement. Néanmoins cette espèce
adultérine ne laisse pas de trouver des partis avantageux, s'ils sont
sans frères et sans sœurs légitimes, ou s'ils sont fils de fort grands
seigneurs qui les aiment, leur postérité perd avec le temps la
flétrissure de son origine, et supplée quelquefois en tout à la
légitime, quoique bien plus rarement que l'autre espèce de simples
bâtards. On en a vu de toutes les deux, ayant des frères légitimes, être
faits grands par le crédit de leurs pères, et fonder alors de plain-pied
des maisons presque pareilles à celles dont ils sortaient par bâtardise,
et dans la suite, leur postérité et la légitime tout à fait confondues.
Il y a encore des exemples récents de ces sortes de grands. Tel est
aujourd'hui un bâtard du duc d'Abrantès, frère du duc de Liñarès, mort
sans enfants vice-roi du Mexique, sous le commencement du règne de
Philippe V, et frère de l'évêque de Cuença, devenu duc d'Abrantès par la
mort de ce frère et de son père, duquel j'ai parlé à propos du plaisant
adieu qu'il fit à l'ambassadeur de l'empereur le jour de l'ouverture du
testament de Charles II. Cet évêque, qu'on n'appelle jamais que le duc
d'Abrantès, a trouvé le crédit à mon départ d'Espagne, c'est-à-dire fort
peu après, de faire faire grand ce frère bâtard, pour soutenir la maison
éteinte, que j'ai expliquée plus haut, et on le nomme le duc de Liñarès.
Ce sont ces usages plus qu'abusifs qui ont donné cette distinction aux
grands mariés comme aux non mariés, que leurs bâtards, et comme tels,
sont admis dans l'ordre de Malte, comme chevaliers de justice, sans
différence des légitimes. Il faut sur cela remarquer qu'après la perte
de Rhodes, cet ordre, devenu errant et prêt à se dissiper, fut protégé
et recueilli par Charles-Quint, qui lui donna l'île de Malte en toute
souveraineté, fors l'hommage annuel de quelques oiseaux pour la chasse,
et qu'encore aujourd'hui l'ambassadeur de Malte ne se couvre point en
aucun cas devant le roi d'Espagne, bien qu'il le reçoive en audience
publique où les grands assistent couverts, et où je me suis trouvé comme
grand avec eux, quoique cet ambassadeur jouisse à Madrid, et par toute
l'Espagne, de toutes les autres prérogatives du caractère d'ambassadeur,
excepté aux chapelles, où il n'a ni place ni fonction. Or, cette
obligation envers la couronne d'Espagne, jointe aux usages particuliers
à ce seul pays sur les bâtards, peut avoir eu grande part à l'admission
de ceux des grands dans l'ordre de Malte. Je dis ce seul pays, les
comtes de Guldenlew ne pouvant faire exemple dans ce recoin du Nord,
demi païen encore dans sa domination, puisque ces bâtards des rois de
Danemark n'en font pas même pour la Suède, ni pour tout le reste du
Nord, qui n'abhorre pas moins la bâtardise qu'on la déteste et qu'on
l'anéantit dans toute l'Allemagne.

Pour les doubles adultérins, ils demeurent dans toute l'Espagne dans une
entière obscurité, faute de ne pouvoir nommer leur mère, et d'avoir
trouvé un jurisconsulte comme Harlay, lors procureur général du
parlement de Paris, qui ait appris à faire reconnaître des enfants sans
mère. Quels que soient ces restes de mœurs, morisques\footnote{Mœurs
  morisques ou mœurs mauresques, des Maures.} qui infectent encore
l'Espagne, elles n'y vont pas jusqu'à connaître ceux-ci, pour lesquels
toute l'horreur et le néant dû à la naissance illégitime s'est rassemblé
sur les doubles adultérins, dont la monstrueuse espèce ne peut être
censée\footnote{Comptée.} dans aucune sorte d'existence.

Les exemples des don Juan, bâtards de filles et de leurs rois,
confirment ce que je viens d'expliquer, et qui s'entendra et
s'expliquera mieux encore par là, en se souvenant que ceux des
particuliers ont les mêmes droits, proportion gardée, qui est ce qui
élève tant ceux des grands, et qui met ceux des rois comme au niveau des
princes légitimes.

Ramassons en deux mots ce qui vient d'être expliqué de l'essence de la
dignité de grand d'Espagne.

Nulle mention d'elle avant Charles-Quint.

Ricos-hombres, ou puissants hommes qui étaient grands et immédiats
feudataires des divers royaumes des Espagnes avec droit de bannière et
de chaudière, y étaient la seule dignité connue jusqu'à nous, parlaient
couverts à leurs rois, et se mêlaient des grandes affaires. Si à titre
de droit ou de puissance, d'usage ou de concession\,; si de succession
ou de besoin que les rois avaient d'eux, obscurité entière. Pareille
obscurité sur leurs autres prérogatives et fonctions.

Se multiplièrent cadets, même collatéraux par femmes, et de femmes en
femmes, par mérite, après service ou besoin, enfin par grandes charges,
sans posséder ces grands fiefs immédiats\,; devenus ricos-hombres,
prirent bannières et chaudières\,; d'où si fréquentes aux armoiries.

Tels étaient-ils devenus sous les rois catholiques.

Leur complaisance pour Philippe le Beau en haine de Ferdinand, coup
mortel à leur dignité.

Puissance de Charles-Quint\,; son adresse à son couronnement impérial
les anéantit, et comme par insensible transpiration, leur substitua sans
concession, sans cérémonie, la nouvelle dignité de grand d'Espagne,
d'abord d'entre les ricos-hombres, puis d'autres\,; leur conserva le
droit de lui parler couverts et leur en procura de grands en Allemagne
et en Italie par politique, et qui subsistent encore par l'appui de
cette même puissance de la maison d'Autriche et de cette même politique.

Cérémonie de la couverture et distinction de deux classes de Philippe
II.

Concessions et patentes de Philippe III, auteur vraisemblable de la
troisième classe, d'où mystère des classes, aisé parmi les grands, et
leur aversion d'aucun rang d'ancienneté entre eux.

Prétention des relis, née des patentes, de la nécessité de leur
consentement pour succéder à la grandesse, même en directe, établie par
l'usage, et la manière de donner part au roi, et d'en recevoir la
réponse, dont la première classe est seule exempte.

De là encore prétention des rois d'en suspendre le rang passée en usage,
dont divers exemples, tant en refusant d'admettre à la couverture, qu'en
autres cas.

Certificat de couverture sans lequel nul rang, même l'ayant faite si le
certificat est perdu, et alors la réitérer, dont exemples. Grands
étrangers habitant hors de l'Espagne exceptés, si ce n'est qu'ils y
aillent, même en passant\,: alors soumis.

Prétention dés rois, née des précédentes, de pouvoir priver de la
grandesse sans crime d'État, ni autre grave, dont exemple en
Vasconcellos et de sa postérité jusqu'à aujourd'hui.

Des patentes et de l'établissement successif de ces prétentions sont nés
les tributs à raison de la dignité. Ils sont trois\,:

Mediannate, qui au moins va à plus de quarante mille livres pour le roi
seul, sans les autres sortes de salaires et d'autres droits\,; se paye
au roi à chaque érection de grandesse\,; se remet quelquefois, et alors
la remise s'exprime dans les patentes mêmes\,; se demande quelquefois,
et est refusée, dont exemples\,;

Annate, qui est un droit annuel plus ou moins fort, mais moindre que la
mediannate\,; il ne se paye point par l'impétrant, et ne se remet jamais
aux successeurs\,;

Mutation, autre droit, moins fort que le premier, plus fort que le
dernier, qui se paye par tout successeur à son avènement à la grandesse,
et ne se remet jamais. Droits contraints par saisie et par suspension de
rang quand il plaît au roi, jusqu'à parfait payement, dont plusieurs
exemples.

Fief le plus petit en tout genre, mais relevant immédiatement du roi,
suffit pour établir une grandesse\,; elle s'établit quelquefois sur le
nom, sans fief, dont exemples existants, à l'imitation des
ricos-hombres, cadets, sans grands fiefs dans leurs décadences\,: en ces
cas, abonnement pour fixer la quotité des tributs susdits.

Indifférence entière, parmi les grands, des titres de duc, marquis et
comte, venue de ce que ces titres s'établirent en Espagne vers la fin
des ricos-hombres, dont la dignité, étant unique, rie reçut rien de ces
titres que la simple dénomination\,; la grandesse ayant été substituée à
la rico-hombrerie pour unique dignité d'Espagne, les titres de duc,
marquis et comte y sont restés de même condition qu'auparavant, encore
que, dans le fait, il ne reste plus aucun duc espagnol qui, par
succession de temps, ne soit devenu grand, espagnol s'entend, et dont le
duché soit en Espagne. De pareille condition de ces trois titres est
celui de prince, qui ne donne et n'ajoute quoi que ce soit par lui-même
en Espagne, et que nul Espagnol naturel n'a encore porté.

Rien de distinct en la succession aux grandesses de la manière de
succéder à tous les autres biens. Les femelles en sont capables en tout
temps en Espagne, et sont préférées aux mâles par la proximité du sang,
et ainsi de femelles en femelles. Appelées de même aux substitutions des
terres ou majorasques, qui sont très fréquentes et toujours à
l'infini\,; d'où naît la difficulté du commerce des terres de toute
espèce qui se trouvent presque toutes substituées, et les autres
soumises aux créances. De là encore cette obscurité presque impénétrable
des vrais noms et des vraies armoiries, qui tombent aux appelés avec les
biens.

Ce qui ajoute encore avec indécence à cette obscurité est l'ancienne
coutume de donner aux Maures et maintenant encore aux juifs qui se
convertissent, et que les grands seigneurs tiennent au baptême, non
seulement leur nom de baptême, mais celui de leur maison, avec leurs
armes qui passent pour toujours dans ces familles infimes, et qui, avec
le temps, les confondent avec les véritables, et les leur substituent
encore plus aisément lorsqu'elles viennent à s'éteindre.

Bâtards en Espagne ont des avantages inconnus chez toutes les autres
nations chrétiennes, venus du mélange avec les Maures qui y a si
longtemps duré.

Peu de différence des bâtards de deux libres d'avec les légitimes un peu
plus de ceux d'une fille et d'un homme marié. Ils héritent et sont
capables de recueillir les substitutions. De là plusieurs maisons de
cette origine, et quelquefois redoublée, qui n'en sont guère moins
considérables. D'autres en nombre dont ce défaut est obscur. Pour ceux
d'une femme mariée, ou les doubles adultérins, leur proscription et
l'infamie de leur origine est telle en Espagne qu'elle devrait être
partout, c'est-à-dire sans espérance et sans exemple d'exception. Ils y
sont sans nom, sans biens, sans existence.

Du fond de la dignité même de grand d'Espagne que je viens d'essayer
d'expliquer, il en faut venir aux usages, et commencer par ceux qui nous
sont connus et qu'ils n'ont pas.

Les grands ni leurs femmes n'ont aucune marque de dignité sur leurs
carrosses ni à leurs armes\,; ce n'est point l'usage en Espagne pour
aucune charge ni dignité que ce soit. Si quelques-uns d'eux conservent
ces anciennes distinctions des bannières et des chaudières des
ricos-hombres, elles sont communes à tous ceux de leur maison qui ne
sont point grands, et se mettent dans l'écu en bordure ou en écartelure.
Il n'y a pas jusqu'aux petits hommes armés et à cheval du connétable de
Castille, et aux ancres de l'amirante qui ne soient en bordure. Il est
pourtant vrai, que quelques-uns, en petit nombre, portent les bannières
en dehors de l'écu, et quelquefois même l'en environnent\,; mais cela ne
tient point lieu de marque de dignité en Espagne. Pour la Toison d'or,
ceux qui l'ont en portent le collier autour de leurs armes, et
pareillement celui du Saint-Esprit, ceux à qui on l'a donné. Depuis que
les ducs de France et les grands d'Espagne fraternisent en rang et en
honneurs, il y a plusieurs de ceux-ci qui, en Espagne et sans en être
jamais sortis, ont pris le manteau ducal\,; peu de grands espagnols
naturels l'ont encore fait. La reine même n'a point de housse.

Les balustres\footnote{Les lits et tables des rois et des grands étaient
  entourés de balustres ou balustrades, qui en fermaient l'accès.} et
les autres distinctions extérieures y sont inconnues, même chez le roi
et la reine, excepté le dais\,; mais ce dais descend chez tous les
\emph{titulados}, dont il y en a quelquefois de fort étranges\,:
j'expliquerai ce que c'est en son temps. Toute la différence est que les
dais de ceux-ci ne sont que de damas tout simple, avec un portrait du
roi dessus, et que ceux des grands sont de velours et riches, sans
portrait, avec quelquefois leurs armes brodées dans la queue. Ainsi les
dais des uns paraissent être pour le portrait, et celui des autres pour
leur dignité et pour eux-mêmes. À l'égard des balustres, peut-être que
l'usage de coucher en des lieux retirés qu'on ne voit point, et de
n'avoir point de ces lits qui ne sont que pour la parade, en a banni la
distinction.

La manière de bâtir en Espagne fait que ce que nous appelons en France
les honneurs du Louvre\footnote{Les \emph{honneurs du Louvre} étaient le
  privilège accordé à certains personnages d'entrer dans la cour du
  Louvre eu carrosse ou à cheval. Favin prétend, dans son \emph{Théâtre
  d'honneur et de chevalerie} (t. Ier, p.~371), que les honneurs du
  Louvre n'étaient accordés primitivement qu'aux princes et princesses
  du sang. Dans la suite, on les étendit aux princes étrangers, au
  connétable, aux cardinaux, enfin à tous les ducs.} n'y peut exister.
Les palais du roi, et tous les autres, ont une grande porte cochère, à
condition qu'aucun carrosse n'y peut entrer\,; mais il y en a une image.
Après, cette porte il y a, au palais de Madrid, un grand vestibule noir
et obscur, couvert, court, mais qui s'étend en deux petites ailes, et
qui aboutit à quelques marches d'une galerie qui sépare deux cours
pavées de grandes pierres plates, avec un grand escalier tout en dehors
au bout de cette galerie. Dans ce vestibule couvert entrent les
carrosses des grands et de leurs femmes, des cardinaux et des
ambassadeurs, et en ressortent dès qu'ils sont descendus à la galerie\,;
ils rentrent de même pour les prendre quand ils veulent remonter pour
s'en aller. Tous les autres hommes et femmes descendent et remontent
devant la grande porte, et tous les carrosses se rangent dans la grande
place du palais. Au Buen-Retiro, entre plusieurs cours, il y en a deux
de suite, comme au Palais-Royal à Paris, mais infiniment plus grandes.
Tous les carrosses entrent dans la première et y restent. Les seuls
grands et leurs femmes, les cardinaux et les ambassadeurs entrent dans
les leurs sous le corps de logis qui sépare les deux cours, et y
descendent dans une galerie ouverte qui conduit au bas du degré, et
leurs carrosses passent outre dans la deuxième cour pour y tourner. Ils
les allaient attendre après dans la première, et entraient comme en
arrivant quand leurs maîtres ou maîtresses voulaient y remonter pour
s'en aller. Maintenant, c'est-à-dire longtemps avant que j'allasse en
Espagne, et je ne sais sous quel règne, leurs carrosses demeurent dans
la seconde cour, et ne font plus qu'avancer pour reprendre leurs maîtres
ou leurs maîtresses où ils les ont descendus. Ce dernier petit avantage
était encore nouveau de mon temps, peut-être sur l'exemple des
ambassadeurs qui l'ont toujours eu.

Il faut se souvenir ici des distinctions extrêmes qu'on a vues plus haut
du président et même du gouverneur du conseil de Castille par-dessus les
grands qui arrêtent devant lui dans les rues, qui n'en ont pas la main
chez lui, et qui n'en sont point visités en quelque occasion que ce
soit, qui est reçu et conduit au carrosse par un majordome quand il va
au palais, et qui y est seul assis en troisième avec le majordome-major
et le sommelier du corps, en attendant que le roi paroisse ou qu'il soit
appelé dans le cabinet, en présence de tous les grands debout\,;

De celle du majordome-major du roi, qui partout les précède tous, et en
place distinguée, et qui est assis à côté du roi, au bal, à la comédie,
aux audiences singulières, les grands debout, et qu'il est comme leur
chef\,;

De celle du majordome-major de la reine, qui chez elle, aux audiences,
les précède tous\,;

De celle des cardinaux, sur eux qui, en présence du roi, sont extrêmes,
mais nulles en son absence. J'aurai occasion d'en parler ailleurs.

Enfin, de celles des ambassadeurs qui, à la vérité, sont peu sensibles
et ne se rencontrent pas souvent.

J'ai remarqué celles des conseillers d'État, même point grands, qui à
leur exclusion, ont le droit d'aller en chaise à porteurs comme les
dames.

À l'égard de celles-ci, toutes celles d'une qualité distinguée, sans
distinction des femmes de grands, se font souvent porter en chaise par
la ville et même au palais, dans l'escalier, jusqu'à la porte extérieure
de l'appartement de la reine, où leurs chaises et leurs porteurs les
attendent, sans \emph{le mezzo-termine}, trouvé à Versailles, de payer
pour faire porter les livrées du roi aux porteurs des personnes qui
n'ont pas les honneurs du Louvre. La vérité est qu'il n'y a guère que
les dames du palais, et fort peu d'autres grandes dames, femmes de
grands, à qui je l'ai vu faire. À propos de livrée, souvent on n'en a
point, puis on en reprend, et jamais presque les mêmes. Jusqu'au fond de
la couleur de la livrée, on la change presque tous les ans dans la même
maison. Elles sont la plupart sombres et toutes fort simples, et les
carrosses et les chaises au-dessous de la simplicité. Les boues de
Madrid l'hiver, sa poussière l'été, et l'air qui résulte de la quantité
et de la nature étrange de ces boues, qui ternit les meubles et jusqu'à
la vaisselle d'argent, est cause de cette grande simplicité, mais qui
n'est pas pour les ambassadeurs.

Les grands n'ont point l'usage de se démettre de leur dignité comme les
ducs, en France\,; mais en Espagne, le successeur direct d'une grandesse
et sa femme ont des honneurs et un rang, en attendant qu'elle leur soit
échue par la mort de celui à qui ils ont droit de succéder. Le comte de
Tessé, en faveur duquel le maréchal son père eut la permission d'en user
comme les ducs à leur exemple ne serait pas traité ni reconnu comme
grand en Espagne du vivant de son père. La chose faite et le rang pris
ici, on en tira un consentement du roi d'Espagne, parce qu'il ne devait
point avoir d'usage en Espagne, où le comte de Tessé ne devait point
aller, et encore ce consentement fut-il difficile et tardif. Philippe V
a pourtant fait deux exceptions à cette règle, que nul autre roi n'avait
enfreinte avant lui.

La première fut en faveur du duc de Berwick, auquel, en récompense de
ses services après la bataille d'Almanza, il donna la grandesse de
première classe, les duchés de Liria et de Quirica, anciens apanages des
infants d'Aragon, pour y établir sa grandesse et jouir, en propriété, de
ses terres de quarante mille livres de rente\,; la liberté d'y appeler
tel de ses enfants qu'il voudrait, pour en jouir même de son vivant et
sa postérité ensuite\,; la faculté de changer ce choix pendant toute sa
vie, et le pouvoir de le changer encore par son testament, toutes grâces
inouïes et proportionnées à l'importance de la victoire d'Almanza. En
conséquence, son fils aîné eut en Espagne la grandesse, les duchés, et
porta le nom du duc de Liria où il s'établit, puissant par son mariage
avec la sœur du duc de Veragua qui en recueillit depuis le vaste et
riche héritage.

L'autre exception fut faite en faveur de la fonction dont je fus honoré
d'aller ambassadeur extraordinaire en Espagne faire la demande de
l'infante pour le roi, conclure le futur mariage, en signer le contrat
et assister de sa part au mariage du prince des Asturies avec une fille
de M. le duc d'Orléans, lors régent du royaume. À l'instant que la
cérémonie en fut achevée, le roi d'Espagne s'avança à moi dans la
chapelle même du château de Lerma, et avec mille bontés me fit l'honneur
de me dire qu'il me donnait la grandesse de la première classe pour moi,
et en même temps pour celui de mes deux fils que je voudrais choisir
pour en jouir dès à présent avec moi, et la Toison d'or à l'aîné. Comme
j'avais la permission de l'accepter, je choisis sur-le-champ le cadet,
et les lui présentai tous deux pour le remercier, avec moi, de ses
grandes grâces, puis à la reine qui ne me témoigna pas moins de bontés,
auxquelles j'eus le bonheur de voir toute la cour applaudir, à laquelle
aussi j'avais tâché de plaire. Comme on retournait deux jours après à
Madrid, on remit à y faire la réception de l'un et la couverture de
l'autre.

Il est bon toutefois de remarquer que ces deux exemples ont été faits en
deux occasions uniques en faveur de deux étrangers à l'Espagne, pour
deux personnes dont la démission ne multipliait rien, parce que, comme
ducs de France, nous avions déjà les mêmes rangs, honneurs et
prérogatives en Espagne que les grands, droit et usage de nous trouver
partout avec et parmi eux, qui étaient bien aises que j'en profitasse
souvent. Ce fut aussi ce qui nous empêcha, M. de Berwick et moi, de
faire pour nous-mêmes la cérémonie de la couverture, parce qu'elle ne
nous donnait rien dont nous ne fussions en possession entière\,; aussi
assistai-je parmi les grands, et couvert comme eux, à la couverture de
mon fils, qui est une cérémonie où les ambassadeurs ne se trouvent
point.

\hypertarget{chapitre-xiv.}{%
\chapter{CHAPITRE XIV.}\label{chapitre-xiv.}}

1701

~

{\textsc{Cérémonie de la couverture et ses différences pour les trois
différentes classes chez le roi d'Espagne, et son plan.}} {\textsc{- La
même cérémonie chez la reine d'Espagne, et son plan.}} {\textsc{- Tout
ancien prétexte de galanterie pour se couvrir aboli.}} {\textsc{-
Distinction de traits et d'attelages.}} {\textsc{- Femmes et
belles-filles aînées de grands seules et diversement assises.}}
{\textsc{- Séance à la comédie et au bal.}} {\textsc{- Grands, leurs
femmes, fils aînés et belles-filles aînées expressément et seuls invités
à toute fête, plaisir et cérémonie, et à quelques-unes les
ambassadeurs}}

~

Après avoir parlé des usages que nous connaissons et que les grands
d'Espagne n'ont pas, il faut venir au rang\,; honneurs et prérogatives
dont ils jouissent, et conclure après, tant de celles qu'ils ont que de
celles qu'ils n'ont pas, quelle idée juste on doit avoir de leur
dignité. Comme la clef du rang et des honneurs dont les grands d'Espagne
jouissent est la cérémonie de leur couverture, comme on l'a vu plus
haut, et que c'est encore où la différence des classes des grands est
presque uniquement sensible, il faut commencer par sa description. Elles
sont toutes semblables suivant leurs classes, tout y est tellement réglé
qu'il n'y a point à s'y méprendre, ni à y accorder ou retrancher quoi
que ce soit. Comme je n'ai vu que celle de mon fils, on ne trouvera donc
pas étrange que ce soit celle-là que je décrive, puisque, de même
classe, toutes sont en tout parfaitement semblables.

D'abord le nouveau grand ou celui qui succède à un autre, car cela est
pareil pour la couverture, visite tous les grands\,; j'y menai mon fils.
Ensuite il en choisit un pour être son parrain. L'amitié, la parenté et
d'autres raisons semblables en font faire le choix, et ce choix lui est
honorable. Je crus en devoir prier un grand et principal seigneur, bien
avec le roi d'Espagne et qui fût agréable à notre cour\,; c'est ce qui
m'engagea à prier le duc del Arco, grand écuyer et favori du roi, qu'il
avait fait grand, de faire cet honneur à mon fils. C'est au parrain à
prendre l'ordre du roi du jour de la cérémonie, d'en faire les honneurs,
tant au palais que chez le nouveau grand, de l'avertir du jour marqué,
et d'en avertir aussi le majordome major du roi, qui a soin d'envoyer un
billet d'avis à tous les grands. Ce dernier, à l'occasion de mon fils,
prétendit que c'était à lui à demander le jour au roi et m'en fit faire
quelque insinuation. J'évitai de l'entendre pour ne pas blesser un si
grand et si respectable seigneur, ni le grand écuyer aussi, et avec lui
tous les glands\,; j'en avertis néanmoins ce dernier qui s'éleva
d'abord, mais qui, en ma considération, l'ignora, et prit cependant
l'ordre du roi d'Espagne qui le donna pour le\footnote{Le mot est en
  blanc dans le manuscrit.}.., et c'est toujours le matin.

Le jour venu, le parrain invite un, deux ou trois grands comme tels, et
qui bon lui semble, pour l'accompagner chez le nouveau grand qu'il va
prendre et qu'il mène au palais dans son carrosse avec eux, et l'en
ramène de même, où tous lui donnent la première place. Ces autres grands
aident au parrain à faire les honneurs, et le nouveau grand se fait
accompagner en cortège.

Le duc del Arco ne prit avec lui que le duc d'Albe, oncle paternel et
héritier de celui qui est mort ambassadeur d'Espagne à Paris, à cause
des places du carrosse que nous remplissions mon fils et moi. Il eut,
comme je l'ai dit ailleurs, la politesse de venir dans son carrosse, et
non dans un du roi dont il se servait toujours, parce que dans celui-là
il ne pouvait donner la main à personne. Je ne pus jamais empêcher, quoi
que je fisse, qu'ils ne se missent tous deux sur le devant, mon fils et
moi eûmes le derrière. Je crus plaire aux Espagnols de marcher à cette
cérémonie avec tout l'appareil de ma première audience, et j'y réussis.
Six de mes carrosses, entourés de ma livrée à pied, suivaient celui du
duc del Arco, où nous étions, et personne autour\,; quinze ou dix-huit
autres seigneurs de la cour marchèrent après les miens remplis de ma
suite\,: tout Madrid était aux fenêtres ou dans les rues.

Nous trouvâmes les gardes espagnoles et wallonnes en bataille dans la
place du Palais, qui rappelèrent à notre passage en arrivant et en
retournant.

À la descente du carrosse nous fûmes reçus par ce qui s'appelle en
Espagne la famille du roi, c'est-à-dire une grosse troupe de bas
officiers de sa maison et une autre d'officiers plus considérables, au
milieu du degré, avec le majordome de semaine, qui était le marquis de
Villagarcias, qui était Guzman et a été depuis vice-roi du Mexique.

L'escalier depuis le bas jusques en haut bordé des hallebardiers sous
les armes avec leurs officiers. Tous ces honneurs ne sont que pour la
première classé. Au haut du degré quelques grands, qui par cette même
distinction descendirent deux marches. Beaucoup de personnes distinguées
dans l'escalier et jusqu'à la porte de l'appartement, et une foule de
grands et de seigneurs nous attendaient dans la première pièce, mais
cela n'est que de civilité\,; la vérité est qu'elle fut extrême, et que
tous me dirent qu'ils ne se souvenaient pas d'avoir vu tant de concours
de grandesse et de noblesse à aucune couverture, et, à ce que, j'y vis,
il fallut le croire.

Les gardes du corps étaient en haie sous les armes à notre passage dans
leur salle et à notre retour. Dans cette première pièce au delà de la
salle des gardes on attend que le roi soit arrivé dans celle qui suit,
et cependant compliments sans fin, et invitation au repas qui suit chez
le nouveau grand\,; lui, son parrain et ses amis particuliers vont
invitant le monde, il fait prier tous les grands, tous leurs fils aînés,
et les maris des filles aînées de ceux qui n'ont point de fils. Cela est
de règle. On peut prier aussi d'autres seigneurs amis ou distingués\,:
on le fait d'ordinaire, et nous en invitâmes plusieurs.

Le roi arrive, la cérémonie commence. Le majordome de semaine sort et
vient avertir le nouveau grand que le roi est entré par l'autre côté.
Tous les grands entrent, saluent le roi et se placent. Les gens de
qualité en font autant\,; les portes s'investissent de curieux\,; et le
nouveau grand entre tout le dernier, ayant son parrain à sa droite et le
majordome de semaine à sa gauche. La marche est fort lente\,: ils font
presque en entrant tous trois de front et tous trois ensemble une
profonde révérence au roi, qui ôte à demi son chapeau et le remet. Il
est debout sur un tapis de pied sous un dais, son capitaine des gardes
en quartier derrière lui, couvert parce qu'il est toujours grand, le dos
à la muraille. Personne du même côté où est le roi que le
majordome-major du roi, qui est couvert, le dos à la muraille, vers le
bout du côté des grands\,; en retour des deux autres côtés jusqu'à la
cheminée qui est vis-à-vis du roi, les grands couverts le dos à la
muraille, d'un seul rang qui ne se redouble point et personne devant
eux. Devant la cheminée, qui est grande, les trois autres majordomes
découverts.

Depuis la porte par où les grands et la cour est entrée, jusqu'à l'autre
vis-à-vis par où le roi est entré, qui fait le quatrième côté de la
pièce où sont les fenêtres, qui sont fort enfoncées et fort larges, sont
tous les gens de qualité de la cour, découverts, pêle-mêle, les uns
devant les autres, tant qu'il y en peut tenir, et le reste regarde par
les deux portes en foule sans avancer dans la pièce. Cette première
révérence faite\,; le parrain quitte le nouveau grand et se va mettre
après tous les grands, entre la porte par où il vient d'entrer et la
cheminée, le dos à la muraille, et s'y couvre, et fait ainsi aux autres
grands les honneurs pour le nouveau grand. Celui-ci s'avance lentement
avec le majordome à sa gauche. Au milieu de la pièce ils font en même
temps, et de front, une deuxième révérence profonde au roi, qui à
celle-là ne branle pas\,; puis, sans partir de la place, salue le
majordome-major et les autres côtés des grands, prenant garde de ne pas
tourner tout à fait le dos au roi. Le majordome-major, le capitaine des
gardes et tous les grands se découvrent entièrement, mais ne laissent
pas tomber leur chapeau fort bas, puis tout de suite se recouvrent.

Le majordome, qui conduit le nouveau grand et qui a fait la même
révérence que lui aux grands, le quitte dès qu'elle est achevée, et se
retire vis-à-vis d'où il se trouve, du côté des fenêtres, un pas au plus
en avant des gens de qualité, à qui le nouveau grand ni lui n'ont point
fait de salut. Le nouveau grand, demeuré seul au milieu de la place,
s'avance de nouveau avec la même lenteur jusqu'au bord du tapis de pied
où est le roi, à qui en arrivant près de lui il fait une profonde et
troisième révérence, à laquelle le roi ne remue pas. Si le grand est de
première classe, le roi prend l'instant qu'il commence à se relever de
sa révérence pour prononcer \emph{cobrios}. Si de la seconde, il le
laisse relever et parler, et faire ensuite la révérence\,; en se
relevant, il prononce \emph{cobrios}, et quand il est couvert le roi lui
répond. Si de la troisième, le roi ne prononce \emph{cobrios} qu'après
avoir répondu, il se couvre un instant, puis se découvre, baise la main
du roi, et le reste comme il va être expliqué. À ceux de première
classe, le roi ayant prononcé \emph{cobrios} comme le grand se relève de
sa troisième révérence, il s'incline de nouveau profondément du corps à
ce mot, mais sans révérence, et en se relevant se couvre avant de
commencer à parler. Les ambassadeurs ne se trouvent point à cette
cérémonie, ni aucune dame.

J'étais à la muraille comme duc de France, ou comme déjà grand, parmi
eux et couvert. On peut croire que je regardais de tous mes yeux par la
curiosité de la cérémonie, et beaucoup plus dans l'inquiétude, comment
mon fils s'en tirerait, qui avec un grand air de respect et de modestie
n'en eut point du tout d'embarras, et fit tout de fort bonne grâce et à
propos, il faut que cela m'échappe. Je remarquai la bonté du roi, qui,
en peine qu'il manquât à se couvrir à temps, lui fit deux fois de suite
signe de le faire comme il se relevait de son inclination après le
\emph{cobrios}. Il obéit, et s'étant couvert, il fit, comme c'est
l'usage, un remerciement au roi de demi-quart d'heure, pendant lequel il
mit quelquefois la main au chapeau et le souleva deux fois, à une
desquelles le roi mit la main au sien. À toutes ces démonstrations qui
ne sont pas pourtant prescrites et qui ne se font qu'en nommant notre
roi, ou quelquefois disant Votre Majesté au roi d'Espagne, tous les
grands les imitèrent en même temps que lui. Il finit en se découvrant,
fit une révérence profonde, et se couvrit en se relevant. Tous les
grands se découvrirent et se recouvrirent en même temps. Aussitôt après,
le roi, toujours couvert, lui répondit en peu de mots\footnote{Le
  personnage dont parle Saint-Simon dans ce passage est son fils cadet,
  Armand-Jean de Saint-Simon, marquis de Ruffec, né le 12 août 1699, et
  reçu grand d'Espagne le 1er février 1722, comme on le verra dans la
  suite de ces Mémoires. Ce fut pour faire obtenir la grandesse à ce
  fils que Saint-Simon demanda à être envoyé ambassadeur extraordinaire
  en Espagne. Il raconte lui-même, à l'année 1721, qu'il dit au duc
  d'Orléans, alors régent de France, qu'il le suppliait «\,de lui donner
  cette ambassade avec sa protection et sa recommandation auprès du roi
  d'Espagne pour faire grand d'Espagne le marquis de Ruffec.\,»}.

Lorsqu'il finit de parler, le nouveau grand se découvre, ploie un genou
tout à fait à terre, prend la main droite du roi, qui est exprès
dégantée, avec la sienne, la baise, se relève et fait une profonde
révérence au roi, qui alors se découvre tout à fait et se recouvre à
l'instant, et le nouveau grand passe au coin du tapis de pied, salue
tous les côtés des grands qui sont découverts et s'inclinent un peu à
lui, et il va pour cette unique fois se placer à la muraille au-dessus
d'eux tous, à côté et au-dessous du majordome-major, sans aucune façon
ni compliment. Là il se couvre et eux tous, et après quelques moments,
le roi se découvre, s'incline un peu aux trois côtés des grands, et se
retire. Tous vont chez la reine, excepté le nouveau grand, sa famille,
son parrain et ses amis particuliers, qui suivent le roi parmi les
félicitations, et à la porte de son cabinet lui font leurs remerciements
de nouveau, mais sans discours en forme, après quoi le nouveau grand,
avec ce qui l'a accompagné, va aussi chez la reine. Le plan fera mieux
entendre toute la cérémonie.

\begin{enumerate}
\def\labelenumi{\arabic{enumi}.}
\item
  Pièce où l'on attend que le roi arrive dans la salle d'audience.
\item
  Porte par où la cour entre\,; fermées avant son arrivée.
\item
  Porte par où le roi entre\,; fermées avant son arrivée.
\item
  Curieux entassés regardant par les portes.
\item
  Le roi debout sous un dais sur un tapis de pied.
\item
  Le capitaine des gardes du corps en quartier.
\item
  Le majordome-major.
\item
  Le nouveau grand lorsqu'il se retire à la muraille.
\item
  Les grands d'Espagne aux murailles.
\item
  La place à peu près où je me trouvai.
\end{enumerate}

{[}11. Cheminée.{]}

\begin{enumerate}
\def\labelenumi{\arabic{enumi}.}
\setcounter{enumi}{11}
\item
  Le parrain.
\item
  Les trois majordomes du roi.
\item
  Gens de qualité.
\item
  Le quatrième majordome du roi lorsque, après la deuxième révérence, il
  a quitté le nouveau grand.
\item
  Première révérence du nouveau grand, après laquelle son parrain le
  quitte et se retire à la muraille.
\item
  Deuxième révérence, après laquelle le majordome de semaine quitte le
  grand, et se va mettre du côté des seigneurs, et prend garde qu'ils ne
  s'avancent pas dans la salle, et que l'enfilade des deux portes
  demeure libre et vide.
\item
  Troisième révérence du nouveau grand seul\,; il se couvre, parle au
  roi, l'écoute, lui baise enfin la main dans cette même place, puis se
  retire à la muraille.
\item
  Personne entre la porte et le roi qui sort par cette même porte, et
  tout ce qui veut sortir par là après lui, au lieu qu'il entre seul par
  là avec ses officiers seulement qui par leurs charges le peuvent.
\end{enumerate}

Chez la reine on attend comme chez le roi dans la pièce qui précède
celle de l'audience, qui est fort singulière au palais de Madrid\,; elle
est fort longue et peu large\,; c'est le double d'une galerie intérieure
qui entre par un bout dans l'appartement de la reine, et par l'autre
dans celui de la princesse des Asturies et dans celui des infants. Cette
salle d'audience communique avec la galerie dans toute leur longueur par
de grandes arcades ouvertes dont elle tire tout son jour, et qui en font
presque une même pièce avec la galerie, qui est pourtant plus longue que
la salle d'audience du côté de l'appartement de la princesse des
Asturies et des infants. Un quart de la longueur de cette salle est
retranché par des barrières à hauteur d'appui et couvertes de tapis du
côté d'en bas, qui ne se mettent que pour ces cérémonies, et qui ne se
mettent que pour ce moment-là. Vis-à-vis au haut de la salle, assez près
de la muraille et en face de la porte et de la barrière, la reine est
assise dans un fauteuil plus haut que les fauteuils ordinaires, avec un
extrêmement gros carreau de velours à grands galons d'or sous ses pieds,
un dais et un grand tapis de pied, ayant derrière son fauteuil un exempt
des gardes du corps découvert, et qui n'est point grand\,; s'il l'était,
car il y en a, il serait couvert. À sa gauche en retour, qui est le côté
de la muraille, une haie de grands couverts, le majordome-major de la
reine à leur tête, et une place vide entre lui et le premier des grands,
pour le nouveau grand quand il se retire à la muraille. Les grands ne
redoublent point, et personne devant eux jusqu'à la barrière. À la
droite, vis-à-vis du majordome-major de la reine, la camarera-mayor, les
dames du palais et d'autres dames. Les femmes et les belles-filles
aînées des grands au-dessus des autres, et à la différence d'elles ayant
chacune un gros carreau devant elles, et les autres, pour grandes dames
qu'elles soient, n'en ont point. Ceux des femmes des grands sont de
velours en toute saison, ceux de leurs belles-filles aînées de damas ou
de satin en toute saison, avec ordinairement de l'or à la plupart,
toutes debout à ces couvertures. Après les daines sont de suite les
\emph{señoras de honor}. Dans l'entrée de la barrière, mais très peu
avant et en face de la reine, des seigneurs et gens de qualité
découverts, les uns devant les autres, et derrière les barrières ceux de
moindre condition. Dans les arcades qui joignent la galerie à la salle
d'audience les caméristes de la reine derrière les dames du palais, et
dans les autres les officiers de la reine.

En attendant que la reine soit arrivée, tous les hommes attendent dans
la pièce qui précède la salle d'audience, où les invitations se
continuent au repas à ceux à qui on pourrait avoir manqué de les faire
chez le roi.

La reine arrivée avec les dames et placée, celui de ses trois majordomes
qui est de semaine ouvre par dedans la porte de la salle d'audience et
vient avertir. Alors tous les grands entrent, se placent à la muraille
et se couvrent. Le parrain n'a point là de fonction, il entre avec les
autres grands, et se place indifféremment parmi eux. Plusieurs seigneurs
et gens de qualité entrent aussi après, mais les uns devant, les autres
après le grand nouveau, à qui on laisse un grand passage libre\,; il
entre lentement avec le majordome de semaine à sa gauche, ils dépassent
la barrière, et quand il s'est avancé quelques pas, il fait à la reine
une profonde révérence avec le majordome qui aussitôt après le quitte,
et se retire quelques pas vers les gens de qualité à gauche. À cette
première révérence la reine se lève en pied et se rassait incontinent\,;
et lors les grands se découvrent et se recouvrent. Ensuite le nouveau
grand s'avance lentement au milieu de la pièce, où il fait à la reine la
deuxième révérence, qui s'incline un peu sans se lever\,; puis, sans
partir de la place, il fait une révérence aux dames entièrement tourné
vers elles, et montrant l'allonger en toute la longueur de leur ligne du
haut en bas, mais pourtant par une seule révérence. Toutes s'inclinent
beaucoup, qui est leur révérence.

Le nouveau grand se tourne ensuite par-devant la reine vers les grands,
toujours sans bouger de la même place, et leur fait une révérence moins
profonde qu'aux dames. Sitôt qu'il se tourne aux grands, ils se
découvrent et se recouvrent lorsque le nouveau grand se tourne vers la
reine après les avoir salués. Il s'avance après jusque sur le tapis de
la reine, et tout auprès de son carreau\,; il y fait sa troisième
révérence, et en se relevant se couvre et fait son compliment et le
reste comme chez le roi, suivant la même différence des classes, mais il
se couvre au temps que la classe dont il est le demande, sans que la
reine le lui dise, parce qu'elle ne fait pas les grands. Il lui baise la
main dégantée comme au roi, un genou à terre, et s'avance pour cela à
côté du carreau. La reine s'incline après à lui, et il se retire à la
muraille\footnote{On peut consulter sur les grands d'Espagne, outre
  Imhof que Saint-Simon a indiqué plus haut, les auteurs suivants\,:J.
  A. de Tapia y Robles, \emph{Ilustracion del nombre de grande,
  principio, grandeza y etimologia, pontífices, santos, emperadores,
  reyes y varones ilustres, que le merecieron en la voz pública de los
  hombres} (Madrid, 1638, in-4)\,;S. M. Marquez, \emph{Tesoro militar de
  cavalleria, s. de ortu statuque equestrium in primis Hispanicorum
  commentarius} (Madrid, 1642, in-folio)\,;Muños, \emph{Discurso sobre
  la anttgüedad de la rica ombria} (Madrid, 1739, in-4)\,;J. Berni,
  \emph{Creacion, antigüedad y privilegios de los títulos de Castilla}
  (Valence, 1769, in-folio)\,;Enfin l'ouvrage intitulé\,: \emph{Noticia
  de las ordenes de caballeria de España, cruzes y medallas de
  distincion} (Madrid, 1815. 4 vol.~in-16).}. Quelques moments après, la
reine s'incline aux grands et aux dames, et se retire, et les grands se
découvrent et s'en vont.

Le plan fera mieux entendre la cérémonie.

1 L'exempt des gardes du corps de semaine chez la reine.

2 La reine.

3 Son majordome-major.

4 Place où le nouveau grand se retire à la muraille.

5 Grands.

6 La camarera-mayor.

7 Les dames du palais et les femmes et belles-filles aînées de grands.

8 Les \emph{señoras de honor} et autres dames de qualité.

9 Seigneurs et gens de qualité.

10 Curieux de moindre distinction.

11 Caméristes.

12 Officiers de la reine.

13 Première révérence du nouveau grand avec le majordome de semaine.

14 Place où se retire le majordome après la première révérence.

15 Deuxième révérence du nouveau grand seul.

16 Troisième révérence du nouveau grand, et place où il se couvre et
parle.

0 Personne eu toutes ces places.

Il faut remarquer que toutes les révérences que le nouveau grand, son
parrain et le majordome de semaine, font à la couverture chez le roi et
chez la reine, sont toutes à la française, même pour les Espagnols, ce
qui s'est apparemment introduit lorsque Philippe V a défendu la golille
et l'habit espagnol en sa présence à tout ce qui n'est ni robe ni
bourgeoisie ni marchands et au-dessous.

Au moment que la reine s'ébranle pour se retirer, le nouveau grand va
faire la révérence et un compliment à chacune de toutes les dames qui
sont à la cérémonie et qui ont l'\emph{excellence}, et point aux autres,
commençant par la camarera-mayor, et ne s'arrêtant qu'un instant devant
chacune, pour avoir le temps d'aller à toutes. Cette nécessité de se
hâter a mis en usage le même compliment, très bref, qui se répète à
toutes, en glissant de l'une à l'autre on leur dit\,: \emph{A los piés
de Vuestra Excelencia} et rien que cela\,; la dame sourit et
s'incline\,: cela se fait plus posément aux unes qu'aux autres suivant
leur qualité, leur faveur ou leur âge. Si la reine n'est pas encore
rentrée, et on se hâte d'avoir fait auparavant, le nouveau grand court à
la porte de la galerie qui donne dans son appartement intérieur et lui
fait là encore un remerciement. Je pris la liberté d'abuser peut-être de
celle qu'elle m'avait bien voulu donner auprès d'elle, je l'appelai pour
l'arrêter, lui faire mon remerciement, et donner le temps à mon fils de
lui venir faire le sien. Cela ne lui déplut pas, et elle nous reçut et
nous répondit avec beaucoup de bonté. Dès qu'elle est rentrée,
compliments pêle-mêle, et félicitations d'hommes et de dames, comme on
ferait en notre cour. Cela dure quelque temps, puis les dames suivent la
reine, d'autres s'en vont chez elles, et les hommes s'écoulent.

Il ne reste plus à la cour d'Espagne trace aucune de cette tolérance de
la vanité prétextée de la galanterie espagnole de l'ancien temps, de
personne qui s'y couvre sans autre droit que celui de son entretien avec
la dame qu'il sert, dont l'amour le transporte au point de ne savoir ce
qu'il fait, si le roi ou la reine sont présents, et s'il est couvert ou
non. Cette tolérance était abolie longtemps avant l'avènement de
Philippe V à la couronne d'Espagne. Il n'en reste pas même d'idée. Il
n'y a occasion ni prétexte qui laisse couvrir personne que les grands,
les cardinaux et les ambassadeurs.

De chez la reine nous allâmes chez le prince des Asturies\,; il n'y a là
aucune sorte de cérémonie. On l'environne en foule, ni lui ni personne
ne se couvre\,; mais le nouveau grand, son parrain, le grand ou les
grands qu'il a menés le prendre, et ses plus familiers qui font les
honneurs de la cérémonie sont les plus près du prince. Cela dure
quelques moments. Il s'y trouva et, s'y trouve toujours en ces occasions
beaucoup de grands et d'autres seigneurs\,; on nous dit que chez la
princesse des Asturies cela se serait passé de même\,; mais un érésipèle
la retenait au lit, et on n'y voit ni princesses ni dames. On ne va
point chez les infants, et nous n'y fûmes point.

Je ne sais si la conduite que nous fit le duc de Popoli, grand d'Espagne
et gouverneur du prince, jusque vers la fin de son appartement, fut un
honneur de politesse pour moi au caractère d'ambassadeur, ou une
distinction due au nouveau grand, car il s'adressa toujours également à
mon fils et à moi sur les compliments de cette reconduite\,; mais je
pense qu'il y eut mélange de tout cela.

Quoique l'appartement du prince soit en bas de plain-pied à la cour, à
quatre ou cinq marches près, nous passâmes en y, entrant et en sortant à
travers une longue haie de hallebardiers sous les armes, et la famille
du roi nous attendait et nous conduisit au carrosse qu'elle vit partir,
comme elle nous avait reçus à la descente, qui sont deux honneurs de la
seule première classe, ainsi que les gardes espagnoles et wallonnes que
nous trouvâmes encore sous les armes dans la place.

Nous retournâmes chez moi en la même manière que nous étions venus, et
parmi tout autant de spectateurs. Il s'y était déjà rendu bonne et
nombreuse compagnie par d'autres rues, presque tous les grands, beaucoup
de leurs fils aînés, quantité de seigneurs et de gens de qualité. Nous
étions plus de cinquante à table, et il y en eut plusieurs autres et
nombreuses d'amis, de familiers et même de grands, de seigneurs et de
gens de qualité qui voulurent s'y mettre. Je me mis à la dernière place.
Le duc del Arco, le duc d'Albe, mon deuxième fils, car l'aîné était
malade, et ceux qui voulurent bien nous aider à faire les honneurs,
comme le duc de Lira, le duc de Veragua, le prince de Masseran, le
prince de Chalais, et d'autres, se placèrent en différents endroits pour
en être plus à portée. On fut content du repas. On y mangea, on y but,
on y parla, on y fit du bruit, comme on aurait pu faire en France. Il
dura plus de trois heures. Un grand nombre s'amusa chez moi jusque fort
tard, et on servit force chocolat et force rafraîchissements. Les jours
suivants tous les grands, leurs fils aînés, et quantité d'autres
seigneurs et de gens de qualité nous vinrent rendre visite, c'est la
coutume\,; et le lendemain, mon fils et moi allâmes remercier le duc del
Arco et le duc d'Albe.

Il faut maintenant venir aux autres distinctions et prérogatives du rang
des grands d'Espagne. Je n'y entamerai rien d'étranger qu'autant qu'il
sera nécessaire pour les mieux expliquer.

Madrid est une belle et grande ville, dont la situation inégale et
souvent en pentes fort roides, a peut-être donné lieu aux sortes de
distinctions dont je vais parler.

J'ai déjà dit que personne, sans exception, hors le roi, la reine, les
infants et le grand écuyer dans les équipages du roi, ne peut aller à
plus de quatre mules dans la ville, mules ou chevaux c'est de même\,;
mais presque personne ne s'y sert de chevaux pour les carrosses. Si on
va ou si on revient de la campagne, on envoie à la porte de la ville
deux ou quatre mules attendre, qu'on y prend et qu'on y laisse de même
lorsqu'on y rentre. Le commun et peu au-dessus ne peut aller qu'à deux
mules, l'étage d'au-dessus à quatre mules, mais sans postillon. Les
\emph{titulados} et plusieurs sortes d'emplois ont un postillon\,; mais
rien n'est plus réglé que ces manières d'aller, et personne ne peut
empiéter au delà de ce qui lui appartient. Ce grand nombre de personnes
qui ont des postillons a peut-être été cause d'une autre sorte de
distinction\,: c'est d'avoir des traits de corde très vilains pour
toutes conditions, mais qui sont courts pour les moindres de ceux qui
ont un postillon\,; longs pour l'étage supérieur, et très longs pour les
grands, lés cardinaux et les ambassadeurs, et fort peu d'autres, comme
les conseillers d'État, les chefs des conseils, et, je crois, les
chevaliers de la Toison, etc.\,; encore ne les ont-ils pas si longs que
les grands. C'est uniquement à la qualité de l'attelage qu'on reconnaît
la qualité des personnes que l'on rencontre dans les rues, et cela
s'aperçait très distinctement, et les cochers ont une adresse qui me
surprenait toujours à tourner court et dans les lieux les plus étroits,
sans jamais empêtrer ni embarrasser leurs traits les plus longs. Je n'ai
point vu que les cochers des grands les menassent tête nue, sinon en
cérémonie, comme à une couverture, ou quelque autre semblable\,; bien
l'ai-je remarqué de ceux des femmes des grands, et du porteur de chaise
de devant des grands, de leurs femmes et de leurs belles-filles aînées.

Chez la reine, les femmes des grands ont un carreau de velours, et leurs
belles-filles aînées un de damas ou de satin, sans or ni argent. Elles
s'asseyent dessus. Toutes les autres, de quelque distinction qu'elles
soient, sont debout ou s'assoient nûment par terre. Mais en Espagne on
ne voit jamais de plancher nulle part\,; tous sont couverts de belles
nattes de jonc qui y sont particulières\,; le feu n'y prend point, elles
sont fort fines, souvent ouvragées de paysages en noir et en jaune, et
d'autres choses faites exprès pour les lieux\,; elles durent toutes une
infinité d'années, et il y en a de fort chères\,; on les balaye,
quelquefois on les ôte pour les secouer, rien n'est plus propre ni plus
commode. Les pièces intérieures ont en tout temps des tapis par-dessus.
Ceux du palais sont de la plus grande beauté, et c'est sur ces tapis que
les dames qui n'ont point de carreau s'assoient et s'en relèvent avec
une souplesse, une grâce et une promptitude, jusque dans les plus
vieilles et sans aucun appui, qui me surprenait toujours.

La coutume de s'asseoir ainsi, même dans les maisons particulières,
avait commencé fort à céder à l'usagé de nos sièges du temps de mon
ambassade. À la comédie, je n'ai vu que des carreaux et les dames qui en
ont droit assises dessus, et les autres tout de suite par terre sur le
tapis après elles. Elles sont comme à Versailles des deux côtés, et le
roi, la reine et les infants sur une ligne vis-à-vis du théâtre, tous
dans des fauteuils, le roi à la droite de tout, puis la reine\,; après,
les infants de suite par rang, le majordome-major du roi, sur un
ployant, joignant le roi à sa droite\,; la camarera-mayor joignant le
dernier infant, à sa gauche, sur un carreau. Derrière les fauteuils, le
capitaine des gardes du corps en quartier, le majordome-major de la
reine, le gouverneur du prince des Asturies, la gouvernante des infants,
assis sur des tabourets. Pas un autre siège, et tous les hommes debout,
grands et autres, quoique les comédies soient fort longues. À la droite
du roi il y a une niche dans la muraille, fermée de jalousies, où on
entre par derrière. Il n'y a là que les ambassadeurs qui y sont assis,
et le nonce du pape, en rochet et camail, à côté duquel j'ai assisté
plus d'une fois à ces comédies, lui jamais vêtu autrement. Au bal, qui
est rangé comme les nôtres à la cour, et qui sont là fort beaux, les
fauteuils et les tabourets derrière sont comme à la comédie\,; le
majordome-major et la camarera-mayor sur son carreau de même, mais il
n'y a point d'autres carreaux, ce sont des tabourets rangés sur une
ligne de chaque côté. Les femmes des grands et leurs belles-filles
aînées sont assises dessus. Après elles et sans mélange toutes les
autres dames\,; les grandes dames entre elles, comme elles arrivent les
premières, puis les \emph{señoras de honor}, enfin les caméristes, mais
toutes assises par terre, le dos appuyé contre les tabourets vides
derrière elles. Les vieilles de tout âge sont là, comme à la comédie, au
premier rang\,; il n'y en a point de second, et on y danse, hommes et
femmes, à tout âge, excepté la véritable vieillesse. Les hommes sont
derrière les tabourets et en face des fauteuils\,; pas un n'est assis,
ni grands ni danseurs. On ménage quelque embrasure de fenêtre, hors de
la vue du roi et de la reine, où il y a des tabourets pour les
ambassadeurs, et, autant qu'on peut, personne ne se tient entre eux et
la vue du bal.

La reine ne danse qu'avec le roi et les infants ni danse réglée ni
contredanse\,; la princesse des Asturies de même. Il est vrai qu'aux
contredanses elles dansent avec tous, mais celui qui est son danseur,
qui la mène, et avec qui principalement elle figure, est le roi ou un
infant. De bal en masques, je n'en ai vu aucun.

Il n'y a point de bal public chez le roi, et il y en avait souvent,
{[}point{]} de comédies au palais, et elles n'y sont pas ordinaires
comme dans notre cour\,; {[}point{]} d'audience publique à des ministres
étrangers, d'audiences publiques aux sujets, et il y en a deux fois la
semaine\,; c'est comme nos placets, excepté que chacun parle au roi\,;
je les expliquerai ailleurs\,; point de fêtes publiques, soit au palais
ou ailleurs auxquelles le roi assiste, point de cérémonie ou de fonction
quelle qu'elle soit, ni que le roi fasse ou qu'il s'y trouve, que les
grands, leurs fils aînés, et leurs femmes n'y soient à chacune
expressément conviés. Si c'est une occasion où on se couvre, les fils
aînés ne le sont pas, ni aux chapelles, parce qu'ils n'y ont point de
place. L'invitation est si fréquente, et en tant de lieux par Madrid,
parce que nul de ceux qui le doivent être n'est omis, même su malade,
que cela se fait assez peu décemment. Le majordome de semaine fait les
billets d'avertissement, datés sans les signer, et les envoie porter par
les hallebardiers de la garde qui en sont chargés. Ils se partagent par
quartiers. Il n'y a que la chose en deux mots, sans compliment ni
cachet, et le dessus mis pour chacun. Lorsqu'il y a quelque cérémonie
purement de grandesse hors du palais, où le roi ne se trouve point, ce
qui est fort rare, quoique j'en aie vu une depuis que je fus grand,
l'avertissement se porte de même en la même forme et par les mêmes
ordres. Je l'étais toujours ainsi comme duc de France, avant que je
fusse grand, même de celles où le roi me faisait lui-même l'honneur de
me commander de me trouver, et de celles encore où je devais assister
par mon caractère et en place d'ambassadeur, hors d'avec les grands,
comme aux chapelles\,; et depuis que mon second fils eut fait sa
couverture, lui et moi fûmes toujours invités, et nous nous sommes
trouvés ensemble parmi les grands, comme grands\,: de cela il résulte
que les grands sont l'accompagnement du roi partout, et son plus naturel
comme son plus illustre cortège. Personne autre n'est jamais invité, si
ce n'est les ambassadeurs en beaucoup d'occasions, comme les fêtes et
les chapelles, et de celle-ci, le plan en expliquera mieux
tout\footnote{On peut ajouter ici, comme complément des indications
  bibliographiques données plus haut, plusieurs ouvrages modernes qui
  retracent les révolutions politiques de l'Espagne et la situation de
  la noblesse de cette contrée\,; entre autres le \emph{Tableau de la
  monarchie espagnole au} XVIe \emph{siècle}, par Léopold Ranke
  (\emph{die Spanische Monarchie, Castilien, Granden}), ouvrage traduit
  en français par Haiber. M. Mignet, dans l'\emph{Introduction aux
  Négociations relatives à la succession d'Espagne}, a exposé
  l'organisation politique de ce pays au XVIIe siècle. Il résume
  rapidement les progrès de la royauté aux dépens de la noblesse
  espagnole\,: «\,Ferdinand le Catholique avait donné l'exemple. Sans
  détruire les ordres de chevalerie de Calatrava, d'Alcantara, de
  Monteza, de Saint-Jacques, qui avaient fait leur temps depuis que les
  Maures étaient expulsés, il leur enleva l'indépendance dont ils
  jouissaient, en devenant lui-même leur grand maître.\,» Charles-Quint
  et Philippe II complétèrent cette révolution. «\,Les grandes familles,
  dit M. Mignet, comme celles des Guzman, des Mendoza, des Enriquez, de
  Pacheco, des Girone, etc., avaient d'immenses richesses, des cours
  constituées sur le modèle des cours féodales au moyen âge, des gardes,
  des sujets en grand nombre et la petite noblesse sous leurs ordres.
  Elles furent laissées à l'écart\,; et les fils des conquérants
  espagnols, réduits au rôle de grands propriétaires, n'aspirèrent
  bientôt plus qu'au privilège de se couvrir devant le roi ou dans sa
  chapelle.\,»}.

1 Sanctuaire fort magnifique derrière l'autel.

2 L'autel, ses marches, son tapis et au-dessous, les trois marches comme
du chœur.

3 Portes du sanctuaire.

4 Table pour le service de l'autel.

5 Bancs nus pour les célébrants.

6 Banc avec un petit tapis pour les évêques.

7 Fauteuil du cardinal patriarche des Indes. 0 Son aumônier.

8 Son petit banc ras de terre avec son tapis et son carreau.

9 Porte de la sacristie.

10 Sommelier de courtine en semaine, debout, c'est-à-dire aumônier.

11 Fauteuil du roi.

12 Son prie-Dieu avec son drap de pied et ses deux carreaux pour les
coudes et pour les genoux.

13 Fauteuil du prince des Asturies.

14 Son prie-Dieu, \emph{idem}, mais qui n'a point de carreau pour les
coudes.

15 Grand tapis commun sous les fauteuils et les prie-Dieu.

16 Grand dais avec sa queue qui les couvre.

17 Banc avec son tapis du capitaine des gardes en quartier.

18 Ployant de velours avec de l'or pour le majordome-major du roi.

19 Banc des grands avec son tapis.

20 Gardes sous les armes.

21 Deux grands chandeliers d'argent qui brûlent jour et nuit.

22 Deux autres pareils qu'on ajoute lorsque le saint sacrement est
exposé.

23 Deux, quatre ou six pages du roi, suivant la solennité, qui viennent
au \emph{Sanctus}, et s'en vont après la communion du prêtre avec de
grands flambeaux allumés de cire blanche.

24 Espèce de croisée de la chapelle.

25 Les quatre majordomes du roi debout.

26 Banc des ambassadeurs.

27 {[}Banc{]} de chapelle avec leur petit banc ras de terre et le tapis
de l'un et de l'autre.

28 La chaire du prédicateur et son petit degré.

29 Banc nu pour les ecclésiastiques et les religieux du premier ordre.

30 Banc, \emph{idem}, pour ceux du deuxième ordre.

31 Vide pour les ecclésiastiques et les religieux du commun debout.

32 Glaces qui servent de fenêtres à la tribune à voir dans la chapelle.

33 Petite porte par où la reine sort de la tribune lorsqu'elle va aux
processions et y rentre.

34 Autre porte de communication pour le prêtre qui vient dire la messe à
la tribune.

35 Place dans la chapelle pour le majordome de la reine en semaine,
debout.

36 Autel de la tribune.

37 Place de la reine sur un prie-Dieu entre deux balustrades.

38 Place des infants.

\hypertarget{chapitre-xv.}{%
\chapter{CHAPITRE XV.}\label{chapitre-xv.}}

1701

~

{\textsc{Séance et cérémonie de tenir chapelle en Espagne.}} {\textsc{-
Cérémonie de la Chandeleur et celle des Cendres.}} {\textsc{- Banquillo
du capitaine des gardes en quartier.}} {\textsc{- Raison pourquoi les
capitaines des gardes sont toujours grands.}} {\textsc{- Places
distinguées à toutes fêtes et cérémonies pour les grands, leurs femmes,
fils aînés et belles-filles aînées.}} {\textsc{- Parasol des grands aux
processions en dehors où le roi assiste et la reine.}} {\textsc{- Cortès
ou états généraux.}} {\textsc{- Traitement par écrit dans les églises,
hors Madrid.}} {\textsc{- Baptême de l'infant don Philippe.}} {\textsc{-
Honneurs civils et militaires partout.}} {\textsc{- Honneurs à Rome.}}
{\textsc{- Rangs étrangers inconnus en Espagne.}} {\textsc{- Égalité
chez tous les souverains non rois.}} {\textsc{- Supériorité de M. le
Prince sur don Juan aux Pays-Bas, et son respect pour le roi fugitif
d'Angleterre, Charles II.}} {\textsc{- Bâtards des rois d'Espagne.}}
{\textsc{- Grands nuls en toutes affaires.}} {\textsc{- Point de
couronnement, nul habit de cérémonie, ni pour les rois d'Espagne, ni
pour les grands.}} {\textsc{- Nulle préférence de rang dans les ordres
d'Espagne, ni dans celui de la Toison d'or.}} {\textsc{- Grands
acceptent des emplois fort petits.}} {\textsc{- Grandesses s'achètent
quelquefois.}} {\textsc{- Autre récapitulation.}} {\textsc{- Nul serment
pour la grandesse.}} {\textsc{- Grand nombre de grands d'Espagne.}}
{\textsc{- Indifférence d'avoir une ou plusieurs grandesses}}

~

Lorsque le roi d'Espagne tient chapelle, ce qui arrive très fréquemment,
dont je parlerai ailleurs, sa cour l'attend à la porte de son
appartement secret. Il passe environ deux pièces, puis se couvre. Les
grands qui marchent sans ordre devant et autour de lui, le prince des
Asturies qui le suit, le capitaine des gardes en quartier qui est
toujours grand, et le patriarche des Indes, s'il est cardinal, qui
marche à côté du capitaine des gardes, se couvrent tous. On fait un long
chemin par de grands et magnifiques appartements, et on arrive ainsi à
la chapelle, où chacun fait la révérence à la reine qui est dans la
tribune\,; puis s'avançant, on la fait à l'autel\,; celle-là est
toujours à l'espagnole, c'est-à-dire comme sont les révérences de nos
chevaliers du Saint-Esprit et de toutes nos cérémonies. Les ambassadeurs
seuls la font à l'ordinaire\,; le roi la fait à l'espagnole vis-à-vis de
sa place, et chacun prend la sienne. Le patriarche, s'il est cardinal,
vis-à-vis du roi, laquelle {[}place{]} j'expliquerai ailleurs, sinon sur
le banc des évêques où il n'y en a presque jamais, parce, que tous
résident très exactement, et que la difficulté de la croix, que la
chapelle ne veut pas souffrir, empêche l'archevêque de Tolède de s'y
trouver. De mon temps c'était le cardinal Borgia qui était patriarche
des Indes.

Tandis que le célébrant commence la messe au bas de l'autel, le cardinal
sort de sa place, où il n'a qu'un aumônier près de lui, debout à sa
droite en surplis, et suivi des quatre majordomes du roi, de front
derrière lui, va au milieu de l'autel sans monter aucune marche, le
salue, puis le roi et le prince des Asturies de suite, se retourne le
dos à l'autel, salue la reine, puis les ambassadeurs qui se lèvent et
s'inclinent à lui, en dernier lieu les grands qui en font de même, et
pour ne le plus répéter, toutes les fois qu'il sort de sa place et qu'il
y revient, il fait les mêmes saluts en se baissant, comme font nos
évêques, et les majordomes derrière lui à l'espagnole dans le même
temps. Il va au prie-Dieu du roi qui est debout, dire l'\emph{Introït} à
voix médiocre, puis revient. Il lui porte l'Évangile à baiser, et au
prince\,; il va les encenser sans en être salué, et il leur porte la
paix, puis à la reine. Lorsqu'il y va et en revient, et c'est toute la
longueur de la chapelle, les ambassadeurs et les grands sont debout. En
sortant de la chapelle, le roi se couvre et les grands, et retournent
comme ils sont venus. Les pages qui portent les flambeaux au
\emph{Sanctus} font, en arrivant à leur place, la révérence à l'autel,
au roi, et au prince en même temps, à la reine, au cardinal et aux
ambassadeurs en même temps, enfin aux grands. C'est à l'espagnole, en
baissant leurs flambeaux tous en même temps et comme en cadence\,: c'est
un vrai exercice. Il y a toujours sermon en espagnol. Le prédicateur
sort de la sacristie, et vient recevoir à genoux la bénédiction du
cardinal, puis fait les révérences susdites, et monte en chaire, en s'en
retournant de même.

Lorsqu'il y a procession, comme à la Chandeleur, il n'y a point
d'ambassadeurs, parce qu'ils ne pourraient marcher que devant le roi ou
après le roi, comme ils font en suite du capitaine des gardes quand on
va et revient des chapelles ordinaires. En avant n'est donc point leur
place. En arrière ils couperaient la reine ou au moins les dames de sa
suite, tellement que ces jours-là ils ne sont point avertis, et ne s'y
trouvent jamais. La bénédiction des cierges finie par le cardinal, le
roi, suivi du prince et de son capitaine des gardes, va au milieu de
l'autel, où le cardinal est, dans un fauteuil sur la plus basse marche,
en sorte que le roi n'en monte aucune. Le majordome-major marche seul à
sa droite, suivi d'un bas officier. Il trouve un majordome vers où est
le cardinal qui lui présente un carreau. Le majordome-major le met
devant le roi, qui reçoit à genoux le cierge du cardinal, le prince
ensuite, puis le majordome-major ôte le carreau, et le rend au
majordome, se met à genoux, reçoit le cierge, après lui le capitaine des
gardes, et retournent en leurs places. Le roi étant déjà en la sienne,
tous les grands ensuite, suivant qu'ils se trouvent placés sur leur
banc, vont prendre le cierge à genoux, et tout de suite le clergé, à qui
il en a été distribué avant le roi, sort de dessus ses bancs, et sort
processionnellement, puis le clergé qui est à l'autel et le cardinal,
après les grands deux à deux, enfin le roi ayant presque de front le
majordome major à sa droite, le prince derrière à côté du capitaine des
gardes\,; tout cela trouve la reine à la porte de sa tribune en dedans,
à. qui le cardinal en passant a donné un cierge, et à tout ce qui est
dans la tribune. Les grands saluent la reine profondément. Le roi la
salue aussi\,; elle laisse un court intervalle entre elle et le prince,
et suit la procession entre son majordome-major et son grand écuyer,
suivie des infants. Après eux marche seule la camarera-mayor, les dames
de la reine deux à deux, puis celles des infants. Le roi et les grands
se couvrent hors la chapelle. Les seigneurs et les gens de qualité
côtaient, les uns les grands les plus près du roi, la plus grande partie
les dames\,; puis le commun suit. Il y a des officiers des gardes du
corps des deux côtés du roi, et celui qui sert auprès de la reine lui
porte la queue. On fait le tour des corridors du palais, ce que
j'expliquerai ailleurs. En toutes les processions c'est le même ordre de
marche. À celle-là mon fils et moi étions sur le banc des grands,
plusieurs entre nous deux, et c'est là où j'ai dit que le hasard fit
qu'il me précéda. Le roi et tous baisent l'anneau du cardinal après
avoir reçu le cierge.

Le jour des Cendres, les ambassadeurs y sont. La bénédiction faite, le
cardinal, suivi du nonce et des majordomes, va au milieu de l'autel,
comme ci-dessus, où tous deux prennent une étole d'un des assistants à
l'autel\,; le célébrant donne des cendres au cardinal seulement incliné,
qui lui en donne ensuite, mais le célébrant à genoux, puis au nonce
incliné qui revient à sa place, après à tout le clergé. Le roi vient
accompagné comme à la distribution des cierges, et le carreau lui est
présenté de même. Lui et le prince en ayant reçu, et le carreau ôté
comme lors des cierges, les ambassadeurs viennent, recevoir les cendres,
puis le majordome-major qui était resté là\,; ensuite le capitaine des
gardes, puis tous les grands, après quoi le cardinal en va porter à la
reine, aux infants et à tout ce qui est dans la tribune. Elle n'assiste
jamais ailleurs à aucune chapelle, les jours ordinaires c'est où le roi
et elle entendent la messe, et où ils communient leurs jours marqués, et
personne n'y entre que leurs grands officiers intérieurs et les dames de
la reine et des infants. Au-dessus est une grande tribune pour la
musique, qui est excellente et nombreuse, et, au-dessus de celle-là, une
autre pour les duègnes et les \emph{criadas}\footnote{Mot espagnol qui
  signifie \emph{servantes} ou \emph{suivantes}.} du palais, ou nul
homme n'entre. Les caméristes sont à l'entrée et au fond de la tribune
de la reine.

Il faut remarquer que les ambassadeurs ni les grands n'ont point de
carreau à la chapelle\,; le tapis de leur banc et de celui des évêques,
et du petit banc ras de terre devant les ambassadeurs, sont jusqu'à
terre et d'assez vilaine tapisserie, la même pour tous. Le petit banc
ras de terré, qui est devant le cardinal, est de velours rouge, et n'est
pas plus étendu que les autres. Sort fauteuil est de bois uni avec les
bras tout droits\,; le siège et le dossier, qui ne lui appuie que les
épaules, est de velours rouge avec un galon d'or et d'argent usé autour,
de forme carrée, avec de larges clous dorés dessus, d'espace en espace,
environné de petits, comme ces anciens fauteuils de château\,; son
carreau est de velours rouge à ses pieds\,; les fauteuils, carreaux et
drap de pied du prie-Dieu du roi et du prince, sont de velours avec
beaucoup d'or ou d'argent, ou d'étoffe magnifique. Ils changent souvent,
mais ceux du roi sont toujours beaucoup plus riches que ceux du prince,
et tournés en biais vers l'autel.

La place du capitaine des gardes du corps fit une grande difficulté.
Philippe V est le premier qui ait eu des gardes du corps et des
capitaines des gardes, sur le modèle de la France. Ses prédécesseurs
n'avaient que des hallebardiers, tels qu'il les a conservés, mais dont
le capitaine n'a point de place nulle part comme tel, et des lanciers en
petit nombre et fort misérables, dont le capitaine n'était rien. Les
grands, qui sont les seuls laïques assis aux chapelles, ne voulurent pas
souffrir que le capitaine des gardes en quartier le fût, ou s'il était
grand, le fût hors de leur banc. Cette difficulté fut réglée pour ne
jamais prendre de capitaine des gardes que parmi les grands. Mais cela
ne les satisfit pas\,; ils voulaient que celui de quartier fût
indifféremment assis avec eux sur leur banc, et le roi d'Espagne, qui
s'en faisait servir sur le modèle de notre cour, prétendit l'avoir assis
derrière son fauteuil. Enfin, par composition, après beaucoup de bruit,
il fut convenu qu'il aurait un \emph{banquillo}, c'est-à-dire un petit
banc à une seule place, couvert comme celui des grands, adossé en biais
à la muraille, à la place où il est marqué dans le plan. À vêpres c'est
la même séance, et au Retiro comme au palais, et en quelque lieu que le
roi tienne chapelle. Il n'y a que la tribune de la reine qui ne peut
être partout placée, ni de plain-pied ni au bout de l'église\,; mais
elle est toujours dans une tribune, et ce changement de sa place n'en
apporte aucun autre. J'ai grossièrement expliqué la chapelle par rapport
seulement aux grands\,; je la détaillerai plus curieusement ailleurs.
Lorsque le roi va en pompe à Notre-Dame d'Atocha, qui est à un dernier
bout de Madrid, il est censé n'y être accompagné que de ses grands
officiers, qui le précèdent ou le suivent dans ses carrosses, et la
reine de même de ses dames. Les grands n'y sont point invités et n'y ont
point de places.

Les fêtes dans la place Mayor, qui est fort grande et qui a cinq étages
égaux, tous à balcons à toutes les fenêtres, sont assez rares. J'y en ai
vu plusieurs à cause des deux mariages, et toutes admirables. J'en
parlerai en leur temps. Il suffit ici de dire qu'il y a au milieu une
maison distinguée pour le roi et sa cour\,; vis-à-vis la largeur de la
place, entredeux, sont les ambassadeurs, et ce même étage, qui est le
premier, est distribué tout autour de la place aux grands et à leurs
femmes, à tous séparément, de façon qu'un grand a du moins quatre
balcons de suite, à quatre ou cinq places chacun, c'est-à-dire quatre au
large et cinq assez aisément, car ils sont, tous égaux et sortent en
dehors trois pieds. Si un grand a une ou plusieurs charges, qui lui
donnent droit de places, on les ajoute de suite à ses balcons comme
grand\,; mais cela est assez rare. Le deuxième, et, s'il le faut, le
troisième étage, sont distribués de même. C'est le majordome-major qui
en donne les ordres, et les balcons désignés dans les billets, en sorte
que chacun sait où aller sans se méprendre. Ce qui reste après de places
jusqu'au cinquième étage est à la disposition du corrégidor de Madrid,
tellement que ceux qui n'ont point de places par grandesses, ou, ce qui
est fort rare, par charges, n'en ont qu'après tous les grands et les
charges, ce qui fait qu'ils en ont de médiocres ou de mauvaises, et même
difficilement par le peu qui en resté pour toute la cour et la ville, de
manière que la plupart des personnes de qualité, hommes et femmes, en
demandent aux grands de leurs amis sur leurs balcons. Les ministres
étrangers en ont avant les seigneurs qui ne sont pas grands, par le
majordome-major. Cela se passe de la sorte dès que la fête est hors du
palais. Quand elle se fait dans la place du palais, où j'en ai vu aussi
d'admirables, les fenêtres se donnent par places aux mêmes, mais avec
moins d'ordre et de commodité, et toujours par les majordomes sous les
ordres dû majordome-major. Aux unes et aux autres la règle y est telle,
qu'il n'y a jamais la plus légère dispute, et qu'on y arrive et qu'on en
sort avec une grande facilité, quoique la foule n'y soit pas moindre que
celle qui fait toujours repentir de la curiosité des spectacles et des
fêtes en France.

Les grands sont invités aux cérémonies avec la même exactitude. Comme il
est des fêtes où on n'en invite point d'autres, encore que toute la cour
s'y trouve, ainsi que je l'ai vu arriver aux bals et aux comédies du
mariage, excepté les ambassadeurs qui le furent aussi, aussi est-il des
cérémonies où on n'invite qu'eux ou presque qu'eux. J'appelle inviter
d'autres, leur faire dire de s'y trouver\,; car, pour l'avertissement en
forme, il ne s'adresse jamais qu'à eux. Ils l'eurent pour la cérémonie
de la signature du contrat de mariage du roi et de l'infante\,; que je
décrirai en son lieu. Il n'y entra qu'eux et les seigneurs les plus
distingués, et les gens de qualité en foule virent entrer et sortir le
roi et les grands du lieu où elle se fit, et le très petit nombre de
charges ou de places indispensables\,; outre les grands qui y furent
admis hors du rang des grands, et bien plus éloigné pour eux de la table
et du roi. Il en fut de même au mariage du prince des Asturies, quoique
célébré à Lerma près de Burgos. Le roi n'y voulut d'abord que sa suite
ordinaire, parce qu'il y alla chasser six semaines auparavant. Mais,
pour le mariage, tous les grands y furent invités\,; eux, leurs femmes,
fils aînés et belles-filles, eurent tous des logements marqués, et
furent les plus près de la cérémonie\,; les femmes et les belles-filles
des grands sur leurs carreaux. Je décrirai en son lieu cette cérémonie.
On y verra aussi, en son temps les audiences publiques aux sujets et aux
ministres étrangers, où les grands sont invités et couverts. Aux
processions, qui se font dehors, où le roi assiste, et où ils sont aussi
invités, ils ont \emph{l'ombrello}, c'est-à-dire le parasol.

Ils sont toujours aussi invités aux \emph{cortes}, c'est ce que nous
appelons en France les états généraux\,; mais ceux d'Espagne ne font
guère que prêter des reconnaissances, des hommages et des serments, et
n'ont pas même les prétentions de ceux de France. Ainsi, y assister,
n'est pas se mêler d'affaires, encore moins prêter du poids et de
l'autorité. En ces assemblées, qui d'ordinaire se font dans la belle
église des Hiéronimites du Buen-Retiro à Madrid, qui sert de chapelle à
ce palais, les grands précédent tous les députés dans la séance et dans
tout le reste.

Le roi, écrivant à un grand, le traite de cousin, et son fils aîné, de
parent\,; de même à leurs femmes.

Dans toutes les villes et lieux où le roi n'est pas, les grands ont à
l'église un tapis à leur place, la première du chœur, un carreau pour
les genoux, et un pour les coudes\,; les fils aînés des grands un
carreau. J'en eus ainsi, et mon deuxième fils, dans la cathédrale de
Tolède, à la grand'messe et au sermon, et le comte de Lorges un carreau.
Mon fils aîné était demeuré malade à Madrid. Ce carreau du comte de
Larges m'en fit demander pour le comte de Céreste, frère du marquis de
Brancas, pour l'abbé de Saint-Simon et pour son frère, et je ne les eus
qu'à grand'peine et par considération pour moi, comme ils me le dirent
nettement. Tous les chanoines étaient en place. On connaît la dignité et
les richesses de cette première église d'Espagne\,; j'en parlerai
ailleurs.

Je remets aussi en son temps à expliquer la cérémonie du baptême de
l'infant don Philippe, où tous les grands et grandes, leurs fils aînés
et belles-filles, furent invités, et les plus près du roi et de la
cérémonie. Je me contenterai ici de remarquer qu'ils eurent le dégoût,
et qui fit du bruit et de grandes plaintes, d'y porter les honneurs, qui
ne le devaient être que par les majordomes.

Ils ont partout les honneurs civils, c'est-à-dire ce que nous appelons
en France le vin, les présents et les compliments des villes et des
notables. Ils ont le canon, la garde et tous les honneurs militaires, la
première visite des vice-rois et capitaines généraux des armées et des
provinces, et la main chez eux pour une seule fois, s'ils sont officiers
ou sujets du pays où le vice-roi commande, chez lequel ils conservent
d'autres sortes de distinctions sur les autres seigneurs des mêmes pays
non grands, et servent suivant leur grade militaire. J'ai expliqué cela
plus haut, ainsi que les honneurs qu'ils ont chez le pape, pareils à
ceux des souverains d'Italie, et dans Rome, semblables en tout aux
distinctions des deux princes du Soglio, qui eux-mêmes sont grands.

Le rang, qui s'est peu à peu introduit en France tel que nous l'y voyons
de prince étranger, soit en faveur des cadets de maisons souveraines,
soit en faveur de maisons de seigneurs français qui l'ont obtenu pièce à
pièce, est entièrement inconnu en Espagne aussi bien que dans tous les
autres pays de l'Europe, qui ont des premières dignités et des charges
qui répondent à nos offices de la couronne. Il n'y a donc de rang en
Espagne que celui des cardinaux, des ambassadeurs et des grands
d'Espagne, celui du chef ou du président du conseil de Castille étant
une chose tout à fait à part, quoique supérieur à tous. On a vu
ci-dessus des princes de maison souveraine attachés au service
d'Espagne, faits grands pour leur vie. C'était le seul moyen de leur
donner un rang dont ils ont joui sans jamais avoir prétendu aucune
distinction particulière ni quoi que ce soit parmi les autres grands.
Ceux-ci se sont soutenus avec le même avantage à l'égard des souverains
qui ont été à Madrid, même les ducs de Savoie. Ceux-là, à la vérité, ne
furent pas faits grands, aussi n'avaient-ils pas à y demeurer, mais ils
n'en précédèrent aucun, et n'osèrent se trouver avec eux. Le seul fils
de Savoie, qui fut depuis le célèbre duc Charles-Emmanuel, y eut quelque
distinction, mais ce ne fut qu'après que son mariage fut arrêté avec
l'infante, et en cette considération\,; encore ces distinctions
au-dessus des grands furent-elles assez médiocres. Du prince de Galles,
qui fut depuis l'infortuné Charles Ier, on n'en parle pas\,: l'héritier
présomptif et direct de la couronne de la Grande-Bretagne est au-dessus
de toutes les règles. La comtesse de Soissons, mère du fameux prince
Eugène, ne put jamais paraître en public à Madrid, ni voir la reine que
dans le dernier particulier malgré sa faveur, ses manèges et ses
privances, qui à la fin aboutirent à l'empoisonner, et à s'enfuir pour
éviter le supplice du à son crime. Lorsque le prince et la princesse
d'Harcourt accompagnèrent la même reine en Espagne, ils n'y purent
obtenir aucun rang, parce que le prince d'Harcourt n'eut le caractère
d'ambassadeur que pour la cérémonie du mariage qui se fit dans un
méchant village, un peu au deçà de Burgos, où j'ai passé. Aucun seigneur
non grand d'Espagne même, ni aucune femme de qualité, ne leur voulut
céder. Charles II ni la fille de Monsieur, sa nouvelle épouse, n'y
trouvèrent rien à reprendre, elle à représenter, ni lui à ordonner.
Ainsi le prince et la princesse d'Harcourt furent contraints de revenir
brusquement pour se tirer de ce qu'ils trouvaient de mortifiant pour
eux. Aussi cette princesse d'Harcourt si insolente de la faveur de
M\textsuperscript{me} de Maintenon, si entreprenante, si forte en
gueule, ne parlait-elle jamais de ce voyage.

Les électeurs et les princes régents d'Allemagne et ceux d'Italie les
traitent en tout chez eux d'égaux et leur donnent la main, et même les
ducs de Savoie, jusqu'au dernier qui, longtemps avant de s'être fait
roi, cessa de les voir ainsi que les cardinaux.

La politique et la puissance de Charles-Quint leur procura tous ces
avantages dans les pays étrangers, que celle de la maison d'Autriche a
su leur y maintenir depuis, comme je l'ai déjà dit. Ils ne se pouvaient
prétexter que par ceux qui leur furent donnés dans leur pays même, et
Charles-Quint et ses successeurs ont toujours cru, à l'exemple des papes
sur les cardinaux, que leur respect et leur grandeur s'accroissait et se
maintenait à la mesure de celle qui émanait d'eux. Tout n'est
qu'exemple, non seulement pour les papes, mais pour ces princes, de la
justesse de cette pensée, que ce n'est pas ici le lieu de pousser.

La stérilité des reines d'Espagne depuis Charles-Quint n'a point laissé
de princes du sang depuis le règne de Charles V. À peine quelque infant
cadet est-il sorti de l'enfance\,; à peine un seul a-t-il atteint
l'adolescence qu'il a été cardinal-archevêque de Tolède, et est mort
promptement après. On n'y a donc vu que presque l'héritier de la
couronne, et jamais de seconde génération. Les nôtres n'ont point voyagé
en Espagne, de manière qu'il n'y a ni règle ni exemple des princes du
sang aux grands. M. le Prince le héros est le seul qu'on puisse citer,
qui, malgré sa situation forcée en Flandre, sut toujours maintenir toute
sa supériorité sur don Juan, gouverneur général des Pays-Bas, général
des troupes, et qui tranchait du prince du sang d'Espagne, quoiqu'il ne
fût que bâtard\,; il la conserva de même sur tous les autres, avec la
gradation de plus de ce qu'il emportait sur le chef des Pays-Bas et des
armées, qui le souffrait très impatiemment, mais qui n'osa jamais lui
rien disputer. Il en usait plus familièrement avec le roi d'Angleterre,
dont l'état, sous l'usurpation de Cromwell, était encore bien plus gêné
et plus réduit à fermer les yeux aux avantages que don Juan en osait
usurper. Cela impatienta M. le Prince, qui, non content de lui avoir
appris à vivre avec lui, lui voulut donner encore la mortification de
lui montrer ce qu'il devait au roi d'Angleterre. Peu de jours après que
ce prince fut arrivé à Bruxelles et qu'il eut remarqué la familiarité
peu décente que don Juan s'avisait de prendre avec lui il les pria l'un
et l'autre à dîner avec tout ce qui était de plus considérable à
Bruxelles. Tous s'y trouvèrent, et quand il fut servi, M. le Prince le
dit au roi d'Angleterre, et le suivit à la salle du repas. Qui fut bien
étonné\,? ce fut don Juan, quand, arrivé en même temps avec la compagnie
qui suivait le roi d'Angleterre et M. le Prince, il ne vit sur une très
grande table qu'un unique couvert avec un cadenas, un fauteuil, et pas
un autre siège. Sa surprise augmenta, si elle le put, quand il vit M. le
Prince présenter à laver au roi d'Angleterre, puis prendre une serviette
pour le servir. Dès qu'il fut à table, il, pria M. le Prince de s'y
mettre avec la compagnie. M. le Prince répondit qu'ils auraient à dîner
dans une autre pièce, et ne se rendit que sur ce que le roi d'Angleterre
le commanda absolument. Alors M. le Prince dit que le roi ordonnait
qu'on apportât des couverts. Il se mit à distance, mais à la droite du
roi d'Angleterre, don Juan à sa gauche, et tous les invités ensuite. Don
Juan sentit toute l'amertume de la leçon, et en fut outré de dépit\,;
mais après cet exemple il n'osa plus vivre avec le roi d'Angleterre
comme il avait osé commencer.

On a vu ci-dessus l'état des bâtards en Espagne. Ceux des rois en ont
profité. Le premier don Juan eut de grands emplois, s'illustra de la
fameuse mais peu fructueuse victoire navale de Lépante, passa de
vice-royauté en vice-royauté, parce que Philippe II avait peur de son
mérite, et le tint tant qu'il put éloigné. Avec tant d'éclat il acquit
l'altesse comme les infants, en prit presque les manières, eut une
maison fort considérable, et alla finir de bonne heure aux Pays-Bas de
la manière que personne n'ignore. Cet exemple fraya le chemin de la
grandeur au second don Juan, qui n'avait pas moins de mérite que le
premier, quoique resserré dans des bornes plus étroites. Il la sut
soutenir par les cabales et un parti qui fit pâlir souvent la reine mère
de Charles II, régente, et qui lui arracha ses plus confidents
serviteurs\,; il n'est donc pas surprenant qu'il ait eu l'altesse et
presque la maison des infants, et que les imitations de beaucoup de
leurs manières lui aient été souffertes par un parti de presque toute
l'Espagne, qui ne se maintenait, ne parvenait, et ne profitait contre la
régente et le gouvernement qu'à l'ombre de sa protection, et qui, à la
majorité de Charles II, chassa cette reine à Tolède, d'où elle ne revint
à la cour qu'en 1679, après la mort de don Juan, qui régna toujours sous
le nom du roi, et qui n'oublia aucun de tous les avantages que peuvent
donner l'exemple et la puissance, et le grand parti qu'il s'était fait.
Tous les deux don Juan moururent sans enfants après avoir été à la tête
des armées et des provinces, le premier à trente-deux ans, l'autre à
cinquante. Je parlerai en son temps de l'altesse et du rang que
M\textsuperscript{me} des Ursins et M. de Vendôme usurpèrent en Espagne,
et qui leur fut à tous deux diversement funeste.

Tels sont à peu près les rangs, les prérogatives, les distinctions, les
honneurs des grands d'Espagne. On n'y voit point leur intervention
nécessaire en rien du gouvernement de l'État, ni de sa police
intérieure, ni leur voix en aucune délibération ni jugement\,; nulle
séance en aucune cour ni tribunal, nulle distinction ni pour leurs
grandesses ni pour leurs personnes dans la manière d'être jugés en aucun
cas. Bien est vrai qu'il y en a toujours eu de conseillers d'État,
c'est-à-dire de ministres jusqu'au commencement de Philippe V, mais
toujours avec d'autres, toujours par avancement personnel, jamais par
nécessité de dignité. Les testaments des rois laissant des fils mineurs
ont quelquefois mis un, grand dans le conseil qu'ils nommaient pour et
au nom de tous les autres, mais par eux exprimé et choisi, et, s'il est
dit comme grand, ce n'est pas, comme on le voit, par ce qui vient d'être
remarqué, qu'un grand comme tel y fût nécessaire, mais par égard pour
eux, et ne sembler pas n'en trouver aucun digne d'y être admis.

Dans le fameux testament de Charles II, qu'on peut dire avoir été
l'ouvrage de quelques grands qui le signèrent, et d'autres grands qui le
surent, et dans la régence qui y fut établie en attendant l'arrivée du
successeur nommé, on voit des égards pour les charges, les places, les
emplois, les personnages, rien ou presque rien donné à la dignité de
grand, dont le concours et l'autorité ne paraît point nécessaire en
dispositions de si grand poids, et qui décidait le sort de cette grande
monarchie. On voit les grands appelés à l'ouverture du testament de
Charles II après sa mort, qui est peut-être la plus auguste et la plus
solennelle action où ils l'aient été. Mais je dis action, et non pas
fonction, puisqu'ils n'y en eurent aucune, et qu'il n'y fut question que
d'apprendre les premiers, et avec décence pour les premiers seigneurs de
la monarchie, en faveur de qui le roi défunt en disposait, la forme de
gouvernement qu'il prescrivait, ceux qu'il admettait, et s'y soumettre
sans aucune forme d'opiner ni de délibérer. Cela fut fait de la sorte
par ceux-là mêmes qui, en les convoquant, savaient bien ce qu'on allait
trouver. Mais un cas unique et sans exemple de la monarchie sans
successeur connu demandait bien une telle formalité en faveur des plus
grands, des plus distingués, et des premiers seigneurs de cette même
monarchie revêtus de la plus grande dignité, pour livrer cette mémé
monarchie à celui que le testateur y avait appelé sans les consulter ni
leur en parler. Ce cas donc si extraordinaire, ou plutôt si unique, ne
constitue point par lui-même aucun droit délibératif ni judiciaire en
quoi que ce soit aux grands, qui même n'y jugèrent et n'y délibérèrent,
mais écoutèrent, apprirent les dispositions, et s'y soumirent sans
qu'aucun entamât aucun discours que d'approbation, et la plupart en un
mot, ou par leur silence. Ainsi rien d'acquis par là ni en matière de
lois intérieures ni en matière d'État. De ce grand et unique exemple,
exemple si signalé, et de tout ce qui a été rapporté auparavant, il faut
donc conclure que la dignité de grand consiste uniquement en
illustration cérémonielle de rangs, prééminences, prérogatives, honneurs
et distinctions, et en accompagnement très privilégié et nécessaire de
décoration du roi.

Depuis les rois catholiques, aucun roi d'Espagne n'a été couronné, aucun
n'a porté d'habit royal ni particulier, en aucune occasion. Les rois
catholiques, c'est-à-dire Ferdinand et Isabelle, l'avaient été, et avant
eux tous les rois particuliers l'étaient dans les Espagnes. Je n'ai
aucune notion que les ricos-hombres eussent, en ces occasions, des
habits propres à leur dignité, ou des fonctions à eux particulières. Ces
royaumes étaient petits, peu puissants, toujours troublés entre eux et
par les Maures\,; il y a lieu de croire que tout s'y passait
militairement et simplement. Quoi qu'il en soit, depuis que le nom et la
dignité de grand a aboli, sous le premier commencement de Charles-Quint,
les ricos-hombres, il n'y a point eu d'habit particulier en aucune
cérémonie ni en aucune occasion, non plus qu'au roi d'Espagne.

Dans les divers ordres d'Espagne et dans celui de la Toison d'or, l'idée
de l'ancienne chevalerie a prévalu à leur dignité, même à celle des
infants. Ces princes ni les grands n'y ont d'autre préférence de rang
que celui de l'ancienneté de la réception, et entre ceux de même
réception, que celui de l'âge. Philippe V est le premier qui ait donné
au prince des Asturies, mort roi d'Espagne, le rang au-dessus de tous
les chevaliers de la Toison, et un carreau sous ses pieds au chapitre,
mais assis à la première place du banc à droite avec les chevaliers, et
coude à coude, sans distance ni distinction du chevalier son voisin, et
faisant la fonction du plus ancien chevalier sans différence, qui est
d'accommoder le collier du nouveau chevalier et l'attacher de son côté,
tandis que le parrain l'attache sur l'autre épaule, et le chancelier de
l'ordre par derrière, puis d'embrasser le nouveau reçu comme tous les
autres chevaliers\,; encore a-t-il fallu que le roi d'Espagne ait
demandé cette préséance et ce carreau aux chevaliers qui l'ont accordé,
et à qui pourtant cela a paru fort nouveau. Sur cet exemple, les autres
infants ont eu le même avantage. J'ai vu ce que je raconte ici à la
réception de mon fils aîné, mais il est vrai qu'à celle de Maulevrier,
qui fut quelque temps après, le prince des Asturies attacha bien un côté
de son collier, mais que, quand ce fut à l'embrassade, il ne se souleva
seulement pas, ne l'embrassa ni n'en fit pas même semblant, et se fit
baiser la main. Au sortir de la cérémonie, la plupart des chevaliers
m'en parlèrent, et s'en parlèrent entre eux comme d'une nouveauté sans
exemple et très offensante, dont ils auraient été bien aises pour
Maulevrier qui était fort haï, mais dont la conséquence pour les
chevaliers qui seraient faits dans la suite, et de là pour l'ordre, les
piquait extrêmement. Je ne sais comme cela se sera passé depuis.

Avec toute la grandeur et la hauteur des grands d'Espagne, ils ne
laissent pas de rechercher des emplois qu'on aurait peine à croire et
qu'on voie rien à quoi cela les puisse mener. Ils en font même
quelquefois des fonctions par eux-mêmes\,; d'autres fois ils subrogent
quelqu'un pour les faire en leur place, en leur absence\,; enfin,
quelques autres ne les ont que par honneur. Ces emplois, sous d'autres
noms, ne sont que des échevinages de villes, même médiocres, avec de
simples gentilshommes et des bourgeois. Il y aura quelquefois deux ou
trois grands, et des plus distingués en tout, échevins de la même
ville\,; il s'en trouve aussi à qui les plus petites défèrent ce bizarre
honneur et qui ne le refusent pas. Mais n'en voilà peut-être que trop
pour donner simplement une juste idée des grands d'Espagne et de leur
dignité, qui n'avait ce semble que frappé les yeux et les oreilles, sans
avoir encore passé fort au delà. Ils reviendront encore plus d'une fois
en propos à l'occasion de différentes choses de mon ambassade.

Je n'ose pourtant finir ce qui regarde cette matière sans dire une
vérité fâcheuse. C'est qu'il n'est pas inouï, il n'est pas nouveau, que
les rois aient accordé la grandesse pour de l'argent. Cette sorte de
marché s'est fait plus d'une fois, et sous plus d'un règne, et j'ai vu
en Espagne plus d'un grand de cette façon. Quand cela se fait c'est tout
uniment. On n'y met ni voile pour le temps ni masque pour l'avenir\,; on
traite tout simplement, on convient de prix, et ce prix est toujours
fort\,; l'argent en est porté dans les coffres du roi, qui, au même
instant, confère la grandesse. Il y en {[}a{]} même de tels de qualité
distinguée, mais ceux de qualité distinguée qui ont acheté lie sont pas
Espagnols.

Récapitulons maintenant ce qui vient d'être dit des usages de la
grandesse, comme nous avons fait pour ce qui en regarde l'essence et le
fond\,: en joignant l'une et l'autre, on aura le précis en peu de lignes
de tout ce qui concerne cette dignité.

Nulle marque extérieure ce la grandesse, aux carrosses ni aux armes. La
reine même n'a point de housse. Depuis la fraternité d'honneurs des ducs
et des grands, plusieurs, même de ceux qui ne sont point sortis
d'Espagne, ont pris le manteau ducal, mais presque aucun Espagnol
naturel. De marques dans leurs maisons, nulles, excepté le dais. Ils
l'ont de velours et souvent leurs armes brodées dans la queue, etc. Les
conseillers d'État et les \emph{titulados}, et il y en a de fort
étranges, en ont aussi, mais de damas, avec un portrait du roi dans la
queue, comme le dais étant là pour le portrait\,; de balustres, le roi
et la reine même n'en ont point.

Démissions des grandesses inconnues, mais les fils aînés des grands ont
des distinctions, et leurs femmes ne diffèrent presque en rien de celles
des grands\,; toutefois deux exemples sous Philippe V, l'un après la
bataille d'Almanza pour le duc de Berwick, l'autre pour moi à l'occasion
du double mariage\,: deux cas uniques, deux étrangers, deux hommes qui,
comme ducs de France, jouissaient déjà de tous les honneurs de la
grandesse, et ces deux exceptions portées par la concession même\,;
inutilité abusive de celui du comte de Tessé.

Couverture d'un grand majestueuse, semblable à la première audience
solennelle d'un ambassadeur.

Différence des trois classes\,:

La première trouve la famille du roi, c'est-à-dire ses bas officiers, à
la descente du carrosse, le majordome de semaine au bas du degré, et le
degré entier bordé des hallebardiers de la garde sous les armes jusqu'à
l'entrée de l'appartement, quelques grands au haut du degré qui en
descendent deux marches\,; se couvre avant de parler au roi, et ayant
fini et fait la révérence, se couvre avant que le roi commence à lui
répondre et l'écoute couvert\,; la garde des régiments des gardes
espagnoles et wallonnes sous les armes dans la place du palais\,; reçoit
les mêmes honneurs en sortant comme en entrant.

La seconde n'en a aucun en entrant ni en sortant, trouve le majordome de
semaine au haut du degré et quelques grands un peu plus loin, parle au
roi découvert, se couvre avant qu'il lui réponde.

La troisième trouve le majordome de semaine à la porte de l'appartement
du roi\,; nuls grands au-devant de lui\,; parle au roi et attend sa
réponse, découvert, qui ne lui dit \emph{cobrios} qu'après lui avoir
baisé la main, et ne se couvre qu'à la muraille. Toutes trois gardent
chez la reine les mêmes différences de se couvrir.

Le roi est debout, la reine chez elle est assise dans un fauteuil, et ne
dit point \emph{cobrios} parce qu'elle ne fait pas les grands. Point de
fonctions de parrain chez elle. Son majordome n'accompagne le grand que
jusqu'à sa première révérence, qu'il fait avec lui\,; à la seconde salue
les dames avant les grands, et point les seigneurs ni les gens de
qualité, non plus que chez le roi\,; va faire un compliment aux dames
qui ont l'excellence lorsque la reine se retire\,; chez le prince des
Asturies, visite de respect sans se couvrir et sans cérémonie.

Nulle cérémonie, nul acte public, nulle fonction, nulle fête publique
que le roi donne au palais ou ailleurs, à laquelle il assiste au dehors,
que les grands ne soient invités, leurs femmes, s'il y a des dames, et
leurs belles-filles aînées, et s'il n'y a point à se couvrir, les maris
de celles-ci, et partout en ces occasions, qui sont très fréquentes, ils
ont tous beaucoup d'avantages en nombre et en distinctions de places.

Eurent, eux et leurs femmes, leurs fils et belles-filles aînées, les
premières et plus proches places au mariage du prince des Asturies, et
conviés d'y venir à Lerma, pareillement à Madrid au baptême de don
Philippe, où j'ai remarqué le dégoût qu'ils eurent d'y porter les
honneurs.

Les grands furent tous mandés et assistèrent, seuls, avec le service le
plus étroit et le plus indispensable, à la lecture et à la signature du
contrat de mariage du roi et de l'infante.

Ont aux chapelles un banc couvert de tapis ensuite du roi, et y sont
salués autant de fois que le roi.

Sont couverts aux audiences solennelles et publiques, et toutes les fois
partout que le roi l'est, sans qu'il le leur dise.

Sont traités de cousins quand le roi leur écrit\,; ont avec différence
des classes des distinctions dans le style de chancellerie\,; en ont
tous aussi dans les lettres ordinaires. Les fils aînés des grands sont
traités par le roi de parents\,; les femmes le sont comme leurs maris.

Ont, hors Madrid et des lieux où le roi se trouve, un tapis à l'église
et double carreau pour les coudes et pour les genoux. Ont tous les
honneurs civils et militaires\,; la première visite du vice-roi et la
main chez lui\,; s'ils sont sujets et habitués dans la vice-royauté, ou
officiers de guerre, une fois et puis plus. Pareillement à l'armée, une
garde et la main chez le général, une seule fois, puis servent de
volontaires ou dans l'emploi qu'ils ont\,; de même font leur cour au
vice-roi avec les honneurs et les distinctions que les grands du pays
ont chez lui.

Les femmes des grands ont chez la reine des carreaux de velours en tout
temps, et leurs belles-filles aînées de damas ou de satin, de même à
l'église pour se mettre à genoux, à la comédie pour s'asseoir\,; et
maintenant des tabourets au bal\,; toutes les autres debout ou par
terre.

Distinction d'aller par la ville à deux et quatre mules avec ou sans
postillon, à traits courts, longs, ou très longs. Ces derniers ne sont
que pour les grands, leurs fils aînés, leurs femmes, les cardinaux, les
ambassadeurs et le président du conseil de Castille.

Leurs cochers les mènent quelquefois tête nue, toujours leurs femmes et
leurs belles-filles aînées, et en chaise, le porteur de devant toujours
découvert aussi pour les grands qu'il porte.

Grande précision et distinction à la réception et conduite des visites.

Les grands ne cèdent à personne, excepté ce que j'ai dit du président ou
gouverneur du conseil de Castille, du majordome-major du roi, et
rarement des cardinaux et des ambassadeurs\,; nul autre rang que le leur
et pour eux, et maintenant donné aux ducs de France. Princes étrangers
faits grands à vie à cause de cela. Souverains sans avantages sur eux en
Espagne, même ducs de Savoie. Ceux qui y furent accordés au célèbre.
Charles-Emmanuel, depuis duc de Savoie, médiocres, et en considération
de son mariage réglé avec l'infante. Prince de Galles, depuis roi
Charles Ier d'Angleterre, hors de pair et d'exemples. Duc d'Orléans
visita toutes leurs femmes, eut le traitement d'infant, traita les
grands comme il traite les ducs de France. Princes du sang de même, et
les infants, comme font les fils de France. On remet à parler de
d'usurpation de la princesse des Ursins et du duc de Vendôme, qui ne
leur fut pas heureuse, à l'exemple des deux don Juan expliqués. Personne
même de ce qui n'était point grand ne voulut céder au prince ni à la
princesse d'Harcourt, qui menèrent la reine, fille de Monsieur. Ils
n'eurent aucuns honneurs particuliers, ni la comtesse de Soissons
depuis.

Les grands sont traités d'égaux chez les électeurs et les autres
souverains, comme les souverains d'Italie chez le pape, et dans Rome
comme les princes du Soglio.

Ont cependant en Espagne plusieurs désavantages qui ont été marqués avec
le gouverneur du conseil de Castille, les cardinaux, les ambassadeurs,
le majordome-major du roi, et en carrosse avec le grand écuyer.

N'ont ni voix ni séance en aucun tribunal, ni part nécessaire aux lois
ni au gouvernement de l'État, ni distinction en la manière d'être jugés
en aucun cas.

Ont séance au-dessus de tous les députés aux cortès ou états généraux,
lesquels ne iront que prêter hommage, et n'ont rien des prétentions de
ceux de France. Un seul pour tous, mais sans nécessité, a quelquefois
été nommé dans les testaments des rois pour être du conseil de régence.
Très peu ont eu part au testament de Charles Il. Tous furent appelés à
son ouverture, et tous sans opiner pour s'y soumettre, cas unique en
singularité et nécessité qui ne leur ajoute aucun droit.

Nul couronnement des rois d'Espagne depuis les rois catholiques, et nul
habillement royal en aucune occasion\,; nul habit distinctif ni
particulier aux grands ni à leurs femmes.

Nul rang ni distinction dans l'ordre de la Toison, ni dans les autres
d'Espagne. Rang avec tous par ancienneté dans l'ordre, et en mémé
réception par âge.

Prennent des emplois municipaux fort au-dessous d'eux, et qui ne les
mènent à rien.

Bâtards devenus grands.

Exemples, et plusieurs, et de plusieurs règnes, et d'Espagnols et
d'étrangers, qui ont acheté et payé fort cher et fort publiquement la
grandesse\,; même entre les étrangers de naissance distinguée, plusieurs
encore existants.

Nul serment pour la dignité de grand d'Espagne, parce qu'elle n'a que
rang, honneurs, etc., et nulle sorte de fonction.

Le nombre des grands d'Espagne beaucoup plus grand en Espagne même que
celui des ducs en France, sans compter les grands établis en Italie et
aux Pays-Bas, même avant l'avènement de Philippe V à la couronne, et
fort augmenté depuis.

Et nombre qui ne diminue presque jamais par la succession à l'infini par
les femelles, en sorte qu'il ne peut guère diminuer que par la chute des
grandesses à d'autres grands par héritage, comme le duc de Medina-Celi
qui en a recueilli seize ou dix-sept qui toutes sont sur sa tête, et qui
toutes ne peuvent passer de lui que sur la même tête, sans que celui qui
en a ce grand nombre ait la moindre préférence en rien par-dessus les
autres grands ni même parmi eux, en sorte qu'il est entièrement
indifférent d'en avoir plusieurs ou de n'en avoir qu'une.

\hypertarget{chapitre-xvi.}{%
\chapter{CHAPITRE XVI.}\label{chapitre-xvi.}}

1701

~

{\textsc{Comparaison des dignités des ducs de France et des grands
d'Espagne.}} {\textsc{- Comparaison du fond des deux dignités dans tous
les âges.}} {\textsc{- Dignité de grand d'Espagne ne peut être comparée
à celle de duc de France, beaucoup moins à celle de pair de France.}}
{\textsc{- Comparaison de l'extérieur des dignités des ducs de France et
des grands d'Espagne.}} {\textsc{- Spécieux avantages des grands
d'Espagne.}} {\textsc{- Désavantage des grands d'Espagne jusque dans les
droits de se couvrir.}} {\textsc{- Abus des grandesses françaises.}}

~

Après cette connaissance de la dignité de grand d'Espagne dans son fond,
dans son origine, et de son état présent, il faudrait en donner une de
celle des ducs de France, pairs, vérifiés et non vérifiés, ou à brevet,
comme on appelle improprement ces derniers. Mais ce n'est pas ici le
lieu des dissertations et des histoires particulières, quelque
obscurcissement qu'on ait pris à tâche de jeter, surtout depuis quelque
temps, sur la première dignité du royaume de France. Elle y est encore
trop connue pour avoir besoin d'entrer dans un détail qui ferait un
volume. Je me contenterai donc de supposer ce qui est vrai, et démontré
par tous les auteurs, et par toutes les images qui restent de la
grandeur de cette dignité, et que l'ignorance, la jalousie\,; l'envie,
la malice, j'ajouterai la folie de ces derniers temps, n'ont pu
étouffer. Il faut se souvenir de l'occasion de cette digression, c'est
l'égalité convenue entre le roi et le roi son petit-fils, des ducs de
France et des grands d'Espagne, et de leur donner réciproquement les
mêmes rangs et honneurs\,; le mémoire présenté au roi d'Espagne, pour
s'en plaindre\,; par les ducs d'Arcos et de Baños, et la punition que
ces deux frères en subirent, et l'examen s'ils ont été bien ou mal
fondés dans cette plainte, ce qui ne se peut faire que par la
comparaison des deux premières dignités des deux monarchies\,; mais en
même temps qu'il ne s'agit pas de faire un livre, ni de s'écarter trop
loin et trop longtemps des matières historiques de ces Mémoires.

En quelque temps que l'on considère la monarchie Française depuis sa
fondation, et les divers États des Espagnes jusqu'à leur réunion, sous
Ferdinand et Isabelle, ou plutôt sous Charles-Quint qui hérita d'eux,
pour ne faire de toutes les Espagnes, excepté le Portugal, qu'une seule
monarchie, elle ne peut entrer en aucune comparaison avec la nôtre. Des
provinces séparées, quoique avec titre de royaume, dont aucun ne l'a
porté que longtemps après la France, n'ont pas plus de similitude avec
ce grand et vaste tout, réuni sous un seul chef, que par la différence
d'antiquité de couronne. Conséquemment nulle proportion entre les grands
vassaux, les vassaux immédiats de la couronne de France, et ceux des
différentes pièces qui composaient les Espagnes sous différents chefs,
connus sous le titre de rois beaucoup plus tard que les nôtres. Quelques
fonctions qu'aient originairement eues ces premiers grands feudataires
des Espagnes, elles n'ont pu être plus importantes et plus relevées que
celles de nos premiers grands vassaux, et la différence en a toujours
été infinie par celle du cercle étroit de chacun de ces petits États ou
royaumes indépendants les uns des autres dans les Espagnes, de la vaste
étendue du royaume de France sous un seul roi dans tous les temps, et la
part que les uns et les autres ont eue aux affaires, soit intérieures,
soit extérieures de l'État dont ils relevaient immédiatement, a été
conforme pour le poids et pour le nombre à l'étendue de ces mêmes États,
ce qui met encore une différence infinie entre les grands vassaux
François et espagnols.

Si de ces temps reculés on descend au moyen âge, on ne voit dans les
Espagnes que la confusion qu'avait faite la domination des Maures, la
nécessité de se défendre et de se soutenir contre eux où étoilent les
rois des différentes provinces des Espagnes, et trop souvent les
usurpations de ces mêmes rois, les uns sur les autres. On ne voit plus
que force, que nécessité, que multiplication sans mesure des
ricos-hombres sans fiefs. Leur part, je dis nécessaire et par droit,
dans les affaires s'évanouit, et depuis il n'en est resté ni ombre ni
vestige, en quoi les grands d'Espagne successeurs de leur dignité ne
sont pas devenus de meilleure condition qu'eux.

Tout au contraire en France. Les grands vassaux ont toujours eu de droit
et de fait part aux grandes affaires du dehors et du dedans. Cette part
est demeurée aux pairs par essence, aux officiers de la couronne qui,
par leurs offices, étaient grands vassaux, puisqu'ils en rendaient foi
et hommage particuliers au roi, à d'autres grands vassaux, mais quand,
et à ceux qu'il plaisait aux rois d'y appeler. Cette transmission dure
jusqu'à nos jours, et sans parler de tant de grands actes de pairie des
temps anciens, il n'y a point de règne qui n'en fourmille jusqu'au
dernier le plus absolu de tous. Témoin tous les lits de justice que le
feu roi a tenus, et en dernier lieu la convocation des pairs par le
grand maître des cérémonies au nom du feu roi, pour l'acte des
renonciations qui a précédé la mort de Sa Majesté de si peu.

De jugement et de nécessité de celui des pairs en certaines affaires, et
de droit en presque toutes, c'est encore une chose qui a toujours été et
qui subsiste encore\,; de même que les formes solennelles pour juger
d'une pairie, ou pour faire le procès criminel à un pair. Rien de tout
cela en Espagne. On ne le voit point des ricos-hombres, on le voit aussi
peu des grands. Leurs grandesses pour la transmission ni pour le
jugement, si contestation arrive, ni leurs personnes, si elles se
trouvent prévenues de crime, n'ont aucune distinction dans la forme de
les juger du moindre héritage ni du moindre particulier. Tout se réduit
pour la seule personne des grands à ne pouvoir être arrêtée que par un
ordre du roi, après quoi plus de distinction dans tout le reste\,; et
jamais en Espagne il n'a été mention d'être jugé par ses pairs,
c'est-à-dire par ses égaux, ce qui, en matière de pairie ou de crime
d'un pair, subsiste encore pour les pairs de France.

En voilà sans doute assez pour démontrer la différence entière des pairs
de France d'aujourd'hui et des grands d'Espagne, et combien il y aurait
peu de justesse de comparer, sous prétexte de convenance, les grands de
la première classe avec les pairs.

Si du fond de la substance de la dignité et de son antiquité transmise
jusqu'à nous, on passe à son inhérence et à sa stabilité, on est
extrêmement surpris de n'en trouver aucune dans la grandesse, et de la
voir non seulement suspendue à chaque mutation, même de père à fils dans
toutes celles, qui ne sont pas de la première classe, du propre aveu de
ces grands, mais suspendue encore par le délai ou le refus de la,
couverture, tant qu'il plaît au roi, pour toutes les trois classes, et
toutes les trois amovibles, et pour toujours, à la volonté du roi, sans
forme aucune, sans crime, sans accusation, sans même de prétexte. On ne
saurait nier qu'une dignité aussi en l'air, autant dans la main du roi,
et d'une manière si absolue et si totalement dépendante, ne soit fort
différente de celles dont l'état est déterminé, fixe, stable, certain à
toujours, et qui, une fois accordées, n'ont plus besoin de nouvelles
grâces, et ne puissent être ôtées qu'avec la vie, pour crime capital, et
avec les formes les plus solennelles.

Il est difficile de n'être pas blessé d'un tribut imposé à une dignité
comme telle, à plus forte raison de tributs redoublés. Ceux qu'on a
expliqués ne ressemblent point aux lods et ventes des terres, ni aux
autres droits de la suzeraineté. Ce n'est point ici une terre qui paye
pour sa mutation, puisque les grandesses attachées aux noms et non aux
terres sont sujettes aux mêmes tributs, et que, faute de payement, ce ne
sont point les terres qui en répondent, mais la dignité qui est
suspendue encore dans ce cas. En France, la noblesse grande, médiocre,
petite, doit le service des armes, mais nul tribut pour elle-même. Ce
qu'elle paye est sur sa consommation, des droits de terre, en un mot
toute autre chose qu'un tribut de noblesse et à cause de sa noblesse.
Combien donc y doit-on être surpris de voir la première et la plus haute
dignité où la noblesse la plus distinguée puisse parvenir en Espagne,
être imposée à divers tributs comme dignité, et pour elle-même, et à
peine de suspension jusqu'à parfait payement\,? Qui peut douter de la
différence que cela met encore entre, la dignité de nos ducs et celle
des grands d'Espagne\,?

Enfin la vénalité de la grandesse, non entre particuliers, mais du roi à
eux, qui l'a quelquefois vendue, depuis Philippe II, sous tous les
règnes, et vendue sans voile et sans mystère. Quelque rares qu'en soient
les exemples, ils sont, et encore une fois il y en a, et de tous les
rois, depuis Philippe II\,; la dignité des ducs a ignoré jusqu'à nos
jours cette manière d'y arriver, qui est commune aux plus petites
charges.

Il résulte donc, de toutes ces différences si essentielles, que la
dignité de grand d'Espagne, pour éclatante qu'elle soit, ne peut être
comparée avec celle de nos ducs, et beaucoup moins encore {[}avec{]}
celle des pairs de France, avec lesquels les grands d'Espagne n'ont
aucune similitude, sont sans fonction, sans avis, sans conseil, sans
jugement, sans faire essentiellement partie de l'État plus que les
autres vassaux immédiats, et sont sans serment et sans foi et hommage
pour cause de leur dignité. Il est donc conséquent que ce n'est à aucun
d'eux à se trouver blessé de la parité convenue, entre le feu roi et le
roi son petit-fils, des ducs de France et des grands d'Espagne, et que
les ducs d'Arcos et de Baños y ont été très mal fondés, et y ont très
peu entendu l'intérêt de leur dignité.

Ce fond des deux premières dignités de France et d'Espagne examiné, il
faut venir à leur extérieur.

Si on est ébloui de certaines choses que les grands d'Espagne ont
conservées par la sage politique de leurs rois, et que les nôtres ont
laissé peu à peu obscurcir dans les ducs, il se trouvera que ceux-ci ont
eu les mêmes avantages, qu'ils les ont presque tous conservés jusque
vers le milieu du dernier règne, et qu'il y en a d'autres où la dignité
des ducs est plus ménagée que ne l'est celle des grands.

Deux choses, l'une au dehors, l'autre au dedans, {[}font{]} briller la
dignité de grand d'Espagne beaucoup plus que celle des ducs de France.
C'est, à qui n'approfondit pas le fond des dignités qui vient d'être
examiné, et à qui n'examine que l'usage\,; présent sans remonter plus
haut, ce qui éblouit le monde en faveur des grands d'Espagne.

Ces deux choses regardent les princes étrangers. On a vu avec quel soin
Charles-Quint établit le rang des grands d'Espagne à Rome, en Italie, en
Allemagne, et partout où s'étendit sa puissance, et avec quelle jalousie
ce même effet de sa politique a été soutenu depuis par les rois
d'Espagne en Italie, à la faveur des grands États qu'ils y ont possédés
depuis Charles-Quint jusqu'à Charles II, et en Allemagne, à l'appui des
empereurs de la même maison d'Autriche. Il ne se trouvera point qu'il en
ait été usé autrement avec les ducs de France jusque vers le milieu du
dernier règne. Sans en discuter les exemples, qui mèneraient trop loin,
il suffit de voir comment le duc de Chevreuse, fils du duc de Luynes, a
été traité à Turin et chez les électeurs, voyageant tout jeune. Ces
voyages font une partie de ceux de Montconis, alors son gouverneur, qui
sont entre les mains de tout le monde, où il touche ce fait sans la
moindre affectation, parce qu'il appartient à ce qu'il raconte. Le duc
de Rohan-Chabot, allant voyager à dix-sept ou dix-huit ans, M. de Lyonne
lui donna une instruction en forme et signée, pour se conduire avec M.
de Savoie également en tout, excepté la main, et pour la prétendre des
électeurs, à plus forte raison de tous les autres souverains d'Allemagne
et d'Italie, et de ne pas voir les électeurs s'ils en faisaient
difficulté. Non seulement les ducs, comme tels, mais les maréchaux de
France, généraux d'armée, ont toujours traité en égalité parfaite avec
les électeurs et tous les autres souverains, comme on le voit par les
lettres du maréchal de Créqui dernier, qui n'était point duc, et de tous
les autres. Une méprise du maréchal de Villeroy à l'égard de l'électeur
de Bavière fit la planche, et de cette planche il a résulté que ce même
électeur, qui ne disputait pas en Hongrie aux princes de Conti, à ce que
M. le prince de Conti m'a dit et raconté plusieurs fois, prétendit, tout
incognito qu'il était, la main chez Monseigneur, et fit si bien qu'il ne
le vit chez lui que dans lés jardins de Meudon, sans mettre le pied dans
la\,: maison, et qu'il montèrent en calèche pour s'y promener tous deux
en même temps par chacun leur portière. Cette égalité avec le Dauphin
n'était pas jusqu'alors entrée dans la tête d'aucun souverain non roi,
et celui-là même avant le profit qu'il sut tirer de la lourde méprise du
maréchal de Villeroy, n'avait pas imaginé de disputer rien à un prince
du sang, non plus que le fameux duc de Lorraine, qui commandait en chef
d'armée de l'empereur, dont il avait l'honneur d'être le beau-frère, et
les princes de Conti volontaires dans cette armée\,; c'est ainsi que des
dignités on entreprend sur leur source, et c'est ce que les papes et les
rois d'Espagne ont sagement prévu et prévenu sur les cardinaux et les
grands.

Dans l'intérieur, la même prévoyance, mais commune à tous les États de
l'Europe, a refusé avec persévérance jusqu'à aujourd'hui tout rang aux
princes étrangers. La seule France les y a établis, et leur a laissé peu
à peu usurper toutes sortes d'avantages\,; ils s'y sont d'abord
introduits sans y en prétendre aucun. Après ils ont ambitionné la
pairie. Ils en ont obtenu après tant qu'ils ont pu. Ils en ont fait
valoir les prérogatives. Devenus puissants, ils ont formé la ligue à la
faveur de laquelle ils ont empiété par degrés, laquelle aurait dû donner
des leçons à n'être pas oubliées. Bien des événements les ont depuis
rafraîchies, mais tout le fruit n'a été que d'augmenter les usurpations
en y associant des branches de maisons de gentilshommes français, de
peur de manquer de princes étrangers vrais ou faux. Il est vrai qu'en
nul lieu ces derniers n'ont précédé les ducs\,; il est vrai encore que
les princes étrangers véritables ne les précèdent encore nulle part, si
ce n'est dans l'ordre du Saint-Esprit, contre les premiers statuts et le
premier exemple de la première promotion que la puissance de la Ligue
fit réformer en deux fois, et que d'étranges causes ont maintenu sans
décision, mais en continuant l'usage. Il est vrai de plus que ceux-là
mêmes, quand ils sont pairs, suivent leur rang d'ancienneté en tous
actes de pairie. Il est donc vrai qu'ils cèdent aux pairs, et qu'ils ne
les précèdent jamais, excepté dans l'ordre, de la façon que je viens de
le dire. Cela suffit pour montrer qu'il n'en était pas ainsi avant le
dernier siècle\,; qu'il y avait déjà des ducs gentilshommes, et que ce
qui s'est introduit depuis n'est qu'usurpation qui laisse la dignité
entière. Mais il faut convenir que la multitude des usurpations, des
distinctions, et de ceux qui en jouissent, l'éclat et les avantages
qu'ils en retirent, la lutte de préséance qu'ils entretiennent à la cour
sur des gens qui s'en lassent et qui n'ont jamais su s'entendre ni se
soutenir, est la chose qui donne le plus spécieux prétexte aux grands
d'Espagne, chez lesquels ces princes n'ont aucun honneur, aucun rang,
aucun établissement, et qui, s'ils s'attachent au service d'Espagne,
n'en peuvent prétendre ni espérer aucun que pour être faits grands
d'Espagne eux-mêmes. Je n'en dirai pas davantage pour ne pas tomber dans
l'inconvénient d'une dissertation contre ces rangs étrangers qui ne sont
soufferts nulle autre part qu'en France.

À ces deux avantages dont il faut convenir quoique en écorce et en
surface sans fond, les grands en ont encore deux autres que les ducs
avaient comme eux les honneurs militaires et civils, dont M. de Louvois
les priva sous prétexte de ménager la poudre, d'où le reste des honneurs
militaires et civils, se sont peu à peu évanouis pour être appropriés
aux ministres qui avant cette insensible époque étaient bien éloignés
d'y prétendre. Cet avantage est donc un de ceux que la dignité de duc a
perdus par l'usage\,; mais qui ne lui est pas moins propre qu'aux
grands, puisqu'ils en ont constamment joui jusqu'à la toute-puissance de
M. de Louvois vers le milieu de son ministère.

Ces quatre avantages que l'usage a conservés aux grands et ôtés aux
ducs, et qui leur ont été également propres, ne consistent donc que dans
lit volonté différente de leurs rois, et dans une différence de volonté
si moderne qu'elle laisse voir le droit et le long usage en faveur des
ducs, et laisse ainsi leur dignité entière, en cela même que le vouloir
des rois y a donné pour la surface l'atteinte dont on ne peut
disconvenir, mais qui ne peut rien opérer de solide contre leur dignité
en faveur de celle des grands, puisque le droit et l'usage, est le même,
et qu'il ne tient qu'à nos rois de le remettre comme il a été en partie
jusqu'à la violence de la Ligue, et en partie jusqu'à M. de Louvois.

Les grands ont encore deux autres avantages\,: l'un n'est qu'un agrément
et une distinction, qui est d'être seuls conviés, ainsi que leurs
épouses, avec leurs fils aînés et les leurs, à tout ce qui se fait de
plus ordinaire et d'extraordinaire en fêtes, divertissements et
cérémonies à la cour ou ailleurs quand le roi s'y trouve, ou qu'ils se
font par ses ordres. Cela fait un accompagnement de grande décoration au
roi, et les nôtres en ont usé de même jusque vers les deux tiers du
règne de Louis XIV\,; ainsi je ne m'arrêterai pas à celui-ci, quoiqu'il
paroisse beaucoup en Espagne, où pour les chapelles, les audiences
publiques et mille occasions, il y en a de continuelles de ces
avertissements aux grands, lesquelles presque toutes n'existent point en
France et y ont toujours été rares de plus en plus.

L'autre avantage des grands en est un effectif\,; la bonne foi veut
qu'on l'avoue, mais il est l'unique à l'égard des ducs. C'est le rang et
les honneurs de leurs fils aînés et des femmes de ces fils aînés, et
quand ils n'ont point de fils, de celui ou de celle à qui la grandesse
doit aller de droit après eux. Les distinctions des fils sont peu
perceptibles, comme l'invitation dont on vient de parler, l'excellence
qui s'est fort multipliée, le traitement de parent quand le roi leur
écrit, et divers autres\,; mais celles de leurs femmes ou de leur fille
aînée, s'ils n'ont point de fils, sont pareilles en tout à celles des
femmes des grands en tout et partout, à l'exception seule de l'étoffe de
leurs carreaux chez la reine pour s'asseoir, ou devant elle à l'église
pour se mettre à genoux (je l'ai dit plus haut), de velours pour les
femmes des grands en toute saison, et de damas ou de satin en toute
saison pour leurs belles-filles aînées. Or, il est vrai que cela n'a
aucune comparaison avec les fils aînés des ducs et leurs femmes\,; cela
est sans doute accordé à ce qu'il n'y a jamais de démission en
Espagne\,; mais quelque anciennes que soient les nôtres qui ont commencé
au dernier connétable de Montmorency, la bonne foi veut encore l'aveu
que nos démissions ne couvrent point cette différence essentielle, parce
que la démission opère un duc, qui par conséquent en a le rang et les
honneurs, que le démis conserve aussi, au lieu que, sans démission, les
fils aînés des ducs n'ont aucune distinction ni leurs femmes, et que les
fils aînés des grands et leurs femmes ont comme tels toutes celles dont
on vient de parler. Mais cet avantage, quelque solide qu'il soit, et qui
est l'unique effectif que les grands aient au-dessus des ducs, ne change
rien au fond de leur dignité\,; il la laisse telle qu'elle a été
montrée\,; il est même un témoignage et un reste de cette multiplication
des ricos-hombres par leurs cadets, et par les cadets de ces cadets,
sans fiefs, qui vers les temps de Ferdinand et d'Isabelle en avaient
défiguré la dignité, et qui à l'habile refonte que Charles-Quint en fit
sous le nom de grands, a été restreinte à des bornes plus raisonnables,
par cet avantage des seuls fils aînés ou successeurs nécessaires des
grandesses au défaut de fils, et de leurs épouses, qui a ôté toute
occasion de démissions.

Après avoir exposé dans toute son étendue les six avantages que les
grands paraissent avoir sur les ducs, je dis paraissent, puisqu'il n'y a
que de l'éblouissant dans les cinq premiers, que les ducs ont eus comme
eux jusqu'au milieu du dernier règne, et comme eux les premières places
partout, dont le feu roi s'est montré si jaloux jusqu'à sa mort\,;
témoin l'aventure de M\textsuperscript{lle} de Melun à un bal, et celle
de M\textsuperscript{me} de Torcy à la table du roi à Marly, les deux
uniques qui s'y soient exposées\,; après avoir avoué de bonne foi la
solidité du dernier et sixième avantage des grands en la personne de
leur fils et belles-filles aînées, il faut venir aux désavantages de ces
mêmes grands comparés aux ducs pour l'extérieur.

Quelques usurpations modernes qu'aient essuyées les ducs du chancelier,
et même du garde des sceaux de France, elles ne vont qu'à la préséance
au conseil, et s'ils ont conservé l'ancienne forme d'écrire et de
recevoir chez eux, que les ducs et les officiers de la couronne ont
perdue, cela ne regarde point les ducs. Mais le président, ni en son
absence le gouverneur du conseil de Castille, ne donne point la main
chez lui aux grands, qui de plus sont obligés, comme tous les autres,
d'arrêter leur carrosse devant le sien, lorsqu'il ne montre pas., par
ses rideaux tirés, qu'il veut être inconnu. Ce respect si grand et si
public est tel en France qu'il n'y est rendu par les ducs qu'au roi, à
la reine et aux fils et filles de France, bien loin de s'étendre jusqu'à
un particulier.

Une seconde différence, et qui est de tous les jours, et n'est pas moins
publique, est l'extrême différence du majordome-major du roi, et comme
tel de tous les grands, lui-même ne le fût-il pas, comme il est
quelquefois arrivé. Non seulement il les précède partout, sans être
jamais mêlé avec eux, mais il a un siège ployant de velours placé à la
chapelle, à la tête de leur banc, et ce siège si distingué d'eux y est
toujours, et il demeure vide, sans pouvoir être occupé, s'il ne l'est
pas par le majordome-major. Il est assis au bal et à la comédie sur ce
même siège, à la droite du roi et le joignant, presque sur la même
ligne, tandis que les grands sont debout\,; et lorsque le roi d'Espagne
reçoit des ambassadeurs sur un trône, comme des Africains et d'autres
nations éloignées, le majordome est assis en pareille place et sur
pareil siège sur le trône, tandis que les grands sont au bas du trône et
debout. Chez la reine, son majordome-major précède tous les grands sans
difficulté, en toutes les cérémonies et les audiences, et le grand
écuyer du roi ne leur donne pas la main dans le carrosse du roi qui est
à son usage. Toutes ces mortifications de charges, publiques et
continuelles, sont entièrement inconnues aux ducs. Bien plus, le
majordome-major du roi, comme tel, et sans être grand, je le répète,
comme il est arrivé quelquefois, jouit de tout le rang et honneurs des
grands\,; et, ce qui est étrange, c'est qu'il est leur chef, et
tellement leur chef, que s'il arrive quelque affaire qui intéresse la
dignité des grands, c'est chez le majordome-major qu'ils s'assemblent et
qu'ils délibèrent, et que c'est par lui que sont portés et présentés au
roi les raisons ou les mémoires qu'ils ont à lui faire entendre, et que
pareillement c'est par le même que le roi s'explique aux grands de ses
décisions ou de ses volontés. Il ne se trouve rien de semblable en
France. J'ai moi-même été témoin de tout cela en Espagne, et pour ce
dernier article, il se passa ainsi au baptême de l'infant don Philippe,
où j'étais, et où le roi voulut que les honneurs fussent portés par les
grands, quoiqu'ils ne l'eussent été jusqu'alors que par les
majordomes\,; les ordres, les remontrances, la décision, tout passa par
le majordome-major, et ce fut chez lui que les grands s'assemblèrent.

Quoique les grands ne cèdent point aux cardinaux, dont j'expliquerai en
son temps le divers rangs en Espagne, et qu'ils ne les voient point chez
eux en public, à cause de la main, les grands essuient néanmoins une
distinction étrange dont la France n'a jamais ouï parler\,: c'est leur
fauteuil à la chapelle, tandis qu'ils n'ont qu'un banc, couvert de
tapisserie, sans petit banc bas devant eux, et les cardinaux et les
ambassadeurs en ont un, celui de ces derniers couvert de tapisserie
comme leur banc, et le petit banc bas des cardinaux couverts de velours
rouge.

Au conseil, lorsque le roi s'y trouve, et qu'il y a des cardinaux\,; ils
y ont un fauteuil comme à la chapelle. Ils sont au-dessus des grands, et
les grands n'y ont que des sièges ployants.

Les grands et le majordome-major même sont nettement précédés par des
ambassadeurs de chapelle à la distribution des cierges à la Chandeleur,
en celle des cendres, et aux autres occasions où ils se trouvent
ensemble qui sont de cérémonie.

Toutes ces choses, la plupart si marquées, si distinctives, si
journalières, sont inconnues aux ducs, et avec raison leur paraîtraient
monstrueuses.

Les infants sont en Espagne comme sont ici les fils et filles de France.

De princes du sang, il n'y en a jamais eu tant que la maison d'Autriche
a régné en Espagne.

M. le duc d'Orléans, petit-fils de France, fut traité en Espagne comme
un infant\,; mais il alla chez toutes les femmes des grands, et traita
les grands comme il traitait ici les ducs.

Pour les bâtards des rois, on a vu ce qui a été dit des deux don Juan,
les deux seuls reconnus en Espagne, et les grands sont fort éloignés de
tout avantage de ce côté-là.

De tout cet extérieur si éblouissant des grands d'Espagne, que leurs
rois leur ont jalousement conservé au dehors et au dedans de l'Espagne,
à l'égard des princes étrangers, et que les ducs ont eu comme eux, il
n'y a de différence que la fermeté des rois d'Espagne par rapport à leur
propre dignité, d'avec l'entraînement des rois de France, dont on a vu,
par l'exemple de l'électeur de Bavière, que leur dignité même a
souffert. Il en est de même des honneurs civils et militaires conservés
jusques au milieu du dernier règne, de l'invitation aux fêtes et aux
cérémonies qui a été de tout temps, et jusqu'à nos jours, pour les ducs
en France, comme en Espagne pour les grands, et de ces distinctions que
je viens de raconter, communes en elles-mêmes aux deux dignités, mais
qui pour la plupart ont cessé au milieu du règne de Louis XIV. Ainsi dès
qu'elles ont été jusqu'alors, rien d'essentiellement distinctif à
l'avantage des grands sur les ducs, puisque la cessation à l'égard de
ces derniers est si moderne, et que, lorsqu'il plaira au roi de France
de penser que sa dignité y est intéressée, toute suréminente qu'elle
est, et de faire réflexion qu'il n'appartient qu'à son sang d'avoir chez
lui des rangs et des distinctions par naissance, inconnues chez toutes
les autres nations, ni à aucune dignité étrangère d'y jouir d'aucun
avantage plus grand que n'en ont celles qu'il donne, cet extérieur sera
bientôt rétabli, et porté au niveau pour le moins de celui qui éblouit
dans les grands d'Espagne, dont le seul avantage réel que n'ont pas les
ducs est celui dont jouissent leurs fils aînés et les femmes de ces
fils.

Pour les désavantages des grands par comparaison aux ducs, on ne compte
point le défaut d'habits particuliers et de marques de dignité aux
armes, quoique cet éclat en soit un fort marqué\,; ni le défaut de
housse, puisque la reine n'en porte point, ni de balustres, parce qu'on
ne voit point leurs lits ni leurs chambres à coucher\,; ni le mélange
dans les ordres, puisque les infants mêmes n'en étaient point exempts
avant Philippe V.

Mais les distinctions étranges du président et même du gouverneur du
conseil de Castille, le fauteuil des cardinaux, la préséance si marquée,
la supériorité aux audiences singulières, et journellement aux bals et
aux comédies, du majordome-major assis à côté du roi où tous les grands
sont debout, sa présidence sur eux par sa charge, même sans être grand,
pour tout ce qui concerne leur dignité, ce sont des choses, pour en
omettre diverses autres, d'un grand contrepoids, et qui toutes sont
parfaitement inconnues aux ducs, et qui ne peuvent pas contribuer à
faire trouver les ducs d'Arcos et de Baños bien fondés dans leurs
plaintes et leur mémoire.

Venons maintenant à ce qui les a le plus frappés et le plus déterminés à
cette démarche\,: c'est que les grands d'Espagne se couvrent devant
leurs rois, et que les ducs de France ne s'y couvrent point\,; que les
princes étrangers s'y couvrent aux audiences des ambassadeurs, et que
ceux de la maison de Lorraine, privativement à tous autres, les
conduisent à l'audience.

Il faut se souvenir de ce qui a été expliqué ci-dessus de l'ancien usage
d'être couvert en France devant le roi sans distinction de dignité, et
de la manière imperceptible dont il a changé par le changement de
coiffures, du chaperon au bonnet, puis à la toque, enfin aux chapeaux.
Lors même qu'on était couvert devant nos rois, nul ne leur parlait
couvert, non pas même les fils de France. Il n'est donc pas étrange que
les ducs n'aient point cet honneur, beaucoup moins depuis que l'usage
d'être couvert devant les rois de France s'est peu à peu aboli, même ne
leur parlant pas. Chaque pays a ses usages particuliers qui se trouvent
souvent la cause primitive et l'origine des distinctions. En France, ni
homme ni femme ne baise la reine\,; ce n'a été qu'au mariage du roi
d'aujourd'hui que cet honneur a été accordé aux princes du sang\,; mais
les duchesses et les princesses étrangères ont celui de s'asseoir devant
elle et les \emph{tabourets} de grâce, et pour les hommes, les fils et
petits-fils de France et les cardinaux, sans que les princes du sang qui
l'ont tenté au mariage du roi d'aujourd'hui y aient pu parvenir, et qui,
jusqu'à la mort du feu roi, ne l'ont jamais prétendu\,; sans qu'en nul
lieu que ce soit les dames assises se soient jamais tenues debout un
instant en leur présence, ce qui aurait été regardé comme un grand
manque de respect, parce qu'il n'y en peut avoir qu'un. Ainsi elles se
levaient lorsqu'un prince du sang arrivait où elles étaient assises, et
se rasseyaient sur-le-champ\,; ce qu'elles faisaient de même pour les
principaux seigneurs. En Angleterre toutes les duchesses baisent la
reine, et pas une n'est assise devant elle\,; tellement que, lorsque les
reines d'Angleterre, femmes de Charles Ier et de Jacques II, sont venues
achever leur vie en France, elles y eurent le choix d'y traiter les
Françaises assises à la manière Anglaise ou française, et elles
choisirent la dernière\,; il est donc vrai de dire que ces honneurs sont
suivant les pays. Aussi a-t-on vu cette multitude de ricos-hombres
cesser de se couvrir devant Philippe le Beau, père de Charles-Quint, par
flatterie pour lui et pour faire dépit à Ferdinand son beau-père, et
l'usage de recouvrir ne revenir que sous Charles-Quint, qui l'établit en
la forme qu'il est demeuré lors de l'abolition de la rico-hombrerie et
de l'établissement de la grandesse.

Il faut se souvenir encore plus de quelle façon s'est introduit l'usage
de se couvrir devant le roi en France. On le peut voir plus haut et y
remarquer que c'est celui des grands d'Espagne qui y donna lieu, par la
liberté qu'un ambassadeur d'Espagne, qui était grand, prit de se couvrir
voyant Henri IV couvert dans ses jardins de Monceaux, et du hasard qui
restreignit cet honneur aux princes du sang, aux princes étrangers et au
duc d'Épernon si éloigné de l'être, parce qu'Henri IV, piqué de voir cet
Espagnol se couvrir, commanda à l'instant de se couvrir à M. le Prince
et aux ducs d'Épernon et de Mayenne, qui par hasard se trouvèrent seuls
à cette promenade. De là, M. de Mayenne prétendit se couvrir aux
audiences où il conduisait les ambassadeurs, et l'obtint\,; les princes
de la maison de Lorraine, de Savoie, de Longueville et de Gonzague, qui
conduisaient aussi les ambassadeurs, se trouvèrent dans le même droit.
Dès qu'ils l'eurent obtenu, il s'étendit aisément à ceux de ces maisons
qui se trouvèrent à ces audiences sans avoir conduit les ambassadeurs,
puisqu'en les conduisant ils se couvraient avec eux\,; à plus forte
raison M. le Prince et les princes du sang, et en même temps M.
d'Épernon, par la bonne fortune de s'être trouvé à cette promenade, où
il se couvrit avec M. le Prince et M. de Mayenne, et comme M. d'Épernon,
ses enfants furent aussi couverts à ces audiences. Ce chapeau vient donc
d'Espagne, et s'est trouvé borné à ceux qu'Henri IV fit couvrir à cette
promenade, et d'eux à leur maison, et aux maisons qui avaient la
conduite des ambassadeurs. Ce n'est que le feu roi qui l'a étendu en
divers temps et à diverses reprises à trois branches de maisons de
gentilshommes, quoiqu'ils ne conduisent pas les ambassadeurs. Le
pourquoi et le comment nous jetterait ici dans une dissertation trop
longue. On en a pu voir ci-dessus quelque chose de MM. de Rohan et de M.
de Monaco\,; de ce dernier il n'a pas passé aux Matignon, qui en ont eu
Monaco avec l'héritière, et l'érection nouvelle du duché-pairie de
Valentinois.

Mais il ne faut pas oublier que cet honneur de se couvrir est
entièrement restreint aux audiences des ambassadeurs, et sans place
distinguée, et sans entrer dans le balustre avec les princes du sang et
l'ambassadeur\,; qu'il ne s'étend à pas une autre sorte d'audience ni de
cérémonies, comme à celle du doge de Gènes, qui se couvrit seul\,; à
l'hommage de MM. de Lorraine, aux audiences des souverains, etc., en
sorte que ce chapeau est uniquement restreint aux audiences des
ambassadeurs, où les cardinaux l'ont aussi obtenu, et ne l'ont nulle
part ailleurs, non plus que leur bonnet devant le roi.

Quel que soit cet honneur, il ne touche point aux ducs, puisqu'il ne
peut être pris en leur présence. Témoin cette audience si solennelle du
cardinal Chigi, légat \emph{a latere} du pape son oncle, pour la
satisfaction de la fameuse affaire des Corses de la garde du pape qui
avaient insulté le duc de Créqui, ambassadeur du roi à Rome. Les princes
du sang ne pouvaient être à cette audience, où le légat eut un fauteuil.
Les ducs s'y devaient trouver, et furent avertis de la part du roi, par
le grand maître des cérémonies, et à cause de leur présence, les princes
étrangers eurent défense de s'y couvrir. Les comtes d'Harcourt, grand
écuyer, et de Soissons, qui tous deux conduisaient le légat à
l'audience, n'oublièrent rien pour avoir permission de se couvrir ou de
n'assister pas à l'audience. Ils ne purent obtenir ni l'un ni l'autre,
et y demeurèrent tout du long et toujours découverts. On peut voir cela
plus au long, et le récit de l'erreur réformée d'une tapisserie (t. II,
p.~80, 81), ou plutôt du mensonge qui les y représente couverts. Il est
donc vrai que la présence nécessaire des ducs fait tomber ce chapeau.
Les deux seuls qui se trouvent aux audiences où on se couvre n'y sont
que par la nécessité de leur charge, l'un en qualité de premier
gentilhomme de la chambre, qui commande dans la chambre, et qui ne s'en
peut absenter alors comme tel\,; l'autre de capitaine des gardes en
quartier, et comme tel, en fonction nécessaire de sa charge, et
nullement comme ducs.

Après ces éclaircissements, ne pourrait-on point remarquer que ce grand
honneur de parler {[}couvert{]} au roi d'Espagne s'affaiblit étrangement
par les conditions qui y sont opposées\,? L'introduction de la nécessité
de faire la couverture, avec toute suspension de rang, honneurs et
distinctions jusqu'à ce qu'elle soit faite, et cependant le pouvoir et
l'usage des rois de la différer tant qu'il leur plaît, et même toujours,
est un grand contrepoids\,; celui d'avoir un certificat de sa couverture
du secrétaire de l'estampille, sous peine, si on le perd, d'avoir à
recommencer et de courir les risques des délais du roi, et en attendant
d'être suspendu de tout rang, honneurs et prérogatives, n'en est pas
moindre, et cela à toute mutation de père même à fils, et même pour la
première classe. En France, \emph{le mort saisit le vif}, sans que le
roi y intervienne\,; et à l'égard des pairs, dont la réception au
parlement de celui en faveur duquel l'érection fixe le rang d'ancienneté
pour lui et pour toute sa postérité, comme l'enregistrement le fixe pour
les ducs vérifiés qui ne sont pas pairs, les successeurs à la pairie ne
dépendent point de leur réception au parlement, ni d'aucune autre chose
pour jouir de tout leur rang, honneurs et prérogatives, soit qu'ils s'y
fassent recevoir tard ou point du tout, et ne préjudicie en aucune sorte
Je choses à leurs successeurs.

En voilà bien assez, ce me semble, pour entendre quelle est la dignité
des grands d'Espagne, {[}tant{]} dans son origine, son essence et son
fond, que dans son écorce et son extérieur\,; et le peu qui a été dit
sur les ducs de France, parce qu'il aurait fallu un volume pour entrer à
fond dans leur dignité, et que j'écris en France où on la doit
connaître, et où on en trouve force mémoires et traités, suffit, ce me
semble, pour montrer que les grands ne peuvent être comparés en rien aux
pairs, et que les ducs d'Arcos et de Baños ont ignoré la dignité des
ducs quand ils se sont plaints de la parité de rang et d'honneurs donnés
aux uns et aux autres dans les deux royaumes.

Mais après cet examen, il faut convenir aussi que l'abus qui s'en est
fait est extrêmement étrange. Lorsque le feu roi et le roi son
petit-fils sont convenus de cette parité, il est manifeste qu'ils n'ont
entendu qu'une fraternité des grands des deux royaumes pour cimenter
mieux celle des deux nations. Au lieu de s'en tenir à un règlement si
raisonnable et si commode pour les ducs et les grands qui vont en
Espagne ou viennent en France, on en a fait des grands d'Espagne
français et en France\,: d'abord une reconnaissance digne du roi
d'Espagne pour le duc de Beauvilliers son gouverneur\,; après, le crédit
des Noailles et du cardinal d'Estrées, aidé de l'amusement que prenait
le roi des enfances de la comtesse d'Estrées, dans la familiarité des
particuliers, des dames du palais, trouve le chausse-pied du passage du
roi d'Espagne de Barcelone en Italie sur une escadre commandée par le
comte d'Estrées pour le faire faire grand d'Espagne, sans qu'il y ait eu
soupçon seulement de la moindre opposition à ce passage. En France, il
ne faut que des exemples\,: sur ceux-là un voyage du comte de Tessé en
Espagne, où ses succès furent nuls à l'armée, avec le manège qui l'a si
bien servi dans les cours, lui procurèrent la grandesse. Je ne parle
point du duc de Berwick, qui, par la bataille d'Almanza, rétablit la
couronne sur la tête du roi d'Espagne\,: c'est en Espagne que les terres
de sa grandesse sont situées, et c'est en Espagne que les grands de sa
postérité se sont fixés. Trois ou quatre seigneurs flamands, grands
d'Espagne, dont les pères ni eux-mêmes n'étaient jamais sortis des
Pays-Bas ou d'Espagne, se viennent fixer à Paris, trouvent plus agréable
d'y jouir du premier rang de l'État et de s'y établir que de demeurer
chez eux. Le duc de Noailles, neveu de M\textsuperscript{me} de
Maintenon, va en Espagne et y est fait grand tout de suite, puis revient
disgracié des deux cours, et, longues années après, fait passer sa
grandesse à son second fils, à quoi d'abord il n'avait pas songé\,;
ainsi, en deux voyages courts, la Toison au premier, la grandesse en
l'autre. M. de Chalais, neveu du premier mari de M\textsuperscript{me}
des Ursins, sans aucun service en France, se dévoue à elle, et est
employé en d'étranges commissions, dont la grandesse est la récompense,
malgré le feu roi, qui, loin de lui permettre de l'accepter, s'en irrita
jusqu'à déclarer qu'il ne souffrirait jamais qu'il en eût le rang ni les
honneurs en France. Croirait-on, après ses aventures à l'égard de M. le
duc d'Orléans, et l'éclat entre ce prince et M\textsuperscript{me} des
Ursins, que ce fut ce prince qui, dans sa régence, lui permit de revenir
en France et d'y jouir du rang et des honneurs\,?

J'avoue que, voyant tant d'abus, je crus en pouvoir profiter comme les
autres, mais sans dissimuler à M. le duc d'Orléans combien je les
désapprouvais. J'ose dire que si, après les grandesses de MM. de
Beauvilliers et de Berwick, il y en a une pardonnable, c'est celle qui
me fut donnée à l'occasion de mon ambassade extraordinaire pour
demander, conclure et signer le mariage du roi avec l'infante.

De là M\textsuperscript{me} de Ventadour, qui fut sa gouvernante, obtint
une grandesse pour le comte de La Mothe, qu'on avait mis à même d'être
fait maréchal de France, et que son incapacité en repoussa toujours, qui
de sa vie n'avait servi l'Espagne, et qui était parfaitement éloigné de
devenir duc. Le mariage arrêté de l'infant avec une fille de M. le duc
d'Orléans fit le grand prieur de France, son bâtard reconnu, grand
d'Espagne. Cette élévation donna de l'émulation à l'électeur de Bavière
pour le sien, attaché au service de France. Il fit si bien valoir tout
ce que lui avait coûté son attachement au service des deux couronnes, et
l'honneur qu'il avait d'être frère de M\textsuperscript{me} la Dauphine,
mère du roi d'Espagne, que le comte de Bavière fut fait grand. Le
maréchal de Villars n'avait jamais servi le roi d'Espagne, ni approché
de ses frontières\,; la Toison ne laissa pas de lui être envoyée, à la
surprise du feu roi et de tout le monde. Pendant la régence, la
grandesse lui plut de même, sans qu'en France ni en Espagne on ait
jamais su pourquoi.

Enfin le marquis de Brancas, à qui un voyage en Espagne avait valu la
Toison, y retourna ambassadeur avec stipulation expresse à M. le
cardinal Fleury et à Chauvelin, lors garde des sceaux et adjoint au
principal ministère, de n'être point grand\,; mais y ayant trouvé sa
belle, il s'y fit faire grand malgré eux, et s'en tira après comme il
put, après avoir essuyé la plus triste disgrâce. Sur cet exemple, le
comte de La Marck, qui lui succéda, y a obtenu aussi la grandesse, et
toutes de première classe. On peut juger si d'autres n'y parviendront
pas. J'oublie M. de Nevers, dont le père était duc à brevet, et qui,
fort mal avec le roi, n'en put jamais obtenir la continuation. Il épousa
la fille unique de Spinola, qui avait acheté la grandesse, et qui,
heureusement pour lui, survécut un peu le feu roi qui s'était déclaré
qu'il ne le laisserait pas jouir du rang. Le régent fut plus indulgent à
la mort de Spinola, et tôt après fit duc et pair le même M. de Nevers
aux instances de la duchesse Sforze sa tante.

Indépendamment des grands d'Espagne qui sont ducs de France, cela fait
douze grands d'Espagne établis à Paris et à la cour, dont pas un n'eût
osé songer à être duc. Il est étrange qu'on parvienne ici au même rang
et aux mêmes avantages par une dignité émanée du roi d'Espagne, quand on
ne peut parvenir à celle que le roi donne, et qu'il souffre qu'un autre
monarque que lui crée, pour ainsi dire des ducs de ses sujets et dans
son royaume. S'il veut élever à la dignité de duc des sujets qui
méritent et qui lui plaisent, n'en est-il pas le maître mais ce qu'il ne
lui plaît pas de faire, il le voit opérer par le roi d'Espagne. Est-ce
là le réciproque du rang des grands des deux royaumes dont les deux rois
sont convenus\,? Cela se présente à l'esprit de soi-même. Le roi
d'Espagne, plus jaloux de ses bienfaits, et les Espagnols plus retenus,
n'ont point encore vu faire de ducs de France en Espagne. Les Espagnols
ont raison de sentir\,: cette inégalité et une profusion si
extraordinaire\,; elle n'est pas moins sentie en France, et si on prend
garde à la mécanique de l'opération, on la trouvera également incroyable
et monstrueuse.

Toutes ces grandesses françaises s'établirent comme les duchés, excepté
qu'en France l'érection précède le rang et les honneurs dont l'impétrant
ne jouit qu'ensuite et en conséquence, au lieu qu'en Espagne ils
précèdent l'érection\,; mais tout tomberait à l'impétrant même si
l'érection ne suivait pas, à moins que, comme la grandesse de
Bournonville, elle ne fût sur le nom même\,; ce qui est très rare en
Espagne, et n'existe en aucun grand français. L'érection faite et passée
au conseil de Castille, il faut des lettres patentes du roi enregistrées
au parlement et en la chambre des comptes, avec un nouvel hommage de
l'impétrant au roi, enfin faire enregistrer ces mêmes lettres patentes
au conseil de Castille\,; la contrariété de ces opérations est
inexplicable. Par l'érection, le roi d'Espagne exerce en France le plus
grand acte de souveraineté sur une terre de la souveraineté du roi, et
se fait un vassal du premier ordre, pour ne pas dire un sujet, d'un
sujet du roi\,; et à quel titre\,? d'une terre située en France, de la
mouvance directe ou indirecte de la couronne, puisque tout fief lui est
reporté, et d'une terre de sa pleine souveraineté, qui n'en est point
pour cela détachée\,; en sorte que le possesseur de cette terre,
primordialement sujet et vassal du roi son seigneur suzerain et
souverain, le devient, au même titre et par la même possession, d'un
autre monarque, dans le royaume duquel il ne vit point, et dans le
royaume duquel cette terre n'est pas située. C'est néanmoins sur cette
opération, à laquelle on ne peut donner de nom, qu'interviennent les
lettres patentes du roi pour l'approuver et la ratifier, qui pour la
France reçoivent leur dernière consommation de leur enregistrement au
parlement et en la chambre des comptes. Ce n'est pas tout, il faut
encore que cette approbation, cette permission du roi, cette
ratification du parlement et de la chambre des comptes, en un mot, que
ces lettres patentes enregistrées soient envoyées en Espagne, pour y
être à leur tour approuvées, ratifiées et enregistrées par le conseil de
Castille, qui, ayant fait la première opération par l'enregistrement de
l'érection, fait aussi la dernière par l'enregistrement de ces lettres
patentes, et de leur enregistrement en France.

Ainsi un grand d'Espagne français fait au roi un nouvel hommage d'une
terre érigée par un roi étranger en dignité étrangère, duquel, à ce
titre, il devient vassal immédiat, pour ne pas dire sujet, et se trouve
avoir deux rois et deux seigneurs suzerains et souverains pour la même
terre, il doit donc à l'un et à l'autre le service des armes. Que
deviendra-t-il donc si ces deux rois viennent à se faire la guerre,
comme il est déjà arrivé, et que deviendraient-ils encore si, à ce qu'à
Dieu ne plaise, le cas funeste des renonciations arrivait\,?

En voilà trop sur cette matière, mais qu'il était bon et curieux de
tirer une bonne fois de l'obscurité, de l'ignorance, et de montrer aux
Français qui admirent tout ce qui est étranger, qui s'en éblouissent, et
qui d'ailleurs se laissent aller au torrent de la plus fausse et de la
plus folle jalousie, ce que c'est en effet que la dignité des pairs de
France, des ducs vérifiés de France, et des trois classes des grands
d'Espagne par rapport de l'une à l'autre, ainsi que l'incroyable abus
des rangs étrangers en France, des grandesses qui s'y sont érigées, et
des Français habitant en France faits grands d'Espagne. J'ai regret à la
longueur de la digression, mais il n'était pas possible de la faire plus
courte sans omettre des parties essentielles des connaissances
nécessaires à y donner. Revenons maintenant d'où nous sommes partis.

\hypertarget{chapitre-xvii.}{%
\chapter{CHAPITRE XVII.}\label{chapitre-xvii.}}

1701

~

{\textsc{Mort du roi Jacques II d'Angleterre.}} {\textsc{- Le prince de
Galles, son fils, reconnu roi d'Angleterre par le roi, et par le roi
d'Espagne et le pape.}} {\textsc{- Visites sur la mort du roi Jacques
II.}} {\textsc{- Voyage de Fontainebleau.}} {\textsc{- Jacques III
reconnu par Philippe V\,; effet de ces reconnaissances\,: signature de
la grande alliance contre la France et l'Espagne.}} {\textsc{- Mouvement
à Naples.}} {\textsc{- Vice-rois changés.}} {\textsc{- Louville à
Fontainebleau pour le voyage du roi d'Espagne en Italie.}} {\textsc{-
Étrange emportement de M. le Duc contre son ami le comte de Fiesque.}}
{\textsc{- La Feuillade\,: son caractère\,; son mariage avec une fille
de Chamillart.}} {\textsc{- Fagon taillé.}} {\textsc{- Harcourt de
retour d'Espagne.}} {\textsc{- Méan doyen de Liège, son frère et leurs
papiers enlevés, et enfermés à Namur.}} {\textsc{- Mort de Bissy\,; sa
prophétie sur son fils depuis cardinal.}} {\textsc{- Mort de M. de
Montespan.}} {\textsc{- Hardiesse de son fils.}} {\textsc{- Duc de
Montfort capitaine des chevau-légers par la démission du duc de
Chevreuse.}}

~

Le voyage du roi d'Angleterre lui avait peu réussi, et il ne traîna
depuis qu'une vie languissante. Depuis la mi-août, elle s'affaiblit de
plus en plus, et, vers le 8 septembre, il tomba dans un état de
paralysie et d'autres maux à n'en laisser rien espérer. Le roi,
M\textsuperscript{me} de Maintenon, toutes les personnes royales le
visitèrent souvent. Il reçut les derniers sacrements avec une piété qui
répondit à l'édification de sa vie, et on n'attendait plus que sa mort à
tous les instants. Dans cette conjoncture, le roi prit une résolution
plus digne de la générosité de Louis XII et de François Ier que de sa
sagesse. Il alla de Marly, où il était, à Saint-Germain, le mardi 13
septembre. Le roi d'Angleterre était si mal que, lorsqu'on lui annonça
le roi, à peine ouvrit-il les yeux un moment. Le roi lui dit qu'il était
venu l'assurer qu'il pouvait mourir en repos sur le prince de Galles, et
qu'il le reconnaîtrait roi d'Angleterre, d'Écosse et d'Irlande. Le peu
d'Anglais qui se trouvèrent présents se jetèrent à ses genoux, mais le
roi d'Angleterre ne donna pas signe de vie. Aussitôt après, le roi passa
chez la reine d'Angleterre, à qui il donna la même assurance. Ils
envoyèrent chercher le prince de Galles, à qui ils le dirent. On peut
juger de la reconnaissance et des expressions de la mère et du fils.
Revenu à Marly, le roi déclara à toute la cour ce qu'il venait de faire.
Ce ne fut qu'applaudissements et que louanges.

Le champ en était beau, mais les réflexions ne furent pas moins
promptes, si elles furent moins publiques. Le roi espérait toujours que
sa conduite si mesurée en Flandre, le renvoi des garnisons hollandaises,
l'inaction de ses troupes, lorsqu'elles pouvaient tout envahir, et que
rien n'y était en état de s'opposer à elles, retiendraient la Hollande
et l'Angleterre, dont la première était si parfaitement dépendante, de
rompre en faveur de la maison d'Autriche. C'était alors pousser cette
espérance bien loin\,; mais le roi s'en flattait encore, et par là de
terminer bientôt la guerre d'Italie, et toute l'affaire de la succession
d'Espagne et de ses vastes dépendances, que l'empereur ne pouvait
disputer avec ses seules forces, et celles même de l'empire. Rien
n'était donc plus contradictoire à cette position, et à la
reconnaissance qu'il avait solennellement faite, à la paix de Ryswick,
du prince d'Orange comme roi d'Angleterre, et que jusqu'alors il n'avait
pas moins solennellement exécutée. C'était offenser sa personne par
l'endroit le plus sensible, et toute l'Angleterre avec lui, et la
Hollande à sa suite\,; c'était montrer le peu de fond qu'ils avaient à
faire sur ce traité de paix, leur donner beau jeu à rassembler avec eux
tous les princes qui y avaient contracté sous leur alliance, et de
rompre ouvertement sur leur propre fait, indépendamment de celui de la
maison d'Autriche. À l'égard du prince de Galles, cette reconnaissance
ne lui donnait rien de solide\,; elle réveillait seulement la jalousie,
les soupçons et la passion de tout ce qui lui était opposé en
Angleterre, les attachait de plus en plus au roi Guillaume, et à
l'établissement de la succession dans la ligne protestante, qui était
leur ouvrage\,; les rendait plus vigilants, plus actifs et plus violents
contre tout ce qui était catholique ou soupçonné de favoriser les
Stuarts en Angleterre, et les ulcérait de plus en plus contre ce jeune
prince et contre la France, qui leur voulait donner un roi, et décider
malgré eux de leur couronne, sans que le roi, qui marquait du moins ce
désir par cette reconnaissance, eût plus de moyen de rétablir le prince
de Galles qu'il n'en avait eu de rétablir le roi son père pendant une
longue guerre où il n'avait pas, comme alors, à disputer la succession
de la monarchie d'Espagne pour son petit-fils.

Le roi d'Angleterre, dans le peu d'intervalles qu'il eut, parut fort
sensible à ce que le roi venait de faire. Il lui avait fait promettre de
ne pas souffrir qu'il lui fût fait la moindre cérémonie après sa mort,
qui arriva sur les trois heures après-midi du 16 septembre de cette
année 1701.

M. le prince de Conti s'était tenu tous ces derniers jours à
Saint-Germain sans en partir, parce que la reine d'Angleterre et lui
étoient enfants des deux sœurs Martinozzi, desquelles la mère était sœur
du cardinal Mazarin. Le nonce du pape s'y était pareillement tenu, par
l'ordre anticipé duquel il reconnut et salua le prince de Galles comme
roi d'Angleterre. Le soir du même jour, la reine d'Angleterre s'en alla
aux Filles de Sainte-Marie de Chaillot, qu'elle aimait fort, et
lendemain samedi, sur les sept heures du soir, le corps du roi
d'Angleterre, fort légèrement accompagné, et suivi de quelques carrosses
remplis des principaux Anglais de Saint-Germain, fut conduit aux
Bénédictins anglais à Paris, rue Saint-Jacques, où il fut rois en dépôt
dans une chapelle comme le plus simple particulier, jusqu'aux temps,
apparemment du moins fort éloignés, qu'il puisse être transporté en
Angleterre\,; et son cœur aux Filles de Sainte-Marie de Chaillot.

Ce prince a été si connu dans le monde duc d'York et roi d'Angleterre,
que je me dispenserai d'en parler ici. Il s'était fort distingué par sa
valeur et par sa bonté, beaucoup plus par la magnanimité constante avec
laquelle il a supporté tous ses malheurs, enfin par une sainteté
éminente.

Le mardi 20 septembre, le roi alla à Saint-Germain, et fut reçu et
conduit par le nouveau roi d'Angleterre, comme il l'avait été par le roi
son père la première fois qu'ils se virent\,; il demeura peu chez lui,
et passa chez la reine d'Angleterre. Le roi son fils était en grand
manteau violet\,; pour elle, elle n'était point en mante, et ne voulut
point de cérémonie. Toute la maison royale et toutes les princesses du
sang vinrent en robe de chambre faire leur visite pendant que le roi y
était, qui y resta le dernier, et qui demeura toujours debout. Le
lendemain mercredi, le roi d'Angleterre, en grand manteau violet, vint
voir le roi à Versailles, qui le reçut et le conduisit, comme il avait
fait la première fois le roi son père, au haut du degré, comme lui-même
en avait été reçu et conduit. Il lui donna toujours la droite\,; ils
furent assis quelque temps dans des fauteuils. M\textsuperscript{me} la
duchesse de Bourgogne le reçut et le conduisit seulement à la porte de
sa chambre, comme elle en avait été reçue et conduite. Il ne vit ni
Monseigneur ni les princes ses fils, qui, dès le matin de ce même jour,
étaient allés à Fontainebleau. Au sortir de cette visite, le roi s'en
alla coucher à Sceaux avec M\textsuperscript{me} la duchesse de
Bourgogne, et de là à Fontainebleau. Incontinent après, le nouveau roi
d'Angleterre fut aussi reconnu par le roi d'Espagne.

Le comte de Manchester, ambassadeur d'Angleterre, ne parut plus à
Versailles depuis la reconnaissance du prince de Galles comme roi
d'Angleterre, et partit, sans prendre congé, quelques jours après
l'arrivée du roi à Fontainebleau. Le roi Guillaume reçut en sa maison de
Loo, en Hollande, la nouvelle de la mort du roi Jacques II et de cette
reconnaissance, pendant qu'il était à table avec quelques princes
d'Allemagne et quelques autres seigneurs\,; il ne proféra pas une seule
parole outre la nouvelle, mais il rougit, enfonça son chapeau et ne put
contenir son visage. Il envoya ordre à Londres d'en chasser Poussin
sur-le-champ, et de lui faire repasser la nier aussitôt après. Il
faisait les affaires du roi en absence d'ambassadeur et d'envoyé, et il
arriva incontinent après à Calais.

Cet éclat fut suivi de près de la signature de la grande alliance
offensive et défensive contre la France et l'Espagne, entre l'empereur,
l'empire, qui n'y avait nul intérêt, mais qui, sous la maison
d'Autriche, n'avait plus de liberté\,; l'Angleterre et la Hollande, dans
laquelle ensuite ils surent attirer d'autres puissances\,; ce qui
engagea le roi de faire une augmentation dans ses troupes.

En même temps le cardinal d'Estrées, qui n'avait plus rien à négocier à
Venise, ni avec les princes d'Italie, s'en retourna à Rome On venait
d'étouffer une révolte à Naples\,: Sassinet, neveu du baron de Lisola,
chargé des procurations de l'empereur, l'avait conduite. Il fait pris.
Le prince de Muccia et le duc de Telena en étaient les principaux chefs,
et se sauvèrent. Le prince de Montesarchio, à quatre-vingts ans, monta à
cheval au premier bruit avec le duc de Popoli, et, avec leurs amis,
dissipèrent la canaille qui s'était assemblée, par où la révolte devait
commencer. Cela contint ceux qui avaient à perdre, et tout fut étouffé
dans l'instant. Le duc de Gaëtano, qui en était, sortit de Rome dans le
carrosse de l'ambassadeur de l'empereur, quoique le pape le lui eût
défendu sous peine de cinquante mille écus d'amende. Le duc de
Medina-Celi, vice-roi, s'y conduisit très bien. Cependant le comte
d'Estrées, qui était à Cadix, eut ordre de mener son escadre à Naples,
où tout fut très promptement mis en sûreté. Le prince Eugène a voit
ordre d'y envoyer dix mille hommes si la révolte avait réussi\,; et pour
achever de suite, le duc de Medina-Celi fut rappelé en Espagne tout à la
fin de l'année, avec la présidence du conseil des Indes, riche et
important emploi. Le duc d'Escalona, plus ordinairement nommé marquis de
Villena, dont il a été parlé souvent à l'occasion du testament de
Charles II, et qui avait été vice-roi de Catalogne, où on l'a vu battu
par M. de Noailles, et après encore par M. de Vendôme, fut envoyé à
Naples vice-roi\,; et le cardinal del Giudice, frère du duc de
Giovenano, grand d'Espagne de troisième classe et conseiller d'État, eut
ordre à Rome d'aller par \emph{interim} vice-roi de Sicile, d'où le duc
de Veragua fut rappelé.

Tout à la fin du voyage de Fontainebleau, Louville y arriva de
Barcelone, où il avait laissé le roi et la reine d'Espagne avec la
princesse des Ursins, et Marsin, ambassadeur de France. Il venait en
apparence pour rendre compte au roi de ce qui s'était passé de plus
intérieur en Espagne pendant la longue et dangereuse maladie du duc
d'Harcourt, surtout du nouveau mariage de Leurs Majestés Catholiques\,;
mais le but effectif de son voyage était d'obtenir que le roi trouvât
bon que le roi son petit-fils passât à Naples sur l'escadre du comte
d'Estrées, qui allait revenir à Barcelone, et qu'au printemps il se naît
à la tête de l'armée des deux couronnes en Italie. Louville eut
plusieurs audiences du roi fort longues, seul avec lui dans son cabinet,
quelquefois chez M\textsuperscript{me} de Maintenon, en sa présence. M.
de Beauvilliers et Torcy l'entretinrent beaucoup, et Mgr le duc de
Bourgogne. Ce qu'il y avait de plus distingué à la cour s'empressa de le
voir. Je m'en saisis à mon tour, et satisfis avec lui ma curiosité à
fond. Je me chargeai de le ramener à Paris le jour que le roi partit,
mais avec une plaisante condition. Le roi d'Espagne l'avait expressément
chargé de faire le tour du canal. Pendant les cinq ou six jours qu'il
avait été à Fontainebleau, il n'en avait pas eu le temps, tellement que
le matin du lundi 14 novembre que nous partîmes, je le menai tête à tête
faire cette promenade. Au retour, nous primes M\textsuperscript{me} de
Saint-Simon et l'archevêque d'Arles, depuis cardinal de Mailly, et nous
nous en allâmes d'une traite à Paris en relais. Je fus ravi de la
promenade pour m'entretenir avec lui plus à mon aise de choses
particulières, et dans le chemin de Paris, je lui fis tant d'autres
questions qu'il arriva sans voix et ne pouvant plus parler.

J'ai ci-devant parlé de la déroute de La Touanne et de Saurion,
trésorier de l'extraordinaire des guerres, et que le roi fit face pour
eux afin de soutenir son crédit. En conséquence, il s'empara de leurs
biens. La Touanne avait à Saint-Maur la plus jolie maison du monde, dont
le jardin donnait dans ceux de la maison de Gourville, que Catherine de
Médicis avait faits, et bâti un beau château. Gourville l'avait donné à
M. le Prince, qui en avait fait présent à M. le Duc. Rien ne lui
convenait davantage que de joindre les jardins de La Touanne aux siens,
et d'avoir sa maison pour en faire à Saint-Maur une petite maison
particulière à ses plaisirs, et souvent une décharge au château quand il
y était avec M\textsuperscript{me} la Duchesse et bien du inonde. Il
l'eut donc pour peu de chose du roi pendant Fontainebleau. Peu après
qu'on en fut revenu, il y fut coucher avec cinq ou six de ses plus
familiers. Le comte de Fiesque en était un depuis fort longtemps. À
table, et avant qu'il pût y avoir de vin sur jeu, il s'éleva une dispute
sur un fait d'histoire entre M. le Duc et le comte de Fiesque. Celui-ci,
qui avait de l'esprit et de la lecture, soutint fortement son opinion,
M. le Duc la sienne, à qui peut-être, faute de meilleures raisons, le
toupet s'échauffa à un tel excès qu'il jeta une assiette à la tête du
comte de Fiesque, et le chassa de la table et du logis. Une scène si
subite et si étrange épouvanta les conviés. Le comte de Fiesque, qui
était veau là pour y coucher, ainsi que les autres, et qui n'avait point
gardé de voiture, alla demander le couvert au curé, et regagna Paris le
lendemain aussi matin qu'il put. On se figure aisément que le reste du
souper et du soir furent fort tristes. M. le Duc, toujours furieux, et
peut-être contre soi-même sans le dire, ne put être induit à chercher à
la chaude à replâtrer l'affront. Il fit grand bruit dans le monde, et
les choses en demeurèrent là plusieurs mois. À la fin, les amis de l'un
et de l'autre s'en mêlèrent. M. le Duc, revenu tout à fait à soi, ne
demanda pas mieux que de faire toutes les avances du raccommodement. Le
comte de Fiesque eut la misère de les recevoir, ils se raccommodèrent,
et ce qu'il y eut de plus merveilleux, c'est qu'ils vécurent tous deux
ensemble depuis comme s'il ne se fût rien passé entre eux.

Le duc de La Feuillade n'avait pu faire revenir le roi sur son compte.
On a vu ci-devant le vol qu'il fit à son oncle\,; la colère où le roi en
fut, qui l'aurait cassé sans Pontchartrain, qui par honneur mit tout son
crédit à l'empêcher. Ses débauches de toutes les sortes, son extrême
négligence pour le service, son très mauvais et très vilain régiment,
son arrivée tous les ans très tard à l'armée, qu'il quittait avant
personne, tout cela le tenait dans une manière de disgrâce très marquée.
Il était parfaitement bien fait, avait un air et les manières fort
nobles, et une physionomie si spirituelle qu'elle réparait sa laideur et
le jaune et les bourgeons dégoûtants de son visage. Elle tenait
parole\,; il avait beaucoup d'esprit et de toutes sortes d'esprit. Il
savait persuader son mérite à qui se contentait de la superficie, et
surtout avait le langage et le manège d'enchanter les femmes. Son
commerce, à qui ne voulait que s'amuser, était charmant\,; il était
magnifique en tout, libéral, poli, fort brave et fort galant, gros et
beau joueur. Il se piquait fort de toute ses qualités, fort avantageux,
fort hardi, grand débiteur de maximes et de morales, et disputait
volontiers pour faire parade d'esprit. Son ambition était sans bornes,
et comme il était sans suite pour rien comme il l'était pour tout, cette
passion et celle du plaisir prenaient le dessus tour à tour. Il
recherchait fort la réputation et l'estime, et il avait l'art de
courtiser utilement lés personnes des deux sexes de l'approbation
desquelles il pouvait le plus espérer, et par cet applaudissement qui en
entraînait d'autres de se faire compter dans le grand monde. Il
paraissait vouloir avoir des amis, et il en trompa longtemps. C'était un
cireur corrompu à fond, une âme de boue, un impie de bel air et de
profession\,; pour tout dire, le plus solidement ruai honnête homme qui
ait paru de longtemps.

Il était veuf sans enfants de la fille de Châteauneuf et sœur de La
Vrillière, secrétaire d'État, avec qui il avait très mal vécu sans
aucune cause, et avec un parfait mépris. Ne sachant où se reprendre dans
un accès d'ambition, il imagina que Chamillart serait en état de tout
faire pour lui en épousant sa seconde fille, Dreux, mari de l'aînée, ne
pouvant par le peu qu'il était lui faire ombrage. Il le fit proposer à
ce ministre, qui s'en trouva d'autant plus flatté que sa fille était
cruellement vilaine. Chamillart en parla au roi, qui l'arrêta tout
court. «\,Vous ne connaissez pas La Feuillade, lui dit-il\,; il ne veut
votre fille que pour vous tourmenter pour que vous me tourmentiez pour
lui\,; or, je vous déclare que jamais je ne ferai rien pour lui, et vous
me ferez plaisir de n'y plus penser.\,» Chamillart se tut tout court, et
demeura fort affligé. La Feuillade ne se rebuta point\,: plus il se vit
sans ressource, plus il sentit que ce mariage seul lui en serait une
unique, et plus il lit presser Chamillart. On ne comprend pas aisément
comment, après un tel refus, il osa quelque temps après retourner à la
charge, et beaucoup moins comment le roi se rendit à ses instances, à
qui l'a connu. Il donna cieux cent mille livres à Chamillart, comme il
faisait à ses ministres, pour ce mariage. Chamillart y en ajouta cent
{[}mille{]} du sien, et le mariage fut conclu. La Feuillade fut mal reçu
du roi, lorsque, la permission accordée à Chamillart, il lui en parla.
Les noces se firent. La Feuillade vécut encore plus mal, s'il est
possible, avec cette seconde femme qu'avec la première, et dès les
commencements\,; mais il avait jeté un charme sur Chamillart à qui il
manqua étrangement quand il ne lui fut plus nécessaire, et qui n'en
demeura pas moins constamment affolé de lui tant qu'il vécut. On verra
dans la suite combien ce mariage a coûté cher à la France.

Fagon, premier médecin du roi, fut taillé par Maréchal, chirurgien
célèbre de Paris, qu'il préféra à tous ceux de la cour et d'ailleurs.
Fagon, asthmatique, très bossu, très décharné, très délicat, et sujet
aux, atteintes du haut mal, était un méchant sujet en termes de
chirurgie\,; néanmoins il guérit par sa tranquillité et l'habileté de
Maréchal, qui lui tira une fort grosse pierre. Cette opération le fit
quelque temps après premier chirurgien du roi. Sa Majesté marqua une
grande inquiétude de Fagon, en qui pour sa santé il avait mis toute sa
confiance. Il lui donna cent mille francs à cette occasion. On a pu voir
quel était Fagon (tome Ier, page 110), tout au commencement de ces
Mémoires.

Le duc d'Harcourt arriva d'Espagne et entretint longtemps le roi et
M\textsuperscript{me} de Maintenon, et dès lors commença à prendre un
grand vol, mais il lui fallait peut-être plus de santé et sûrement plus
de mesure.

Le comte de Montrevel, qui, à la prière de l'électeur de Cologne, évêque
de Liège, s'était saisi de la citadelle de Liège, et avait prévenu de
fort peu les Hollandais, fit par ordre du roi et du même électeur
enlever le baron de Mean, doyen du chapitre de Liège, et son frère avec
tous leurs papiers, et les fit conduire dans Je château de Namur.
C'étaient deux hommes d'une grande ambition, surtout le doyen qui avait
beaucoup d'esprit et de hardiesse, et qui excellait en projets, en
menées et en intrigues. Ils étaient fort attachés au roi Guillaume qui
s'en servait beaucoup, et en dernier lieu il avait voulu débaucher le
gouverneur d'Huy avec sa place, et fait le projet de l'occupation de
Liège par les Hollandais. Ce fut un grand cri de tous les alliés contre
la France, outrés de se voir privés de deux instruments si utiles, et
encore plus de ce qu'on verrait de leurs desseins par leurs papiers. On
n'en était plus aux mesures, on laissa crier, et on resserra bien les
deux prisonniers.

Le vieux Bissy, ancien lieutenant général et commandant depuis longtemps
en chef en Lorraine et dans les Trois-Évêchés, mourut à Metz fort
regretté par son équité, sa discipline et la netteté de ses mains. Ce
fut un de ces militaires de bas aloi que M. de Louvois fit chevalier de
l'ordre à la fin de 1688. Il s'appelait Thiard, d'une famille qui a
donné des conseillers et des présidents aux parlements de Dijon et de
Besançon, et un évêque de Chalon-sur-Saône, grand poète, ami de Ronsard,
de Desportes, du cardinal du Perron, et savant d'ailleurs, qui mourut
tout au commencement du dernier siècle. Bissy, par ce commandement de
Lorraine, trouva à marier son fils aîné à une Haraucourt, qui longues
années après devint héritière par la mort de ses frères sans enfants. Il
était aussi père de l'abbé de Bissy, à qui il procura l'évêché de Toul,
et qui depuis est devenu cardinal et a fait un étrange bruit dans le
monde. Étant allé tout jeune homme et presque du collage voir son père à
Nancy, ce fut à qui le louerait le plus. Le père qui était galant homme,
bon citoyen et vrai, s'en impatienta. «\,Vous ne le connaissez pas, leur
dit-il\,; voyez-vous bien ce petit prestolet-là qui ne semble pas savoir
l'eau troubler, c'est une ambition effrénée qui sera capable, s'il peut,
de mettre l'Église et l'État en combustion pour faire fortune.\,» Ce
vieux Bissy n'a été que trop bon prophète. Il y aura lieu de parler plus
d'une fois de ce prestolet qui en conserva l'air toute sa vie.

M. de Montespan mourut clans ses terres de Guyenne, trop connu par la
funeste beauté de sa femme, et par ses nombreux et plus funestes fruits.
Il n'en avait eu qu'un fils unique avant l'amour du roi, qui était le
marquis d'Antin, menin de Monseigneur, lequel sut tirer un grand parti
de la honte de sa maison. Dès que son père fut mort, il écrivit au roi
pour lui demander de faire examiner ses prétentions à la dignité de duc
d'Épernon. Tous les enfants de sa mère en supplièrent le roi après son
souper, ou de le faire duc, M. le duc d'Orléans portant la parfile.
Cette folie d'Épernon fut en effet son chausse-pied, mais les moments
n'en étaient pas venus, un obstacle invincible l'arrêtait encore
M\textsuperscript{me} de Montespan vivait, et M\textsuperscript{me} de
Maintenon la haïssait trop pour lui donner le plaisir de voir
l'élévation de son fils.

Malgré elle, M. de Chevreuse fut plus heureux, par la permission qu'il
obtint de donner sa charge de capitaine des chevau-légers de la garde au
duc de Montfort son fils. Elle ne put jamais revenir de l'affaire de M.
de Cambrai à l'égard de ses anciens et persévérants amis qui l'avaient
tant été d'elle-même\,; elle haïssait surtout le duc de Chevreuse et la
duchesse de Beauvilliers. M. de Beauvilliers, elle le supportait
davantage quoiqu'elle ne l'aimât guère mieux\,; M\textsuperscript{me} de
Chevreuse était la moins dans sa disgrâce\,: mais le roi était si
parfaitement revenu pour tous les quatre, que M\textsuperscript{me} de
Maintenon ne put jamais leur donner d'atteinte. Ainsi finit cette année
et tout le bonheur du roi avec elle.

\hypertarget{chapitre-xviii.}{%
\chapter{CHAPITRE XVIII.}\label{chapitre-xviii.}}

1702

~

{\textsc{Année 1702.}} {\textsc{- Bals à la cour et comédies chez
M\textsuperscript{me} de Maintenon et chez la princesse de Conti.}}
{\textsc{- Longepierre.}} {\textsc{- Mort de la duchesse de Sully.}}
{\textsc{- Mort étrange de Lopineau.}} {\textsc{- Mort et aventures de
l'abbé de Vatteville.}} {\textsc{- Mariage de Villars et de
M\textsuperscript{lle} de Varangeville.}} {\textsc{- Délibération sur le
voyage de Philippe V en Italie.}} {\textsc{- Brillante situation
d'Harcourt qui lui fait espérer d'être ministre.}} {\textsc{- Position
brillante d'Harcourt en Espagne.}} {\textsc{- Son embarras entre les
deux.}} {\textsc{- Caractère d'Harcourt.}} {\textsc{- Conférence très
singulière.}} {\textsc{- Raison pour et contre le voyage.}} {\textsc{-
Harcourt arrête la promotion des maréchaux de France.}} {\textsc{- Son
imprudence.}} {\textsc{- Il se perd auprès du roi d'Espagne et se ferme
après le conseil.}} {\textsc{- M\textsuperscript{me} la duchesse de
Bourgogne et Tessé.}} {\textsc{- Le voyage résolu et Louville dépêché au
roi d'Espagne.}}

~

L'année commença par des bals à Versailles\,; il y en eut quantité en
masques. M\textsuperscript{me} du Maine en donna plusieurs dans sa
chambre toujours gardant son lit, parce qu'elle était grosse, ce qui
faisait un spectacle assez singulier. Il y en eut aussi à Marly, mais la
plupart de ceux-là sans mascarades. M\textsuperscript{me} la duchesse de
Bourgogne s'amusa fort à tous. Le roi vit en grand particulier, mais
souvent et toujours chez M\textsuperscript{me} de Maintenon, des pièces
saintes, comme \emph{Absalon, Athalie}, etc.\,; M\textsuperscript{me} la
duchesse de Bourgogne, M. le duc d'Orléans, le comte et la comtesse
d'Ayen, le jeune comte de Noailles, M\textsuperscript{lle} de Melun,
poussée par les Noailles, y faisaient les principaux personnages en
habits de comédiens fort magnifiques. Le vieux baron excellent acteur,
les instruisait et jouait avec eux, et quelques domestiques de M. de
Noailles. Lui et son habile femme étaient les inventeurs et les
promoteurs de ces, plaisirs intérieurs pour s'introduire de plus en plus
dans la familiarité du roi, à l'appui de l'alliance de
M\textsuperscript{me} de Maintenon. Il n'y avait de place que pour
quarante spectateurs. Monseigneur et les deux princes ses fils,
M\textsuperscript{me} la princesse de Conti, M. du Maine, les dames du
palais, M\textsuperscript{me} de Noailles et ses filles y furent les
seuls admis. Il n'y eut que deux ou trois courtisans en charge et en
familiarité, et pas toujours. Madame y fut admise avec son grand habit
de deuil\,: le roi l'y convia, parce qu'elle aimait fort la comédie, et
lui dit qu'étant de sa famille si proche, son état ne la devait pas
exclure de ce qui se faisait en sa présence dans un si grand
particulier. Cette faveur fut fort prisée. M\textsuperscript{me} de
Maintenon voulut lui marquer qu'elle avait oublié le passé.

Longepierre, celui même qui avait été chassé de chez M. du Maine pour
avoir entêté M. le comte de Toulouse d'épouser M\textsuperscript{lle}
d'Armagnac, dont la mère et la fille furent longtemps exclues de tout,
et ne se seraient pas sauvées de la plus profonde disgrâce sans l'amitié
du roi pour M. le Grand, Longepierre, dis-je, était enfin revenu,
s'était accroché aux Noailles, et avait fait une pièce fort singulière
sous le titre d'\emph{Électre} qui fut jouée sur un magnifique théâtre
chez lime la princesse de Conti à la ville avec le plus grand succès.
Monseigneur et toute la cour qui s'y empressa, la vit plusieurs fois.
Cette pièce était sans amour, mais pleine des autres passions et des
situations les plus intéressantes. Je pense qu'elle avait été faite
ainsi dans l'espérance de la faire voir au roi, mais il se contenta d'en
entendre parler, et les représentations en furent bornées à l'hôtel de
Conti. Longepierre ne la voulut pas donner ailleurs. C'était un drôle,
intrigant de beaucoup d'esprit, doux, insinuant, et qui, sous une
tranquillité, une indifférence et une philosophie fort trompeuse, se
fourrait et se mêlait de tout ce qu'il pouvait pour faire fortune. Il
fit si bien qu'il entra chez M. le duc d'Orléans où nous le
retrouverons, et où, avec tout son art et son savoir-faire, il montra
vilainement la corde et se lit honteusement chasser. D'ailleurs il
savait entre autres {[}choses{]} force grec, dont il avait aussi toutes
les mœurs.

La mort de la duchesse de Sully priva les bals du meilleur et du plus
noble danseur de son temps, le chevalier de Sully, son second fils, et
que le roi faisait danser, quoique d'âge à y avoir renoncé. Sa mère
était fille de Servien, surintendant des finances, à qui était Meudon où
il avait tant dépensé. Elle était pauvre, quoiqu'elle eût eu huit cent
mille livres, et que par l'événement elle fût devenue héritière. Mais
Sablé, son frère, s'était ruiné dans la plus vilaine crapule et la plus
obscure, quoique fort bien fait et avec beaucoup d'esprit, et l'abbé
Servien, son autre frère, qui n'en avait pas moins, et avait été
camérier du pape\,; ne fut connu que par ses débauches, et le goût
italien qui lui attira force disgrâces. Ainsi périssent en bref, et
souvent avec honte, les familles de ces ministres si puissants et si
riches, qui semblent dans leur fortune les établir pour l'éternité.

Lopineau, commis de Chamillart pour dresser les arrêts de finance, était
perdu depuis trois mois. C'était un homme doux et poli, bien que commis
principal, et homme à mains nettes, quoique de tout temps employé aux
finances. Il était aimé et estimé de tout le monde, et n'était point
marié. Étant à Paris, et sorti une après-dînée seul à pied, il ne revint
plus, et son corps fut enfin trouvé près du pont de Neuilly dans la
rivière. Ce pauvre homme apparemment fut pris par des scélérats pour le
rançonner et détenu longtemps, puis assassiné et jeté dans la rivière,
sans que, quelque soin qu'on ait pris de le chercher puis de faire
toutes les perquisitions possibles de ce crime, on en ait pu rien
apprendre.

La mort de l'abbé de Vatteville fit moins de bruit, mais le prodige de
sa vie mérite de n'être pas omis. Il était frère du baron de Vatteville,
ambassadeur d'Espagne en Angleterre, qui fit à Londres, le 10 octobre
1661, une espèce d'affront au comte, depuis maréchal d'Estrades,
ambassadeur de France, pour la préséance, dont les suites furent si
grandes, et qui finirent par la déclaration que fit au roi le comte de
Fuentès, ambassadeur extraordinaire d'Espagne, envoyé exprès, que les
ambassadeurs d'Espagne, en quelque cour que ce fût n'entreraient jamais
en concurrence avec les ambassadeurs de France. Cela se passa le 24 mars
1662, en présence de toute la cour et de vingt-sept ministres étrangers,
dont on tira acte.

Ces Vatteville sont des gens de qualité de Franche-Comté. Ce cadet-ci se
fit chartreux de borane heure, et après sa profession fut ordonné
prêtre. Il avait beaucoup d'esprit, mais un esprit libre, impétueux, qui
s'impatienta bientôt du joug qu'il avait pris. Incapable de demeurer
plus longtemps soumis à de si gênantes observances, il songea à s'en
affranchir. Il trouva moyen d'avoir des habits séculiers, de l'argent,
des pistolets, et un cheval à peu de distance. Tout cela peut-être
n'avait pu se pratiquer sans donner quelque soupçon. Son prieur en eut,
et avec un passe-partout va ouvrir sa cellule, et le trouve en habit
séculier sur une échelle, qui allait sauter les murs. Voilà le prieur à
crier l'autre, sans s'émouvoir, le tue d'un coup de pistolet, et se
sauve. À deux ou trois journées de là, il s'arrête pour dîner à un
méchant cabaret seul dans la campagne, parce qu'il évitait tant qu'il
pouvait de s'arrêter dans des lieux habités, met pied à terre, demande
ce qu'il y a au logis. L'hôte lui répond\,: «\,Un gigot et un chapon.
--- Bon, répond mon défroqué, mettez-les à la broche.\,» L'hôte veut lui
remontrer que c'est trop des deux pour lui seul, et qu'il n'a que cela
pour tout chez lui. Le moine se fâche et dit qu'en payant c'est bien le
moins d'avoir ce qu'on veut, et qu'il a assez bon appétit pour tout
manger. L'hôte n'ose répliquer et embroche. Comme ce rôti s'en allait
cuit, arrive un autre homme à cheval, seul aussi, pour dîner dans ce
cabaret. Il en demande, il trouve qu'il n'y a quoi que ce soit que ce
qu'il voit prêt à être tiré de la broche. Il demande combien ils sont
là-dessus, et se trouve bien étonné que ce soit pour un seul homme. Il
propose en payant d'en manger sa part, et est encore plus surpris de la
réponse de l'hôte, qui l'assure qu'il en doute à l'air de celui qui a
commandé le dîner. Là-dessus le voyageur monte, parle civilement à
Vatteville, et le prie de trouver bon que, puisqu'il n'y a rien dans le
logis que ce qu'il a retenu, il puisse, en payant, dîner avec lui.
Vatteville n'y veut pas consentir\,; dispute\,; elle s'échauffe\,; bref,
le moine en use comme avec son prieur, et tue son homme d'un coup de
pistolet. Il descend après tranquillement, et au milieu de l'effroi de
l'hôte et de l'hôtellerie, se fait servir le gigot et le chapon, les
mange l'un et l'autre jusqu'aux os, paye, remonte à cheval et tire pays.

Ne sachant que devenir, il s'en va en Turquie, et pour le faire court se
fait circoncire, prend le turban, s'engage dans la milice. Son reniement
l'avance, son esprit et sa valeur le distinguent, il devient hacha, et
l'homme de confiance en Morée, où les Turcs faisaient la guerre aux
Vénitiens. Il leur prit des places, et se conduisit si bien avec les
Turcs, qu'il se crut en état de tirer parti de sa situation, dans
laquelle il ne pouvait se trouver à son aise. Il eut des moyens de faire
parler au généralissime de la république, et de faire son marché avec
lui. Il promit verbalement de livrer plusieurs places et force secrets
des Turcs, moyennant qu'on lui rapportât, en toutes les meilleures
formes, l'absolution du pape de tous les méfaits de sa vie, de ses
meurtres, de son apostasie, sûreté entière contre les chartreux, et de
ne pouvoir être remis dans aucun autre ordre, restitué plénièrement au
siècle avec les droits de ceux qui n'en sont jamais sortis, et
pleinement à l'exercice de son ordre de prêtrise, et pouvoir de posséder
tous bénéfices quelconques. Les Vénitiens y trouvèrent trop bien leur
compte pour s'y épargner, et le pape crut l'intérêt de l'Église assez
grand à favoriser les chrétiens contre les Turcs\,; il accorda de bonne
grâce toutes les demandes du bacha. Quand il fut bien assuré que toutes
les expéditions en étaient arrivées au généralissime en la meilleure
forme, il prit si bien ses mesures qu'il exécuta parfaitement tout ce à
quoi il s'était engagé envers les Vénitiens. Aussitôt après, il se jeta
dans leur armée, puis sur un de leurs vaisseaux qui le porta en Italie.
Il fut à Rome, le pape le reçut bien\,; et pleinement assuré, il s'en
revint en Franche-Comté dans sa famille, et se plaisait à morguer les
chartreux.

Des événements si singuliers le firent connaître à la première conquête
de la Franche-Comté. On le jugea homme de main et d'intrigue\,; il en
lia directement avec la reine mère, puis avec les ministres, qui s'en
servirent utilement à la seconde conquête de cette même province. Il y
servit fort utilement, mais ce ne fut pas pour rien. Il avait stipulé
l'archevêché de Besançon, et en effet, après la seconde conquête, il y
fut nommé. Le pape ne put se résoudre à lui donner des bulles, il se
récria au meurtre, à l'apostasie, à la circoncision. Le roi entra dans
les raisons du pape, et il capitula avec l'abbé de Vatteville, qui se
contenta de l'abbaye de Baume, la deuxième, de Franche-Comté, d'une
autre bonne en Picardie, et de divers autres avantages. Il vécut depuis
dans son abbaye de Baume, partie dans ses terres, quelquefois à
Besançon, rarement à Paris et à la cour\,; où il était toujours reçu
avec distinction.

Il avait partout beaucoup d'équipage, grande chère, une belle meute,
grande table et bonne compagnie. Il ne se contraignait point sur les
demoiselles, et vivait non seulement en grand seigneur et fort craint et
respecté, mais à l'ancienne mode, tyrannisant fort ses terres, celles de
ses abbayes, et quelquefois ses voisins, surtout chez lui très absolu.
Les intendants pliaient les épaules\,; et, par ordre exprès de la cour,
tant qu'il vécut, le laissaient faire et n'osaient le choquer en rien,
ni sur les impositions, qu'il réglait à peu près comme bon lui semblait
dans toutes ses dépendances, ni sur ses entreprises, assez souvent
violentes. Avec ces mœurs et ce maintien qui se faisait craindre et
respecter, il se plaisait à aller quelquefois voir les chartreux, pour
se gaudir d'avoir quitté leur froc. Il jouait fort bien à l'hombre, et y
gagnait si souvent codille\footnote{\emph{Gagner codille}, locution du
  jeu d'hombre, signifiait gagner sans avoir fait jouer.}, que le nom
d'\emph{abbé Codille} lui en resta. Il vécut de la sorte, et toujours
dans la même licence et dans la même considération, jusqu'à près de
quatre-vingt-dix ans. Le petit-fils de son frère a, longues années
depuis, épousé une sœur de M. de Maurepas, du second lit.

Villars, aux portes de la fortune, fit un riche mariage. Il épousa
M\textsuperscript{lle} de Varangeville, belle et de fort grand air, sœur
cadette de la femme de Maisons président à mortier, fort belle aussi,
mais moins agréable. Elles n'étaient qu'elles deux, sans frère\,; et par
l'événement M\textsuperscript{me} de Villars a tout eu, le fils unique
de M\textsuperscript{me} de Maisons étant mort fort jeune, et son fils
unique très promptement après lui encore en enfance, tellement que cela
a joint des biens immenses à ceux, que Villars avait amassés.
Varangeville s'appelait Rocq, était de Normandie, et moins que rien.
Courtin, doyen du conseil, si bien avec le roi, si connu par ses
ambassades, duquel on a souvent parlé ici, n'avait qu'un fils abbé, qui
prit le petit collet par paresse et par débauche, avec lequel il est
mort, et deux filles. Le président de Rochefort, du parlement de
Bretagne, en épousa une\,; Varangeville obtint l'autre par ses
richesses, belli, et vertueuse, avec de l'esprit et de la conduite, qui
demeura toujours avec son père veuf, dont elle gouvernait la maison, et
par lui se mit très agréablement dans le monde.

L'affaire du jour était alors la résolution à prendre sur le voyage du
roi d'Espagne en Italie. Mais comme le mérite des affaires n'est pas,
toujours ce qui en forme la décision, l'intrigue avec laquelle celle-ci
fut contredite et soutenue mérite bien quelque détail. Louville, plus
instruit que personne des affaires d'Espagne par la confiance des deux
cours, et par l'influence que lui donnait sur toutes la faveur et la
confiance entière du roi d'Espagne, était celui qui avait imaginé ce
voyage d'Italie, qui l'avait fait goûter à M. de Beauvilliers et à
Torcy, et qui, une fois assuré de leur approbation, l'avait mis en tête
au roi d'Espagne dès avant son départ de Madrid. Louville était plein
d'esprit et de sens, ardent, mais droit, et persuadé une fois, rien ne
le faisait démordre et aussi peu s'arrêter. L'engouement où la vivacité
et l'abondance des pensées et des raisons le jetaient quelquefois,
exposait ce feu à des indiscrétions. Il en commit en rendant compte au
roi des affaires d'Espagne, et du désir et des raisons du roi d'Espagne
pour aller en Italie\,; il s'échappa sur l'état de l'Espagne, sur les
Espagnols et sur quelques personnages considérables. Chargé de rendre
compte du mariage du roi d'Espagne, il ne put taire ce qui s'y était
passé, de l'incartade des dames espagnoles au souper du jour des noces,
des pleurs et de l'enfance de la reine, qui cette nuit-là ne voulut
jamais coucher avec le roi et ne parlait que de s'en retourner en
Piémont, enfin de tout ce que j'ai raconté sur ces noces. Outre qu'il
devait ce compte au roi, inutilement lui aurait-il voulu cacher une
aventure si publique au souper, et le reste connu de tout l'intérieur du
palais, en particulier de M\textsuperscript{me} des Ursins et de Marsin,
qui n'auraient osé n'en pas écrire. Mais Louville parlait au roi en
présence de M\textsuperscript{me} de Maintenon, qui de plus savait par
le roi ce qu'il apprenait de Louville clans son cabinet tête à tête.

Louville était créature du duc de Beauvilliers, ami intime de Torcy et
très bien avec le duc de Chevreuse, et il se donnait pour tel, dans le
compte qu'il rendait et les questions que le roi lui lit entre quantité
d'affaires, de choses et de détails particuliers, inconnus la plupart,
les autres {[}connus{]} seulement par leur superficie au duc d'Harcourt,
qui sitôt après l'arrivée à Madrid, et si longtemps, avait été à la mort
et fort longtemps après encore à se remettre à la Sarçuela, éloigné du
bruit de la cour et de L'embarras des affaires. Tout cela aliéna
M\textsuperscript{me} la duchesse de Bourgogne, qu'on entêta que
Louville avait rendu de mauvais services à la reine sa sœur. Plusieurs,
de ses dames, ennemies de M. de Beauvilliers, par des intrigues de cour
ou pour plaire à M\textsuperscript{me} de Maintenon, firent et
excitèrent encore plus de bruit contre Louville, et tous les amis de M.
d'Harcourt firent chorus.

On a vu en son lieu la haine de M\textsuperscript{me} de Maintenon pour
les ducs de Chevreuse et de Beauvilliers d'autant plus grande que, sur
le point de les chasser, elle s'était trouvée impuissante, et ces deux
seigneurs, peu à peu revenus eux et leurs femmes mieux et plus
familièrement que jamais auprès du roi. On a vu encore l'affection que
M\textsuperscript{me} de Maintenon portait à M. d'Harcourt, et combien
elle l'avait servi\,; et on en a vu aussi l'impure mais puissante
source, et combien il en avait su profiter. Ce délié courtisan comptait
bien en tirer un plus grand parti. Sa santé, moins que ses vues lui
avait fait demander son congé et presser son retour\,; sa réception les
avait confirmés\,; il s'agissait de ne pas laisser refroidir de si
favorables dispositions. M\textsuperscript{me} de Maintenon le
conduisait par la main. Sous prétexte des affaires d'Espagne, elle lui
procurait des entretiens fréquents avec le roi, et comme les affaires
d'Espagne influaient sur toutes les autres, Harcourt, par sort conseil,
passait avec le roi des unes aux autres, et par cet appui en était
écouté.

Si Beauvilliers et Torcy étaient dans sa disgrâce, il s'en fallait peu
que le chancelier ne se trouvât au même point. On a vu qu'après leur
grande liaison il lui était devenu pesant aux finances, et que le désir
qu'elle eut d'y avoir un contrôleur général tout à elle avait plus que
toute autre raison poussé Pontchartrain à la place de chancelier, qu'il
désirait lui-même infiniment, et pour la grandeur de la charge et pour
se défaire des finances qu'il abhorrait. La cessation d'occasion de
mécontentement avait d'autant moins ramené M\textsuperscript{me} de
Maintenon à lui, qu'il ne s'était jamais soucié de s'en rapprocher, et
que son mépris marqué pour son successeur aux finances, et pour toutes
les opérations qu'il y faisait, avait formé un éloignement entre eux qui
fomenta l'ancien levain de lame de Maintenon, protectrice déclarée de
Chamillart. De cette sorte, de quatre ministres qui formaient le conseil
d'État, elle n'en avait qu'un à elle. Elle voulait donc y faire entrer
Harcourt, accoutumer le roi à lui, et l'y disposer par ces conversations
fréquentes qui se tournaient en consultations.

Elle l'avait lié avec M. du Maine et avec les plus accrédités valets du
roi de sa dépendance, et surtout avec Chamillart. Lui, de son côté,
avait gagné, à force de souplesses et de respects bien ménagés, la
roguerie sauvage de M. de La Rochefoucauld, qui, envieux né de tous et
de tout, haïssait MM. de Chevreuse et de Beauvilliers sans savoir
pourquoi. Harcourt avait gagné le peu de gens que leurs privances
approchaient du roi, et s'en était rendu ainsi tous les accès
favorables. Le grand vol qu'on lui voyait prendre et que nul autre homme
de qualité n'avait pu jusqu'alors atteindre, lui frayait le chemin à
toutes ces unions, et il devenait d'un air distingué d'être en liaison
avec lui. Il n'en faut pas tant dans les cours pour avoir à en choisir.
Telle était la position de M. d'Harcourt à Versailles.

La sienne à Madrid n'était pas moins riante. De Saint-Jean de Luz à
Madrid, et dans le peu qu'il fut en santé, le roi d'Espagne l'avait fort
goûté. Un peu avant le départ, il lui avait confié son désir d'aller en
Italie\,; il l'avait prié de le servir auprès du roi son grand-père sur
ce dessein\,; enfin, il l'avait pressé d'y venir lui mettre les armes à
la main, et de le conduire pendant la campagne. Non content d'une
ouverture si flatteuse, il lui avait écrit plusieurs fois, depuis, les
mêmes choses, et avec le plus grand empressement de l'avoir avec lui à
l'armée, et de s'y gouverner par ses conseils, et il le demandait au
roi. Tant de faveurs et de brillante fortune passait les bornes, non de
l'ambition d'Harcourt, qui était sans bornes, mais de la route qu'il,
s'était destinée. Rien de plus contradictoire que d'entrer ici dans le
conseil, et d'être celui du roi d'Espagne à l'armée d'Italie, commandée
sous lui par MM. de Villeroy et de Vaudemont, dont il connaissait le
crédit et les appuis. Ce fut donc un embarras d'autant plus grand pour
Harcourt, qu'il se voulait ménager l'Espagne pour ressource, si les
obstacles pour entrer dans le conseil se trouvaient trop forts. En ce
cas, son projet était de retourner en Espagne quand Philippe V y serait
de retour, et de prendre de là un vol nouveau et des forces nouvelles,
pour forcer à son retour ici la, porte du conseil. Il ne se fallait donc
pas montrer contraire au voyage d'Italie, pour ne pas perdre la
confiance du roi d'Espagne et la ressource qu'il méditait\,; mais, étant
si à portée d'arriver dès lors au comble de ses désirs, il avait surtout
à se garder d'une absence si étrangement à contretemps, et engagé comme
il se trouvait à ne pas quitter la personne du roi d'Espagne en Italie,
il fallait sur toutes choses lui rompre ce voyage, et encore plus le
rompre avec assez d'adresse pour qu'il n'en pût pas être accusé ou du
moins convaincu. Ce n'était pas une conduite aisée, surtout vis-à-vis
d'un homme aussi avisé, aussi pénétrant que Louville, convaincu de
l'importance de faire faire ce voyage, et chargé de le persuader à notre
cour, ardent\,: d'ailleurs et fortement appuyé du duc de Beauvilliers,
de Torcy, et du chancelier qu'il avait gagné par ses raisons, quoique
mal avec M. de Beauvilliers et très enclin aux avis contraires aux
siens.

Harcourt, avec les manières les plus polies, les plus affables, les plus
engageantes, les plus ouvertes, était l'homme du monde le plus haut, le
plus indifférent, excepté à sa fortune, le plus méprisant, avec
toutefois le bon esprit de consulter, soit pour gagner des gens, soit
pour faire sien ce qu'il en tirait de bon. Il avait beaucoup d'esprit,
juste, étendu, aisé à se retourner et à prendre toutes sortes de formes,
surtout séduisant, avec beaucoup de grâces dans l'esprit. Sa
conversation la plus ordinaire était charmante, personne n'était de
meilleure compagnie\,; ployant, doux, accessible, facile à se faire tout
à tous, et par là s'était fait extrêmement aimer partout et s'était fait
une réputation. Il parlait d'affaires avec, une facilité et une
éloquence naturelle et simple. Les expressions qui entraînaient
coulaient de source\,; la force et la noblesse les accompagnaient
toujours. Il ne fallait pas toutefois s'y fier si les affaires étaient
mêlées avec ses vues, il ne souffrait pas patiemment ce qui les
contredisait. Le sophisme le plus entrelacé et le mieux poussé lui était
familier. Il savait y donner un air simple et vrai, et jeter force
poudre aux yeux par des interrogations hardies, et quelquefois par des
disparates quand il en avait besoin. L'écorce du bien public et de la
probité, qu'il montrait avec celle de la délicatesse pour persuader sans
avoir l'air de s'en parer, n'avait rien qui le pût contraindre. Jamais
elle ne lui passa l'épiderme. Il avait l'art d'éviter d'y être pris,
mais s'il lui arrivait de se prendre dans le bourbier, une plaisanterie
venait au secours, un conte, une hauteur, en un mot il payait
d'effronterie et, ne se détournait pas de son chemin. Il mariait
merveilleusement l'air, le langage et les manières de la cour et du
grand monde, avec le propos, les façons et la liberté militaire, qui
l'une à l'autre se donnaient du prix. Droit et franc quand rien ne l'en
détournait\,; au moindre besoin la fausseté même et la plus profonde, et
toujours plein de vues pour soi, et de desseins personnels.
Naturellement gai, d'un travail facile, et jamais incommode par
inquiétude, ni à la guerre, ni dans le cabinet\,; jamais impatient,
jamais important, jamais affairé, toujours occupé et toujours ne
paraissant rien à faire\,; sans nul secours domestique pour le dehors et
pour sa fortune\,: en tout un homme très capable, très lumineux, très
sensé\,; un bel esprit net, vaste, judicieux, mais avare, intéressé,
rapportant tout à soi, fidèle uniquement à soi, d'une probité beaucoup
plus qu'équivoque, et radicalement corrompu par l'ambition la plus
effrénée. Il était l'homme de la cour le plus propre à devenir le
principal personnage, le plus adroit en détours, le plus fertile en
souterrains et en `manèges, que le liant de son esprit entretenait avec
un grand art, soutenu par une suite continuelle en tout ce qu'il se
proposait.

Il avait eu l'habileté de persuader au roi qu'il était l'homme le plus
instruit de l'Espagne, et le seul qui en connût les affaires et les
Personnages à fond. Il était pourtant vrai que fort délaissé, fort
suspect et fort éloigné de tout à sa première ambassade jusqu'au moment
que la reine voulût traiter avec lui, ou peut-être l'amuser et le
tromper par l'amirante, et qu'ayant eu défense d'écouter rien de cette
part, le dépit qu'il eut le fit retirer à la campagne à tirer des lapins
jusqu'à son rappel, lorsqu'on voulut faire déclarer le traité de partage
à Charles II, et n'y pas exposer la personne et le caractère de
l'ambassadeur. M. d'Harcourt n'avait donc pu revenir de cette première
ambassade bien instruit et au fait des choses d'Espagne\,; et à sa
seconde, à peine fut-il arrivé à Madrid, qu'il tomba dans cette grande
maladie qui dura en grand danger, ou à se rétablir à la Sarçuela, loin
de la cour et des affaires jusqu'au départ du roi d'Espagne pour la
Catalogne, et au sien pour revenir. Ce n'était donc pas pour être fort
instruit, et néanmoins il persuada au roi tout ce qu'il voulut
là-dessus, parce qu'il convenait aux vues de M\textsuperscript{me} de
Maintenon sur lui que le roi le crût tel qu'il se vantait à lui d'être.

Dans cette opinion, le roi en peine de se déterminer sur le voyage du
roi d'Espagne en Italie entre Louville et le duc d'Harcourt qui l'en
dissuadait de toutes ses forces, chacun soutenu de ses appuis, on vit
avec surprise un phénomène nouveau à la cour. Le roi ordonna à ses
ministres, c'est-à-dire au duc de Beauvilliers, à Torcy et à Chamillart
de s'assembler chez le chancelier, et au duc d'Harcourt de s'y trouver
pour y débattre le pour et le contre de ce voyage d'Italie, et lui faire
le rapport des avis. Jamais une pareille assemblée de ministres hors du
conseil et de la présence du roi, beaucoup moins personne admis à
délibérer avec eux, et ce qui était de plus surprenant, un seigneur que
sa qualité de seigneur en excluait plus constamment et plus radicalement
que nul autre. Aussi une telle distinction apporta-t-elle une extrême
considération à Harcourt, et le fit-elle regarder comme celui qui avait
levé le charme, et qui était tout contre d'entrer dans le conseil.
Louville, avec M\textsuperscript{me} de Maintenon contraire, n'était pas
bastant pour être de la conférence. Beauvilliers et Torcy étaient pleins
et persuadés de ces raisons\,; il ne fut pas seulement question de l'y
admettre.

En faveur du voyage on alléguait l'indécence de l'oisiveté d'un prince
de l'âge et de là santé du roi d'Espagne, tandis que toute l'Europe
s'armait pour lui ôter ou lui conserver ses couronnes\,; le peu de
prétextes qu'on pouvait prendre de la nécessité de veiller lui-même au
gouvernement de ses États, et son peu d'expérience et de
connaissances\,; l'influence fâcheuse qu'en recevrait sa réputation et
le respect de sa personne dans tous les temps\,; le plein repos où on
devait être sur la fidélité de l'Espagne et des ministres qui
gouverneraient en son absence, et sur lesquels tout portait, même en sa
présence, dans la jeunesse de son âge et la nouveauté de son arrivée\,;
l'importance de l'éloigner de bonne heure de l'air de fainéantise et de
paresse des trois derniers rois d'Espagne, qui n'étaient jamais sortis
de la banlieue de Madrid, et s'en étaient si mal trouvés\,; l'approcher
au contraire de l'activité de Charles-Quint, et le former de bonne heure
par le spectacle des différents pays, des divers génies des nations à
qui il avait à commander, et par l'apprentissage de la guerre et de ses
différentes parties, dont il aurait à entendre parler et à décider toute
sa vie. Enfin l'exemple de tous les rois, dont aucun, excepté ces trois
derniers d'Espagne, ne s'était dispensé d'aller à la guerre\,; sur quoi
celui du roi n'était pas oublié. On ajoutait la nécessité de montrer à
Milan, et surtout à Naples, avec ce qu'il venait d'y arriver, un jeune
roi dont ils n'avaient vu aucun depuis Charles-Quint, et un roi qui
commençait une lignée nouvelle, dont la présence lui attacherait de plus
en plus ces différents États par le soin qu'il prendrait à leur plaire
et par quelques bienfaits répandus à propos qui sortiraient sur les
lieux immédiatement de sa main.

À ces raisons on opposait le danger d'abandonner l'Espagne presque
aussitôt que le roi s'y était montré\,; l'embarras et le danger de sa
personne dans l'armée d'Italie\,; enfin le peu d'argent à employer à ses
dépenses plus indispensables qu'à une pompe de voyage et de campagne qui
ne se pouvait éviter en les faisant faire au roi d'Espagne et qui
coûterait infiniment.

Louville ne demeurait pas court à ces objections. Il répondait à la
première, que loin qu'il y eût du danger de tirer Philippe V de Madrid,
la gloire de l'occasion en plairait à toute l'Espagne\,; que dans ce
commencement d'arrivée et d'engouement, il y fallait accoutumer les
seigneurs, qui dans d'autres temps ne seraient pas si maniables à ce
qu'ils regarderaient comme une nouveauté, et qu'il n'était que très bon
de faire éprouver à Madrid l'éclipse d'un soleil dont la présence le
rendait heureux et abondant, et dont le retour après et la présence y
serait bien plus goûtée et chérie. À la seconde objection, que la
gloire, la réputation, le respect et l'attachement personnel
s'acquéraient très principalement et très solidement par les travaux et
les périls, lesquels étaient bien moindres pour les rois que pour les
autres hommes, et qui souvent faisaient un heureux bruit à bon marché\,;
enfin sur la dépense, qu'il n'y en avait aucune plus utile ni plus
nécessaire que celle qui allait à remplir des vues si principales\,; que
la dépense même se pouvait beaucoup modérer avec la plus grande
bienséance, et qu'un jeune prince n'en était que plus aimé et plus
estimé, en retranchant les pompes, les fêtes et tout l'inutile pour ne
pas fouler ses peuples et employer ses finances à les protéger et à les
défendre\,; qu'un voyage de guerre n'était pas celui d'un mariage ou
d'une entrevue, et que le simple nécessaire, réduit à la juste mesure de
la dignité d'un jeune roi qui ne va qu'en passant visiter ses nouveaux
sujets pour se mettre à la tête de son armée et y faire ses premières
armes, n'était pas si coûteux qu'on se le voulait persuader.

Ces raisons pour et contre, leurs subdivisions, leurs suites, leurs
conséquences, c'est ce qui fut débattu chez le chancelier. Harcourt, à
qui il était capital d'empêcher ce voyage, n'y oublia rien dans cette
conférence, appuyé de Chamillart\,; les deux autres, d'un sentiment
contraire, entraînèrent à demi le chancelier, qui ne se souciait plus de
faire sa cour à M\textsuperscript{me} de Maintenon. Il avait toujours
ménagé Monseigneur et lui avait fait tous les plaisirs qu'il avait pu
tandis qu'il avait eu les finances. Harcourt, qui n'oubliait rien,
commençait à se lier avec les deux sœurs Lislebonne. Il avait entretenu
Monseigneur, mais ce prince avait donné des audiences à Louville\,; il
aimait le roi d'Espagne\,; tel qu'il était, il sentait que son
empressement d'aller en Italie était appuyé de bonnes raisons, et que sa
gloire personnelle y était intéressée. Il en avait embrassé le sentiment
et l'appuyait. Le compte qui fut rendu au roi de la conférence ne lui
apprit rien de nouveau. Son goût par son propre exemple penchait au
voyage. M\textsuperscript{me} de Maintenon et Chamillart le retenaient
en suspens.

Dans ce même temps, le roi, qui méditait une grande promotion
d'officiers généraux, eut envie de faire des maréchaux de France en même
temps. Il est certain qu'il en écrivit quatre de sa main auxquels il se
voulait borner, qui étaient Rose ri, Huxelles, Tallard et Harcourt. Il
s'ouvrait alors de beaucoup de choses à Harcourt\,; il lui parla de la
promotion d'officiers généraux, il lui fit sentir quelque chose de celle
des maréchaux de France\,: Harcourt, qui mourait de peur de l'être,
parce qu'il sentait bien qu'on l'enverrait servir, et qu'il ne voulait
pas s'éloigner, sur le point qu'il se croyait d'entrer dans le conseil,
dissuada le roi d'en faire. Ce qui ne se comprend pas d'un homme
d'autant d'esprit, c'est que sa vanité le porta à s'en vanter jusqu'au
marquis d'Huxelles, à qui il en parla dans un coin de la galerie,
peut-être en lui répondant sur ce que l'autre le sondait pour hâter
cette promotion. Huxelles, surpris et encore plus outré du propos
d'Harcourt\,: «\,Mort.., lui dit-il, si vous n'étiez pas duc, vous vous
en seriez bien gardé\,;» et lui tourna le dos en furie.

Pendant tous ces manèges, Harcourt avec le meilleur visage du monde se
plaignait de coliques la nuit, d'insomnies et de toutes sortes de maux
qui ne paraissaient point, pour se tenir une porte ouverte à refuser de
servir et de s'éloigner\,; et toujours porté par sa protectrice, avait
de fréquents entretiens avec le roi, dans lesquels il frondait toujours
l'avis de ses ministres. La plupart de ces entretiens roulaient sur
l'Espagne ou sur la guerre.

Cette opposition d'Harcourt revint souvent par le roi même à Chamillart.
Soit que les ducs de Beauvilliers et de Chevreuse, ses amis
particuliers, lui fissent faire des réflexions, soit qu'il en fit de
lui-même, il ouvrit les yeux sur le risque personnel dont le menaçait
l'entrée d'Harcourt au conseil. Il comprit que, parvenu à ce comble de
ses désirs, et n'ayant plus rien à craindre, il ne songerait qu'à
empiéter la principale autorité, qu'étant homme de guerre et surtout de
détail, ce serait à ses dépens qu'il s'autoriserait\,; qu'il aurait
peine à résister à un homme aussi entreprenant, qui partageait au moins
avec lui la faveur et l'appui de M\textsuperscript{me} de Maintenon, et
qui, avant que de se voir dans le conseil, ne craignait pas de faire
contre aux ministres, et à lui-même dans les entretiens qu'il avait avec
le roi. Il pensa donc sérieusement à éviter ce péril, et à éloigner
Harcourt en le faisant maréchal de France, et servir en cette qualité.
Mais le roi incertain par ce que Harcourt lui avait représenté, on
prétend qu'un événement fortuit acheva d'empêcher qu'il n'y eût des
maréchaux de France\,; je dis on prétend, parce que, encore que j'aie eu
alors tout lieu de croire l'anecdote que je vais raconter, je n'en suis
pas assuré avec certitude. Voici le fait\,:

M\textsuperscript{me} la duchesse de Bourgogne qui, par ses grâces, ses
manières flatteuses et amusantes, et son attention de tous les instants
à plaire au roi et à M\textsuperscript{me} de Maintenon, s'était rendue
familière avec eux jusqu'à usurper toutes sortes de libertés, remuant un
soir les papiers du roi, sur sa petite table, chez M\textsuperscript{me}
de Maintenon, trouva cette liste des quatre maréchaux de France\,: en la
lisant, les yeux lui rougirent, elle s'écria en s'adressant au roi qu'il
oubliait Tessé, qui en mourrait de douleur et elle aussi. Elle se
piquait d'aimer Tessé, parce qu'il avait fait la paix de Savoie et son
mariage, et elle s'apercevait bien que par cette raison cela plaisait au
roi. Il fut fâché cette fois qu'elle eût vu ce papier, et soit qu'il eut
déjà résolu de ne point faire de maréchaux de France, ou qu'il fût butté
alors à ne pas faire Tessé, il répondit avec émotion à la princesse
qu'elle ne s'affligerait pas et qu'il n'en ferait aucun.

Cependant le roi d'Espagne écrivait lettres sur lettres au roi, sur son
voyage d'Italie. Le temps s'avançait. Il fallait se déterminer.
Chamillart, tout doucement détaché d'Harcourt, cessa ses oppositions par
rapport aux finances, comme entrant dans les raisons du voyage et dans
le goût que le roi y montrait. Il fut résolu, et Louville dépêché pour
en informer le roi d'Espagne.

Harcourt alors se sentit perdu avec lui, et sa ressource de retourner en
Espagne, si besoin lui en était, évanouie. Il avait tergiversé et
s'était caché tant qu'il avait pu sur ce voyage\,; mais la conférence
chez le chancelier lui avait forcé la main\,; il sentit bien que
Louville ne cacherait pas son opposition au roi d'Espagne, et le refus
dont je parlerai bientôt, que le duc de Beauvilliers ne lui laisserait
point ignorer, et beaucoup moins Torcy. Cela le résolut à redoubler
d'efforts pour entrer dans le conseil, et profiter de sa situation
présente.

Je ne sais si la vanité le trahit, ou s'il crut imposer à ceux qu'il
craignait par un raffinement de politique. Quoi qu'il en soit, il ne
craignit pas de plaisanter, avec un air de hauteur et d'assurance, de la
peur des ministres de le voir entrer dans le conseil, qui n'en fermaient
pas l'œill d'inquiétude, disait-il, tandis qu'il dormait les nuits tout
d'un somme, et il eut ou l'imprudence, ou la fausse politique de tenir
ce propos-là même à Louville, dans les derniers jours qu'il demeura pour
recevoir les dernières instructions par rapport au voyage arrêté
d'Italie. Harcourt disait très vrai pour la moitié, mais pour la
tranquillité de son sommeil, elle n'était pas aisée à persuader. Ses
entretiens continuaient sur le même pied, jusqu'à ce qu'enfin sa trop
grande assurance y mit Lin, et renversa pour lors son espérance.

Il avait pris à tâche d'être toujours diamétralement opposé aux avis des
ministres\,; il avait commencé à s'expliquer sur eux au roi, avec un
mépris moins couvert, et à lui montrer des abus, et à lui proposer des
réformes. Un jour que le roi insistait avec lui sur l'opinion de, ses
ministres, et qu'Harcourt la contredisait fortement, il lui échappa de
dire que ces gens-là n'étaient pas capables de la moindre bagatelle.
Cette parole mit fin aux entretiens et aux consultations du roi avec
lui, et lui ferma la porte du conseil déjà entrouverte. Le roi, jaloux
de ses choix, et qui n'avait pas dessein de changer son conseil, comprit
alors qu'en y admettant Harcourt, il aurait à essuyer une division
continuelle, une diversité d'avis sur tout, à la fin des querelles et
des prises qui le gêneraient autant que ce qu'il en avait éprouvé entre
Louvois et Colbert. Dès lors il résolut de n'augmenter point son conseil
d'un personnage qui y serait si fâcheux à ses ministres, dont
l'importunité retomberait sur lui, aussi bien que l'embarras à se
déterminer entre des avis toujours opposés.

Les matières d'Espagne qui avaient servi de chausse-pied à ces
entretiens étaient épuisées avec Harcourt, la confiance sur les autres
affaires cessait avec la pensée de le faire ministre\,; avec elle aussi
tombèrent les entretiens et les consultations. En vain Harcourt
chercha-t-il à se raccrocher, en vain M\textsuperscript{me} de Maintenon
essaya-t-elle de le rapprocher, et tous deux de faire naître des
prétextes et des occasions de nouveaux entretiens, tout fut inutile. Le
roi avait pris son parti, et tint ferme à n'avoir plus de particuliers
avec lui, mais d'ailleurs le traitant bien et même avec distinction. Ce
changement l'affligea au dernier point. Il avait évité le bâton de
maréchal de France, comme le plus dangereux écueil, avec tout le soin
possible\,; il avait également échoué à s'entretenir avec le roi
d'Espagne, et à rompre son voyage d'Italie, et il se voyait frustré de
ce grand but auquel il voulait atteindre, et dont il s'était trouvé si
longtemps tout près. M\textsuperscript{me} de Maintenon, qui pour ses
vues particulières n'en fut pas moins désolée que lui, le soutint et le
consola par l'espérance de profiter plus heureusement, pour ne pas dire
plus sagement, d'autres conjonctures qui pourraient naître, et qui
pourraient le porter de nouveau au même but, auquel pour lors il n'était
plus possible de songer.

\hypertarget{chapitre-xix.}{%
\chapter{CHAPITRE XIX.}\label{chapitre-xix.}}

1702

~

{\textsc{Retour de Catinat.}} {\textsc{- Promotion d'officiers
généraux.}} {\textsc{- Ma réception au parlement.}} {\textsc{- Visites
qui la précèdent\,; pièges que j'y évite.}} {\textsc{- Je quitte le
service.}} {\textsc{- Bagatelles qui caractérisent.}} {\textsc{-
Bougeoir.}} {\textsc{- Soupers de Trianon.}} {\textsc{- Duc de Villeroy
arrivé d'Italie.}} {\textsc{- Journée de Crémone.}} {\textsc{- Situation
de Crémone et qui y commandait.}} {\textsc{- Maréchal de Villeroy
pris.}} {\textsc{- Aventure de Montgon.}} {\textsc{- Villeroy hautement
protégé du roi et traité en favori.}} {\textsc{- Revel chevalier de
l'ordre.}} {\textsc{- Praslin lieutenant général.}}

~

Catinat arrivé d'Italie, où sa patience avait essuyé de si cruels
dégoûts, salua le roi à son dîner, un jour qu'il avait pris médecine\,;
le roi lui fit un air assez gracieux, lui dit quelques mots, mais ce fut
tout\,; nul particulier\,; le roi ne lui dit pas même qu'il l'entretien
droit, et le modeste maréchal ne montra pas seulement qu'il le désirât,
et s'en retourna tranquillement à Paris.

La promotion d'officiers généraux dont j'ai parlé se fit enfin. Elle fut
prodigieuse. Dix-sept lieutenants généraux, cinquante maréchaux de camp,
quarante et un brigadiers d'infanterie, et trente-huit de cavalerie.
Avant que d'expliquer où elle me conduisit, il faut dire que je me fis
recevoir ce même hiver au parlement. Le roi qui sur ses bâtards a
toujours commencé de fait toutes les distinctions qu'il leur a données,
avant que de les leur accorder par des brevets, des lettres, des
déclarations et des édits, et qui depuis longtemps avait établi qu'aucun
pair n'était reçu au parlement, sans lui en demander la permission,
qu'il ne refusait jamais, s'était mis à la différer si le pair n'avait
pas vingt-cinq ans, pour mettre peu à peu une différence d'âge entre ses
enfants naturels et eux, par un usage qu'il pût après tourner en règle.
Je le savais, et j'avais exprès différé ma réception plus d'une année au
delà des vingt-cinq ans, sous prétexte de négligence.

Il fallut aller chez le premier président Harlay qui m'accabla de
respects, chez les princes du sang, chez les bâtards. M. du Maine se fit
répéter le jour marqué, puis, d'un air de joie mal contenue par celui de
la politesse et de la modestie\,: «\,Je n'aurai garde d'y manquer, me
dit-il, ce m'est un honneur trop grand d'y assister et trop sensible que
vous veuilliez bien que j'y sois, pour ne pas m'y trouver,\,» et avec
mille compliments me conduisit jusqu'au jardin, car c'était à Marly où
j'étais ce voyage. Le comte de Toulouse et M. de Vendôme me répondirent
plus simplement, mais ne parurent pas moins contents, ni moins polis ni
attentifs à remplir tout ce qu'ils devaient, comme avait fait M. du
Maine. Depuis que le cardinal de Noailles avait reçu la pourpre romaine,
il ne venait plus au parlement, parce qu'il n'y pouvait prendre sa place
qu'au rang de l'ancienneté de sa pairie. Je pris le temps de son
audience publique pour l'aller convier. «\,Vous savez, me dit-il, que je
n'ai plus de place. --- Et moi, monsieur, lui répondis-je, qui vous y en
connais une fort belle, je viens vous supplier de la venir prendre à ma
réception.\,» Il se mit à sourire et moi aussi. Nous nous entendions
bien tous deux\,; puis me vint conduire au haut de son degré, les
battants des portes ouverts, et passant tous deux de front, moi à sa
droite. M. de Luxembourg fut le seul duc qui n'entendit pas parler de
moi à cette occasion. J'avais toujours sur le cœur l'étrange arrêt qu'il
avait obtenu, et dont j'ai assez parlé ci-devant pour n'en rien répéter.
Je me flattais que nous y pourrions revenir quelque jour, et je ne
voulus pas donner atteinte à cette espérance, par une reconnaissance
solennelle et personnelle du droit qu'il lui avait acquis. Je n'étais
point raccommodé avec lui, ainsi je ne lui en fis faire aucune
honnêteté.

Dongois, qui faisait la fonction de greffier en chef du parlement, à qui
ses accès et sa capacité avait donné autorité en beaucoup de choses dans
le parlement, était par là connu et recherché. Je le connaissais fort,
et pris langue avec lui du détail de ce que j'avais à faire. Tout
obligeant et honnête homme qu'il était, le bonhomme me tendit trois
pièges. Il ne fallait pas s'attendre à moins de sa robe, mais je les
sentis tous trois et tout d'abord, et je me préservai de tous les trois.
Il me dit donc qu'il convenait pour le respect du parlement d'y paraître
cette première fois en habit tout noir, sans dorure\,; que pour celui
des princes du sang, dont le manteau court descendait plus bas que
l'habit, le mien ne débordât pas mon justaucorps, et que pour celui du
premier président, j'allasse, comme c'est la coutume, le matin même
après ma réception, le remercier, mais avec mon habit du parlement. Ces
trois respects ne me furent pas si grossièrement dits, mais insinués
avec esprit. Je n'en fis pas semblant, mais je fis directement le
contraire, et instruit de la sorte, j'en avertis ceux qui furent reçus
dans la suite, qui s'en gardèrent comme j'avais fait, et c'est par ces
sortes de ruses, pour le dire en passant, que sont venues tant de choses
à l'égard des ducs dont l'accès affermi a de quoi plus que surprendre.

Je devrais ajouter ici ce qu'il se passa en cette occasion entre M. de
La Rochefoucauld et moi, qui nous disputions la préséance. Je réserve à
le raconter de suite au temps qu'il fut question de la juger. Il ne vint
point à ma réception, et tout se passa alors avec toute l'amitié qui
s'était entretenue entre nous, depuis la liaison que le procès contre M.
de Luxembourg y avait formée, et que la qualité de gendre de M. le
maréchal de Lorges, son plus ancien et intime ami, ne gâtait pas.

Dreux, père du grand maître des cérémonies, nouvellement monté à la
grand'chambre, fut le rapporteur que je choisis, parce que c'était un
vrai et intègre magistrat, que je le connaissais plus que les autres, et
qu'ils sont flattés de rapporter nos réceptions. Je lui envoyai le matin
même, suivant l'usage, ainsi qu'au premier président et procureur
général, un service de vaisselle d'argent. Lamoignon, premier président,
commença celui de ne le point accepter qui a toujours duré depuis lui.
Dreux, nouveau venu à la grand'chambre, et tout enterré dans ses sacs,
ignorait parfaitement l'un et l'autre usage. Il trouva fort mauvais que
je lui eusse envoyé un présent, et demanda pour qui on le prenait. Il le
renvoya comme une offense qui lui était faite, et n'apprit qu'après que
ce n'était qu'une formalité.

La réforme qui suivit la paix de Ryswick fut très grande et faite très
étrangement. La bonté des régiments, surtout dans la cavalerie, le
mérite des officiers, ceux qui les commandaient, Barbezieux jeune et
impétueux n'eût égard à rien, et le roi le laissa le maître. Je n'avais
aucune habitude avec lui. Mon régiment fut réformé, et comme il était
fort bon, il fit présent de ses débris à des royaux, au régiment de
Duras, et jusqu'à ma compagnie fut incorporée dans celui du comte
d'Uzès, son beau-frère, dont il prenait un soin particulier. Ce me fut
un sort commun avec beaucoup d'autres qui ne m'en consola pas. Ces
mestres de camp réformés sans compagnie furent mis à la suite d'autres
régiments\,; j'échus à celui de Saint-Moris. C'était un gentilhomme de
Franche-Comté que je n'avais vu de ma vie, dont le frère était
lieutenant général et estimé. Bientôt après, la pédanterie, qui se
mêlait toujours avec la réalité du service, exigea deux mois de présence
aux régiments à la suite desquels on était. Cela me parut fort sauvage.
Je ne laissai pas d'y aller, mais comme j'avais eu diverses
incommodités, et qu'on m'avait conseillé les eaux savonneuses de
Plombières, je demandai la permission d'y aller, et y passai trois ans
de suite le temps d'exil à un régiment où je ne connaissais personne, où
je n'avais point de troupes et où je n'avais rien à faire. Le roi ne
parut point le trouver mauvais. J'allai souvent à Marly\,; il me parlait
quelquefois, qui était chose bien marquée et bien comptée\,; en un mot
il me traitait bien, et mieux que ceux de mon âge et de ma sorte.

Cependant on remplaça quelques mestres de camp de mes cadets\,;
c'étaient d'anciens officiers qui avaient obtenu des régiments à force
de services et de temps\,; je me payai de cette raison. La promotion
dont on parlait ne me réveilla point. On n'était plus dans un temps à se
prévaloir de dignités ni de naissance. Excepté des actions, et
sur-le-champ, personne n'était distingué de l'ordre du tableau. J'avais
trop d'anciens pour songer à être brigadier\,; tout mon objet était un
régiment, et de servir à la tête, puisque la guerre s'ouvrait, pour
n'avoir pas le dégoût de la commencer pour ainsi dire aide de camp de
Saint-Moris et sans troupes, après avoir été préféré par distinction en
arrivant de la campagne de Neerwinden pour en avoir un, l'avoir bien
rétabli, et y avoir, je l'ose dire, commandé avec application et
réputation les quatre campagnes suivantes qui avaient fini la guerre.

La promotion se déclara, qui surprit tout le monde par le grand
nombre\,; jamais à beaucoup près il n'y en avait eu de pareille\,; je
parcourus avidement les brigadiers de cavalerie pour voir si mon tour
approchait de près. Je fus bien étonné quand j'en vis cinq à la queue
mes cadets. Leur nom n'est jamais sorti de ma mémoire et y est toujours
demeuré très présent. C'était d'Ourches, Vandeuil, Streff, le comte
d'Ayen et Ruffé. Il est difficile de se sentir plus piqué que je le fus.
Je trouvais l'égalité confuse de l'ordre du tableau suffisamment
humiliante, la préférence du comte d'Ayen malgré son népotisme, et celle
de quatre gentilshommes particuliers me parut insupportable. Je me tus
cependant, pour ne rien faire de mal à propos dans la colère. M. le
maréchal de Lorges fut outré et pour moi et pour lui-même\,; M. son
frère ne le fut guère moins, et par l'inconsidération pour eux, et telle
qu'il fut volontiers pour tout le monde. Il avait pris de l'amitié pour
moi. Tous deux me proposèrent de quitter. Le dépit m'en donnait grande
envie\,; la réflexion de mon âge, de l'entrée d'une guerre, de renoncer
à toutes les espérances du métier, l'ennui de l'oisiveté, la douleur des
étés à ouïr parler de guerre, de départs, d'avancements de gens qui s'y
distinguent, qui s'y élèvent, qui acquièrent de la réputation me
retenait puissamment. Je passai ainsi deux mois dans ce déchirement,
quittant tous les malins, et ne pouvant bientôt après m'y résoudre.

Poussé enfin à bout de cet état avec moi-même, et pressé par les deux
maréchaux, je me résolus à prendre des juges à l'avis desquels je me
rendrais, et à les prendre en des états différents. Je choisis le
maréchal de Choiseul sous qui j'avais servi, et bon juge en ces
matières, M. de Beauvilliers, M. le chancelier, et M. de La
Rochefoucauld. Je leur avais déjà fait mes plaintes\,; ils étaient
indignés de l'injustice, mais les trois derniers en courtisans. C'était
mon compte. Ce génie était propre à tempérer leur conseil, et comme je
n'en cherchais qu'un bon qui fût approuvé dans le monde, de gens de
poids et qui approchaient du roi, surtout qui ne fût pas sujet à
légèreté, imprudence ni repentir, ce fut à ceux-là que je déterminai
d'abandonner la décision de ma conduite.

Je me trompai, les trois courtisans furent du même avis que les trois
maréchaux\,; tous me dirent avec force qu'il était honteux et
insoutenable {[}qu'{]} un homme de ma naissance, de ma dignité, qui
avait servi avec quelque honneur, assiduité et approbation quatre
campagnes à la tête d'un beau et bon régiment, réformé jusqu'à sa
compagnie, sans raison, demeuré dans une aussi nombreuse promotion, et y
voir cinq de ses cadets avec la dernière injustice, recommençât la
guerre non seulement sans brigade, mais sans régiment, mais sans troupes
et sans compagnie, avec pour toute fonction d'être à la suite de
Saint-Moris\,; qu'un duc et pair de ma naissance établi d'ailleurs comme
je l'étais, et ayant femme et enfants, n'allait point servir comme un
haut-le-pied dans les armées, et y voir tant de gens si différents de ce
que j'étais, et qui pis était de ce que j'y avait été, tous avec des
emplois et des régiments\,; qu'après une si nombreuse promotion
j'attendrais longtemps un régiment vacant aboyé des familles et des
officiers, encore plus longtemps une brigade, avec tous les dégoûts de
la situation où je me trouvais, que cette injustice faite, mon beau-père
et son frère vivants maréchaux de France, ducs et tous deux capitaines
des gardes du corps, que pouvais-je espérer quand ils ne seraient
plus\,? Ils ajoutèrent toute la différence de quitter par paresse ou par
pis, d'avec quitter par des raisons aussi évidentes après avoir vu, fait
et servi avec distinction\,; qu'à tout compter il y avait bien loin et
bien des dégoûts et des hasards de fortune à essuyer entre ce que
j'étais et le but qui me retiendrait au service, outre que l'injustice
qui m'était faite me reculait beaucoup, et influait sur le délai de tous
les autres pas\,: en un mot, tous six séparément m'accablèrent des mêmes
raisons, comme s'ils les avaient concertées ensemble.

Je ne les avait pas pris pour juges, pour appeler après de leur
décision. Je pris donc mon parti\,; mais je crus souvent l'avait bien
pris que je sentais que je balançais encore\,; j'eus besoin de ma colère
et de mon dépit, et de me rappeler encore ce que j'avais vu arriver à M.
le maréchal de Lorges à la tête de l'armée du Rhin, par les intendants
La Fonds et La Grange, soutenus de la cour, et au maréchal de Choiseul
dans le même emploi, que j'ai l'un et l'autre racontés en leur lieu,
sans compter tout ce qui se trouve à essuyer de ce genre, avant que
d'arriver au commandement des armées. Près de trois mois se passèrent
dans ces angoisses intérieures jusqu'à ce que je pusse me déterminer.
Finalement je le fis, et lorsqu'il en fallut venir à l'exécution, je
suivis encore le conseil des mêmes personnes\,: je ne laissai point
échapper de paroles de mécontentement, et content du public, et surtout
du militaire sur mon oubli dans la promotion, je le laissai dire. Pour
moi, la colère du roi était inévitable. Ces messieurs m'y avaient
préparé, et je m'y étais bien attendu. Oserai-je dire qu'elle ne m'était
pas indifférente\,? Il s'offensait quand on cessait de servir. Il
appelait cela le quitter, encore plus des gens distingués. Mais ce qui
le piquait au vif, c'était de quitter sur une injustice, et il le
faisait toujours du moins longtemps sentir. Mais les mêmes personnes ne
mirent jamais de proportion entre cette suite de quitter, qui après
tout, à mon âge avait son bout, et la honte et le dégoût de servir dans
la situation où j'étais. Ils crurent cependant que le respect et la
prudence voulaient également tout le ménagement qui s'y pouvait
apporter.

Je fis donc une lettre courte au roi, par laquelle sans plainte aucune,
ni la moindre mention d'aucun mécontentement, et sans parler de régiment
ni de promotion, je lui marquais mon déplaisir que la nécessité de ma
mauvaise santé m'obligent à quitter son service, dont je ne pouvais me
consoler que par une assiduité auprès de sa personne, qui me procurerait
l'honneur de la voir, et de lui faire ma cour plus continuellement. Ma
lettre fut approuvée, et le mardi de la semaine sainte, je la lui
présentai moi-même à la porte de son cabinet, comme il y rentrait de la
messe. J'allai de là chez Chamillart, que je ne connaissais point du
tout. Il sortait pour aller au conseil. Je lui fis de bouche le même
compliment, sans le mêler de rien qui pût sentir le mécontentement, et
tout de suite je m'en allai à Paris.

J'avais mis gens de plusieurs sortes en campagne, hommes et femmes de
mes amis, pour être informé de ce qu'il échapperait au roi, où que ce
fût, sur ma lettre. Je demeurai huit jours à Paris, et ne retournai à
Versailles que le mardi de Pâques. Je sus du chancelier que, le conseil
appelé et entrant le mardi saint dans le cabinet du roi, qu'il lisait ma
lettre et qu'il appela aussitôt après Chamillart, auquel il parla un
moment en particulier. Je sus d'ailleurs qu'il lui avait dit avec
émotion\,: «\,Eh bien\,! monsieur, voilà encore un homme qui nous
quitte,\,» et que tout de suite il, lui avait raconté ma lettre mot pour
mot. D'ailleurs, je n'appris point qu'il lui fût rien échappé. Ce mardi
de Pâques, je reparus devant lui, pour la première fois depuis ma
lettre, à la sortie de son souper. J'aurais honte de dire la bagatelle
que je vais raconter si dans la circonstance elle ne servait à le
caractériser.

Quoique le lieu où il se déshabillait fût fort éclairé, l'aumônier de
jour, qui tenait, à sa prière du soir, un bougeoir allumé, le rendait
après au premier valet de chambre, qui le portait devant le roi venant à
son fauteuil. Il jetait un coup d'œill tout autour, et nommait tout haut
un de ceux qui y étaient, à qui le premier valet de chambre donnait le
bougeoir. C'était une distinction et une faveur qui se comptait, tant le
roi avait l'art de donner l'être à des riens. Il ne le donnait qu'à ce
qui était là de plus distingué en dignité et en naissance, extrêmement
rarement à des gens moindres, en qui l'âge et les emplois suppléaient.
Souvent il me le donnait, rarement à des ambassadeurs, si ce n'est au
nonce, et dans les derniers temps à l'ambassadeur d'Espagne. On ôtait
son gant, on s'avançait, on tenait ce bougeoir pendant le coucher, qui
était fort court, puis on le rendait au premier valet de chambre qui, à
son choix, le rendait à quelqu'un du petit coucher. Je m'étais exprès
peu avancé, et je fus très surpris, ainsi que l'assistance, de
m'entendre nommer, et dans la suite je l'eus presque aussi souvent que
je l'avais eu jusque-là. Ce n'était pas qu'il n'y eût à ce coucher force
gens très marqués à qui le donner, mais le roi fut assez piqué pour ne
vouloir pas qu'on s'en aperçût.

Ce fut aussi tout ce que j'eus de lui trois ans durant qu'il n'oublia
aucune bagatelle, faute d'occasions plus importantes, de me faire sentir
combien il était fâché. Il ne me parla plus\,: ses regards ne tombaient
sur moi que par hasard\,; il ne dit pas un mot de ma lettre à M. le
maréchal de Lorges, ni de ce que je quittais. Je n'allai plus à Marly,
et après quelques voyages, je cessai de lui donner la satisfaction du
refus.

Il faut épuiser ces misères. Quatorze ou quinze mois après, il fit un
voyage à Trianon. Les princesses avaient accoutumé de nommer chacune
deux dames pour le souper, et le roi ne s'en mêlait point pour leur
donner cet agrément. Il s'en lassa. Les visages qu'il voyait à sa table
lui déplurent, parce qu'il n'y était pas accoutumé. Les matins il
mangeait seul avec les princesses et leurs dames d'honneur, et il
faisait une liste lui-même et fort courte des dames qu'il voulait le
soir, et l'envoyait à la duchesse du Lude chaque jour pour les faire
avertir. Ce voyage était du mercredi au samedi\,: ainsi trois soupers.
Nous en usâmes, M\textsuperscript{me} de Saint-Simon et moi, pour ce
Trianon-là comme pour Marly\,; et ce mercredi que le roi y allait, nous
fûmes dîner chez Chamillart à l'Étang, pour aller de là coucher à Paris.
Comme on s'allait mettre à table, M\textsuperscript{me} de Saint-Simon
reçut un message de la duchesse du Lude pour l'avertir qu'elle était sur
la liste du roi pour le souper de ce même jour. La surprise fut
grande\,; nous retournâmes à Versailles. M\textsuperscript{me} de
Saint-Simon se trouva seule de son âge à beaucoup près à la table du
roi, avec M\textsuperscript{me}s de Chevreuse et de Beauvilliers, la
comtesse de Grammont et trois ou quatre autres espèces de duègnes
favorites ou dames du palais nécessaires, et nulle autre. Le vendredi,
elle fut encore nommée et avec les mêmes dames\,; et depuis, le roi en
usa toujours ainsi aux rares voyages de Trianon. Je fus bientôt au fait
et j'en ris. Il ne nommait point M\textsuperscript{me} de Saint-Simon
pour Marly, parce que les maris y allaient de droit quand leurs femmes y
étaient\,; ils y couchaient, et personne n'y voyait le roi que ce qui
était sur la liste. À Trianon liberté entière à tous les courtisans d'y
aller faire leur cour à toutes les heures de la journée\,; personne n'y
couchait que le service le plus indispensable, pas même aucune dame. Le
roi voulait donc marquer mieux par cette déférence que l'exclusion
portait sur moi tout seul, et que M\textsuperscript{me} de Saint-Simon
n'y avait point de part.

Nous persévérâmes dans notre assiduité ordinaire sans demander pour
Marly\,: nous vivions agréablement avec nos amis, et
M\textsuperscript{me} de Saint-Simon continua de jouir à l'ordinaire des
agréments qui ne se partageaient point avec moi, et que le roi et
M\textsuperscript{me} la duchesse de Bourgogne avaient commencé
longtemps avant ceci à lui donner, et qui s'augmentèrent toujours. J'ai
voulu épuiser cette matière de suite qui, par rapport au caractère du
roi, a sa curiosité\,: reprenons maintenant où nous en sommes demeurés.
J'ajouterai seulement ici qu'après la promotion, le roi donna force
pensions militaires, et qu'il fit la galanterie à M. le maréchal de
Lorges de lui mander qu'il avait choisi le plus beau de tous les
régiments de cavalerie gris que la promotion mettait en vente, pour en
donner la préférence à son fils, depuis assez peu capitaine de
cavalerie.

Le duc de Villeroy arriva, le 6 février, envoyé par son père pour rendre
compte au roi de bien des détails et de projets qui auraient emporté
trop de temps par des dépêches. Bien lui prit de ce voyage, trois jours
après il eut tout lieu de le sentir.

La promotion si nombreuse dont j'ai parlé, et qui me fit quitter vers
Pâques, s'était faite et déclarée le 29 janvier. Le mercredi 8 février,
on alla à Marly, où il y eut des bals. Nous fûmes du voyage,
M\textsuperscript{me} de Saint-Simon et moi, comme souvent nous en
étions. Le lendemain jeudi 9, Mahoni, officier irlandais de beaucoup
d'esprit et de valeur, arriva d'Italie avec la plus surprenante nouvelle
dont on eût ouï parler en ces derniers siècles. L'action s'était passée
le 1er février.

Le prince Eugène, qui en savait plus que le maréchal de Villeroy,
l'avait obligé d'hiverner au milieu du Milanais, et l'y tenait fort
resserré, tandis que lui-même avait établi ses quartiers fort au large
avec lesquels il inquiétait fort les nôtres\footnote{Voy., page 10 des
  Pièces la lettre du maréchal de Villeroy au cardinal d'Estrées
  (\emph{Notes de Saint-Simon}.)}. Dans cette situation avantageuse il
conçut le dessein de surprendre le centre de nos quartiers, et par ce
coup de partie qui le mettait au milieu de notre armée et de notre pays,
de dissiper l'une, et de se rendre maître de l'autre, et par là se
mettre en état ensuite de prendre Milan et le peu de places de ce pays,
toutes en fort mauvais ordre, et d'achever ainsi sûrement et brusquement
sa conquête.

Crémone était ce centre\,; il y avait un gouverneur espagnol et une fort
grosse garnison\,: quelques autres troupes y étaient encore entrées à la
fin de la campagne, avec Crenan, lieutenant général, pour y commander
tout. Praslin, dont j'ai parlé quelquefois, y commandait la cavalerie
comme brigadier\,; il venait d'être fait maréchal de camp, mais la
promotion n'était pas encore parvenue jusqu'à eux, et Fimarcon
commandait les dragons. Vers les derniers jours de janvier, Revel,
premier lieutenant général de l'armée, était arrivé à Crémone, et par
son ancienneté y commanda au-dessus de Crenan.

Il reçut ordre du maréchal de Villeroy, qui visitait ses quartiers,
d'envoyer un gros détachement à Parme, que le duc de ce nom lui
demandait pour sa sûreté, et qu'on eut lieu de soupçonner depuis de
l'avoir fait de concert avec le prince Eugène, pour dégarnir Crémone
d'autant. Sur les nouvelles de différents mouvements des ennemis, Revel,
en homme sage, se contenta de faire et de tenir le détachement prêt sans
le faire partir. Le maréchal de Villeroy finit sa promenade par Milan,
où il conféra avec le prince de Vaudemont, d'où il arriva le dernier
janvier à Crémone d'assez bonne heure. Revel alla au-devant de lui, lui
rendit compte des raisons qu'il avait de retenir le détachement qu'il
lui voit ordonné d'envoyer à Parme. Il en fut fort approuvé du maréchal,
qui soupa en nombreuse compagnie, où il parut fort rêveur. Il ne laissa
pas de jouer après une partie d'hombre, mais on remarqua que ce ne fut
pas sans distractions, et il se retira de fort bonne heure.

Le prince Eugène était informé qu'il y avait à Crémone un ancien aqueduc
qui s'étendait loin à la campagne, et qui répondait dans la ville à une
cave d'une maison occupée par un prêtre\,; que cet aqueduc avait été
nettoyé depuis assez peu de temps, et cependant ne conduisait que peu
d'eau et que la ville avait été autrefois surprise par ce même aqueduc.
Il en fit secrètement reconnaître l'entrée dans la campagne\,; il gagna
le prêtre chez qui il aboutissait, et qui était voisin d'une porte de la
ville qui était murée et point gardée\,; il fit couler dans Crémone ce
qu'il put de soldats choisis, déguisés en prêtres et en paysans, qui se
retirèrent dans la maison amie, où on se pourvut le plus et le plus
secrètement qu'on put de haches. Tout bien et promptement préparé, le
prince Eugène donna un gros détachement au prince Thomas de Vaudemont,
premier lieutenant général de son armée, et fils unique du gouverneur
général du Milanais pour le roi d'Espagne\,: il lui confia son
entreprise, et, le chargea de s'aller rendre maître d'une redoute qui
défendait la tête du pont du Pô, pour venir par le pont à son secours,
quand on serait aux mains dans la ville. Il détacha cinq cents hommes
d'élite avec des officiers entendus pour se rendre par l'aqueduc chez le
prêtre, où les gens qu'il y avait fait couler les attendaient, et
devaient avoir bien reconnu les remparts, les postes, les places et les
rues de la ville, et avec eux, aller ouvrir la porte murée au reste des
troupes\,: en même temps il marcha en personne et en force pour se
rendre à cette porte.

Tout concerté avec justesse fut exécuté avec précision et tout le secret
et le bonheur possible. Le premier qui s'en aperçut fut le cuisinier de
Crenan, qui, allant à la provision à la première petite pointe du jour,
vit les rues pleines de soldats dont les habits lui étaient inconnus. Il
se rejeta dans la maison de son maître qu'il courut éveiller\,; ni lui
ni ses valets n'en voulaient rien croire\,; mais, dans l'incertitude,
Crenan s'habilla en un moment, sortit et n'en fut que trop tôt assuré.
En même temps le régiment des vaisseaux se mettait en habille dans une
place, par un bonheur qui sauva Crémone. D'Entragues, gentilhomme
particulier de Dauphiné, en était colonel\,: c'était un très honnête
garçon, fort appliqué, fort valeureux, qui avait une extrême envie de
faire et de se distinguer, et qui avait appris et retenu la vigilance du
maréchal de Boufflers, dont il avait été aide de camp, et qui, lui ayant
trouvé de l'honneur et des talents, le protégeait beaucoup. D'Entragues
voulait faire la revue de ce régiment, et la commençait avec le petit
jour. À cette clarté encore faible, et ses bataillons déjà sous les
armes et formés, il aperçut confusément des troupes d'infanterie se
former au bout de la rue, en face de lui. Il savait, par l'ordre donné
la veille, que personne ne devait marcher, ni autre que lui faire de
revue. Il craignit donc tout aussitôt quelque surprise, marcha
sur-le-champ à ces troupes qu'il trouva impériales, les charge, les
renverse, soutient le choc des nouvelles qui arrivent, engage un combat
si opiniâtre, qu'il donne le temps à toute la ville de se réveiller, et
à la plupart des troupes de prendre les armes et d'accourir, qui sans
lui eussent été égorgées endormies.

À cette même pointe du jour, le maréchal de Villeroy écrivait déjà tout
habillé dans sa chambre\,; il entend du bruit, demande un cheval, envoie
voir ce que c'est, et, le pied à l'étrier, apprend de plusieurs à la
fois que les ennemis sont dans la ville. Il enfile la rue pour gagner la
grande place, où est toujours le rendez-vous en cas d'alarme. Il n'est
suivi que d'un seul aide de camp et d'un seul page. Au détour de la rue,
il tombe dans un corps de garde qui l'environne et l'arrête. Lui
troisième sentit bien qu'il n'y avait pas à se défendre\,; il se jette à
l'oreille de l'officier, se nomme, lui promet dix mille pistoles et un
régiment, s'il veut le lâcher, et de plus grandes récompenses du roi.
L'officier se montre inflexible, lui répond qu'il n'a pas servi
l'empereur jusqu'alors pour le trahir, et de ce pas le conduit au prince
Eugène, qui ne le reçut pas avec la même politesse qu'il l'eût été de
lui en pareil cas. Il le laissa quelque temps à sa suite, pendant lequel
le maréchal voyant amener Crenan prisonnier et blessé à mort, il s'écria
qu'il voudrait être en sa place. Un moment après ils furent envoyés tous
deux hors de la ville, et ils passèrent la journée à quelque distance,
gardés dans le carrosse du prince Eugène.

Revel, seul lieutenant général désormais, et commandant en chef par la
prise du maréchal de Villeroy, tâcha de rallier les troupes. Chaque rue
fournissait un combat, {[}les troupes{]} la plupart dispersées,
quelques-unes en corps, plusieurs à peine armées, et jusqu'à des gens en
chemise qui tous combattaient avec la plus grande valeur, mais la
plupart repoussées et réduites pied à pied à gagner les remparts, ce qui
les y rallia toutes naturellement. Si les ennemis s'en fussent emparés,
ou qu'ils n'eussent pas laissé à nos troupes le temps de s'y reconnaître
et de s'y former avec toutes leurs forces, le dedans de la ville n'eût
jamais pu leur résister. Au lieu donc de faire effort ensemble pour
chasser nos troupes des remparts, ils ne s'attachèrent qu'au dedans de
la ville.

Praslin, ne voyant point Montgon, maréchal de camp, s'était mis à la
tête des bataillons irlandais, qui sous lui firent des prodiges. Ils
tinrent dans la place et nettoyèrent les rues voisines. Quoique
continuellement occupé à défendre et à attaquer, Praslin s'avisa que le
salut de Crémone, si on la pouvait sauver, dépendait de la rupture du
pont du Pô, pour empêcher les Impériaux d'être secourus par là et
rafraîchis. Il le répéta tant de fois que Mahoni l'alla dire à Revel qui
n'y avait pas songé, qui trouva l'avis si bon qu'il manda à Praslin de
faire tout ce qu'il jugerait à propos. Lui, à l'instant, envoya retirer
ce qui était dans la redoute à la tête du pont. Il n'y avait pas une
minute à perdre. Le prince Thomas de Vaudemont paraissait déjà,
tellement qu'on n'eut que le loisir de retirer ses troupes et de rompre
le pont, ce qui fut exécuté en présence même du prince Thomas de
Vaudemont, qui avec toute sa mousqueterie ne le put empêcher.

Il était lors trois heures après midi. Le prince Eugène était à l'hôtel
de ville à prendre le serment des magistrats. Sortant de là, et en peine
de voir ses troupes, faiblir en la plupart des lieux, il monta avec le
prince de Commercy au clocher de la cathédrale pour voir d'un coup
d'œill ce qu'il se passait dans tous les endroits de la ville, et en
peine aussi de ne voir point arriver le secours qu'amenait le prince
Thomas de Vaudemont. À peine furent-ils au haut du clocher qu'ils virent
son détachement au bord du Pô, et le pont rompu qui rendait ce secours
inutile. Ils ne furent pas plus satisfaits de ce qu'ils découvrirent
dans tous les différents lieux de la ville et des remparts. Le prince
Eugène, outré de voir son entreprise en si mauvais état après avoir
touché de si près à la conquête, hurlait et s'arrachait les cheveux en
descendant. Il pensa dès lors à la retraite, quoique supérieur en
nombre.

Fimarcon faisait merveilles cependant avec les dragons, qu'il avait fait
mettre pied à terre. En même temps Revel, qui voyait ses troupes
accablées de faim, de lassitude et de blessures, et qui depuis la
première pointe du jour n'avaient pas eu un instant de repos ni même de
loisir, songeait aussi de son côté à les retirer, ce qu'il pourrait, au
château de Crémone, pour s'y défendre au moins à couvert, et y obtenir
une capitulation\,; de sorte que les deux chefs opposés pensaient en
même temps à se retirer.

Les combats se ralentirent donc sur le soir en la plupart des lieux dans
cette pensée commune de retraite, lorsque nos troupes firent un dernier
effort pour chasser les ennemis d'une des portes de la ville qui leur
ôtait la communication du rempart où étaient les Irlandais, et pour
avoir cette porte libre pendant la nuit et pouvoir par là recevoir du
secours. Les Irlandais secondèrent si bien cette attaque par leur
rempart, que le dessus de la porte fut emporté\,; les ennemis
conservèrent le bas de la porte de plain-pied à la rue. Un calme assez
long succéda à ce dernier combat. Revel cependant songeait à faire
retirer doucement les troupes au château, lorsque sur ce long calme
Mahoni lui proposa d'envoyer voir ce qui se passait partout, et se
proposa lui-même pour aller aux nouvelles et lui en venir rendre compte.
Il faisait déjà obscur\,; les batteurs d'estrade en profitèrent. Ils
virent tout tranquille, et reconnurent que les ennemis s'étaient
retirés. Cette grande nouvelle fut portée à Revel, qui fut longtemps, et
beaucoup d'autres avec lui, sans le pouvoir croire. Persuadé enfin, il
laissa tout au même état jusqu'au grand jour, qu'il trouva les rues et
les places jonchées de morts et remplies de blessés. Il donna ordre à
tout, et dépêcha Mahoni au roi, qui y avait fait merveilles.

Le prince Eugène marcha toute la nuit avec le détachement qu'il avait
amené, et se fit suivre fort indécemment par le maréchal de Villeroy,
désarmé et mal monté, qu'il envoya à Ustiano, et, depuis, sur les ordres
de l'empereur, à Inspruck, qui le fit après conduire à Gratz, en Styrie.
Tous ses gens et son équipage lui fut envoyé à Ustiano et le suivit
depuis. Crenan mourut dans le carrosse du maréchal de Villeroy, allant
le joindre à Ustiano. D'Entragues, à la revue et à la valeur duquel on
fut redevable du salut de Crémone, ne survécut pas à une si glorieuse
journée. Le gouverneur espagnol fut tué avec la moitié de nos troupes.
Les Impériaux y en perdirent un plus grand nombre et manquèrent un coup
qui finissait en bref en leur faveur la guerre d'Italie.

Montgon, maréchal de camp, essuya là une aventure qui ne rétablit pas sa
réputation. Il sortit à pied au premier grand bruit, et il rentra
incontinent chez lui. Il prétendit avoir été jeté par terre et foulé aux
pieds des chevaux des ennemis. Il se dit fort blessé et se mit au lit,
d'où il envoya se rendre prisonnier au plus voisin corps de garde, et
demander d'être mis en sûreté. Il passa ainsi cette terrible journée
dans le repos entre deux draps. Il y apprit Crémone prise, puis
reprise\,; alors sa sauvegarde eut besoin qu'il lui en servît, et il
obtint de Revel de la renvoyer libre. Le fâcheux fut qu'il ne se trouva
sur Montgon aucune blessure. Le prince Eugène le réclama comme
prisonnier, et lui ne demandait pas mieux. Nos généraux prétendirent
qu'il avait recouvré sa liberté avec la place. Le roi voulut avoir
l'avis des maréchaux de France, et toutefois avant de l'avoir eu il
manda que ce n'était pas la peine de disputer. On ne disputait plus, le
prince Eugène s'était rendu. Montgon ne laissa pas de l'aller trouver,
mais le prince Eugène, qui ne voulait point, de prisonniers incertains,
le renvoya libre. Cette aventure qui fit grand bruit et grand tort à
Montgon, l'eût perdu auprès du roi sans M\textsuperscript{me} de
Maintenon, protectrice déclarée de tout temps de sa femme, de la vieille
Heudicourt, sa belle-mère.

J'appris cette nouvelle, dans ma chambre, par M. de Lauzun. Aussitôt
j'allai au château où je trouvai une grande rumeur et force pelotons de
gens qui raisonnaient. Le maréchal de Villeroy fut traité comme le sont
les malheureux qui ont donné de l'envie\footnote{Les recueils de
  chansons de cette époque sont remplis de couplets sur l'affaire de
  Crémone. On n'en peut guère extraire que ces vers souvent
  cités\,:François, rendez grâce à Bellone\,:Votre bonheur est sans
  égal\,;Vous avez conservé CrémoneEt perdu votre général.}. Le roi prit
hautement son parti et publiquement. Il témoigna, en dînant, à
M\textsuperscript{me} d'Armagnac combien il était sensible au malheur de
son frère, et l'excusa en montrant même de l'aigreur contre ceux qui
tombaient sur lui. La vérité est que ce n'était pas à lui, qui arrivait
à Crémone la veille de la surprise, à savoir cet aqueduc et cette porte
murée, ni, s'il y avait déjà des soldats impériaux introduits et cachés.
Crenan et le gouverneur espagnol étaient ceux qui en devaient répondre,
et le maréchal ne pouvait mieux que d'aller au premier bruit à la grande
place, ni répondre de sa capture au détour d'une rue en s'y portant.

Son fils, qui était à Marly avec sa femme, l'amena à cette nouvelle à
Versailles, où était la maréchale de Villeroy. J'étais extrêmement de
leurs amis. Je les trouvai le lendemain dans la plus morne douleur. La
maréchale, qui avait infiniment de sens et d'esprit, et du plus aimable,
n'avait point été la dupe de l'éclat de l'envoi de son mari en Italie.
Elle le connaissait et elle craignait les événements. Celui-ci
l'accabla, et {[}elle{]} fut longtemps sans vouloir voir personne que
ses plus intimes, ou des gens indispensables. La duchesse de Villeroy ne
revint plus à Marly à cause des bals, dont M\textsuperscript{lle}
d'Armagnac ne perdit aucun, quoique son père et ses oncles prissent feu
pour le maréchal de Villeroy et toutes sortes de mesures pour lui.

Au sortir du dîner du jour de l'arrivée de Mahoni, le roi s'enferma seul
avec lui dans son cabinet. Cependant la cour était nombreuse dans sa
chambre, et ce qui surprit fut d'y voir Chamillart y attendre comme les
autres en proie aux questions. Il vanta fort les principaux officiers,
et le gros des autres et les troupes, et il s'étendit sur les merveilles
de Praslin, et sur sa présence d'esprit d'avoir fait rompre le pont. On
a vu ci-devant, en son lieu, qu'il était extrêmement de mes amis.
Quoique alors je ne connusse point du tout Chamillart, je ne pus
n'empêcher de lui dire que cet important service méritait une grande
récompense. Au bout d'une heure le roi sortit de son cabinet. En
changeant d'habits, pour aller dans ses jardins, il parla fort de
Crémone en louange, et surtout des principaux officiers\,; il prit
plaisir à s'étendre sur Mahoni, et dit qu'il n'avait jamais ouï personne
rendre un si bon compte de tout, ni avec tant de netteté d'esprit et de
justesse, même si agréablement. Il ajouta avec complaisance qu'il lui
donnait mille francs de pension et un brevet de colonel. Il était major
du régiment de Dillon.

Le soir, comme nous entrions au bal, M. le prince de Conti nous dit que
le roi donnait l'ordre à Revel, et faisait Praslin lieutenant général.
La joie que j'en eus me fit le lui demander encore pour en être plus
sûr. Les autres officiers principaux furent avancés à proportion de
leurs grades, et beaucoup eurent des pensions. Revel eut encore le
gouvernement de Condé\,; et le marquis de Créqui, quoiqu'il n'eût pas
été à Crémone, eut la direction de l'infanterie\,; c'était la dépouille
de Crenan.

\hypertarget{chapitre-xx.}{%
\chapter{CHAPITRE XX.}\label{chapitre-xx.}}

1702

~

{\textsc{Harcourt refuse l'armée d'Italie.}} {\textsc{- Vendôme
l'accepte et part.}} {\textsc{- Grand prieur refusé de servir.}}
{\textsc{- Feuquières refusé de servir\,; son étrange caractère.}}
{\textsc{- Colandre colonel avec choix.}} {\textsc{- La Feuillade
maréchal de camp tout à coup.}} {\textsc{- M\textsuperscript{me} de
Chambonas dame d'honneur de la duchesse du Maine.}} {\textsc{-
Changement chez Madame.}} {\textsc{- Maréchale de Clérembault.}}
{\textsc{- Comtesse de Beuvron.}} {\textsc{- Mort de Fouquet, évêque
d'Agde.}} {\textsc{- P. Camille se fixe en Lorraine\,; son caractère.}}
{\textsc{- Sourdis.}} {\textsc{- Mariage de sa fille avec le fils de
Saint-Pouange.}} {\textsc{- Mariage du duc de Richelieu avec la marquise
de Noailles.}} {\textsc{- Mort du bailli d'Auvergne.}} {\textsc{-
Médailles du roi.}} {\textsc{- Jalousie sur Louis XIII.}} {\textsc{-
Comte de Toulouse pour la mer avec le comte d'Estrées.}} {\textsc{- Mgr
le duc de Bourgogne en Flandre avec le maréchal de Boufflers et le
marquis de Bedmar.}} {\textsc{- Le maréchal d'Estrées en Bretagne.}}
{\textsc{- Chamilly à la Rochelle, etc.}} {\textsc{- Catinat sur le
Rhin.}} {\textsc{- Son sage et curieux éclaircissement avec le roi et
Chamillart.}} {\textsc{- Jugement arbitral du pape entre l'électeur
palatin et Madame qui proteste.}}

~

La principale {[}dépouille{]} tenait en grande attention\,: c'était le
commandement de l'armée d'Italie. Il était pressé d'y pourvoir. Le
lendemain, vendredi, le roi, au sortir de sa messe, entra chez
M\textsuperscript{me} de Maintenon, où Chamillart fut quelque temps en
tiers. Tout ce qui était à Marly était dans les salons, attendant le
choix du général qu'on voyait bien qui s'allait déclarer. Ma curiosité
m'y porta comme les autres. Chamillart sortit, vit M. le prince de
Conti, alla lui dire un mot. Chacun le crut l'élu\,; on applaudit, mais
l'erreur ne dura guère. Chamillart fut fort court avec lui, s'avança
lentement cherchant des yeux, et, apercevant Harcourt, alla droit à lui.
Alors on ne douta plus, et tous les yeux s'arrêtèrent sur eux. Rien ne
se mariait mieux avec le désir du roi d'Espagne d'aller en Italie, et
d'y avoir ce général sous lui. Mais Harcourt en était alors à cet assaut
du conseil dont je viens de parler, et au plus fort de ses espérances
que lui-même n'avait pas encore détruites en parlant avec ce grand
mépris des ministres au roi, comme il fit depuis. Il n'eut donc garde
d'accepter un commandement qui anéantissait toutes ses mesures si
avancées pour entrer dans le conseil. Il se défendit sur sa santé et
refusa. Lui et Chamillart parlèrent à l'écart assez longtemps avec
action. Tout ce qu'il y avait là d'yeux n'en perdaient aucune, et virent
enfin ces deux hommes se séparer, et Chamillart seul retourner chez
M\textsuperscript{me} de Maintenon. Il y fut peu et ressortit. La
curiosité était plus allumée. Il s'avança, chercha des yeux, et fut
joindre M. de Vendôme. Leur conversation fut très courte. Tous deux
ensemble allèrent chez M\textsuperscript{me} de Maintenon. Alors on fut
assuré du choix et de l'acceptation. Il fut déclaré lorsque le roi passa
dans son appartement. Le soir il fut longtemps chez
M\textsuperscript{me} de Maintenon avec le roi et Chamillart, prit congé
et s'en alla à Paris pour partir le surlendemain pour l'Italie. Le roi
lui donna quatre mille louis pour son équipage.

Le dépit de M. le duc d'Orléans et des princes du sang fut extrême et
fort marqué. Ils n'en tombèrent que plus rudement sur le maréchal de
Villeroy, que le roi en toutes occasions prit à tâche de défendre,
jusqu'à dire en publie qu'on ne l'attaquait que par jalousie de ce qu'il
avait beaucoup d'amitié pour lui. Le mot de favori, qui n'était jamais
sorti de sa bouche, lui échappa même une fois. Il lui écrivit une
lettre, la plus obligeante qu'il fût possible, et, la lui envoya
ouverte, pour que les ennemis n'en eussent pas de soupçon, et
qu'eux-mêmes vissent quelle était son estime et son amitié pour lui.
Quoiqu'il n'eût aucune familiarité avec la maréchale de Villeroy, il lui
fit dire mille choses agréables par son fils, par M. le Grand et par
d'autres, et, après Marly, la vit en particulier longtemps et la combla
de bontés. Il la vit plusieurs fois de la sorte pendant l'absence de son
mari, dont il ne se lassa point de se montrer le défenseur.

Mais l'envie est une cruelle passion\,; Praslin l'éprouva. Des plus
grandes louanges on passa au regret de la récompense. Il fut lieutenant
général avant que d'avoir pu savoir qu'il était maréchal de camp. De
raisons on n'en pouvait dire\,; les femmes criaient en place de
raisons\,; et la comtesse de Roucy, entre autres, qui en était furieuse,
fut de meilleure foi, car l'ayant poussée à bout, elle me répondit,
acculée et dans l'excès de sa colère, qu'enfin Praslin était lieutenant
général, et que son mari ne l'était pas, lequel mari était lors à la
cour.

M. le duc d'Orléans et les princes du sang n'en eurent pas moins contre
M. de Vendôme. Ils sentaient, il y avait longtemps, la résolution du roi
à ne se servir d'aucun d'eux, et sa préférence pour la naissance
illégitime. Cette dernière les outra. Vendôme, qui le comprit dans le
peu d'heures qu'il demeura à Marly et à Paris, entre sa nomination et
son départ, ne cessa de répandre qu'il ne devait son choix qu'au refus
d'Harcourt, et d'émousser ainsi le dépit des princes, tandis qu'il se
fit un mérite de ne refuser rien, même le reste d'un autre, pour montrer
son attachement à la personne du roi, et son désir d'essayer à
contribuer au bien de l'État.

Le grand prieur, intimement uni avec son frère, eut la douleur de n'être
point employé, et d'essuyer même le refus d'aller servir sous lui en
Italie. Sa crapule journalière, sa vie honteuse, plusieurs frasques
qu'il avait hasardées sur la faveur de sa naissance et sur celle de son
frère, reçurent enfin ce coup de caveçon dont il eut grande peine à
revenir dans la suite.

Feuquières, lieutenant général, reçut le même refus. C'était un homme de
qualité, d'infiniment d'esprit et fort orné, d'une grande valeur, et à
qui personne ne disputait les premiers talents pour la guerre, mais le
plus méchant homme qui fût sous le ciel, qui se plaisait au mal pour le
mal, et à perdre d'honneur qui il pouvait, même sans aucun profit.
Dangereux au dernier point pour un général d'armée, qui ne se pouvait
fier ni à ses conseils ni à son exécution, tant il était hardi à faire
échouer les entreprises pour la malice d'en perdre quelqu'un, comme il
fit Bullonde à Coni, comme il ne tint pas à lui à la bataille de
Neerwinden, où il ne chargea ni ne branla jamais, comme je l'ai remarqué
ailleurs, et comme le duc d'Elbœuf le lui reprocha devant toute l'armée,
parce qu'il voulait perdre M. de Luxembourg, en lui faisant perdre la
bataille, lequel l'avait demandé pour le remettre sur l'eau, et qui avec
raison n'en voulut jamais plus. Il avait joué les mêmes tours aux autres
généraux d'armée\,; pas un d'eux n'en voulait, et avec d'autant plus de
raison que sa capacité n'était qu'à craindre. M. le maréchal de Lorges
l'avait aussi tiré de l'oisiveté\,; il en reçut la même reconnaissance
que M. de Luxembourg. Il ne tint pas à lui qu'il ne fit battre son armée
à ne s'en pas relever\,; et la chose devint par le hasard si grossière,
et le cri si général, que, pour peu que M. le maréchal de Longes eût
voulu, sa tête aurait couru grand risque. Les Mémoires qu'il a laissés,
et qui disent avec art tout le mal qu'il peut de tous ceux avec qui et
surtout sous qui il a servi, sont peut-être le plus excellent ouvrage
qui puisse former un grand capitaine, et d'autant plus d'usage qu'ils
instruisent par les examens et les exemples, et font beaucoup regretter
que tant de capacité, de talents, de réflexions se soient trouvés unis à
un cœur aussi corrompu et à une aussi méchante âme, qui les ont tous
rendus inutiles par leur perversité. Il avait épousé l'héritière
d'Hocquincourt, qui la devint par l'événement. Il acheva sa vie
abandonné, abhorré, obscur et pauvre. Son fils, unique mourut sans
enfants, sa fille fut misérablement mariée.

Colandre, lieutenant aux gardes, qui s'était distingué partout où il
s'était trouvé, et dont la figure intéressait les dames, eut l'agrément
d'un régiment et traita de celui de la Reine infanterie\,; mais le roi
arrêta le marché, et trouva que Colandre, fils de Le Gendre, riche
négociant de Rouen, n'était pas fait pour être colonel de régiments de
cette sorte. Les maximes ont changé depuis, c'est ce qui m'a engagé à ne
pas omettre ce fait, que je pourrais grossir de beaucoup d'autres et
plus marqués encore à l'égard d'autres corps.

La Feuillade ne tarda pas à profiter de l'alliance qu'il venait de
contracter. Chamillart le fit faire maréchal de camp sous la cheminée,
et partir pour l'Italie, et aussitôt après il fut déclaré. Ainsi, il ne
fut point brigadier, et fit tomber encore son régiment à un Aubusson.

M\textsuperscript{me} du Maine et M\textsuperscript{me} de Manneville,
fille de Montchevreuil et sa dame d'honneur, se lassèrent l'une de
l'autre. La princesse peu à peu avait secoué tous ses jougs, même celui
du roi et de M\textsuperscript{me} de Maintenon, qui enfin la laissèrent
vivre à son gré. Ce reste de lien lui déplut\,; M. du Maine tremblait
devant elle. Il mourait toujours de peur que la tête ne lui tournât.
Elle prit M\textsuperscript{me} de Chambonas, que personne ne
connaissait, et, dont le mari était déjà à M. du Maine, capitaine de ses
gardes, comme gouverneur de Languedoc.

En même temps Madame fit un changement chez elle, dans lequel le roi
entra, et qui se régla chez elle à Marly, dans une visite que le roi lui
rendit un matin en revenant de la messe. Elle congédia ses filles
d'honneur avec leur gouvernante en leur donnant des pensions, et prit
auprès d'elle, mais sans titre ni nom, la maréchale de Clérembault et la
comtesse de Beuvron, qu'elle avait toujours fort aimées, mais sur
lesquelles Monsieur, qui les haïssait, l'avait toujours fort contrainte.
Toutes deux étaient veuves, la comtesse de Beuvron pauvre, et toutes
deux n'avaient rien de mieux à faire. Elle leur donna quatre mille
livres de pension à chacune. Le roi leur donna un logement à
Versailles\,; elles suivirent Madame partout, et furent, sans demander,
de tous les voyages de Marly.

La maréchale de Clérembault était fille de Chavigny, secrétaire d'État,
dont j'ai parlé au commencement de ces Mémoires, à l'occasion de mon
père, et sœur entre autres de l'évêque de Troyes, de la retraite duquel
j'ai parlé, et qui reviendra encore sur la scène. Elle était gouvernante
de la reine d'Espagne, fille de Monsieur, qui se prit à elle de diverses
choses et la chassa assez malhonnêtement. Elle était parente assez
proche et fort amie de M. et de M\textsuperscript{me} la chancelière, et
allait souvent à Pontchartrain avec eux. C'est où je l'ai fort vue et
chez eux à la cour. C'était une vieille très singulière, et quand elle
était en liberté, et qu'il lui plaisait de parler, d'excellente et de
très plaisante compagnie, pleine de traits et de sel qui coulait de
source, sans faire semblant d'y toucher et sans aucune affectation. Hors
de là des journées entières sans dire une parole\,; étant jeune, elle
avait pensé mourir de la poitrine, et avait eu la constance d'être une
année entière sans proférer un mot. Avec sa tranquillité, son
indifférence, sa froideur naturelle, l'habitude lui en était restée. On
ne saurait plus d'esprit qu'elle en avait, ni d'un tour plus singulier.
Quoique venue fort tard à la cour, elle en était passionnée et instruite
à surprendre de tout ce qui s'y passait, dont, quand elle daignait en
prendre la peine, les récits étaient charmants\,; mais elle ne se
laissait aller que devant bien peu de personnes et bien en particulier.

Avare au dernier point, elle aimait le jeu passionnément, et ces
conversations particulières et resserrées, et rien du tout autre chose.
Je me souviens qu'à Pontchartrain, par le plus beau temps du monde, elle
se mettait, en revenant de la messe, sur le pont qui conduit aux
jardins, s'y tournait lentement de tous côtés, puis disait à la
compagnie\,: «\,Pour aujourd'hui, me voilà bien promenée, oh\,! bien,
qu'on ne m'en parle plus, et mettons-nous à jouer tout à l'heure\,;» et
de ce pas prenait des cartes qu'elle n'interrompait que le temps des
deux repas, et trouvait mauvais encore qu'on la quittât à deux heures
après minuit. Elle mangeait peu, souvent sans boire, au plus un verre
d'eau. Qui l'aurait crue, on eût fait son repas sans quitter les cartes.
Elle savait beaucoup et en histoire et en sciences\,; jamais il n'y
paraissait. Toujours masquée en carrosse, en chaise, à pied par les
galeries\,: c'était une ancienne mode qu'elle n'avait pu quitter, même
dans le carrosse de Madame. Elle disait que son teint s'élevait en
croûte sitôt que l'air le frappait\,; en effet, elle le conserva beau
toute sa vie, qui passa quatre-vingts ans, sans d'ailleurs avoir jamais
prétendu en beauté. Avec tout cela, elle était fort considérée et
comptée. Elle prétendait connaître l'avenir par des calculs et de petits
points, et cela l'avait attachée à Madame, qui avait fort ces sortes de
curiosités\,; mais la maréchale s'en cachait fort.

Il faut donner le dernier trait à cette espèce de personnage. Elle avait
une sœur religieuse à Saint-Antoine à Paris, qui, à ce qu'on disait,
avait pour le moins autant d'esprit et de savoir qu'elle\,: c'était la
seule personne qu'elle aimât. Elle l'allait voir très souvent de
Versailles\,; et, quoique très avare mais fort riche, elle l'accablait
de présents. Cette fille tomba malade\,; elle la fut voir et y envoya
sans cesse. Lorsqu'elle la sut fort mal et qu'elle comprit qu'elle n'en
reviendrait pas\,: «\,Oh bien, dit-elle, ma pauvre sœur, qu'on ne m'en
parle plus.\,» Sa sœur mourut, et oncques depuis elle n'en a parlé ni
personne à elle. Pour ses deux fils, elle ne s'en souciait point, et
n'avait pas grand tort, quoiqu'en grande mesure avec elle\,; elle les
perdit tous deux, il n'y parut pas et dès les premiers moments.

La comtesse de Beuvron était une autre femme à qui, non plus qu'à la
maréchale de Clérembault, il ne fallait pas déplaire, et qui était
extrêmement de mes amies. Elle était fille de condition de Gascogne\,;
son père s'appelait le marquis de Théobon, du nom de Rochefort. Elle
était fille de la reine lorsqu'elle épousa le comte de Beuvron, frère de
la duchesse d'Arpajon et du comte de Beuvron, père du duc d'Harcourt,
desquels j'ai parlé plus d'une fois. Le comte de Beuvron était capitaine
des gardes de Monsieur, dont j'ai fait mention à propos de la mort de la
première femme de ce prince. Elle en était veuve, en 1688, sans enfants
et était pauvre. Des intrigues du Palais-Royal la firent chasser par
Monsieur au grand déplaisir de Madame, qui fut plusieurs années sans
avoir permission de la voir, et qui ne la vit enfin que rarement et à la
dérobée dans des couvents à Paris. Elle lui écrivait tous les jours de
sa vie, et en recevait réponse par un page qu'elle envoyait exprès. Elle
était intimement unie avec la famille de son mari, et notre liaison avec
la comtesse de Roucy, fille unique de la duchesse d'Arpajon, où elle
était sans cesse, forma la nôtre avec elle\,; mais elle n'était revenue
à la cour qu'à la mort de Monsieur, qui la lui avait fait défendre.
C'était une femme qui avait beaucoup d'esprit et de monde, et qui, à
travers de l'humeur et une passion extrême pour le jeu, était fort
aimable et très bonne et sûre amie.

L'évêque d'Agde mourut vers ce temps-ci fort riche en bénéfices. Il
était frère du surintendant Fouquet, mort à Pignerol en 1680, après
vingt années de prison, de l'archevêque de Narbonne et de l'abbé Fouquet
si connu en son temps, mort deux mois avant son frère, à la disgrâce
duquel ses imprudences et ses folies avaient eu grande part. Il fut en
1656 chancelier de l'ordre, et en même temps Guénégaud, secrétaire
d'État, fut garde des sceaux de l'ordre qu'on désunit de la charge de
chancelier qu'ils achetèrent de M. Servien. La disgrâce du surintendant
leur frère les dépouilla des marques de l'ordre, fit réunir la charge de
chancelier aux sceaux de l'ordre, entre les mains de Guénégaud en 1661,
et confina ses frères dans un exil. M. d'Agde changea souvent de lieu,
et eut enfin permission de demeurer à Agde sans en sortir le reste de
ses jours. Il fut chancelier de l'ordre sur la démission de son frère en
1659.

Carlingford, milord irlandais, qui avait été gouverneur de M. de
Lorraine de la main de l'empereur, à qui il était fort attaché, avait
suivi son pupille dans ses États à la paix de Ryswick\,; il était grand
maître de sa maison et à la tête de son conseil. Devenu feld-maréchal de
l'empereur, il désira retourner à Vienne. M. le Grand, qui avait
beaucoup d'enfants et peu de patrimoine, trouva jointure à mettre le
prince Camille à la place de Carlingford pour la charge et pour de plus
fortes pensions encore. Il le fit trouver bon au roi, et le prince
Camille s'alla fixer en Lorraine, où il ne fut pas plus goûté qu'il
l'était ici. C'était un homme de peu d'esprit, fort glorieux,
particulier, qui avala toute sa vie beaucoup de vin fort tristement\,;
une espèce de fagot d'épine, mais ruminant toujours à part soi la
grandeur de sa maison, et qui n'avait des Guise, qu'il regrettait, que
la valeur et la volonté. Il avait toujours servi et n'était point marié,
du reste honnête homme.

Saint-Pouange fit un grand mariage pour son fils avec la fille unique de
Sourdis, chevalier de l'ordre, dont il avait toute sa vie été ami
intime. La débauche les avait unis, et cette amitié suppléa au mérite
pour l'avancement. Sourdis se fit battre auprès de Neuss avec tant
d'ignorance, et s'en tira si honteusement à l'ouverture de la guerre
précédente, en 1689, que M. de Louvois, n'osant plus l'employer dans les
armées, mais pressé par Saint-Pouange, l'envoya commander en Guyenne. Il
s'y conduisit avec tant de crapule, et si misérablement d'ailleurs,
qu'il ne put y être soutenu davantage. Le commandement de la province
lui fut ôté, et un successeur envoyé à sa place. Sourdis, enchanté de sa
maîtresse à soixante-dix ans, ne put quitter Bordeaux parce qu'elle y
voulait demeurer, et y survécut ainsi à lui-même. À la fin la honte de
sa vie obligea à l'en faire sortir. Il ne put s'en éloigner et se
confina dans une de ses terres en Guyenne. Un homme si peu soigneux de
son honneur donna sa fille au fils de son ancien ami et protecteur, sans
compter pour rien l'inégalité du mariage de son héritière à qui il
devait laisser de grands biens qu'elle eut en effet, et qu'il ne lui fit
pas longtemps attendre. Il mourut en grand affaiblissement d'esprit, et
fort vieux et veuf depuis longues années sans s'être remarié.

Le duc de Richelieu, vieux et veuf deux fois, épousa en troisièmes noces
une Rouillé, veuve du marquis de Noailles, frère du duc, du cardinal et
du bailli de Noailles, dont elle avait une fille unique. Elle était fort
riche et voulait un tabouret. M. de Richelieu, qui l'était fort aussi,
mais qui, avec des biens substitués et une conduite toujours
désordonnée, en était toujours aux expédients, lui donna le sien pour se
remettre à flot, et n'avait aussi qu'un fils unique. En s'épousant, ils
arrêtèrent le mariage de leurs enfants, dont ils passèrent et signèrent
le contrat en attendant qu'ils fussent en âge de se marier. Le vieux
couple avait de l'esprit, mais l'humeur de part et d'autre peu
concordante, qui donna des scènes au monde. Malgré ce second mariage de
la duchesse de Richelieu, elle demeura toute sa vie dans l'union la plus
intime avec la famille de son premier mari, surtout avec le cardinal de
Noailles.

Celle du comte d'Auvergne, et lui-même, se trouvèrent fort soulagés par
la mort du bailli d'Auvergne, son fis aîné, que l'indignité de toute la
suite de sa vie, et celle de son combat avec Caylus dont j'ai parlé en
son temps, avaient chassé du royaume, fait déshériter et jeté malgré lui
dans l'ordre de Malte, menaçant souvent de réclamer contre ses vœux.

Il sembla que les flatteurs du roi prévissent alors que le terme des
prospérités de son règne fût arrivé, et qu'ils n'auraient désormais à le
louer que de sa constance. Ce grand nombre de médailles frappées en
toutes sortes d'occasions, où les plus communes n'étaient pas même
oubliées, fut ramassé, gravé et destiné à une, histoire métallique.
L'abbé Tallemant, Tourel\footnote{Ce membre de l'Académie française,
  dont le nom est ainsi écrit par Saint-Simon, s'appelait Jacques de
  Toureil.} et Dacier, trois savants principaux de l'Académie française,
avaient été chargés de l'explication de ces médailles, à mettre à côté
de chacune dans un gros volume de la plus magnifique impression du
Louvre. Il fallut une préface, et comme cette sorte d'histoire
commençait à la mort de Louis XIII, sa médaille fut nécessairement mise
à la tête du livre, et engageait ainsi à dire quelque chose de ce prince
dans cette préface. Quelqu'un de leur connaissance s'avisa de ma juste
reconnaissance, et crut qu'elle me prêterait ce que je n'avais pas de
moi-même pour le morceau de la préface qui devait regarder Louis XIII,
ou pour mettre sous sa médaille, qui devait être à la tête de celles de
Louis XIV. On me proposa de le faire. L'esprit fut la dupe du cœur, et,
sans consulter mon incapacité, j'y consentis, à condition qu'on m'en
épargnerait le ridicule dans le monde, et qu'on m'en garderait
fidèlement le secret.

Je le fis donc, et je m'y tins en garde contre moi-même, toujours occupé
de ne pas obscurcir le fils par le père dans un ouvrage tout à la gloire
du premier et où le second n'entrait que par accident et par nécessité
de l'introduction\footnote{Voy., ce court éloge, page 13 des Pièces.
  (\emph{Note de Saint-Simon}.)}. Mon thème fait, et il ne me fallut
guère qu'une matinée, parce qu'il ne devait pas être fort étendu, je le
donnai. J'eus le sort des auteurs\,; ma\,; pièce fut louée, et ne parut
excéder en rien. Je m'en applaudis, ravi d'avoir consacré deux ou trois
heures à ma juste reconnaissance, car je n'y en mis pas davantage.

Quand ce fut à l'examen pour l'insérer, ces messieurs furent effrayés.
Il est des vérités dont la simplicité sans art jette un éclat qui efface
tout le travail d'une éloquence qui grossit ou qui pallie\,: Louis XIII
fournit de celles-là en abondance. Je m'étais contenté de les montrer,
mais ce crayon ternissait les tableaux suivants, à ce qu'il parut à ceux
qui les ornaient. Ils s'appliquèrent donc à élaguer, à affaiblir, à
voiler tout ce qu'ils purent pour n'obscurcir pas leur héros par une
comparaison qui se faisait d'elle-même. Ce travail leur fut ingrat\,;
ils s'aperçurent enfin que ce n'était pas moi qu'ils avaient à corriger,
mais la chose même dont le lustre naissant de soi-même ne se pouvait
éteindre que par la suppression\,; ils sentirent le mensonge de cette
sorte de correction\,; que, taisant certains faits, certaines vérités,
ils ne pouvaient les omettre toutes, et toutes à leurs yeux étaient de
nature à offusquer leur sujet. Cet embarras, grossi de l'esprit dominant
de l'adulation, les détermina enfin à donner leur ouvrage avec la
médaille sèche de Louis XIII en tête, sans parler de ce prince qu'en
deux mots et uniquement pour marquer que sa mort fit place à son fils
sur le trône. Les réflexions sur ce genre d'iniquité mèneraient trop
loin. Elle ne fut pas étendue à mon égard\,; je demeurai sous le silence
qui m'avait été promis.

Chamillart faisait affaires sur affaires\,: il fallait fournir aux
dépenses immenses des armées. Vendôme, conduit par M. du Maine, qui
l'était lui-même par M\textsuperscript{me} de Maintenon, envoyait
continuellement des courriers pour vanter sa vigilance, ses projets, et
surtout pour grossir les bagatelles que le voisinage des quartiers
ennemis produisait assez souvent, et toujours fort légèrement avec les
nôtres. Le comte d'Estrées, revenu de Naples à Toulon, vint faire un
tour de huit jours à Paris. Il reçut les ordres du roi pour aller
prendre le roi d'Espagne à Barcelone, et le conduire à Naples, revenir
incontinent après à Toulon, où le comte de Toulouse devait se rendre
pour aller à la mer et faire pour la première fois sa charge d'amiral.
Cette déclaration, qui pourtant n'était qu'une suite de sa charge, et
qui n'avait rien de commun avec la terre, ne laissa pas d'être un
renouvellement de douleur pour M. le duc d'Orléans et les deux princes
du sang. En même temps, le maréchal de Boufflers fut choisi pour
commander l'armée de Flandre sous Mgr le duc de Bourgogne, où le marquis
de Bedmar commanda les troupes d'Espagne. Le maréchal d'Estrées fut
envoyé en Bretagne\,; et Chamillart, ami de Chamilly, ou plutôt leurs
deux femmes, prit occasion de l'oisiveté où on le laissait avec
injustice, pour le remettre à flot, et lui procura le commandement de la
Rochelle et des provinces voisines jusqu'au Poitou inclus, chacun avec
quelques officiers généraux sous eux. Beuvron et Matignon allèrent en
Normandie.

Pour l'armée du Rhin, il fallut avoir recours à Catinat. Il était
presque toujours depuis son retour d'Italie à sa petite maison de
Saint-Gratien, par delà Saint-Denis, où il ne voyait que sa famille et
ses amis particuliers en très petit nombre, portant l'injustice avec
sagesse et le peu de compte qu'on avait tenu de lui depuis son retour
d'Italie. Chamillart lui manda, qu'il avait ordre du roi de
l'entretenir. Catinat vint chez lui à Paris\,; il y apprit sa
destination\,; il s'en défendit\,; la dispute fut longue\,; il ne se
rendit qu'avec une extrême peine et par la nécessité seule de
l'obéissance. Le lendemain matin, 11 mars, il se trouva à la fin du
lever du roi, qui le fit entrer dans son cabinet. La conversation fut
amiable de la part du roi, sérieuse et respectueuse de celle de Catinat.
Le roi, qui s'en aperçut bien, le voulut ouvrir davantage, lui parla
d'Italie et le pressa de s'expliquer avec lui à cœur ouvert de ce qu'il
s'y était passé. Catinat s'en excusa, répondit que c'étaient toutes
choses passées, très inutiles maintenant à son service, uniquement
bonnes à lui donner mauvaise opinion de gens dont il avait paru qu'il
aimait à se servir, et au reste à nourrir les inimitiés éternelles. Le
roi admira cette sagesse et cette vertu, mais il voulut néanmoins
approfondir certaines choses, tant par rapport à. justifier son propre
mécontentement du maréchal que pour démêler qui de lui ou de son
ministre avait eu tort, pour les rapprocher ensuite dans la nécessité du
commerce que le commandement de l'armée leur allait donner ensemble. Il
allégua donc à Catinat des faits importants, les uns dont il n'avait
rendu aucun compte, d'autres qu'il avait entièrement tus et qui lui
étaient revenus d'ailleurs.

Catinat, qui par sa conversation de la veille avec Chamillart avait eu
soupçon que le roi lui en dirait quelque chose, avait apporté ses
papiers à Versailles. Sûr de son fait, il maintint au roi qu'il ne lui
avait rien tu, ni manqué à rendre à lui-même ou à Chamillart un compte
détaillé de ces mêmes choses dont le roi lui parlait alors, et le
supplia avec instance de permettre à un de ces garçons bleus qui sont
toujours dans les cabinets d'aller chez lui chercher sa cassette sans
que lui-même en sortît, d'où il lui tirerait les preuves des vérités
qu'il avançait, et que Chamillart, s'il était présent, n'oserait
désavouer. Le roi le prit au mot et envoya quérir Chamillart.

Le roi en tiers leur remit ce qui venait de se passer entre lui et
Catinat. Chamillart répondit d'une voix assez embarrassée qu'il n'était
pas besoin d'attendre la cassette de Catinat, parce qu'il convenait
qu'il accusait vrai en tout et partout. Le roi bien étonné lui reprocha
l'infidélité de son silence, et d'avoir causé par sa confiance en lui
l'extrême mécontentement qu'il avait eu de Catinat. Chamillart, les yeux
bas, laissa dire, mais comme il sentit que la colère s'allumait\,:
«\,Sire, dit-il, vous avez raison, mais ce n'est pas ma faute. --- Et de
qui donc\,? reprit le roi vivement\,; est-ce la mienne ? --- Non plus,
sire, continua Chamillart en tremblant, mais j'ose vous dire avec la
plus exacte vérité que ce n'est pas aussi la mienne.\,» Le roi insistant
il fallut bien accoucher, et Chamillart lui dit qu'ayant montré les
lettres de Catinat à M\textsuperscript{me} de Maintenon, parce qu'il
jugeait que leur contenu, le même dont le roi reprochait le silence ou
la négligence, lui ferait beaucoup de peine et d'embarras, elle n'avait
jamais voulu qu'elles allassent jusqu'à Sa Majesté, et que lui ayant
insisté qu'il y allait de sa fidélité à ne rien supprimer et à ne rien
ordonner de soi-même, comme venant du roi, et de sa perte si cette faute
si principale venait jamais à être découverte, M\textsuperscript{me} de
Maintenon lui avait répondu de tout, et défendu si étroitement de donner
au roi la moindre connaissance de ces lettres, qu'il n'avait jamais osé
passer outre. Il ajouta que M\textsuperscript{me} de Maintenon n'était
pas loin, et qu'il suppliait le roi de lui demander la vérité de cette
affaire.

À son tour, la roi, plus embarrassé que Chamillart, baissant aussi la
voix, dit qu'il n'était pas concevable jusqu'où M\textsuperscript{me} de
Maintenon portait ses inquiétudes, pour aller au-devant de tout ce qui
pouvait le fâcher\,; et sans plus rien trouver mauvais, se tourna au
maréchal, et lui dit qu'il était ravi d'un éclaircissement qui lui
faisait voir que personne n'avait tort\,; ajouta en général mille choses
gracieuses au maréchal, le pria de bien vivre avec Chamillart, et se
hâta de les quitter, et d'entrer dans ses derniers cabinets.

Catinat, plus honteux de ce qu'il venait de voir et d'entendre, que
content d'une justification si entière, fit des honnêtetés à Chamillart,
qui, encore hors de lui d'une explication si périlleuse, les reçut et
les rendit du mieux qu'il put. Ils ne les prolongèrent pas, ils
sortirent ensemble du cabinet, et le choix de Catinat pour l'armée du
Rhin fut déclaré. Les réflexions se présentent ici d'elles-mêmes. Le roi
vérifia le fait le soir avec M\textsuperscript{me} de Maintenon. Ils
n'en furent que mieux ensemble. Elle approuva Chamillart, mis au pied du
mur, d'avoir tout avoué, et ce ministre n'en fut que mieux traité de
l'un et de l'autre.

Le pape, de qui le roi avait lieu d'être extrêmement content sur Naples
et Sicile, quoiqu'il n'en eût pas encore voulu donner l'investiture au
roi d'Espagne, rendit un jugement dont on ne fut pas satisfait, entre
Madame et l'électeur palatin. Ce prince, chef de la branche palatine de
Neubourg, et frère de l'impératrice, avait succédé au frère de Madame,
mort sans enfants, à l'électorat palatin. Madame était héritière, tant
du mobilier qui allait fort loin, que de ce que l'électeur son frère
pouvait laisser de fiefs féminins. La discussion durait depuis
longtemps, et n'ayant pu être terminée par la paix de Ryswick, le
jugement y avait été renvoyé à l'empereur et au roi, et au cas qu'ils ne
pussent convenir, au pape, pour prononcer la confirmation de la sentence
arbitrale de l'un ou de l'autre monarque. L'abbé de Thésut, frère du
secrétaire des commandements de feu Monsieur, et de M. le duc d'Orléans
ensuite, était à Rome, à la suite de cette affaire, sur laquelle il
avait été diversement prononcé à Vienne et ici, et de sept consulteurs
nommés par le pape, trois furent d'avis de confirmer la sentence rendue
par le roi, et les quatre autres de réduire Madame, pour toutes ses
prétentions, à toucher de l'électeur palatin trois cent mille écus
romains, en défalquant même ce qu'elle pouvait avoir déjà reçu de ce
prince. Le pape embrassa ce dernier avis et y confirma sa sentence
arbitrale. On prétendit ainsi qu'il avait passé son pouvoir, et l'abbé
de Thésut, au nom et comme procureur de Madame, protesta contre ce
jugement d'une manière solennelle.

\hypertarget{chapitre-xxi.}{%
\chapter{CHAPITRE XXI.}\label{chapitre-xxi.}}

1702

~

{\textsc{Mort du roi Guillaume III d'Angleterre.}} {\textsc{- Le roi ne
prend point le deuil du roi Guillaume, et défend aux parents de ce
prince de le porter.}} {\textsc{- Mariage du frère de Chamillart.}}
{\textsc{- Époque d'un usage ridicule.}} {\textsc{- Mort de la marquise
de Gesvres.}} {\textsc{- Mort du comte Bagliani.}} {\textsc{- Mort de
Jean Bart et de La Freselière\,; son caractère.}} {\textsc{- Mort du
marquis de Thianges.}} {\textsc{- États de Catalogne.}} {\textsc{-
Départ du roi d'Espagne pour l'Italie et de la reine pour Madrid par
l'Aragon.}} {\textsc{- Comte d'Estrées grand d'Espagne.}} {\textsc{-
Autres grâces de Philippe V.}} {\textsc{- Cardinal Borgia et sa bulle
d'Alexandre VI.}} {\textsc{- Philippe V à Naples.}} {\textsc{- Cardinal
Grimani.}} {\textsc{- Louville à Rome obtient un légat a latere vers
Philippe V.}} {\textsc{- Cardinal de Médicis.}} {\textsc{- Conspiration
contre la personne de Philippe V.}} {\textsc{- Entrevue de Philippe V et
de la cour de Toscane à Livourne, qui traite le grand-duc d'Altesse.}}
{\textsc{- Entrevue de Philippe V et de la cour de Savoie à
Alexandrie.}} {\textsc{- Fauteuil manqué.}} {\textsc{- Philippe V à
Milan.}} {\textsc{- États d'Aragon.}} {\textsc{- La reine d'Espagne à
Madrid.}} {\textsc{- Junte.}} {\textsc{- Comte de Toulouse va à la
mer.}} {\textsc{- Mgr le duc de Bourgogne va en Flandre.}} {\textsc{-
Ruse en faveur du duc du Maine.}} {\textsc{- Honteux accompagnement de
Mgr le duc de Bourgogne.}} {\textsc{- Passage de Mgr le duc de Bourgogne
par Cambrai.}} {\textsc{- Cent cinquante mille livres au maréchal de
Boufflers.}} {\textsc{- Cinquante mille à Tessé.}} {\textsc{- Bedmar
fait grand d'Espagne\,; son caractère\,; son extraction.}}

~

Le roi Guillaume, tout occupé d'armer l'Europe entière contre la France
et l'Espagne, avait fait un voyage en Hollande, pour mettre la dernière
main à ce grand ouvrage, entamé par lui, dès l'instant qu'il fut informé
des dernières dispositions de Charles II, et il était dans sa maison de
chasse de Loo, au plus fort de cette grande occupation, lorsqu'il y
apprit la mort du roi son beau-père, de la manière que je l'ai racontée,
et la reconnaissance que le roi avait faite du prince de Galles, en
qualité de roi d'Angleterre, qui donna toute liberté au roi Guillaume
d'éclater partout, et d'agir à découvert. Il prit le deuil en violet,
drapa, se hâta d'achever en Hollande tout ce qui assurait cette
formidable ligue, à laquelle ils donnèrent le nom de grande alliance, et
s'en retourna en Angleterre animer la nation, et chercher des secours
pécuniaires dans son parlement.

Ce prince, usé avant l'âge, des travaux et des affaires, qui firent le
tissu de toute sa vie, avec une capacité, une adresse, une supériorité
de génie qui lui acquit la suprême autorité en Hollande, la couronne
d'Angleterre, la confiance, et, pour en dire la vérité, la dictature
parfaite de toute l'Europe, excepté la France, était tombé dans un
épuisement de forces et de santé qui, sans attaquer ni diminuer celle de
l'esprit, ne lui fit rien relâcher des travaux infinis de son cabinet,
et dans une difficulté de respirer qui avait fort augmenté l'asthme
qu'il avait depuis plusieurs années. Il sentait son état, et ce puissant
génie ne le désavouait pas. Il fit faire des consultations aux plus
célèbres médecins de l'Europe sous des noms feints, entre autres une à
Fagon, sous celui d'un curé, lequel, y donnant de bonne foi, la renvoya
sans ménagement et sans conseil autre que celui de se préparer à une
mort prochaine. Le mal augmentant ses progrès, Guillaume consulta de
nouveau, mais à découvert. Fagon, qui le fut, reconnut la maladie du
curé. Il ne changea pas d'avis, mais il fut plus considéré, et
prescrivit avec un savant raisonnement les remèdes qu'il jugea les plus
propres, sinon pour guérir, au moins pour allonger. Ces remèdes furent
suivis et soulagèrent\,; mais enfin, les temps étaient arrivés où
Guillaume devait sentir que les plus grands hommes finissent comme les
plus petits, et voir le néant de ce que le monde appelle les plus
grandes destinées. Il se promenait encore quelquefois à cheval, et il
s'en trouvait soulagé, mais n'ayant plus la force de s'y tenir, par sa
maigreur et sa faiblesse, il fit une chute qui précipita sa fin par sa
secousse. Elle fut aussi peu occupée de religion que l'avait été toute
la suite de sa vie. Il ordonna de tout, et parla à ses ministres et à
ses familiers avec une tranquillité surprenante et une présence d'esprit
qui ne l'abandonna point jusqu'au dernier moment. Quoique accablé de
vomissements et de dévoiement dans les derniers jours de sa vie,
uniquement rempli des choses qui la regardaient, il se vit finir sans
regret avec la satisfaction d'avoir consommé l'affaire de la grande
alliance, à n'en craindre aucune désunion par sa mort, et dans
l'espérance du succès des grands coups que par elle il avait projetés
contre la France. Cette pensée, qui le flatta jusque dans la mort, même,
lui tint lieu de toute consolation\,; consolation frivole et cruellement
trompeuse, qui le laissa bientôt en proie Ci d'éternelles vérités. On le
soutint les deux derniers jours par des liqueurs fortes et des choses
spiritueuses. Sa dernière nourriture fut une tasse de chocolat. Il
mourut le dimanche, 19 mars, sur les dix heures du matin.

La princesse Anne, sa belle-sœur, épouse du prince Georges de Danemark,
fut en même temps proclamée reine. Peu de jours après elle déclara son
mari grand amiral et généralissime, rappela les comtes de Rochester, son
oncle maternel, et de Sunderland, fameux par son esprit et ses
trahisons, dans son conseil, et envoya le comte de Marlborough, si connu
dans la suite, suivre en Hollande tous les plans de son prédécesseur.
Portland s'y retira dès le lendemain de la mort de son maître, et ne
vécut depuis qu'obscurément.

Le roi n'apprit cette mort que le samedi matin suivant par La Vrillière,
à qui il était arrivé un courrier de Calais. Une barque s'était échappée
malgré la vigilance qui avait fermé les ports. Le roi en garda le
silence, excepté à Monseigneur et à M\textsuperscript{me} de Maintenon,
à qui il le manda à Saint-Cyr. Le lendemain la confirmation arriva de
toutes parts, et le roi n'en fit plus un secret, mais il en parla peu et
affecta beaucoup d'indifférence. Dans le souvenir de toutes les folies
indécentes de Paris, lorsque dans la dernière guerre on le crut tué à la
bataille de la Boyne en Irlande, on prit par ses ordres les précautions
nécessaires pour ne pas retomber dans le même inconvénient.

Il déclara seulement qu'il n'en prendrait pas le deuil, et il défendit
aux ducs de Bouillon, aux maréchaux de Duras et de Lorges, et par eux à
tous les parents, de le porter, chose dont il n'y avait pas encore eu
d'exemple. Le prince de Nassau, gouverneur héréditaire de Frise, nommé
héritier par le testament du roi Guillaume, fut, par voie de fait,
frustré de la plus grande partie par l'électeur de Brandebourg, qui
eurent là-dessus des contestations dont les États généraux, exécuteurs
testamentaires, prirent connaissance. L'héritier n'y eut pas beau jeu
contre un prince puissant et avide, et tout à cet égard n'est pas encore
fini entre eux. Le gros de l'Angleterre le pleura et presque toutes les
Provinces-Unies. Quelques bons républicains seulement respirèrent en
secret, dans la joie d'avoir recouvré leur liberté. La grande alliance
fut très sensiblement touchée de cette perle\,; mais elle se trouva si
bien cimentée, que l'esprit de Guillaume continua de l'animer, et
Heinsius, sa créature la plus confidente, élevé par lui au poste de
Pensionnaire de Hollande, le perpétua, et l'inspira à tous les chefs de
cette république, à leurs alliés et à leurs généraux, tellement qu'il ne
parut pas que, Guillaume ne fût plus. M. le prince de Conti, M.
d'Isenghien et plusieurs seigneurs français se présentèrent comme
créanciers ou héritiers de la succession du roi Guillaume, comme prince
d'Orange, qui, outre Orange, avait des terres en Franche-Comté et
ailleurs. Le roi leur permit de suivre leurs prétentions, dont il se
forma plusieurs procès entre eux avec peu de profit pour aucun.

Je ne mettrais pas ici une chose aussi peu considérable que le mariage
du frère de Chamillart, s'il ne servait d'époque à quelque chose
d'extrêmement ridicule, mais que le monde, si souvent glorieux mal à
propos et toutefois toujours si bas et si rampant devant la faveur et la
puissance, a parfaitement adopté en tous les imitateurs depuis de cette
même sottise. Chamillart avait deux frères, qu'on peut dire qui
excellaient en imbécillité\,: l'évêque de Dol, à qui il fit donner
Sentis ensuite, et à qui il fallait donner Condom, et ne l'en laisser
jamais sortir, mais le meilleur homme du monde\,; l'autre, méchant
autant que sa sottise le lui pouvait permettre, et à qui la faveur et le
ministère avaient tourné la tête de vanité. Il s'appelait le chevalier
Chamillart, et il était, je ne sais comment, devenu capitaine de
vaisseau. Son frère, déjà mal avec Pontchartrain, le tira de la marine,
le fit maréchal de camp tout d'un coup, et lui fit épouser la fille
unique de Guyet, maître des requêtes, très riche et très bien faite,
dont il fit le père intendant des finances, qui n'en était pas plus
capable que le marin son gendre des fonctions de maréchal de camp.
Depuis longtemps tout cadet usurpe le nom de chevalier. Il ne pouvait
être porté par un homme marié, celui-ci s'appela donc le comte de
Chamillart. Le \emph{de} s'usurpait aussi par qui voulait depuis quelque
temps, mais de marquiser ou comtiser son nom bourgeois de famille, c'en
fut le premier exemple. En même temps Dreux, gendre de Chamillart,
s'appela le marquis de Dreux. Il eut tort, il fallait prendre le titre
de comte, cela se fût mieux incrusté sur les comtes de Dreux sortis de
la maison royale\,; ce fut sans doute une modestie dont il lui fallut
savoir gré. On en rit tout bas, mais tout haut personne n'osait omettre
les titres ni les \emph{de}, ni leur disputer même dès lors d'être des
capitaines. Maints autres bourgeois ont depuis suivi cet exemple, qui
dans la suite est devenu attaché aux frères des présidents à mortier des
parlements de provinces\,: c'est un apanage apparemment comme Orléans
l'est du frère du roi. Ceux de Paris, qui ne font pas comparaison avec
eux, ont été du temps sans les imiter, quelques-uns enfin se sont
laissés aller à cette friandise.

Le marquis de Gesvres perdit sa femme fort riche et peu heureuse, qui
lui laissa plusieurs enfants. Ce mariage, dans lequel le roi était entré
par bonté pour le marquis de Gesvres, qui n'avait rien, et que son père
haïssait et ruinait, avait tiré Boisfranc, son beau-père, d'affaires
très fâcheuses avec Monsieur, dont il avait été longtemps surintendant,
et d'autres encore de finances avec le roi qui ne valaient pas mieux.

Je perdis aussi en même temps un ancien ami de mon père, le comte
Bagliani qui, depuis près de quarante ans, était envoyé du duc de
Mantoue sans être jamais sorti d'ici. C'était une espèce de colosse en
hauteur et en grosseur, mais d'où sortait tout l'esprit du monde, et
l'esprit le plus délicat et le plus orné. Nos ministres en avaient
toujours fait un cas particulier. Il avait beaucoup d'amis, et il
s'était acquis une considération personnelle fort distinguée de la
médiocrité du caractère dont il était revêtu. Il entendait parfaitement
les intérêts divers de l'Europe\,; il en connaissait les cours et les
intrigues, sans avoir bougé d'ici, et nos ministres lui parlaient
volontiers confidemment en particulier. C'était d'ailleurs un homme
droit, fort à sa place, plein d'honneur, et, sans qu'il y parût, d'une
grande piété depuis grand nombre d'années. Ce fut le dernier des amis
particuliers de mon père, que je cultivai tous jusqu'à leur mort avec
grand soin, et que je regrettai beaucoup.

Le roi fit une perte en la mort du célèbre Jean Bart, qui, a si
longtemps et si glorieusement fait parler de lui à la mer, qu'il n'est
pas besoin que je le fasse connaître. Sa Majesté en fit une autre en la
personne du bonhomme La Freselière, lieutenant général et lieutenant
général de l'artillerie. J'en ai parlé ailleurs\,: il servait encore à
quatre-vingts ans avec la vigilance d'un jeune homme et une capacité
très distinguée. C'était d'ailleurs un homme plein d'honneur et de
valeur, modeste et très homme de bien. Jeunes et vieux le respectaient à
l'armée, et il était si aimable qu'il avait toujours chez lui la
meilleure compagnie de tous âges\,: c'est un rare éloge à quatre-vingts
ans.

Un homme de meilleure maison, et d'une situation bien singulière, mourut
aussi en même temps chez lui en Bourgogne, le marquis de Thianges, du
nom de Damas, dont le père était chevalier de l'ordre. Il avait épousé,
en 1655, la fille aînée du premier duc de Mortemart, sœur du maréchal
duc de Vivonne, de M\textsuperscript{me} de Montespan, qui ne fut mariée
qu'en 1663, et de l'abbesse de Fontevrault. Je réserve ailleurs à parler
de cette famille, pour n'avoir rien à rappeler. Il suffira ici de dire
qu'ayant eu de son mariage un fils et la duchesse de Nevers, sa femme
l'abandonna pour s'attacher à la honteuse faveur de sa saur, dont elle
partagea au moins l'autorité et la confiance sans que leur intimité en
fût jamais blessée, et qu'elle l'imita en n'entendant jamais plus parler
de son mari, dont elle quitta les armes et les livrées pour porter les
siennes seules, comme M\textsuperscript{me} de Montespan avait fait. M.
de Thianges, sans aucune raison commune avec celles de son beau-frère,
mais sentant le mépris d'une femme altière et puissante, se confina chez
lui, où il s'enterra dans l'oisiveté et l'obscurité. Devenu veuf en
1693, et M\textsuperscript{me} de Montespan hors de la cour, il ne crut
pas que ce fût la peine de revenir à Paris, après une absence de tant
d'années, ni de changer une vie où il avoir eu tout le temps de
s'accoutumer. Ses filles n'étaient pas élevées à penser qu'elles avaient
un père\,; lui aussi avait oublié ses filles et son gendre. Son fils
l'allait voir souvent\,; ainsi M. de Thianges mourut dans son château
avec aussi peu de bruit qu'il y avait vécu.

Louville était arrivé à Barcelone, où il avait trouvé les états de
Catalogne finis, ce qui n'était pas arrivé depuis plus d'un siècle.
Après force disputes, ils avaient accordé au roi ce qu'il leur avait
demandé, et s'étaient désistés de plusieurs privilèges qu'ils avaient
tâché d'obtenir. La joie du roi d'Espagne fut grande de n'avoir plus
qu'à se préparer à passer en Italie. La reine partit en même temps qu'il
s'embarqua\,; M\textsuperscript{me} des Ursins la suivit elle passa au
célèbre monastère de Notre-Dame de Mont-Serrat, allant à Saragosse tenir
les états d'Aragon.

Le comte d'Estrées reçut le roi d'Espagne avec tous les honneurs
possibles. Sa petite flotte arbora pavillon d'Espagne. Le vice-amiral
n'avait pas perdu son temps dans les huit jours qu'il avait été à la
cour. Aidé des Noailles et des enfances de sa femme, il avait disposé le
roi à trouver bon qu'il fût fait grand d'Espagne à cette occasion.
Louville était fort bien avec eux tous, et ne fut pas indifférent à se
les acquérir de plus en plus par un si grand service\,: Philippe V en
partant disposa de la vice-royauté du Pérou en faveur de Castel dos
Rios, son ambassadeur, qu'il avait laissé en France, et le roi eut
grande part à cette grâce. L'amirante de Castille, fort suspect, fut
nommé pour le venir relever en la même qualité à Paris\,; et la Toison
fut envoyée à Harcourt et au comte d'Ayen, qui leur était promise il y
avait déjà du temps. En la leur envoyant, ils furent avertis de la
porter au cou, pendue à un ruban couleur de feu ondé, comme on l'a
toujours portée depuis. Quelque mal qu'Harcourt se sentît avec le roi
d'Espagne depuis son retour en France, il s'opiniâtra à ne prendre point
la Toison qu'il voulait faire passer à Cezane, son frère, fort jeune, et
Louville réussit enfin à y faire consentir le roi d'Espagne.

Le cardinal Borgia était du voyage et patriarche des Indes. C'était un
homme très ignorant, fort bas courtisan et tout à fait extraordinaire.
Louville était sur le même bâtiment. Il fut prié à dîner par ce cardinal
le vendredi saint. Jamais homme plus surpris qu'il le fut, lorsque, se
mettant à table, il n'y vit que de la viande. Le cardinal, qui le
remarqua, lui dit qu'il avait dans sa maison une bulle d'Alexandre VI
qui leur donnait la permission de manger de la viande et d'en faire
manger chez eux à tout le monde en quelque jour que ce fût, et
spécialement le vendredi saint. L'autorité d'un si étrange pape, et
aussi étrangement employée, n'imposa pas à la compagnie. Le cardinal se
mit en colère\,; il prétendit que douter du pouvoir de sa bulle était un
crime qui faisait tomber dans l'excommunication. Le respect du jour
l'emporta sur celui de la bulle et sur l'exemple du cardinal, qui mangea
gras et en fit manger à qui il put à force de persécution, de colère et
de menaces d'encourir les censures\,: un abus de ce genre est au-dessus
de toutes les réflexions.

Le samedi saint, Marsin, pour éviter la dépense de l'entrée, prit
caractère à son audience publique sur le vaisseau, pour pouvoir assister
aux chapelles et à toutes les cérémonies. Le jour de Pâques, le roi
débarqua à Pouzzol, donna la clef d'or à Louville, et lit le comte
d'Estrées grand de la première classe. Il y trouva le duc d'Escalona,
vice-roi de Naples, ou, comme on l'appelait souvent, le marquis de
Villena, avec tout ce qu'il y avait de plus distingué à Naples, où le
roi arriva sur ses galères jusque sous son palais. Il se montra sur un
balcon à un peuple infini accouru dans la place, et alla ensuite à une
église voisine, où le \emph{Te Deum} fut chanté. Le cardinal Cantelmi,
archevêque de Naples, et le duc de Popoli, son frère, furent extrêmement
bien recueillis. Ce dernier, venait de recevoir la permission en même
temps que Revel de porter l'ordre du Saint-Esprit, en attendant qu'ils
pussent être reçus. On a vu la part qu'il eut à étouffer dans sa
naissance la révolte de Naples. Torcy en ce même temps alla interroger
le prince de La Riccia à Vincennes et le baron de Sassina à la Bastille,
qui y était extrêmement resserré.

L'empereur avait à Rome chargé de ses affaires le cardinal Grimani, qui,
avec beaucoup d'esprit et de manège, était un scélérat du premier ordre,
et qui ne prenait pas même la peine de se cacher d'être capable de
toutes sortes de crimes et de n'y être pas apprenti, avec cela l'homme
du monde le plus violent, et le plus furieux partisan de la maison
d'Autriche. Tout était à craindre de ses menées. Le prétexte dont lui et
Lisola s'étaient servis pour soulever Naples était que ces peuples ne
pouvaient reconnaître pour leur roi, ni être tenus à. fidélité à un
prince qui n'avait pas l'investiture du pape, d'un royaume qui était
fief de l'Église, quoique le pape eût enjoint aux évêques de ce royaume
de prêcher, faire publier et afficher qu'il reconnaissait Philippe pour
roi de Naples, et qu'il ordonnait à tous les sujets de ce royaume de lui
être fidèles, et lui obéir comme à leur roi légitime, et, tout comme
s'il avait eu déjà son investiture. Il était toujours dangereux qu'un
peuple aussi naturellement léger et séditieux, poussé par beaucoup de
seigneurs puissants aussi légers et aussi amateurs de trouble que ce
peuple, et appuyés et dirigés, par le cardinal Grimani, ne donnât encore
beaucoup d'inquiétude et peut-être d'occupations au dedans, tandis que
les armées en avaient tant en Lombardie.

Ces considérations faisaient extrêmement désirer l'envoi d'un légat
\emph{a latere} dont l'éclat et la solennité fermât la bouche à tous
ceux qui remuaient sous prétexte du défaut d'investiture. Le duc
d'Uzeda, ambassadeur d'Espagne à Rome, sollicitait fortement cette
affaire, le cardinal Grimani et toute sa faction s'y opposait avec
violence et menaces, et le pape, embarrassé, ne pouvait se déterminer.
Louville fut envoyé à Rome pour la presser de la part du roi d'Espagne,
et pour saluer le pape sur l'arrivée de ce prince à Naples et son
voisinage du pape, que l'embarras du cérémonial et les affaires qui
l'appelaient en Lombardie empêchaient de venir lui rendre ses respects
en personne comme il l'eut bien désiré. Louville vint descendre chez le
duc d'Uzeda, qui, pour le mieux appuyer à Rome, l'y donna comme un
favori et comme celui qui avait toute la confiance du roi d'Espagne. Il
fut reçu sur ce pied-là du pape et des cardinaux. Grimani redoubla ses
menaces et ses fureurs jusqu'à dire qu'il ferait poignarder Louville.
S'il crut l'effrayer, il se trompa. Louville en prit occasion de parler
de ce cardinal avec toute la hauteur et l'insulte qu'il méritait, et que
protégeait le caractère de l'autre, de montrer combien ces menaces
étaient injurieuses au pape traité, et retenu avec violence, et à quel
point aussi l'honneur du roi d'Espagne se trouvait engagé dans une
affaire si audacieusement traitée par les Impériaux et en maîtres du
pape et de Rome. En peu de jours il obtint un légat \emph{a latere}. Le
cardinal Grimani menaça de faire des protestations en plein consistoire.
Le pape lui fit dire que si c'était comme ministre de l'empereur,
c'était à lui, non au consistoire qu'il devait s'adresser\,; que si
c'était comme cardinal il lui ordonnait de se taire. Cela l'arrêta tout
court, mais l'ambassadeur de l'empereur sortit de Rome et se retira à
San-Quirico. Le cardinal Charles Barberin, petit-neveu d'Urbain VIII,
fut choisi comme très agréable à la France, où sa famille s'était
réfugiée pendant, la persécution que lui fit Innocent X {[}Pamphile{]},
et où elle fut comblée de grâces et de biens, et d'ailleurs un cardinal
très riche et très magnifique. Il reçut la croix de légat \emph{a
latere} en plein consistoire et partit deux jours après. Le cardinal de
Janson, qui faisait alors les affaires du roi à Rome, servit en cette
affaire avec grande dextérité et une grande fermeté. Le légat fit son
entrée solennelle à Naples entre le cardinal de Médicis et lui.

Médicis était frère du grand-duc\,; c'était le meilleur homme du monde,
le plus sans aucune façon et le plus attaché à la France. Il était venu
à Naples voir Philippe V dès qu'il y fut arrivé. Ils furent si contents
l'un de l'autre, que l'amitié et jusqu'à la familiarité se mit entre
eux. Le roi le traitait avec toutes sortes d'égards, et le cardinal
vivait en courtisan avec lui et avec sa cour. Il ne portait jamais sa
calotte, était vêtu presque en cavalier\,; ses bas rouges étaient toute
sa marque. On ne le voyait que malgré lui vêtu en cardinal et seulement
aux cérémonies. Il ne put quitter Naples tant que Philippe V y fut\,; il
ne se sépara de lui qu'avec larmes à Livourne jusqu'où il l'avait suivi,
et il le revit encore depuis lorsque le roi d'Espagne s'en retourna par
Gènes en quittant l'Italie. Il n'avait point d'ordres sacrés, et, voyant
son neveu sans enfants, il quitta le chapeau dans la suite et se maria à
une Gonzague, sœur du duc de Guastalla. Le légat fut reçu avec tous les
honneurs qui depuis longtemps leur ont été prodigués. Philippe V le
visita, tout se passa avec la plus grande satisfaction réciproque. Comme
il ne s'agissait que de démonstration et d'aucune affaire dans cette
légation, Barberin demeura peu de jours à Naples. Sa venue avait différé
le départ du roi d'Espagne\,; il était pressé d'aller en Lombardie\,; il
partit incontinent après le légat pour aller à Milan et se mettre à la
tête de l'armée.

Cette légation si marquée et si fort emportée malgré l'empereur n'eut
pas le succès pour lequel principalement `on l'avait désirée. Tandis que
Philippe V n'était occupé qu'à répandre des grâces sur les seigneurs et
sur les peuples du royaume de Naples, les privilèges confirmés, les
dettes remises, il se brassait une conspiration conçue à Vienne, tramée
à Rome et prête d'éclater à Naples\,; il ne s'agissait de rien moins que
d'assassiner le roi d'Espagne. Un des conjurés qui le vit le lendemain
de son arrivée fut tellement touché de compassion en le considérant, ou
plutôt si touché par celui qui veille à la conservation des rois, qu'il
prit sur-le-champ la résolution de découvrir le complot. Il s'adressa à
un des officiers de la cour et demanda à parler au roi pour une affaire
très importante et très pressée. On résolut de l'admettre. Il trouva le
roi accompagné seulement de Marsin, des deux seigneurs du
\emph{despacho} et de Louville, et, en leur présence, révéla toute la
conjuration et ceux qui en étaient. Il donna les lettres qu'il avait
apportées, il indiqua des gens travestis en moines et des moines aussi
qui devaient arriver le lendemain par différentes portes. Effectivement,
ils arrivèrent et ils furent arrêtés en entrant dans la ville avec les
lettres dont ils étaient chargés, qui vérifièrent tout ce que leur
camarade avait révélé. On se saisit de plusieurs seigneurs, un plus
grand nombre prit la fuite, les prisons furent remplies de criminels.
Cependant on avait secrètement dépêché à Rome, où on se saisit de la
cassette du baron de Lisola, que l'empereur y tenait avec une sorte de
caractère. Il s'y trouva tant de choses précises sur le projet et
l'exécution, que la cour de Vienne n'osa crier contre cette violence.
Les plus coupables, de toutes qualités, de ceux qu'on avait arrêtés
furent exécutés dans les châteaux de Naples, d'autres envoyés aux Indes,
plusieurs bannis\,; on fit grâce au grand nombre. Tout ce qui n'était
point de la conjuration, seigneurs et peuple\,; en témoigna la plus
grande indignation.

On crut sur cette disposition publique éteindre toute mauvaise volonté
par la clémence, la confiance et les bienfaits. Ils furent poussés
jusqu'à former un régiment des gardes entièrement composé de
Napolitains, officiers et soldats, auxquels le roi déclara qu'il voulait
confier la garde de sa personne. Il fut incontinent sur pied, et le roi
en prit une partie sur le bâtiment qu'il monta et qui le porta à Final.
Je ne sais qui fut auteur de ce conseil et d'une confiance si outrée.
Elle pensa être funeste\,; M. de Vendôme découvrit, par des lettres
interceptées, que des officiers de ce régiment avaient traité avec le
prince Eugène de lui livrer le roi d'Espagne mort ou vif, en le
conduisant à l'armée, appuyés de deux mille chevaux que ce général
devait envoyer secrètement au-devant d'eux, soutenus d'un plus gros
corps pour s'emparer de sa personne. Sur cet avis, quelques-uns de ces
officiers furent observés, pour les arrêter tous\,; mais la crainte
d'être découverts qui les occupait sans cesse leur donna du soupçon.
Presque tous s'enfuirent, on n'en put saisir que peu qui avouèrent
d'abord tout ce que M. de Vendôme avait mandé et ne laissèrent rien
ignorer de cet horrible complot. Le régiment fut aussitôt cassé et
dispersé, et on veilla plus que jamais à la conservation du roi
d'Espagne. J'ai voulu rapporter cette suite sans interruption.

Le roi d'Espagne s'arrêta à Livourne sans coucher à terre où le
grand-duc et toute sa cour l'attendait, et lui fit des présents dignes
d'un grand roi. Il fut reçu avec toutes les marques possibles d'amitié
et de distinction, jusque-là que le roi lui donna l'Altesse. La grande
princesse surtout témoigna une joie extrême et la plus tendre pour ce
prince son neveu. Elle était sœur de M\textsuperscript{me} la Dauphine
sa mère. Philippe V lui témoigna les plus grands égards, beaucoup
d'amitié, et la vit tête à tête. Il ne s'assit en aucune de ces
occasions, et ils se séparèrent avec regret de se quitter. Ce fut là où
le cardinal de Médicis, venu avec le roi et sur son même bâtiment, de
Naples, prit congé de lui. Ils s'en retournèrent tous à Florence charmés
et comblés de tout ce que le roi avait fait dans cette entrevue.

Celle qui suivit ne réussit pas si bien\,: la cour d'Espagne ayant enfin
mis pied à terre à Final, le roi en chaise de posté prit le chemin
d'Alexandrie, où la cour de Savoie s'était rendue. M. de Savoie vint à
quelques, milles au-devant de lui et mit pied à terre dès qu'il aperçut
sa chaise. Le roi le voyant tout proche descendit et l'embrassa après
d'assez courts compliments. Le roi lui fit excuse de ne pouvoir lui
offrir une place dans une si petite voiture, et ajouta qu'il espérait le
revoir dans peu, et lui donner à souper le soir même. Le duc fut
d'autant plus aise de cette invitation, qu'il compta consolider par là
d'une manière plus authentique et plus publique l'usurpation qu'il
s'était adroitement ménagée.

Marsin n'était pas né pour être instruit du cérémonial. Il était poli
jusqu'à la bassesse et, de plus, fort étourdi. M. de Savoie, en le
faisant pressentir sur la manière dont il serait reçu, et ne mettant pas
en doute qu'il n'eût qu'un fauteuil, fit valoir sa déférence de ne
prétendre pas la main, quoique le fameux Charles-Emmanuel eût eu l'une
et l'autre en Espagne où il alla en personne épouser la fille de
Philippe II. Marsin gagné, les deux seigneurs du despacho n'osèrent
s'opposer à son consentement, mais tous trois en firent un secret à
Louville.

Le prince de Vaudemont attendait aussi le roi d'Espagne à Alexandrie. Il
fut averti du fauteuil comme ce prince arrivait et un moment après il
s'en alla chez lui. Il rencontra Louville. En entrant dans
l'appartement, blessé à l'excès de ce fauteuil à cause du duc de
Lorraine son père, pour qui il n'en avait jamais été question en
Espagne, il attaqua Louville là-dessus\,; celui-ci n'en voulait rien
croire, et ne se rendit que lorsque, avançant tous deux dans
l'appartement, ils virent les deux fauteuils préparés, Louville entra
dans le cabinet du roi d'Espagne, où il apprit ce que je viens de
raconter\,; piqué pour la grandeur de son maître peut-être encore du
secret qu'on lui avait fait, {[}il{]} représenta au roi d'Espagne la
différence de la maison de France dont pas un prince du sang ne cédait
aux électeurs ni aux ducs de Savoie comme il était arrivé au même
Charles-Emmanuel à Lyon et à Paris avec le prince de Condé sous Henri
IV, duquel il n'avait jamais prétendu le fauteuil, d'avec la maison
d'Autriche qui ne connaît point, dès qu'on s'assied, de distinction de
sièges, qui donne le fauteuil aux infants, et qui avait traité
Charles-Emmanuel en infant à cause de son mariage\,; que l'électeur de
Bavière à qui M. de Savoie cédait, et avait toujours cédé à Venise où
ils s'étaient trouvés tout un carnaval ensemble, n'avait jamais eu qu'un
tabouret devant le roi Guillaume sans avoir prétendu mieux, quoique
l'empereur lui donnât un fauteuil\,; que ce serait dégrader et sa maison
et sa couronne que d'être la dupe des artifices de M. de Savoie, et de
fonder par cette faiblesse la même prétention pour les électeurs\,; et
sans doute pour d'autres souverains qui ne l'imaginaient pas jusqu'à
cette heure. Avec ces raisons très pertinentes, Louville convainquit le
roi d'Espagne qui ordonna d'ôter les deux fauteuils.

Un demi-quart d'heure après, M. de Savoie arriva, et fut reçu debout\,;
et comme le roi d'Espagne ne parla point de s'asseoir, il sentit bien
qu'il y avait du changement\,; il le voulut sonder jusqu'au bout par le
souper auquel il avait été convié, mais dans le courant de la
conversation, le roi l'en éconduisit par des excuses, sous prétexte que
ses officiers n'étaient pas arrivés. Alors le duc de Savoie comprit
qu'il n'avait plus de fauteuil à espérer. Il ne fit aucun semblant de
s'en apercevoir, abrégea sa visite et s'en alla outré de dépit. Le
lendemain, le roi l'alla voir, et les deux duchesses, avec lesquelles
tout se passa le plus poliment, et même avec une sorte d'ouverture,
surtout avec la fille de Monsieur. M. de Savoie parut respectueux et
fort mesuré. Les quatre ou cinq jours de séjour se passèrent de la
sorte, toujours debout et sans jamais aucun particulier. Au départ du
roi, la cour de Savoie prit congé de lui\,; M. de Savoie lui fit ses
excuses de ne pouvoir faire la campagne comme il l'avait projeté, et
même de ne pouvoir fournir autant de troupes que l'année précédente. Ce
prince ne mit guère {[}de temps{]}, dans Alexandrie même, à découvrir
d'où lui était venu le coup, et il n'oublia rien pour piquer Marsin et
les seigneurs du despacho contre Louville, qui de sa part leur fit
goûter ses excuses de n'avoir pas eu le temps de les avertir avant de
détourner le roi de ce fauteuil. Les deux seigneurs du despacho, qui
n'avaient cédé à Marsin que par crainte, étaient ravis ainsi que tous
les autres grands que ce fauteuil eût avorté, et le bas et timide Marsin
n'osa trouver rien mauvais du favori du roi d'Espagne qui avait toute la
confiance de notre cour. Nous verrons en son lieu que M. de Savoie,
n'ayant pu réussir avec eux, prit d'autres mesures pour se venger de
Louville. Il en fut averti par Phélypeaux, ambassadeur de France à
Turin, sur la fin de la campagne\,; mais la partie fut si bien liée,
qu'eu lieu de la récompense qu'il méritait, il se trouva perdu comme je
le rapporterai en son temps.

M. de Vaudemont suivit le roi d'Espagne à Milan, dont il lui fit
splendidement les honneurs. Ce fut en cette ville que le roi d'Espagne
apprit par M. de Vendôme la conjuration ourdie par ce régiment des
gardes napolitaines que j'ai déjà racontée, l'éclat qui en suivit, et
qui retombait si à plomb sur la cour de Vienne et sur le prince Eugène,
engagea ce dernier à se justifier comme il put par une grande lettre
qu'il écrivit à M. de Vendôme, qu'il lui envoya par un trompette. M. de
Vendôme lui répondit du verbiage honnête, qu'il finit par ces mots
remarquables\,: «\,Qu'il avait trop bonne opinion de lui pour pouvoir
soupçonner qu'il fût capable d'exécuter un si horrible complot quand
bien même il en eût reçu les ordres.\,» Le roi, averti du danger, fit
choisir dans toutes ses troupes six officiers de distinction,
lieutenants-colonels, majors et capitaines, qu'il envoya au roi son
petit-fils pour être toujours autour de lui. C'était en effet des gens
de valeur, de conduite et de probité, et d'une fidélité éprouvée, et
même des gens d'esprit dont quelques-uns l'avaient orné, et tous fort
capables au delà de leur grade. Il est étrange que pas un d'eux n'ait
fait la moindre fortune. C'était don Gaëtano Coppala, prince de
Montefalcone qui était colonel des gardes napolitaines. J'ai voulu
raconter de suite tout ce qui regarde le roi d'Espagne depuis Barcelone
jusqu'à Naples et à Milan.

J'ajouterai que la reine d'Espagne obtint à peu près, ce qu'elle voulut
des états d'Aragon à Saragosse, qui protestèrent sur ce qu'ils ne
devaient être, tenus que par des rois et non par une reine. Elle s'en
alla de là à Madrid, où pour la forme elle fut à la tête de la junte du
gouvernement dont le cardinal Portocarrero était le véritable régent. Ce
fut un grand accueil entre lui et la princesse des Ursins, son ancienne
amie, qui, sous prétexte de former la reine au sérieux et aux affaires,
commença elle-même à s'y initier. Il ne se peut rien ajouter à l'esprit,
aux grâces, à l'affabilité que cette jeune reine montra pendant son
voyage et à son arrivée à Madrid. Le naturel y eut grande part, et la
princesse des Ursins grand honneur par les soins qu'elle prit à la
former. Elle ne s'en donna pas moins à la gagner, et elle y réussit au
delà de ses espérances\,; elle ne fut pas moins heureuse à lui inspirer
le goût du crédit et des affaires. Dans une si grande jeunesse, elle
assista tous les jours à la junte, qui était composée du cardinal
Portocarrero, don Manuel Arias, gouverneur du conseil de Castille, le
duc de Medina-Celi, le marquis de Villafranca, de Mancera et du comte de
Monterey. J'ai parlé suffisamment de tous ces personnages pour les faire
connaître\,; retournons maintenant sur nos pas.

Le comte d'Estrées revenu à Toulon, M. le comte de Toulouse partit pour
s'y rendre accompagné d'O, qui fut fait chef d'escadre. Cheverny,
attaché comme d'O à Mgr le duc de Bourgogne, n'avoir depuis beaucoup
d'années aucune santé pour l'accompagner à la guerre ni pour monter même
un moment à cheval. Tellement que le roi leur joignit en quatrième
Gamaches qu'on avait longtemps appelé Cayeux, qu'il avait mis auprès de
M. le duc d'Orléans avant la mort de Monsieur, et qui depuis était à
louer, parce que ce prince avait une maison, et presque toute celle de
feu Monsieur. Le choix parut encore plus sauvage que la première fois,
mais au moins celui-là avait de l'honneur et de la valeur, il avait été
toute sa vie à la guerre, et y était arrivé au grade de lieutenant
général. Il suivit donc Mgr le duc de Bourgogne avec Saumery, aussi
attaché à. lui, et qui avait été son sous-gouverneur.

Le roi, qui fit servir M. du Maine dans son armée où son ancienneté le
faisait le second lieutenant général, rusa pour qu'il fût le premier\,;
il fit entrer Rosen dans son cabinet qui était le premier et mestre de
camp général de cavalerie, et lui dit qu'il le destinait à être attaché
à la personne de son petit-fils, et à lui servir de conseil pour sa
conduite. Cette proposition, qui ne put être accompagnée que de force
cajoleries, flatta Rosen qui l'accepta. C'était un Allemand rusé et fort
délié sous une apparence et même une affectation de grossièreté et de
manière de reître, qui vit bientôt après à quoi il devait ce choix, et
qui se repentit bien de s'être laissé duper. Il voulait être maréchal de
France\,; il commandait l'aile droite comme premier lieutenant général,
et toute la cavalerie comme mestre de camp général\,: c'était encore lui
que regardaient de droit les détachements considérables qui se pouvaient
faire par des corps séparés. Tout cela le conduisait au bâton, et tout
cela était incompatible avec l'état de mentor du jeune prince qui de
plus avait beaucoup d'épines du côté de la cour et de l'armée. Réflexion
faite, il alla trouver le roi et s'excusa sur son incapacité de
l'honneur qu'il lui voulait faire, et s'en tira si dextrement que le roi
ne put lui savoir mauvais gré.

En sa place le roi mit Artagnan, homme \emph{désinvolte}\footnote{Ce
  mot, traduit de l'italien, désigne un homme dont le corps et l'esprit
  ont une allure vive et dégagée. Le substantif \emph{désinvolture} est
  resté avec le même sens, dans la langue française.}**, et qui
n'entendait pas moins bien les souterrains de la cour que son détail du
régiment des gardes et de major général. Ainsi accompagné, l'héritier
nécessaire de la couronne partit pour la Flandre, n'ayant que Moreau,
son premier valet de chambre, pour l'y servir, y commander et lui
présenter tout le monde. Cette indécence parut si grande à M. de La
Rochefoucauld, que, libre comme il était avec le roi, il ne put
s'empêcher d'en parler au roi à son lever qui ne répondit pas une
parole.

Il était moins occupé de la décoration de son petit-fils que de la
nécessité de son passage par Cambrai, qui ne se pouvait éviter sans
affectation. Il eut de sévères défenses non seulement d'y coucher, mais
de s'y arrêter même pour manger, et pour éviter le plus léger
particulier avec l'archevêque, le roi lui défendit de plus de sortir de
sa chaise. Saumery eut ordre de veiller de près à l'exécution de cet
ordre. Il s'en acquitta en argus avec un air d'autorité qui scandalisa
tout le monde. L'archevêque se trouva à la poste, il s'approcha de la
chaise de son pupille dès qu'elle arriva, et Saumery, qui venait de
mettre pied à terre et lui avait signifié les ordres du roi, fut
toujours à son coude. Le jeune prince attendrit la foule qui
l'environnait par le transport de joie qui lui échappa à travers\,;
toute sa contrainte en apercevant son précepteur. Il l'embrassa à
plusieurs reprises et assez longuement pour se parler quelques mots à
l'oreille, malgré l'importune proximité de Saumery. On ne fit que
relayer, mais sans se presser. Nouvelles embrassades, et on partit sans
qu'on eût dit un mot que de santé, de route et de voyage. La scène avait
été trop publique et trop curieusement remarquée pour n'être pas rendue
de toutes parts. Comme le roi avait été exactement obéi, il ne put
trouver mauvais ce qui s'était pu dérober parmi les embrassades, ni les
regards tendres et expressifs du prince et de l'archevêque. La cour y
fit grande attention et encore plus celle de l'armée. La considération
de l'archevêque qui, malgré sa disgrâce, avait su s'en attirer dans son
diocèse et même dans les Pays-Bas, se communiqua à l'armée, et les gens
qui songeaient à l'avenir prirent depuis leur chemin par Cambrai plus
volontiers que par ailleurs pour aller ou revenir de Flandre.

Mgr le duc de Bourgogne s'arrêta à Bruxelles sept ou huit jours, où tout
ce qu'il y avait de considérable des sujets d'Espagne s'empressa à lui
faire la cour. Enfin il alla se mettre à la tête de l'armée. Mais comme
si on eût voulu accumuler toutes les indécences, ses équipages ne l'y
joignirent, que quinze jours après, en sorte que, depuis son arrivée à
Bruxelles, il fut toujours, lui et son peu de suite, chez le maréchal de
Boufflers et à ses dépens. Le roi lui donna vingt-cinq mille écus pour
cette dépense extraordinaire, et en même temps cinquante mille livres à
Tessé pour la dépense qu'il avait faite pendant le blocus de Mantoue,
duquel je parlerai bientôt.

Bedmar, capitaine général et gouverneur général des Pays-Bas espagnols
par intérim, en l'absence de l'électeur de Bavière, qui était dans ses
États, commandait un corps vers la mer. Il agissait de concert avec le
maréchal de Boufflers, mais au vrai sous ses ordres, quoique cela ne
parût pas, et Mgr le duc de Bourgogne qui avait une patente de
généralissime du roi son frère, commandait en apparence à tous les deux.
Bedmar, bien qu'Espagnol d'illustre naissance, a voit servi toute sa vie
avec beaucoup de valeur, et avait acquis de la capacité à forte d'années
hors de son pays, parmi des italiens et surtout des Flamands où il avait
presque toujours vécu. Il n'avait conservé de sa nation que la probité,
le courage et la dignité, la libéralité et la magnificence\,; du reste
doux, affable, prévenant, poli, ouvert, du commerce le plus commode et
le plus agréable, avec beaucoup d'esprit, et toujours gracieux et
obligeant, il s'était fait aimer et estimer partout, et adorer des
Français depuis qu'ils étoient sous ses ordres. Parfaitement uni avec le
maréchal de Boufflers, bien avec tous les commandants et intendants de
nos frontières, il avait tellement plu au roi, qu'il obtint, sans lui en
avoir rien laissé pressentir, la grandesse de première classe pour lui,
en même temps que le comte d'Estrées reçut la même grâce. Bedmar était
de la maison de Benavidès, mais il portait le nom de La Cueva par cette
coutume des majorasques et des alliances espagnoles dont j'ai parlé à
l'occasion de la grandesse d'Espagne. L'une et l'autre maison ont des
grands. Le duc d'Albuquerque est La Cueva\,; mais il faut remarquer que
cette maison castillane est éteinte depuis bien des siècles, et que
toute la maison de La Cueva descend du mariage de Marie La Cueva avec
Hugues Bertrand qui était François, et dont les enfants quittèrent leur
nom et leurs armes pour prendre le nom seul et les armes
pleines\footnote{Les armes pleines sont celles qui sont entières, d'une
  pièce, sans divisions, brisures, ni mélanges.} de La Cueva. Un
Français de ce nom, qui épouse une telle héritière, pourrait bien être
de cette ancienne maison déjà illustre longtemps avant le maréchal
Robert Bertrand septième du nom, sous le règne de Philippe de Valois. Je
me suis étendu sur le marquis de Bedmar, parce que je l'ai fort vu et
connu en Espagne.

\hypertarget{chapitre-xxii.}{%
\chapter{CHAPITRE XXII.}\label{chapitre-xxii.}}

1702

~

{\textsc{Kaiserswerth assiégé.}} {\textsc{- Déclaration de guerre de
l'Angleterre et de la Hollande.}} {\textsc{- Marlborough, sa femme et
leur fortune.}} {\textsc{- Canonnade de Nimègue, etc.}} {\textsc{-
Places perdues.}} {\textsc{- Retour de Mgr le duc de Bourgogne et du duc
du Maine.}} {\textsc{- Retour du comte de Toulouse.}} {\textsc{-
Varennes commandant de Metz, etc., enlevé, rendu et déplacé.}}
{\textsc{- Blainville lieutenant général, et Brancas, brigadier sortent
de Kaiserswerth.}} {\textsc{- Rouen soustrait à la primatie de Lyon.}}
{\textsc{- Aubercourt et les jésuites condamnés.}} {\textsc{- Grand
prieur veut rendre ses bénéfices, et va servir sous Catinat avec vingt
mille livres de pension.}} {\textsc{- Cinq grands d'Espagne chevaliers
de l'ordre.}} {\textsc{- Rude chute de M. de La Rochefoucauld à la
chasse.}} {\textsc{- M. de Duras perd une prétention contre M. de
Noailles.}} {\textsc{- Époque de mon intime liaison avec M. le duc
d'Orléans.}} {\textsc{- Avances inutiles vers moi de M. et de
M\textsuperscript{me} du Maine.}} {\textsc{- Philippe V à Crémone.}}
{\textsc{- Combat de Luzzara.}} {\textsc{- Marquis de Créqui tué\,: son
caractère.}} {\textsc{- Prince de Commercy fils tué.}} {\textsc{- Autre
conspiration découverte à Naples.}} {\textsc{- Descente inutile de dix
mille Anglais dans l'île de Léon, près Cadix.}} {\textsc{- M. de Vendôme
chevalier de la Toison.}} {\textsc{- Philippe V à Milan et à Gênes,
suivi du cardinal d'Estrées, donne l'Altesse au doge et fait couvrir
quelques sénateurs, à l'exemple de Charles-Quint.}} {\textsc{- Abbé
d'Estrées va en Espagne.}} {\textsc{- Maréchal de Villeroy libre.}}
{\textsc{- Marquis de Legañez vient se purger de soupçon à Versailles.}}
{\textsc{- Amirante de Castille se retire en Portugal.}} {\textsc{-
Cienfuegos, jésuite.}} {\textsc{- Retour des galions {[}qui sont{]}
brûlés par les Anglais dans le port de Vigo, et quinze vaisseaux
français.}} {\textsc{- La reine d'Espagne se fait garder à Madrid,
quoique sans exemple.}}

~

La campagne de Flandre fut triste. L'électeur de Brandebourg et le
landgrave de Hesse assiégèrent Kaiserswerth de bonne heure. Blainville
le défendit à merveille\,: il y eut force combats. L'Angleterre et la
Hollande déclarèrent solennellement la guerre aux deux couronnes\,: leur
armée unie fut commandée par le comte d'Athlone pour les États généraux,
et par le comte de Marlborough pour les Anglais.

C'était milord Churchill, favori du roi Jacques, qui fit son élévation
de très simple gentilhomme qu'il était, et frère de sa maîtresse dont il
eut le duc de Berwick. Jacques lui donna le titre de comte de
Marlborough et une compagnie de ses gardes du corps. Il lui confia aussi
le commandement de ses troupes lors de l'invasion du prince d'Orange,
auquel il l'aurait livré si le comte de Feversham, aussi capitaine de
ses gardes, et frère des maréchaux de Duras et de Lorges, ne l'eût
empêché d'aller à son camp faire une revue, où il eut avis que le piège
était tendu. La femme de Marlborough était de tout temps attachée à la
princesse de Danemark dont elle était favorite et dame d'honneur lorsque
la princesse parvint à la couronne. Elle la confirma dans cette charge,
envoya en même temps son mari en Hollande comme son ambassadeur et,
comme général de l'armée qu'elle y allait former, le fit duc et
chevalier de la Jarretière bientôt après. Il n'y aura que trop
d'occasions de parler de lui dans la suite, à qui nos malheurs donnèrent
un si grand nom.

M. de Boufflers fut accusé d'avoir par incertitude manqué une occasion
heureuse de le battre au commencement de la campagne\,: elle ne se
retrouva plus\,; on subsista dans leur pays. On crut les tenir aux
environs de Nimègue\,: on prétendit qu'on aurait pu encore avoir là un
grand avantage sur eux\,; rien n'en séparait ou presque rien. La
canonnade dura tout le jour\,; on leur prit quelques chariots et
quelques munitions, et on leur tua quelque monde\,: peu à peu ils se
retirèrent sous Nimègue et passèrent de l'autre côté. Kaiserswerth,
Venloo, Ruremonde, la citadelle de Liège et divers petits postes perdus
furent les fruits de leur campagne et les prémices de leur bonheur. Mgr
le duc de Bourgogne marqua beaucoup d'affabilité, d'application et de
valeur\,; mais en tutelle, il ne put que se laisser conduire, se
présenter au feu du canon de bonne grâce, et proposer divers partis qui
marquaient son envie de faire. L'armée n'étant plus en état d'imposer
aux ennemis, il fut rappelé à Versailles, après une autre canonnade
aussi peu décisive que la première, et M. du Maine le suivit de prés. Il
avait eu lieu et occasion de faire valoir sa situation de premier
lieutenant général de l'armée, à quoi Rosen eût été un léger obstacle\,;
M. de Boufflers l'avait espéré, mais elle ne s'y trompa pas. Le roi en
eut une douleur qui renouvela les précédentes\,; il comprit enfin que
les lauriers s'offriraient ingratement à ce fils bien-aimé\,: il prit
avec amertume la résolution de ne le plus exposer à des hasards si peu
de son goût.

Le comte de Toulouse se promena sur la Méditerranée. De la hauteur de
Civita-Vecchia, il envoya d'O complimenter le pape, qui en fut très bien
reçu. Il fut de là passer quelque temps à Palerme et à Messine, où on
lui fit de grands honneurs\,; il y passait les journées à terre, mais il
coucha toujours à bord. Le pape y envoya le complimenter à son tour, sur
ce qu'il fut trouvé que don Juan avait reçu un pareil honneur autrefois.
Le roi y fut fort sensible, et fit tôt après revenir le comte de
Toulouse.

Il le fut fort aussi à l'aventure de Varennes, qui commandait à Metz et
dans tout le pays, et qui, allant sans précaution à Marsal sur la foi de
la neutralité de la Lorraine, fut pris par un parti. On contesta
longtemps de part et d'autre sur cette capture\,: le roi prétendit que
c'était à M. de Lorraine à le faire rendre, qui à la fin en craignit les
suites et obtint sa liberté comme ayant été pris mal à propos. C'était
une manière d'ennuyeux important qui, parce qu'il était fort proche du
maréchal d'Huxelles et de M. le Premier chez qui il logeait, et qui le
protégeaient, avait tout fait et tout mérité, et qui à la valeur près ne
méritait que l'oubli. Il trouva son poste rempli par Locmaria, et ne
servit plus depuis.

Blainville, après plusieurs assauts et un siège soutenu au double de ce
qu'on en devait attendre, à bout d'hommes, de vivres et de munitions, et
ouverts de toutes parts, rendit Kaiserswerth, qu'on n'essaya pas même de
secourir. Il fut fait lieutenant général et le marquis de Brancas
brigadier, à qui nous verrons faire une rare fortune. Il avait fort bien
fait dans cette place à la tête du régiment d'Orléans, où il avait passé
depuis peu de lieutenant de galère qu'il avait été assez longtemps.

Le roi jugea deux procès singuliers. Colbert, archevêque de Rouen,
prétendit soustraire sa métropole à la primatie de Lyon, reconnue par
celles de Tours, de Sens et de Paris\,; Saint-Georges, archevêque de
Lyon, défendit sa juridiction. Les deux prélats étaient savants, et
leurs factums furent curieux, historiques et pièces de bibliothèque.
Pontcarré, maître des requêtes, depuis premier président du parlement de
Rouen, rapporta l'affaire devant des conseillers d'État commissaires,
puis devant le roi, qui y donna deux conseils entiers en un même jour,
et gain de cause à l'archevêque de Rouen.

L'autre affaire fut rapportée par le même aussi devant le roi. Le P.
d'Aubercourt, sorti des jésuites après plusieurs années depuis ses vœux
faits, se prétendit restitué au siècle, et demandait sa portion
héréditaire à sa famille. Les jésuites qui seuls dans l'Église, parmi
les réguliers qui font des vœux, en ont un quatrième qu'ils ne font
faire qu'à qui d'entre eux il leur plaît, et qui y demeure tellement
caché, que le gros des jésuites même ignore ceux qui y ont été admis,
prétendaient n'être point liés à leurs confrères, tandis qu'ils
l'étaient à eux, c'est-à-dire que les jésuites ayant fait les trois vœux
ne pouvaient plus demander à sortir de la compagnie, mais qu'en tout
temps elle était en droit de renvoyer ceux que bon lui semblait, pourvu
qu'ils n'eussent pas fait le quatrième vœu. Conséquemment, que ces
jésuites renvoyés quelquefois, au bout de quinze et vingt années,
étaient en droit de se faire rendre compte du partage de leur bien et de
rentrer en possession de ce qui leur aurait appartenu s'ils fussent
demeurés dans le siècle. Ils avaient tiré d'Henri IV, en 1604, une
déclaration qui semblait favoriser cette prétention. Ils en avaient
toujours su tirer parti lorsque le cas s'en était présenté. La famille
d'Aubercourt se montra plus difficile\,; ils intervinrent pour
Aubercourt et eurent le crédit de faire évoquer l'affaire devant le roi,
où ils crurent mieux trouver leur compte\,; en effet, ils ne se
trompaient pas. Le roi fut tout à fait favorable aux jésuites, et voulut
bien que les juges s'en aperçussent. Pontcarré, qui d'ailleurs était
porté de bonne volonté pour eux et qu'ils avaient eu l'adresse de faire
nommer rapporteur, ne remplit pas leur attente\,; ni lui ni la pluralité
ne chercha point en cette occasion à plaire. La subversion des familles
par ces retours surannés à partage, l'incertitude ruineuse de toutes
celles où il y aurait des jésuites, les détermina. Le chancelier sur
tous parla si fortement, qu'Aubercourt et les jésuites furent condamnés,
et que, pour couper toute racine de prétention, l'édit de 1604 fut
révoqué. Le roi ne voulut pis user d'autorité sur le fond d'un jugement
si important à l'état des familles, mais ne put s'empêcher d'en montrer
son déplaisir à plusieurs reprises, et à la fin de succomber au moins en
quelque chose à son affection pour les jésuites, en faisant ajouter, en
prononçant et de sa pleine puissance, que les jésuites renvoyés de la
compagnie auront une pension viagère de leur famille, statuée par les
juges des lieux. Ce fut néanmoins une grande douleur aux jésuites que
cet arrêt. Aubercourt leur demeura toujours fort attaché, et bientôt
après ils obtinrent pour lui des bénéfices et une abbaye.

Le grand prieur, noyé de dettes, voulut rendre les siens au roi à
condition qu'il y serait mis un économe chargé de payer tout ce qu'il
devait, même après sa mort, jusqu'à parfait acquit. Il fallait le
consentement de Rome pour une condition si étrange. Cela dura et varia
fort longtemps.

M\textsuperscript{me} de Maintenon, par M. du Maine, s'employa si bien
pour lui, qu'il arracha, mais sourdement, une pension de vingt mille
livres, et qu'il obtint vers le milieu de l'été d'aller servir de
lieutenant général dans l'armée du maréchal Catinat.

Le Jour de la Pentecôte, le roi déclara au chapitre cinq grands
d'Espagne chevaliers de l'ordre. Il crut à propos de répandre cet
honneur sur les seigneurs les plus distingués de cette cour par leur
attachement au roi son petit-fils et par leurs charges, et il dit que ce
prince les lui avait demandés. Il fit même pour le cardinal Portocarrero
ce qui était jusqu'alors sans exemple, et qui n'en a pas eu depuis, et
il est vrai qu'il n'y avait point de règle qui ne dût faire hommage à
ses services. Il fut nommé d'avance à la première place de cardinal
vacante qui étaient lors toutes quatre remplies, avec la permission de
porter l'ordre en attendant. Cette distinction fut accompagnée d'une
croix de l'ordre, que le roi lui envoya, de plus de cinquante mille
écus. Les quatre chevaliers furent le marquis de Villafranca,
majordome-major\,; le duc de Medina-Sidonia, grand écuyer\,; le comte de
Benavente, sommelier du corps, c'est-à-dire grand chambellan, et le duc
d'Uzeda, ambassadeur d'Espagne à Rome. J'ai suffisamment parlé des
quatre premiers ci-devant\,; je n'aurai que trop d'occasions de faire
connaître le dernier dans la suite. Je me contenterai présentement de
dire qu'il était Acuña y Pacheco y Sandoval et beau-frère du duc de
Medina-Celi.

M. de La Rochefoucauld, emporté par son cheval à la chasse à Marly, fut
désarçonné et se cassa le bras gauche entre le coude et l'épaule, qu'il
avait eu rompue autrefois au passage du Rhin. Le roi et Monseigneur y
accoururent avec toute sorte d'amitié. Félix lui raccommoda le bras, et
il en fut quitte pour le mal. C'était vers la mi-juillet. M. de
Noailles, premier capitaine des gardes, avait lors le bâton qu'il avait
continué après son quartier pour M. de Duras qui y entrait après lui,
mais qui était malade à Paris, et dont le quartier finissait le dernier
juin. Le quartier de juillet était celui du maréchal de Villeroy qui
avait eu la charge de M. de Luxembourg\,; tellement que M. de Duras,
accoutumé en leur absence à continuer le quartier de juillet après le
sien, se disposait à se trouver à Versailles au retour de Marly
{[}pour{]} y prendre le bâton. C'était entre les grands officiers à qui
servirait, et cet empressement leur tournait à grand mérite. M. de
Noailles, averti du dessein de M. de Duras, représenta au roi qu'ayant
commencé le quartier qui n'était pas celui de M. de Duras, le bâton lui
devait demeurer\,; il avait raison, le roi le jugea ainsi, et manda à M.
de Duras de ne point venir et de ne songer qu'à sa santé il entendit le
français et demeura à Paris.

Je ne m'arrêterais pas à la bagatelle que je vais raconter, si elle
n'était une époque très considérable dans ma vie, et ne marquait de plus
comment des riens ont quelquefois les plus grandes suites. Sur la fin de
ce même mois de juillet, le roi fit un voyage à Marly.
M\textsuperscript{me} la duchesse d'Orléans, ravie de la liberté et de
la grandeur personnelle, qu'elle trouvait par la mort de Monsieur, eut
envie d'en jouir et d'aller tenir une cour à Saint-Cloud. Le roi
l'approuva, pourvu qu'elle y eût une compagnie honorable et point mêlée,
sinon de ce reste de la cour la plus particulière de feu Monsieur qui ne
se pouvait exclure. Il y avait déjà longtemps que ce projet était fait,
et entre les dames de la cour qu'elle engagea à être de ce voyage, elle
en pressa M\textsuperscript{me} de Saint-Simon qui le lui promit.
Cependant nous voulûmes aller à la Ferté y passer six semaines.
M\textsuperscript{me} la duchesse d'Orléans, qui sur l'arrangement des
Marly avait enfin ajusté à peu près son voyage de Saint-Cloud, vit qu'il
se trouverait pendant le nôtre, et ne voulut point laisser partir
M\textsuperscript{me} de Saint-Simon qu'elle ne lui eût promis de
revenir de la Ferté à Saint-Cloud le jour même qu'elle irait, dont elle
la ferait avertir. En effet la duchesse de Villeroy lui écrivit de sa
part à la Ferté et M\textsuperscript{me} de Saint-Simon se rendit à
Saint-Cloud comme elle l'avait promis. La compagnie était bien choisie,
les plaisirs et les amusements furent continuels. M. et
M\textsuperscript{me} la duchesse d'Orléans firent très poliment les
honneurs de ce beau lieu\,; la magnificence et la liberté rendirent le
séjour charmant, et pour la première fois Saint-Cloud se vit sans
tracasseries. On a vu au commencement de ces Mémoires, que, dès ma plus
petite jeunesse, j'avais fort vu M. le duc d'Orléans. Cette familiarité
dura jusqu'à ce qu'il fût tout à fait entré dans le monde, et même
jusqu'après la campagne de 1693, où il commandait la cavalerie de
l'armée de M. le duc de Luxembourg où je servais. Plus il avait été tenu
de court, plus il se piqua de libertinage. La vie peu réglée de M. le
Duc et de M. le prince de Conti lui donna une triste émulation\,; les
débauchés de la cour et de la ville s'emparèrent de lui\,; le dégoût
d'un mariage forcé et si inégal lui fit chercher à se dédommager par
d'autres plaisirs, et le dépit qu'il conçut de se voir éloigné du
commandement des armées et trompé sur ce qui lui avait été promis de
gouvernements et d'autres grâces acheva de le précipiter dans une
conduite fort licencieuse, qu'il se piqua de porter au plus loin pour
marquer le mépris qu'il faisait de son épouse et de la colère que le roi
lui en témoignait. Cette vie qui ne pouvait cadrer avec la mienne me
retira de ce prince\,: je ne le voyais plus qu'aux occasions rares et
des moments, par bienséance. Depuis six ou sept ans, je le rencontrais
peu dans les mêmes lieux. Quand cela se trouvait, il avait toujours pour
moi un air ouvert, mais ma vie ne lui convenait pas plus qu'à moi la
sienne, tellement que la séparation était devenue entière. La mort de
Monsieur, qui par nécessité l'avait ramené au roi et à
M\textsuperscript{me} sa femme, n'avait pu rompre, ses engagements de
plaisirs. Il se conduisait plus honnêtement avec elle et plus
respectueusement avec le roi, mais le pli de la débauche était pris,
elle lui était entrée dans la tête comme un bel air qui convenait à son
âge et qui lui donnait un relief opposé au ridicule qu'il concevait dans
une vie moins désordonnée. Il admirait les plus outrés et les plus
persévérants dans la plus forte débauche, et ce léger changement à
l'égard de la cour n'en apporta ni à ses mœurs ni à ses parties obscures
à Paris, où elles le faisaient aller et venir continuellement. Il n'est
pas temps encore de donner une idée de ce prince que nous verrons si
fort sur le théâtre du monde, et en de si différentes situations.

M\textsuperscript{me} de Fontaine-Martel était à Saint-Cloud\,: c'était
une de ces dames de l'ancienne cour familière de Monsieur, et toute sa
vie extrêmement du grand monde. Elle était femme du premier écuyer de
M\textsuperscript{me} la duchesse d'Orléans, frère du feu marquis
d'Arcy, dernier gouverneur de M. le duc d'Orléans, pour qui il se piqua
toujours d'une estime, d'une amitié et d'une reconnaissance qu'il
témoigna par une considération toujours soutenue pour toute sa famille,
et même jusqu'à ceux de ses domestiques qu'il avait connus, il leur fit
du bien. M\textsuperscript{me} de Fontaine-Martel, par la charge de son
mari, goutteux, qu'on ne voyait guère, passait sa vie à la cour. Elle
était des voyages, et même quelquefois de ceux de Marly\,; elle soupait
souvent chez M. le maréchal de Lorges, qui tenait soir et matin une
table grande et délicate, où sans prier il avait toujours nombreuse
compagnie et de la meilleure de la cour, et M\textsuperscript{me} la
maréchale de Lorges l'y attirait beaucoup par son talent particulier de
savoir tenir et bien faire les honneurs d'une grande maison sans tomber
dans aucun des inconvénients qui, par la nécessité du mélange que fait
un grand abord, rendent une maison moins respectée par des facilités qui
n'eurent jamais entrée dans celle-là. J'y étais poli à tout le monde,
mais tout le monde ne me revenait pas, ni moi par conséquent à chacun. À
force de nous voir, M\textsuperscript{me} de Fontaine-Martel et moi,
nous nous accommodâmes l'un de l'autre et cette amitié dura toujours
depuis. Elle me demandait quelquefois pourquoi je ne voyais plus M. le
duc d'Orléans, et disait toujours que cela était ridicule de part et
d'autre, parce que, malgré la diversité de notre vie, nous nous
convenions l'un et l'autre par mille endroits. Je riais et la laissais
dire. Un beau jour à Saint-Cloud, elle attaqua M. le duc d'Orléans sur
la même chose\,; tandis qu'il causait avec elle, la duchesse de Villeroy
et M\textsuperscript{me} de Saint-Simon, tous trois se mirent à dire
mille choses obligeantes de moi, et M. le duc d'Orléans ses regrets de
ce que je le trouvais trop libertin pour le voir, et son désir de
renouer avec moi. Cela fut poussé le reste du voyage jusqu'à regretter
qu'il fût trop près de sa fin pour me convier d'y venir et pour se
promettre à mon retour à Versailles de vaincre, comme disait M. le duc
d'Orléans, mon austérité. M\textsuperscript{me} de Saint-Simon fut priée
de m'en écrire\,; je répondis, comme je le devais. Elle revint à la
Ferté, et me dit que les choses étaient au point de ne pouvoir m'en
défendre.

J'avais pris tout cela comme une fantaisie de M\textsuperscript{me} de
Fontaine-Martel, et une politesse de M. le duc d'Orléans, comme de ces
parties ou de ces projets qui ne s'exécutent point\,; et la différence
de goût et de vie me persuadait que ce prince et moi ne nous convenions
plus, et que je ferais bien de m'en tenir où j'étais, en faisant tout au
plus à mon retour une visite de remerciement et de respect\,: je me
trompai. Cette visite qu'à mon retour je différais toujours, et dont M.
le duc d'Orléans faisait des reproches à ces dames chez
M\textsuperscript{me} la duchesse d'Orléans, fut reçue, avec
empressement. Soit que l'ancienne amitié de jeunesse eût repris, soit
désir d'avoir quelqu'un à voir familièrement à Versailles, où il se
trouvait fort souvent désœuvré, tout se passa de si bonne grâce de sa
part, que je crus me retrouver en notre ancien Palais-Royal. Il me pria
de le voir souvent\,; il pressa mes visites, oserai-je dire qu'il se
vanta de mon retour à lui, et qu'il n'oublia rien pour me rattacher. Le
retour de l'ancienne amitié de ma part fut le fruit de tant d'avances
dont il m'honorait, et la confiance entière en devint bientôt le sceau
qui a duré jusqu'à la fin de sa vie sans lacune, malgré les courtes
interruptions qu'y ont quelquefois mises les intrigues, quand il fut
devenu le maître de l'État. Telle fut l'époque de cette liaison intime
qui m'a exposé à des dangers, qui m'a fait figurer un temps dans le
monde, et que j'oserai dire avec vérité qui n'a pas été moins utile au
prince qu'au serviteur, et de laquelle il n'a tenu qu'à M. le duc
d'Orléans de tirer de plus grands avantages.

Il faut ici ajouter une autre bagatelle, parce que j'ai cru lui devoir
des suites directement contraires à celles dont je viens de parler, et
qui ont fort croisé ma vie\,; quoiqu'elle soit d'une date un peu
postérieure, je la raconterai tout de suite, parce que ces différentes
suites ont eu un contraste d'un continuel rapport dans beaucoup de
choses ou curieuses ou importantes, qui se verront ici dans la suite. M.
de Lauzun, toujours occupé de la cour, et toujours affligé profondément
de se voir éloigné de son ancienne faveur, ne se lassait point de remuer
toutes pierres pour s'en rapprocher\,; il mit en œuvre ses anciennes
liaisons avec M\textsuperscript{me} d'Heudicourt du temps de
M\textsuperscript{me} de Montespan, et ses cessions à M. du Maine, pour
sortir de Pignerol, dans l'esprit de se servir d'eux auprès de
M\textsuperscript{me} de Maintenon, et par elle auprès du roi. Il essaya
de faire l'une la gouvernante et la protectrice de la Jeunesse de sa
femme, pour fa mettre de tout à la cour, et l'initia chez
M\textsuperscript{me} du Maine. Outre les agréments qu'il comptait lui
procurer et qui réussirent pour elle, il se flattait d'arriver lui-même
à son but. Sa femme, jeune, gaie, sage, aimable, fut fort goûtée. Le
gros jeu qu'il lui faisait jouer, et où elle fut heureuse, la rendait
souvent nécessaire. M\textsuperscript{me} du Maine ne s'en pouvait
passer, et elle était sans cesse à Sceaux avec elle. M. du Maine
cherchait à lui attirer bonne compagnie\,: il voulut faire en sorte
d'accrocher aussi M\textsuperscript{me} de Saint-Simon par sa sœur.
C'était un moyen de plaire, elle s'y laissa aller, mais non pas avec
assiduité. J'eus lieu de croire que M. et M\textsuperscript{me} du Maine
avaient formé le projet de me gagner\,; ils n'ignoraient pas combien
leur rang me déplaisait. Par moi-même je n'étais rien moins qu'à
craindre\,; mais la politique qui, dans l'inquiétude de ce qui peut
arriver, cherche à tout gagner, leur persuada, je pense, de s'ôter en
moi une épine qui pourrait peut-être les piquer un jour. Ils se mirent
sur mes louanges avec ma femme et ma belle-sœur, ils leur témoignèrent
le désir qu'ils avaient de me voir à Sceaux, enfin ils leur proposèrent
tantôt à l'une tantôt à l'autre de m'y amener, et les pressèrent de m'en
convier de leur part.

Surpris d'une chose si peu attendue de la part de gens avec qui je
n'avais jamais eu le moindre commerce, je me doutai de ce qui les
conduisait, et cela même me tint sur mes gardes. Je ne pouvais
m'accommoder de ce rang nouveau\,; je sentais en moi-même un désir de le
voir éteindre, qui me donnait celui de pouvoir y contribuer un jour\,;
je le sentais tel à n'y pouvoir résister. Comment donc lier un commerce
et se défendre de le tourner en amitié, avec des gens qui me faisaient
tant d'avances, et en apparence si gratuites, en situation de me
raccommoder avec le roi, et que tout me faisait sentir qu'ils se
voulaient acquérir sur moi des obligations à m'attacher à eux, et
comment céder à leur amitié et se soumettre à en recevoir des marques,
en conservant cette aversion de leur, rang et cette résolution de le
faire renverser si jamais cela se trouvait possible\,? La probité, la
droiture rie se pouvait accommoder de cette duplicité. J'eus beau me
sonder, réfléchir sur ma situation présente, nulle faveur ne m'était
comparable à consentir à la durée de ce rang et à renoncer à l'espérance
de travailler à m'en délivrer. Je demeurai donc ferme dans mes
compliments et mes refuites. Je tins bon contre les messages en forme
qu'ils m'envoyèrent, contre les reproches les plus désireux que m'en fit
M\textsuperscript{me} du plaine, à qui jamais je n'avais parlé, et qui
s'arrêta à moi dans l'appartement du roi, et je les lassai enfin dans
leurs poursuites. Ils sentirent que je ne vouloir me prêter à aucune
liaison avec eux\,; ils en furent d'autant plus piqués qu'ils n'en
firent aucun semblant et redoublèrent, au contraire, à l'égard de
M\textsuperscript{me} de Saint-Simon.

J'ai toujours cru que M. du Maine me voulut nuire dès lors, qu'il me mit
mal dans l'esprit de M\textsuperscript{me} de Maintenon, de qui je
n'étais connu en aucune sorte, et que je n'ai su que depuis la mort du
roi, qu'elle me haïssait parfaitement. Ce fut Chamillart qui me le dit
alors\,; et qu'il en avait eu des prises avec elle, pour me remettre en
selle auprès du roi par des Marly et des choses de cette nature. Je me
doutais bien par tout ce qui me revenait qu'elle m'était peu favorable,
mais je ne sus pas, tant que le roi vécut, ce que j'en appris depuis.
Chamillart sagement ne me voulut pas donner d'inquiétude, ni moins
encore m'ouvrir la bouche trop facile et trop libre sur ceux que je
croyais ne devoir pas aimer, et peu retenu par leur grandeur ni leur
puissance. Pour achever ce qui me regardé, pour lors avec M. du Maine,
assez longtemps après, M\textsuperscript{me} la duchesse de Bourgogne
retint à Marly M\textsuperscript{me} de Lauzun à jouer le jour qu'on en
partait, et que, venue avec M\textsuperscript{me} du Maine, elle devait
s'en retourner avec elle. Cette excuse qu'elle allégua n'arrêta point
M\textsuperscript{me} la duchesse de Bourgogne, qui lui dit de mander à
M\textsuperscript{me} du Maine\,: qu'elle la ramènerait.
M\textsuperscript{me} du Maine eut la folie de s'en piquer assez pour en
faire le lendemain une telle sortie à la duchesse de Lauzun, qu'elle
sortit de chez elle pour n'y rentrer de sa vie. M. du Maine vint chez
elle aux pardons. M. le Prince aux excuses. Ils tournèrent M. de Lauzun
de toutes les façons, il était presque rendu, mais sa femme ne put être
persuadée.

Je fus ravi d'une occasion si naturelle et si honnête pour
M\textsuperscript{me} de Saint-Simon de se tirer d'un lieu où la
compagnie peu à peu s'était plus que mêlée, et où sûrement depuis ce que
j'ai raconté, il n'y avait rien à gagner pour nous, et depuis ce
temps-là elle ne vit plus M\textsuperscript{me} du Maine qu'aux
occasions, quoiqu'elle et M. du Maine n'eussent rien oublié pour
l'empêcher de se retirer d'eux à cette occasion. Je pense qu'elle acheva
de me mettre mal avec eux, s'il y a voit lors à y ajouter. Depuis cette
aventure, M\textsuperscript{me} la duchesse de Bourgogne mena toujours
M\textsuperscript{me} de Lauzun à Marly\,; c'était une distinction et
qui piqua extrêmement M\textsuperscript{me} du Maine. Enfin, quelques
années après, M. du Maine et M. de Lauzun voulurent finir cette
brouillerie, et convinrent que M\textsuperscript{me} du Maine ferait des
excuses à M\textsuperscript{me} de Lauzun chez M\textsuperscript{me} la
Princesse à Versailles, qu'elles seraient reçues honnêtement, et que
deux jours après M\textsuperscript{me} de Lauzun irait chez
M\textsuperscript{me} du Maine\,: cela fut exécuté de la sorte et bien.
M. du Maine se trouva chez M\textsuperscript{me} sa femme lorsque
M\textsuperscript{me} de Lauzun y vint, pour tâcher d'ôter l'embarras et
d'égayer la conversation\,; M\textsuperscript{me} de Lauzun en demeura à
cette visite, et la vit depuis uniquement aux occasions\,; conséquemment
M\textsuperscript{me} de Saint-Simon de même. Tout ce narré, qui semble
maintenant inutile, retrouvera dans la suite un usage important.

De Milan où le duc de Saint-Pierre régala le roi d'Espagne d'un opéra
superbe à ses dépens, ce prince vint à Crémone, où M. de Vendôme le vint
saluer le 14 juillet. M. de Mantoue et le duc de Parme y vinrent aussi
lui faire la révérence\,; tous trois y firent peu de séjour. Les deux
derniers retournèrent à Casal et à Parme, le premier à son armée, dans
le dessein de la mener vis-à-vis de Casal-Maggiore et d'y faire un pont,
tant pour la communication avec le prince de Vaudemont que pour y faire
passer le roi d'Espagne pour se mettre à la tête de l'armée de M. de
Vendôme. Les marches, le passage du Crostolo, l'exécution de venir à
bout de faire lever le long blocus de Mantoue, retardèrent l'arrivée de
M. de Vendôme au rendez-vous, qui fut même changé, et le pont fait un
peu plus bas que sa destination première. Le 29 juillet, jour que le roi
d'Espagne devait joindre l'armée avec neuf escadrons. M. de Vendôme
surprit Visconti, campé avec trois mille chevaux à Santa-Vittoria, le
culbuta, le défit, prit ses barrages et son camp tout tendu, fit un
grand carnage, force prisonniers, et presque tout le reste qui s'enfuit
se précipita de fort haut dans un gros ruisseau qui en fut comblé. Le
roi d'Espagne, qui avait hâté sa marche, laissa sa cavalerie derrière
pour arriver plus vite au feu qu'il entendait, et ne le put que tout à
la fin de l'action. Les mouvements de nos armées obligèrent le prince
Eugène de quitter le Serraglio. Zurlauben sortit de Mantoue, rasa leurs
forts et leurs retranchements, et acheva de mettre cette place en
liberté.

Pendant ces divers campements, Marsin, toujours occupé de plaire, fit
déclarer par le roi d'Espagne M. de Vendôme conseiller d'État,
c'est-à-dire ministre, et le fit asseoir au despacho au-dessus de tous.
Cette séance ne plut pas aux grands d'Espagne\,; le duc d'Ossone et
quelque autre s'était dispensé de suivre le roi d'Espagne à la fin de
l'action de ces trois mille chevaux dont je viens de parler\,; presque
tous les autres Espagnols s'y distinguèrent, et le duc de Mantoue, qui
était revenu faire sa cour au roi d'Espagne et l'accompagner jusqu'à
l'armée, y frit aussi fort bien, quoiqu'on pût croire qu'il ne
s'attendait pas à cette aventure, et qu'il s'en serait très bien passé.
Le roi d'Espagne manda au roi ce fait du duc d'Ossone, des autres
Espagnols et de M. de Mantoue.

Après plusieurs campements de part et d'autre, et la jonction de Médavy
avec un gros détachement des troupes du prince de Vaudemont, M. de
Vendôme voulut prendre le camp de Luzzara, petit bourg au pied d'un fort
long rideau. Le prince Eugène, qui avait le même dessein, y marcha de
son côté, tellement que le 15 août les deux armées arrivèrent sur les
quatre heures après midi, chacune au pied de ce rideau, sans avoir le
moindre soupçon l'une de l'autre, ce qui paraît un prodige, et ne
s'aperçurent que lorsque de part et d'autre les premières troupes
commencèrent à monter la pente peu sensible de ce rideau. Qui attaqua
les premiers, c'est ce qui ne se peut dire, mais dans un instant tout
prit poste des deux côtés et se chargea pour s'en chasser. Jamais combat
si vif, si chaud, si disputé, si acharné\,; jamais tant de valeur de
toutes parts, jamais une résistance si opiniâtre, jamais un feu ni dés
efforts si continuels, jamais de succès si incertain\,; la nuit finit le
combat, chacun se retira un très petit espace et demeura toute la nuit
sous les armes, le champ de bataille demeurant vide entredeux et Luzzara
derrière notre armée, mais tout proche.

Le roi d'Espagne se tint longtemps au plus grand feu avec une
tranquillité parfaite\,; il regardait de tous côtés les attaques
réciproques dans ce terrain étroit et fort coupé, où l'infanterie même
avait peine à se manier, et où la cavalerie derrière elle ne pouvait
agir. Il riait assez souvent de la peur qu'il croyait remarquer dans
quelques-uns de sa suite\,; et ce qui est surprenant, avec une valeur si
bien prouvée, sans curiosité d'aller çà et là voir ce qui se passait en
différents endroits. À la fin Louville se proposa de se retirer plus bas
sous des arbres, où il ne serait pas si exposé au soleil, mais en effet
parce qu'il y serait plus à couvert du feu. Il y alla et y demeura avec
le même flegme. Louville, après l'y avoir placé, s'en alla voir de plus
près ce qui se passait, et tout à la fin revint au roi d'Espagne, à qui
il proposa de se rapprocher, et qui ne se le fit pas dire deux fois,
pour se montrer aux troupes. Marsin ne demeura pas un moment auprès de
lui, prit son poste de lieutenant général et s'y distingua fort. Les
deux généraux opposés y firent merveilles. L'émulation les transportait,
et la présence du roi d'Espagne fut un aiguillon au prince Eugène, qui
dans le souvenir de la bataille de Pavie, lui fit faire des prodiges.

Le carnage fut grand de part et d'autre, et fort peu de prisonniers. Le
marquis de Créqui, lieutenant général, y fut tué. C'était le seul fils
du feu maréchal de Créqui et gendre du duc d'Aumont, sans enfants. Sa
probité ni sa bonté ne le firent regretter de personne, mais bien ses
talents à la guerre, où il était parvenu à une grande capacité par son
application et son travail\,; sa valeur était également solide et
brillante, son coup d'œill juste et distinctif. Tout se présentait à lui
avec netteté, et, quoique ardent et dur, il ne laissait pas d'être sage.
C'était un homme qui touchait au bâton et qui l'aurait porté aussi
dignement que son père. Il avait été fort galant, et on voyait encore
qu'il avait dû l'être. Avec cela beaucoup d'esprit, plus d'ambition
encore, et tous moyens bons pour la satisfaire. Les Impériaux y
perdirent les deux premiers généraux de leur armée après le prince
Eugène, le prince de Commercy fut tué, et le prince Thomas de Vaudemont
survécut deux ans à sa blessure. Ils n'étaient point mariés, tous deux
feld-maréchaux, et le dernier, fils unique du prince de Vaudemont,
gouverneur général du Milanais pour le roi d'Espagne, à qui ce fut une
grande douleur. Celle de M\textsuperscript{me} de Lislebonne et de ses
deux filles fut extrême. Il n'avait devant lui que le prince Eugène. Il
y avait plus de vingt ans qu'elles ne l'avaient vu, et selon toute
apparence ne le devaient jamais revoir. Monseigneur prit des soins
d'elles qui relevèrent encore leur considération. Il ne l'ut occupé qu'à
les consoler. Quelque accoutumé qu'on doive être dans les cours aux
choses singulières, ce soin du Dauphin d'une douleur qui devait demeurer
cachée se fit fort remarquer. Ce fut le duc de Villeroy qui en apporta
la nouvelle, et qui peu de jours après retourna en Italie lieutenant
général.

Sitôt que le jour parut, le lendemain de l'action, les armées, se
trouvèrent si proches qu'elles se mirent à se retrancher, et qu'il y eut
encore bien des tués et des blessés de coups perdus. Aucune des deux ne
voulut se retirer devant l'autre. Chaque jour augmentait les
retranchements et les précautions. Il fallut même changer le roi
d'Espagne de chambre, parce qu'il n'y était pas en sûreté du feu, et il
ne fut question que de subsistances chacun par ses derrières, et de
s'accommoder le mieux qu'on put dans les deux camps, où les deux armées
subsistèrent longtemps avec un péril et une vigilance continuels. On
compta avoir perdu trois mille hommes et les ennemis beaucoup plus. Ce
combat fut enfin suivi d'un cartel en Italie.

J'oubliais de dire, sur la conspiration que j'ai rapportée contre la
personne du roi d'Espagne, que le vice-roi de Naples en découvrit une à
Naples qui devait s'exécuter en cadence de l'autre. Un envoyé de Venise
très suspect, et gagné par le cardinal Grimani, l'avait tramée, et
venait d'être rappelé à la prière du roi à sa république. Force moines
furent arrêtés, et le duc de Noja Caraffa et le prince de Trebesaccio
qui en étaient les chefs. Ils avaient vingt-cinq complices, chacun de
quelque considération dans leur état. Le projet était de se saisir
d'abord du tourion\footnote{Petite tour, tourelle.} des Carmes. Le duc
de Medina-Celi, qui, en revenant de Naples en Espagne, était venu faire
la révérence au roi, et que M. de Torcy avait fort entretenu, lui avait
nommé plusieurs seigneurs napolitains suspects qui se trouvèrent depuis
de cette conspiration, qui fut d'abord étouffée et plusieurs complices
punis.

Pour continuer de suite la même matière d'Espagne, le duc d'Ormond, avec
une grosse escadre, essaya de surprendre Cadix fort dégarni. Il s'y jeta
fort à propos quelques bâtiments français chargés pour l'Amérique. Les
ennemis débarquèrent, et, ne trouvant rien devant eux, s'établirent dans
l'île de Léon, dix mille hommes, et leurs vaisseaux demeurés à la rade.
Ils firent des courses et par leur pillage, surtout des églises,
achevèrent d'indisposer le pays. On ne saurait croire avec quel zèle
tout s'offrit, tout monta à cheval, tout marcha contre eux. Ils y
subsistèrent pourtant près de deux mois, espérant émouvoir le pays et
ramasser les partisans de la maison d'Autriche. Qui que ce soit ne
branla. Enfin, Villadarias y marcha avec ce qu'on put ramasser de
troupes, don, l'ardeur était extrême. Le 27 octobre, les Anglais et les
Hollandais regagnèrent leurs vaisseaux, vivement poursuivis dans leur
retraite. Ils y perdirent assez de monde, et beaucoup en maraude et de
maladies pendant leur séjour. Cette expédition leur fut inutile. Ils
retournèrent en leurs ports fort déchargés d'hommes et d'argent et fort
désabusés des espérances que M. de Darmstadt leur avait données d'un
soulèvement général en Espagne, dès qu'on les y verrait en état de
l'appuyer, et qui était avec eux.

Il se passe peu de choses en Italie le reste de la campagne. M. de
Vendôme prit Guastalla, où le roi d'Espagne vit fort les travaux. Le 28
septembre il partit pour aller à Milan, et, en disant adieu à M. de
Vendôme, il lui donna le collier de l'ordre de la Toison d'or. Le
cardinal d'Estrées vint de Rome joindre le roi d'Espagne, qui s'embarqua
à Gênes pour la Provence, et de là aller par terre en Espagne suivi du
même cardinal, où l'abbé d'Estrées son neveu eut ordre d'aller le
trouver, pour y être chargé sous lui des affaires du roi en la place de
Marsin, qui avait instamment demandé son retour, et qui quitta le roi
d'Espagne à Perpignan, dont il refusa la grandesse et la Toison, pour
que cela ne tirât pas à conséquence pour les autres ambassadeurs de
France, à ce qu'il écrivit au roi. Il n'était point marié, était fort
pauvre, très nouveau lieutenant général\,; il voulait une fortune en
France\,; il l'espéra de ce refus\,; on verra bientôt qu'il n'y fut pas
trompé. À Gênes, Philippe V, sur l'exemple de Charles-Quint, traita le
doge d'Altesse, et fit couvrir quelques sénateurs.

Le roi eut en ce même temps nouvelle du maréchal de Villeroy qu'il
allait être libre en conséquence du cartel, dont Sa Majesté témoigna une
grande joie. Il donna aussi une longue audience au marquis de Legañez,
venu exprès d'Espagne pour se justifier sur son attachement à la maison
d'Autriche, et beaucoup de choses qui lui avaient été imputées en
conséquence, sur lesquelles le roi parut d'autant plus content de lui,
que la lenteur de son voyage avait fait douter de son arrivée. Celle de
l'amirante de Castille n'eut pas la même issue. J'ai ailleurs fait
connaître ce seigneur, et il n'y a pas longtemps que j'ai dit que les
soupçons qu'on avait toujours sur lui l'avaient fait choisir pour
succéder à l'ambassadeur d'Espagne en France, nommé vice-roi du Pérou.
L'amirante accepta, fit de grands et lents préparatifs, partit le plus
tard qu'il put, et marcha à pas de tortue. Il était accompagné de son
bâtard, de plusieurs gentilshommes de sa confiance et du jésuite
Cienfuegos, son confesseur. Il avait pris avec lui toutes ses
pierreries, ce qu'il avait pu d'argent, et mis à couvert argent et
effets. Comme il approcha de la Navarre, il disparut avec ceux que je
viens de nommer, et par des routes détournées où il avait secrètement
disposé des relais, il gagna la frontière de Portugal avant que la
nouvelle de sa fuite, portée à Madrid, eût donné le temps de le pouvoir
rattraper. Il eut tout lieu de se repentir d'avoir pris ce conseil, et
son jésuite de se remercier de l'avoir donné. Il lui valut enfin la
pourpre, l'archevêché de Montréal en Sicile et la
comprotection\footnote{Les principaux États avaient à Rome un cardinal
  \emph{protecteur}, qui était chargé de la défense de leurs intérêts.
  Lorsqu'un cardinal était associé au protecteur d'un État, on appelait
  sa charge \emph{comprotection}.} d'Allemagne, dont il jouit près de
vingt ans.

Cependant les galions, retardés de près de deux années, étaient désirés
avec une extrême impatience. Châteaurenauld les était allé chercher. Il
les trouva très richement chargés, et les amena avec son escadre. Il
envoya aux ordres, et voulait entrer dans nos ports. On craignit la
jalousie des Espagnols, qui néanmoins étaient de toutes les nations
commerçantes celle qui avait le moindre intérêt à leur changement\,; on
n'osa les confier au port de Cadix, et ils furent conduits dans le port
de Vigo, qui n'en est pas éloigné, et qu'on avait fortifié de plusieurs
ouvrages. Renauld, dont je parlerai en son lieu, eut beau représenter le
danger de ce lieu et la facilité d'y recevoir le plus fatal dommage, et
soutenir la préférence de Cadix, il ne fut pas écouté, et on ne pensa
partout qu'à se réjouir de l'heureux retour si désiré des galions, et
des richesses qu'ils apportaient. On ne laissa pas de prendre la sage
précaution de transporter le plus tôt qu'on put tout l'or, l'argent, et
les effets les plus précieux et les plus aisés à remuer, à plus de
trente lieues dans les terres, à Lugo.

On y était encore occupé, lorsque les ennemis arrivèrent, débarquèrent,
s'emparèrent des forts qu'on avait faits à Vigo, et des batteries qui en
défendaient l'entrée, forcèrent l'estacade qu'on y avait faite,
rompirent la chaîne qui fermait le port, brûlèrent les quinze vaisseaux
de Châteaurenauld, à la plupart desquels lui-même avait fait mettre le
feu, et tous ceux que les Espagnols y avaient ramenés des Indes, dont
quelques-uns, en petit nombre, furent coulés à fond. Il n'y avait point
de troupes ni de moyens d'empêcher ce désastre\,; il était bien demeuré
encore pour huit millions de marchandises sur ces vaisseaux. Ce malheur
arriva le 23 octobre, et répandit une grande consternation.
Châteaurenauld ramassa ce qu'il put de matelots de la flotte, de milices
et quelques soldats du pays à Saint-Jacques de Compostelle, pour se
jeter dans les défilés entre Vigo et Lugo, d'où on transporta tout à
Madrid avec une infinité de bœufs et de mulets.

La reine d'Espagne, quelque temps auparavant, s'était trouvée fort
inquiétée plusieurs nuits de beaucoup de bruits dans le palais de
Madrid, et jusqu'autour de son appartement. Elle s'en plaignit à la
junte, et demanda des gardes pour sa sûreté. Jamais les rois d'Espagne
n'avaient eu que quelques hallebardiers dans l'intérieur du palais, qui
le plus souvent y demandaient l'aumône, et quand ils sortaient en
cérémonie, quelques lanciers fort mal vêtus. Cette nouveauté de donner
des gardes à la reine reçut donc beaucoup de difficultés, mais enfin lui
fut accordée.

\hypertarget{note-i.-portraits-du-roi-philippe-v-de-la-reine-louise-de-savoie-et-des-principaux-seigneurs-du-conseil-de-philippe-v-tracuxe9s-par-le-duc-de-grammont-alors-ambassadeur-en-espagne.}{%
\chapter{NOTE I. PORTRAITS DU ROI PHILIPPE V, DE LA REINE LOUISE DE
SAVOIE ET DES PRINCIPAUX SEIGNEURS DU CONSEIL DE PHILIPPE V, TRACÉS PAR
LE DUC DE GRAMMONT, ALORS AMBASSADEUR EN
ESPAGNE.}\label{note-i.-portraits-du-roi-philippe-v-de-la-reine-louise-de-savoie-et-des-principaux-seigneurs-du-conseil-de-philippe-v-tracuxe9s-par-le-duc-de-grammont-alors-ambassadeur-en-espagne.}}

Voici\footnote{Bibl. imp. du Louvre, ms, F, 325, t. XXI, pièce 29. Copie
  du temps. En lisant ces portraits, tracés par le duc de Grammont, on
  ne doit pas oublier ce que Saint-Simon dit du caractère de cet
  ambassadeur et de son peu de succès en Espagne. Cela contribue à
  expliquer la causticité de Grammont. On trouve d'ailleurs dans ces
  portraits la confirmation de ce que dit Saint-Simon de son antipathie
  contre la princesse des Ursins.} le portrait juste et au naturel du
roi d'Espagne, de la reine et de la plupart des grands que j'ai connus à
Madrid\,:

«\,Le roi d'Espagne a de l'esprit et du bon sens. Il pense toujours
juste, et parle de même\,; il est de naturel doux et bon, et incapable
par lui-même de faire le mal\,; mais timide, faible et paresseux à
l'excès. Sa faiblesse et sa crainte pour la reine sont à tel point que,
bien qu'il soit né vertueux, il manquera sans balancer à sa parole, pour
peu qu'il s'aperçoive que ce soit un moyen de lui plaire. Je l'ai
éprouvé en plus d'une occasion. Ainsi l'on peut m'en croire, et tabler
une fois pour toutes que, tarit que le roi d'Espagne aura la reine, ce
ne sera qu'un enfant de six ans, et jamais un homme.

«\,La reine a de l'esprit au-dessus d'une personne de son âge. Elle est
fière, superbe, dissimulée, indéchiffrable, hautaine, ne pardonnant
jamais. Elle n'aime, à seize ans, ni la musique, ni la comédie, ni la
conversation, ni la promenade, ni la chasse, en un mot aucun des
amusements d'une personne de son âge\,; elle ne veut que maîtriser
souverainement, tenir le roi son mari toujours en brassière, et dépendre
le moins qu'il lui est possible du roi son grand-père\,: voilà son génie
et son caractère. Quiconque la prendra différemment ne l'a jamais
connue.

«\,Veragua est la superbe même\footnote{Voy. p.~5 et 91 de ce volume.}\,;
il est ingénieux, plein d'artifice et d'esprit, et tel qu'il convient
d'être pour parvenir au grade de favori de la princesse. Il hait la
France souverainement, et autant que l'Espagne le méprise, qui est tout
dire.

«\,Aguilar est à peu près de ce même caractère, et pour qu'il fût
content et bien à son aise, il faudrait que la nation française fût
éteinte en Espagne\footnote{\emph{Ibid}., p.~125 et 126.}.

«\,Medina-Celi a la gloire de Lucifer, la tête pleine de vent et d'idées
chimériques. De son mérite, je n'en parle pas, j'en laisse le soin aux
historiens de Naples. Il se dit attaché au roi et à la France\,; mais sa
conduite tous les jours le dément.

\footnote{\emph{Ibid}. p.~7.}

«\,Monterey ne manque pas aussi de sens pour les affaires\,; mais c'est
une girouette qui tourne à tous vents, qui condamne tout et ne remédie à
rien\footnote{\emph{Ibid}., p.~126.}. Il a beaucoup de confrères en ce
monde.

«\,Mancera est un des plus raffinés ministres que j'aie connus\,; mais
rien ne tient contre quatre-vingt-douze ans, et il faut bien à la fin
que l'esprit et le bon sens cèdent à l'extrême vieillesse\footnote{\emph{Ibid}.,
  p.~6, 7 et 121.}.

«\,Arias est une des meilleures têtes qu'il y ait en Espagne. Il est
incorruptible et sa vertu est toute romaine. Il aime l'État et la
personne du roi d'Espagne, et a une vénération toute particulière pour
le roi\footnote{\emph{Ibid}., p. 6.}. Il vit comme un ange dans son
diocèse, et est généralement aimé et respecté de tout le monde dans
Séville. Son seul mérite est la cause de sa disgrâce.

«\,Le cardinal Portocarrero est un homme de talents fort médiocres, mais
d'une grande probité, fidèle et uniquement attaché à son maître, haut et
ferme pour le bien de l'État, allant toujours à ce qui peut contribuer à
sa conservation, esclave de sa parole, et qui mérite une grande
distinction à tous égards possibles\footnote{\emph{Ibid}., p.~4 et 5.}.
C'est celui qui a mis la couronna sur la tête du roi, qui, envers et
contre tous, la lui a conservée, et celui qui, pour avoir eu le malheur
de déplaire à M\textsuperscript{me} des Ursins, est traité avec honte et
ignominie\,; ce qui fait gémir le peuple et la noblesse.

«\,Medina-Sidonia\footnote{\emph{Ibid}., p.~7.} ne manque pas
d'intelligence\,; il est très galant homme, incorruptible et attaché au
roi d'Espagne de même que l'ombre l'est au corps. Il est à naître qu'il
ait reçu des grâces, et sa persécution est extrême, parce que l'on a
imaginé que sa femme, qui n'y a jamais songé, aspirait à être
camarera-mayor. L'on jugera aisément de l'effet que cela produit.

«\,San-Estevan est un petit finesseux, plein de souterrains, et
attendant le parti le plus fort pour s'y déterminer et s'y
joindre\footnote{Voy., p.~5 de ce volume.}.

«\,Benavente est un homme plein d'honneur, ennemi de cabale et
d'intrigue, ne connaissant que son devoir et son maître\footnote{\emph{Ibid}.,
  p.~7.}.

«\,L'Infantado est un jeune homme qui ne se mêle de rien. L'on peut dire
de lui qu'il n'est ni chair ni poisson, et je suis très persuadé qu'il
n'a jamais mérité les bottes qu'on lui a données. Il ne veut que la paix
et le repos, et n'est pas capable d'autre chose.

«\,Villafranca est un des Espagnols les plus vertueux qu'il y ait
ici\footnote{\emph{Ibid}., p.~6.}. Il est vrai en tout, plein de zèle et
de fidélité pour le roi son maître. Personne ne désire plus ardemment
que lui, ni avec plus de sagesse, que l'entier gouvernement de cette
monarchie passe promptement des mains où il est, en celles du roi, et
que rien ne se décide que par sa volonté absolue. C'est là le bon
sens\,; tout le reste n'étant que plâtrage et ne conduisant qu'à
perdition.

«\,Lémos est une bête brute, tout à fait incapable de l'emploi qu'il
exerce, et que la faveur de sa femme auprès de M\textsuperscript{me} des
Ursins lui a fait obtenir\footnote{\emph{Ibid}., p.~90, sur la maison de
  Lémos.}.

«\,Rivas est capable d'un grand travail. Il a des talents, de l'esprit
et de l'intelligence, beaucoup de facilité pour les affaires, de la
pénétration et une mémoire étonnante. Avec ces dispositions, il semble
qu'il pourrait servir très utilement\,; mais les qualités de son cœur
entraînent peut-être malgré lui celles de son esprit. Il est né fourbe,
et ne sait ce que c'est que de se conduire en rien avec droiture\,; il
dorme des paroles, mais il ne fait pas profession de les garder, et
quand la chose doit servir à ses intérêts, il ne se fait pas scrupule de
nier qu'il les ait données. Il est fort intéressé, et l'intérêt du roi
et celui de l'État ne peuvent jamais entrer en considération avec le
sien. Uniquement occupé de son élévation et de son opulence, il perd
aisément de vue les intérêts de son maître. Ce qui a fait que, dans bien
des rencontres, il a paru travailler contre lui\,; et, tout compté,
comme le mauvais qui est en sa personne est bien plus dangereux que son
bon ne peut être utile, je conclus par décider que gens de son caractère
ne peuvent jamais être mis en place.

«\,Voilà le caractère fidèle des principaux personnages qui composent
cette cour, que j'ai connus à fond et fort pratiqués.\,»

\hypertarget{note-ii.-intendants-lieutenants-civil-criminel-de-police-pruxe9vuxf4t.-des-marchands.}{%
\chapter{NOTE II. INTENDANTS, LIEUTENANTS CIVIL, CRIMINEL, DE POLICE,
PRÉVÔT. DES
MARCHANDS.}\label{note-ii.-intendants-lieutenants-civil-criminel-de-police-pruxe9vuxf4t.-des-marchands.}}

Saint-Simon parle, souvent dans ses Mémoires, et notamment dans ce
volume, des intendants des généralités, des lieutenants civil, criminel,
de police, des prévôts des marchands, etc. Comme ces termes ne sont plus
en usage et qu'ils ne présentent pas toujours au lecteur un sens précis,
il ne sera pas inutile de rappeler l'origine de ces dignités et les
fonctions qui y étaient attachées.

I. INTENDANTS.

Les intendants étaient des magistrats que le roi envoyait dans les
diverses parties du royaume pour y veiller à tout ce qui intéressait
l'administration de la justice, de la police et des finances, pour y
maintenir le bon ordre et y exécuter les commissions que le roi ou son
conseil leur donnaient. C'est de là que leur vint le nom
d'\emph{intendants de justice, de police et finances et commissaires
départis dans les généralités du royaume pour l'exécution des ordres du
roi}\footnote{Voy., pour les détails, le \emph{Traité des Offices} de
  Guyot., t. III. p.~119.}

L'institution des intendants ne date que du ministère de Richelieu.
Cependant on en trouve le principe dans les maîtres des requêtes, qui
étaient chargées, au XVIe siècle, de faire, dans les provinces, les
inspections appelée \emph{chevauchées}. Un rôle du 23 mai 1555 prouve
que les maîtres des requêtes étaient presque tous employés à ces
chevauchées. En effet, de vingt-quatre qu'ils étaient alors, le roi n'en
retint que quatre auprès de lui\,; les vingt autres furent envoyés dans
les provinces. Le titre de ce rôle mérite, d'être cité\,: \emph{C'est le
département des chevauchées que MM. les maîtres des requêtes de l'hôtel
ont à faire en cette présente année, que nous avons départis par les
recettes générales, afin qu'ils puissent plus facilement servir et
entendre à la justice et aux finances, ainsi que le roi le veut et
entend qu'ils fassent}.

Ce fut seulement à l'époque de Richelieu que le nom d'\emph{intendants}
commença à être donné aux maîtres des requêtes chargés de l'inspection
des provinces. On trouve, dès 1628, le maître ales requêtes Servien,
désigné sous le nom d'\emph{intendant de justice et police} en Guyenne,
et chargé de faire le procès à des Rochelais accusés de lèse-majesté, de
piraterie, de rébellion et d'intelligence avec les Anglais. Le parlement
de Bordeaux s'opposa à la juridiction de l'intendant, et rendit, le 5
mai, un arrêt, par lequel il fit défense à Servien et à tous les autres
officiers du roi de prendre la qualité d'\emph{intendants de justice et
police} en Guyenne, et d'exercer, dans le ressort de la cour, aucune
commission, sans au préalable l'avoir fait signifier et enregistrer au
parlement. Servien n'en continua pas moins l'instruction du procès.
Alors intervint un nouvel arrêt du parlement de Bordeaux, en date du 17
mai 1628, portant que Servien et le procureur du roi de l'amirauté de
Languedoc seraient assignés à comparaître en personne, pour répondre aux
conclusions du procureur général. Ce nouvel arrêt n'eut pas plus d'effet
que le précédent. Le 9 juin, le parlement de Bordeaux en, rendit un
troisième, portant que «\,certaine ordonnance du sieur Servien, rendue
en exécution de son jugement, serait lacérée et brûlée par l'exécuteur
de la haute justice, et lui pris au corps, ses biens saisis et annotés,
et qu'où il ne pourrait être appréhendé, il serait assigné au poteau.\,»
Le conseil du roi cassa ces trois arrêts comme attentatoires à
l'autorité royale, et ceux qui les avaient signés furent cités à
comparaître devant le roi, pour rendre compte de leur conduite.

À Paris, les parlements firent retentir leurs plaintes jusque dans
l'assemblée des notables. Ils disaient au roi\footnote{Ces doléances des
  parlements se trouvent dans un manuscrit de la Bibliothèque de
  l'Université, H, I, 8, fol.~205.} en 1626\,: «\,Reçoivent vos
parlements grand préjudice d'un nouvel usage d'intendants de justice,
qui sont envoyés ès ressort et étendue desdits parlements près MM. les
gouverneurs et lieutenants généraux de Votre Majesté en ces provinces,
ou qui, sur autres sujets, résident en ficelles plusieurs années,
fonctions qu'ils veulent tenir à vie\,; ce qui est, sans édit, tenir un
chef et officier supernuméraire de justice créé sans payer finance,
exauctorant (abaissant) les chefs des compagnies subalternes,
surchargeant vos finances d'appointements, formant une espèce de
justice, faisant appeler les parties en vertu de leurs mandements et
tenant greffiers, dont surviennent divers inconvénients, et, entre
autres, de soustraire de la juridiction, censure et vigilance de vos
parlements, les officiers des sénéchaussées, bailliages, prévôtés et
autres juges subalternes. Ils prennent encore connaissance de divers
faits, dont ils attirent à votre conseil les appellations au préjudice
de la juridiction ordinaire de vos parlements. C'est pourquoi Votre
Majesté est très humblement suppliée de les révoquer, et que telles
fonctions ne soient désormais faites sous prétexte d'intendance ou
autrement, sauf et sans préjudice du pouvoir attribué par les
ordonnances aux maîtres des requêtes de votre hôtel faisant leurs
chevauchées dans les provinces, tant que pour icelles leur séjour le
requerra.\,» Heureusement Richelieu avait l'âme trop ferme et l'esprit
trop pénétrant pour céder à ces remontrances. Il lui tallait dans les
provinces des administrateurs qui dépendissent immédiatement de son
pouvoir\,; il les trouva dans les intendants, dont il rendit
l'institution permanente à partir de 1635.

Les intendants n'appartenaient pas, comme les gouverneurs, à des
familles puissantes\,; ils pouvaient être révoqués à volonté, et
étaient, par conséquent, les instruments dociles du ministre dans les
provinces. De là la haine des grands et des parlements, qui, à l'époque
de la Fronde, réclamèrent vivement et obtinrent la suppression des
intendants (déclaration du 13 juillet 1648). Mais la cour, qui n'avait
cédé qu'à la dernière extrémité, se sentait par cette suppression
\emph{blessée à la prunelle de l'œil}, comme dit le cardinal de Retz\,;
elle maintint des intendants en Languedoc, Bourgogne, Provence,
Lyonnais, Picardie et Champagne. Rétablis en 1654, les intendants furent
institués successivement dans toutes les généralités. Le Béarn et la
Bretagne furent les dernières provinces soumises à leur
administration\,: le Béarn en 1682, la Bretagne en 1689. Saint-Simon, en
parlant de Pomereu ou Pomereuil, rappelle que ce fut le premier
intendant «\,qu'on ait hasardé d'envoyer en Bretagne, et qui trouva
moyen d'y apprivoiser la province.\,» Avant la révolution, de 1789, il y
avait en France trente-deux intendances, savoir\,: Paris, Amiens,
Soissons, Orléans, Bourges, Lyon\,; Dombes, la Rochelle, Moulins, Riom,
Poitiers, Limoges, Tours, Bordeaux, Auch, Montauban, Champagne, Rouen,
Alençon, Caen, Bretagne, Provence, Languedoc, Roussillon, Bourgogne,
Franche-Comté, Dauphiné, Metz, Alsace, Flandre, Artois, Hainaut.

Les intendants avaient de vastes et importantes attributions\,: ils
avaient droit de juridiction et l'exerçaient dans toutes les affaires
civiles et criminelles, pour lesquelles ils recevaient une commission
émanant du roi. On pourrait citer un grand nombre de procès jugés par
les intendants\,; je me bornerai à renvoyer aux notes placées à la fin
du cinquième volume de cette édition des Mémoires de Saint-Simon. On y
verra que le procès de B. Fargues fut instruit par l'intendant Machaut,
qui le jugea en dernier ressort et le condamna à la peine capitale.
Guyot cite, dans son \emph{Traité des Offices}\footnote{T. III, p.~134
  et suiv.}**, beaucoup d'autres procès qui furent jugés par les
intendants. Du reste, ces magistrats n'exerçaient les fonctions
judiciaires que temporairement, en en vertu des pouvoirs extraordinaires
que leur conférait la royauté. Leurs attributions ordinaires étaient
surtout administratives.

Ils étaient chargés de surveiller les protestants, et, depuis la
révocation de l'édit de Nantes (1685), ils avaient l'administration des
biens des religionnaires qui quittaient le royaume. Les juifs, qui
légalement n'étaient tolérés que dans la province d'Alsace, étaient
aussi soumis à la surveillance des intendants. Ces magistrats
prononçaient sur toutes les questions concernant les fabriques des
églises paroissiales, et étaient chargés de pourvoir à l'entretien et à
la réparation de ces églises, ainsi qu'au logement des curés. Toutes les
questions financières qui touchaient aux églises étaient de leur
compétence. Ils avaient la surveillance des universités, collèges,
bibliothèques publiques. L'agriculture et tout ce qui s'y rattache,
plantation de vignes, pépinières royales, défrichements et
dessèchements, haras, bestiaux, écoles vétérinaires, eaux et forêts,
chasse, etc., commerce, manufactures\,; arts et métiers, voies
publiques, navigation, corporations industrielles, imprimerie,
librairie, enrôlement des troupes, revues, approvisionnement des armées,
casernes, étapes, hôpitaux militaires, logement des gens de guerre,
transport des bagages, solde des troupes, fortifications des places et
arsenaux, génie militaire, poudres et salpêtres, classement des marins,
levée et organisation des canonniers garde-côtes, désertion, conseils de
guerre, milices bourgeoises, police, service de la maréchaussée,
construction des édifices publics, postes, mendicité et vagabondage,
administration municipale, nomination des officiers municipaux,
administration des biens communaux, conservation des titres des villes,
revenus municipaux, domaines, aides, finances, amendes, droits de
greffe, droits du sceau dans les chancelleries, contrôle des actes et
exploits, etc.\,: tels étaient les principaux objets dont s'occupaient
les intendants. On peut juger par là de la puissance de ces magistrats,
auxquels Saint-Simon compare les corrégidors de Madrid, en ajoutant que
ces magistrats espagnols réunissaient à des fonctions si importantes
celles des lieutenants civil, criminel et de police, des maires ou
prévôts des marchands.

\begin{enumerate}
\def\labelenumi{\Roman{enumi}.}
\setcounter{enumi}{1}
\tightlist
\item
  LIEUTENANTS CIVIL, CRIMINEL, DE POLICE\,; PRÉVÔT DES MARCHANDS.
\end{enumerate}

Le mot \emph{lieutenant} désignait souvent, dans l'ancienne monarchie,
un magistrat qui présidait un tribunal subalterne (présidial, bailliage,
etc.), en l'absence du bailli, prévôt ou sénéchal. Ces derniers étaient
presque toujours des hommes d'épée, qui, dans l'origine avaient cumulé
les fonctions militaires, financières et judiciaires\,; mais, à mesure
que l'administration était devenue plus compliquée, une seule personne
n'avait pu remplir des fonctions aussi diverses. Les baillis, prévôts ou
sénéchaux, avaient conservé la présidence nominale des tribunaux, mais
on leur avait adjoint des lieutenants qui devaient être gradués en droit
et qui rendaient la justice en leur nom. Les lieutenants civil et
criminel tiraient leur nom de ce qu'ils présidaient l'un la chambre
civile, l'autre la chambre criminelle du Châtelet.

Le lieutenant général de police, qui fut établi par édit du mois de mars
1667, était chargé de veiller à la sûreté de la ville de Paris et de
connaître des délits et contraventions de police. Le premier lieutenant
général de police fut La Reynie. Fontenelle a caractérisé l'importance
et les difficultés de cette charge avec l'ingénieuse précision de son
style\,: «\,Les citoyens d'une ville bien policée jouissent de l'ordre
qui y est établi, sans songer combien, il en coûte de peine à ceux qui
l'établissent ou le conservent, à peu près comme tous les hommes
jouissent de la régularité des mouvements célestes, sans en avoir aucune
connaissance\,; et même plus l'ordre d'une police ressemble par son
uniformité à celui des corps célestes, plus il est insensible, et par
conséquent il est toujours d'autant plus ignoré qu'il est plus parfait.
Mais qui voudrait le connaître, l'approfondir, en serait effrayé.
Entretenir perpétuellement dans une ville telle que Paris une
consommation immense, dont une infinité d'accidents peuvent toujours
tarir quelques sources\,; réprimer la tyrannie des marchands à l'égard
du public, et en même temps animer leur commerce\,; empêcher les
usurpations naturelles des uns sur les autres, souvent difficiles à
démêler\,; reconnaître dans une foule infinie ceux qui peuvent si
aisément y cacher une industrie pernicieuse, en purger la société ou ne
les tolérer qu'autant qu'ils peuvent être utiles par des emplois dont
d'autres qu'eux ne se chargeraient ou ne s'acquitteraient pas si bien\,;
tenir les abus nécessaires dans les bornes précises de la nécessité,
qu'ils sont toujours prêts à franchir\,; les renfermer dans l'obscurité
à laquelle ils doivent être condamnés et ne les en tirer pas même par
des châtiments trop éclatants\,; ignorer ce qu'il vaut mieux ignorer que
punir, et ne punir que rarement et utilement\,; pénétrer par des
souterrains dans l'intérieur des familles et leur garder les secrets
qu'elles n'ont pas confiés, tant qu'il n'est pas nécessaire d'en faire
usage, être présent partout sans être vu\,; enfin, mouvoir ou arrêter à
son gré une multitude immense et tumultueuse, et être l'âme toujours
agissante et presque inconnue de ce grand corps\,: voilà quelles sont en
général les fonctions du magistrat de police. Il ne semble pas qu'un
homme seul y puisse suffire ni par la quantité des choses dont il faut
être instruit, ni par celle des vues qu'il faut suivre, ni par
l'application qu'il faut apporter, ni par la variété des conduites qu'il
faut tenir et des caractères qu'il faut prendre.\,»

Le prévôt des marchands était, à Paris et à Lyon, le chef de
l'administration municipale que l'on nommait maire dans la plupart des
villes. Pendant longtemps ce magistrat fut élu par les bourgeois de
Paris\,; il avait, tant que durait sa charge, le soin de veiller à la
défense de leurs privilèges et de protéger leurs intérêts. Mais les
magistrats royaux diminuèrent peu à peu l'autorité du prévôt des
marchands et ne lui laissèrent enfin que la police municipale. Assisté
des quatre échevins qui formaient le bureau de la ville, le prévôt des
marchands jugeait tous les procès en matière commerciale jusqu'à
l'époque où le chancelier de L'Hôpital établit les juges consuls, qui
formèrent de véritables tribunaux de commerce. C'était le prévôt des
marchands qui répartissait l'impôt de la capitation, fixait le prix des
denrées arrivées par eau et avait la police de la navigation. Les
constructions d'édifices publiés, de ponts, fontaines, remparts,
dépendaient du prévôt des marchands. Enfin ce magistrat portait le titre
de chevalier et avait un rôle important dans les cérémonies publiques et
spécialement aux entrées des rois. Dans ces circonstances il portait,
ainsi que les échevins qui l'accompagnaient, un costume qui rappelait,
par sa singularité, les vêtements du moyen âge. Leurs robes étaient de
deux couleurs, ou, comme on disait alors, mi-parties de rouge et de
violet. Un journal inédit de la Fronde par Dubuisson-Aubenay (Bibl.
Maz., ms. in-fol., H, 1719) nous montre le prévôt des marchands et les
échevins allant dans ce costume à la rencontre de Louis XIV, le 18 août
1649\,: «\,Sur les trois heures, le prévôt des marchands, le sieur
Féron, à cheval en housse de velours, avec sa robe de velours rouge
cramoisi, mi-partie de velours violet cramoisi du côté gauche, précédé
de deux huissiers de l'hôtel de ville aussi à cheval, en housse, vêtus
de robes de drap ainsi mi-parties, et suivi de cinq ou six échevins,
pareillement en housse comme lui et vêtus de robes de velours plein
ainsi mi-parties, et des procureurs du roi et greffier de l'hôtel de
ville, vêtus l'un d'une robe de velours violet cramoisi plein, l'autre
d'une de velours rouge cramoisi plein, aussi en housse, et de près de
cent principaux bourgeois de la ville, aussi à cheval et en housse,
allèrent par ordre jusqu'à la croix qui penche près de Saint-Denys,
au-devant de Sa Majesté.\,»

Aux XVIIe et XVIIIe siècles, l'élection du prévôt des marchands n'était
plus qu'une formalité, comme on peut s'en convaincre en lisant le récit
d'une de ces élections dans le journal de l'avocat Barbier, à la date du
17 août 1750\footnote{T. III, p.~160, du \emph{Journal historique et
  anecdotique} de l'avocat Barbier, publié par la \emph{Société de
  l'Histoire de France}.}.

\hypertarget{note-iii.-mort-de-madame.}{%
\chapter{NOTE III. MORT DE MADAME.}\label{note-iii.-mort-de-madame.}}

Saint-Simon dit que personne n'a douté que Madame (Henriette
d'Angleterre, première femme du frère de Louis XIV) \emph{n'eût été
empoisonnée et même grossièrement}. Il raconte, dans la suite du
chapitre, les détails de ce prétendu empoisonnement. Il est de l'équité
historique de ne pas oublier les témoignages opposés. Nous ne pouvons
que les indiquer rapidement, mais cette note suffira pour prouver que le
doute est, au moins, permis. On y verra aussi que Saint-Simon a eu tort
d'affirmer, comme il le fait p.~225, que Madame était alors \emph{d'une
très bonne santé}.

Presque tous les contemporains, de la mort de Madame, arrivée en 1670,
cinq ans avant la naissance de Saint-Simon, attestent que cette
princesse mourut des suites d'une imprudence qui brisa sa constitution,
depuis longtemps débile et profondément altérée. Les médecins, dont nous
avons les rapports, s'étonnèrent qu'elle n'eût pas succombé plus tôt aux
vices de son organisation, qu'aggravait encore un mauvais régime. Ils
appellent \emph{cholera-morbus} la maladie qui l'emporta en quelques
heures. Valot, premier médecin du roi, soutenait que, depuis trois ou
quatre ans, \emph{elle ne vivait que par miracle}. Ce sont les paroles
mêmes de la dépêche adressée par Hugues de Lyonne à l'ambassadeur de
France en Angleterre\footnote{Voy., cette dépêche et plusieurs autres où
  il est question du même événement dans les \emph{Négociations
  relatives à la succession d'Espagne}, publiées par M. Mignet, t. III,
  p.~207 et suiv. (\emph{Docum. inédits relatifs à l'hist. de France}).}.
Le témoignage des médecins qui furent chargés de faire l'ouverture du
corps de Madame et de rechercher les causes de sa mort fut unanime. Ils
déclarèrent qu'il n'y avait point eu d'empoisonnement, sans quoi
l'estomac en aurait porté des traces, tandis qu'on le trouva en état
excellent.

Si l'attestation officielle des médecins et des ministres paraît
suspecte, on ne peut rejeter le témoignage de contemporains
désintéressés. Mademoiselle\footnote{\emph{Mémoires de Mademoiselle}, à
  l'année 1670.} répète la déclaration des médecins\,: «\,Sur les bruits
que je viens de dire, l'on fit assembler tous les médecins du roi, de
feu Madame et de Monsieur, quelques-uns de Paris, celui de l'ambassadeur
d'Angleterre, avec tous les habiles chirurgiens qui ouvrirent Madame.
Ils lui trouvèrent les parties nobles bien saines ce qui surprit tout le
monde, parce qu'elle était délicate et quasi toujours malade. Ils
demeurèrent d'accord qu'elle était morte d'une bile échauffée.
L'ambassadeur d'Angleterre y était présent, auquel ils firent voir
qu'elle ne pouvait être morte que d'une colique qu'ils appelèrent un
\emph{cholera-morbus}.\,»

Un magistrat, qui notait jour par jour les événements remarquables avec
une complète impartialité, Olivier d'Ormesson, parle de même des causes
de la mort de Madame\,: après avoir rappelé les principaux détails de
cet événement, il écrit dans son \emph{Journal\,:} «\,L'on parla
aussitôt de poison, par toutes les circonstances de la maladie et par le
mauvais ménage qui était entre Monsieur et elle, dont Monsieur était
fort offensé et avait raison. Le soir, le corps fut ouvert en présence
de l'ambassadeur d'Angleterre et de plusieurs médecins qu'il avait
choisis, quelques-uns Anglais avec les médecins du roi. Le rapport fut
que la formation de son corps était très mauvaise, un de ses poumons
attaché au côté et gâté, et le foie tout desséché sans sang, une
quantité extraordinaire de bile répandue dans tout le corps et l'estomac
entier\,; d'où l'on conclut que ce n'est pas poison\,; car l'estomac
aurait été percé et gâté. »

Mais ce qui est plus remarquable, c'est qu'un médecin qui n'avait pas de
caractère officiel et qui, par humeur, était plus porté à soupçonner le
mal qu'à croire au bien, Gui Patin, attribue aussi la mort de Madame à
une cause naturelle. Un mois après l'événement, il
écrivait\footnote{Cette lettre, en date du 30 juillet 1670, ne se trouve
  pas dans l'édition récente du docteur Reveillé-Parise, mais dans celle
  de la Haye (1725, 3 vol.~in-12). Elle a été citée par M. Floquet dans
  ses \emph{Études sur la vie de Bossuet}, t. III, p.~410. On trouve
  réunis dans ce savant ouvrage tous les documents relatifs à la mort de
  Madame.}\,: «\,On parle encore de la mort de M\textsuperscript{me} la
duchesse d'Orléans. Il y en a qui prétendent par une fausse opinion
qu'elle a été empoisonnée. Mais la cause de sa mort ne vient que d'un
mauvais régime de vivre, et de la mauvaise constitution de ses
entrailles. Il est certain que le peuple, qui aime à se plaindre et à
juger de ce qu'il ne connaît pas, ne doit pas être cru en telle matière.
Elle est morte, comme je vous ai dit, par sa mauvaise conduite (mauvais
régime), et faute de s'être bien purgée selon le bon conseil de son
médecin, auquel elle ne croyait guère, ne faisant rien qu'à sa tête. »
Ce qui donne plus d'autorité au témoignage de Gui Patin, c'est que six
ans avant la mort de Madame, dans une lettre du 26 septembre 1664, il
parlait déjà de la mauvaise constitution de cette princesse\,:
«\,M\textsuperscript{me} la duchesse d'Orléans, écrivait-il à Falconet,
est fluette, délicate et du nombre de ceux qu'Hippocrate dit avoir du
penchant à la phtisie. Les Anglais sont sujets à leur maladie de
consomption, qui en est une espèce, une phtisie sèche ou un
flétrissement de poumon.\,» L'autorité de Gui Patin suffirait pour
prouver combien le doute est permis en pareille matière. On peut y
ajouter la \emph{Relation de la maladie, mort et ouverture du corps de
Madame}, par l'abbé Bourdelot\footnote{Cette relation a été publiée par
  Poncet de La Grave dans ses \emph{Mémoires intéressants pour servir à
  l'histoire de France}, t. III, p.~411.}, et l'opinion de Valot sur les
causes de la mort de Madame\footnote{Bibl. de l'Arsenal, ms Conrart, t.
  XIII, p.~779.}. Ces médecins, avec lesquels Gui Patin est si rarement
d'accord, rejettent, comme lui, l'empoisonnement, parmi les contes
populaires.

Je terminerai l'énumération des autorités contemporaines qui repoussent
le bruit de l'empoisonnement de Madame par la lettre de Bossuet, qui
assista cette princesse à ses derniers moments\,; elle est datée de
juillet 1670\footnote{Cette lettre a été publiée par M. Floquet, d'abord
  dans la \emph{Bibliothèque de l'École des Chartes} (2e série, t. Ier,
  p.~114), et ensuite dans ses \emph{Études sur la vie de Bossuet} (t.
  III, p.~416 et suiv.). Elle est tirée des Mémoires manuscrits de
  Philibert de La Mare. Je n'en cite que la partie qui a trait à la
  question examinée dans cette note.}\,: «\,Je crois que vous avez su
que je fus éveillé, la nuit du dimanche au lundi, par ordre de Monsieur,
pour aller assister Madame, qui était à l'extrémité, à Saint-Cloud, et
qui me demandait avec empressement. Je la trouvai avec une pleine
connaissance, parlant et faisant toutes choses sans trouble, sans
ostentation, sans effort et sans violence, mais si bien et si à propos,
avec tant de courage et de piété que j'en suis encore hors de moi. Elle
avoir déjà reçu tous les sacrements, même l'extrême-onction, qu'elle
avait demandée au curé qui lui avait apporté le viatique, et qu'elle
pressait toujours, afin de les recevoir avec connaissance. Je fus une
heure auprès d'elle, et lui vis rendre les derniers soupirs en baisant
le crucifix\footnote{Daniel de Cosnac, qui avait été aumônier de Madame,
  confirme ces détails dans une relation étendue de sa mort. Voy.,
  l'introduction à ses \emph{Mémoires}, publiés par la \emph{Société de
  l'Histoire de France, t}. Ier, p.~xlvii et suiv.}, qu'elle tint à la
main, attaché à sa bouche, tant qu'il lui resta de force. Elle ne fut
qu'un moment sans connaissance. Tout ce qu'elle a dit au roi, à Monsieur
et à tous ceux qui l'environnaient était court, précis, et d'un sens
admirable. Jamais princesse n'a été plus regrettée ni plus admirée\,; et
ce qui est plus merveilleux est que, se sentant frappée, d'abord elle ne
parla que de Dieu, sans témoigner le moindre regret. Quoiqu'elle sût que
sa mort allait être assurément très agréable à Dieu, comme sa vie avoir
été très glorieuse par l'amitié et la confiance de deux grands rois,
elle s'aida, autant qu'elle put, en prenant tous les remèdes avec
cœur\,; mais elle n'a jamais dit un mot de plainte de ce qu'ils
n'opéraient pas, disant seulement \emph{qu'il fallait mourir dans les
formes}.

«\,On a ouvert son corps, avec un grand concours de médecins, de
chirurgiens et de toute sorte de gens, à cause qu'ayant commencé à
sentir des douleurs extrêmes, en buvant trois gorgées d'eau de chicorée,
que lui donna la plus intime et la plus chère de ses femmes, elle avait
dit d'abord \emph{qu'elle était empoisonnée}. M. l'ambassadeur et tous
les Anglais qui sont ici l'avaient presque cru\,; mais l'ouverture du
corps fut une manifeste conviction du contraire, puisque l'on n'y trouva
rien de sain que l'estomac et le cœur, qui sont les premières parties
attaquées par le poison\,; joint que Monsieur, qui avait donné à boire à
M\textsuperscript{me} la duchesse de Meckelbourg\footnote{Élisabeth-Angélique
  de Montmorency-Bouteville, sœur du maréchal de Luxembourg\,; elle
  avait épousé en premières noces Gaspard de Coligny, duc de Châtillon,
  et, en secondes noces, Christian-Louis, duc de Mecklembourg. On disait
  au XVIIe siècle Meckelbourg. Saint-Simon parle plusieurs fois de cette
  personne dans ses Mémoires. Voy., entre autres, t. Ier, p.~81.}, qui
s'y trouva, acheva de boire le reste de la bouteille, pour rassurer
Madame\,; ce qui fut cause que son esprit se remit aussitôt, et qu'elle
ne parla plus de poison que pour dire \emph{qu'elle avait cru d'abord
être empoisonnée par méprise}. Ce sont les propres mots qu'elle dit à M.
le maréchal de Grammont.\,»

De ces témoignages, auxquels on doit joindre celui de
M\textsuperscript{me} de La Fayette, la compagne assidue et l'amie
intime d'Henriette d'Angleterre, dont elle a écrit la vie, on doit
conclure que la duchesse d'Orléans était d'une santé depuis longtemps
altérée, et que la plupart des contemporains ont rejeté le bruit
d'empoisonnement adopté par la crédulité populaire. Ainsi Saint-Simon a
eu tort d'affirmer que personne n'a douté de l'empoisonnement, et
d'ajouter que Madame était alors d'une très bonne santé.

\end{document}
